% Options for packages loaded elsewhere
\PassOptionsToPackage{unicode}{hyperref}
\PassOptionsToPackage{hyphens}{url}
%
\documentclass[
]{book}
\usepackage{amsmath,amssymb}
\usepackage{lmodern}
\usepackage{iftex}
\ifPDFTeX
  \usepackage[T1]{fontenc}
  \usepackage[utf8]{inputenc}
  \usepackage{textcomp} % provide euro and other symbols
\else % if luatex or xetex
  \usepackage{unicode-math}
  \defaultfontfeatures{Scale=MatchLowercase}
  \defaultfontfeatures[\rmfamily]{Ligatures=TeX,Scale=1}
\fi
% Use upquote if available, for straight quotes in verbatim environments
\IfFileExists{upquote.sty}{\usepackage{upquote}}{}
\IfFileExists{microtype.sty}{% use microtype if available
  \usepackage[]{microtype}
  \UseMicrotypeSet[protrusion]{basicmath} % disable protrusion for tt fonts
}{}
\makeatletter
\@ifundefined{KOMAClassName}{% if non-KOMA class
  \IfFileExists{parskip.sty}{%
    \usepackage{parskip}
  }{% else
    \setlength{\parindent}{0pt}
    \setlength{\parskip}{6pt plus 2pt minus 1pt}}
}{% if KOMA class
  \KOMAoptions{parskip=half}}
\makeatother
\usepackage{xcolor}
\usepackage{longtable,booktabs,array}
\usepackage{calc} % for calculating minipage widths
% Correct order of tables after \paragraph or \subparagraph
\usepackage{etoolbox}
\makeatletter
\patchcmd\longtable{\par}{\if@noskipsec\mbox{}\fi\par}{}{}
\makeatother
% Allow footnotes in longtable head/foot
\IfFileExists{footnotehyper.sty}{\usepackage{footnotehyper}}{\usepackage{footnote}}
\makesavenoteenv{longtable}
\usepackage{graphicx}
\makeatletter
\def\maxwidth{\ifdim\Gin@nat@width>\linewidth\linewidth\else\Gin@nat@width\fi}
\def\maxheight{\ifdim\Gin@nat@height>\textheight\textheight\else\Gin@nat@height\fi}
\makeatother
% Scale images if necessary, so that they will not overflow the page
% margins by default, and it is still possible to overwrite the defaults
% using explicit options in \includegraphics[width, height, ...]{}
\setkeys{Gin}{width=\maxwidth,height=\maxheight,keepaspectratio}
% Set default figure placement to htbp
\makeatletter
\def\fps@figure{htbp}
\makeatother
\setlength{\emergencystretch}{3em} % prevent overfull lines
\providecommand{\tightlist}{%
  \setlength{\itemsep}{0pt}\setlength{\parskip}{0pt}}
\setcounter{secnumdepth}{5}
\usepackage[margin=1in]{geometry}
\usepackage{booktabs}
\usepackage{enumitem}

% Without these you get ! LaTeX Error: Too deeply nested.
\setlistdepth{10}
\renewlist{enumerate}{enumerate}{10}
\setlist[itemize]{labelsep=.5em}

% Correct for the way articles/sections are defined and pretty the TOC
\usepackage{fancyhdr}
\usepackage{tocbasic}

\DeclareTOCStyleEntry[%
  entryformat=\bfseries,
  pagenumberformat=\bfseries,
]{tocline}{chapter}
\DeclareTOCStyleEntries[
  pagenumberbox=\hbox,
  dynnumwidth
]{tocline}{%
  chapter,section,subsection,subsubsection,paragraph,subparagraph,%
  figure,table
}
\DeclareTOCStyleEntries[
  dynindent
]{tocline}{subsection,subsubsection,subparagraph}

\renewcommand{\chaptername}{Article}
\renewcommand{\thechapter}{\Roman{chapter}}
\renewcommand{\appendixname}{Exhibit}
\ifLuaTeX
  \usepackage{selnolig}  % disable illegal ligatures
\fi
\usepackage[]{natbib}
\bibliographystyle{plainnat}
\IfFileExists{bookmark.sty}{\usepackage{bookmark}}{\usepackage{hyperref}}
\IfFileExists{xurl.sty}{\usepackage{xurl}}{} % add URL line breaks if available
\urlstyle{same} % disable monospaced font for URLs
\hypersetup{
  pdftitle={NBA Collective Bargaining Agreement - 1995},
  pdfauthor={Robert},
  hidelinks,
  pdfcreator={LaTeX via pandoc}}

\title{NBA Collective Bargaining Agreement - 1995}
\author{Robert}
\date{2023-04-01}

\begin{document}
\maketitle

{
\setcounter{tocdepth}{1}
\tableofcontents
}
\hypertarget{preface}{%
\chapter*{Preface}\label{preface}}
\addcontentsline{toc}{chapter}{Preface}

\emph{If you are only interested in a pdf or epub of the CBA, then please click on the download icon at the top left (it is next to the ``A'') and select the format you wish to download.}

This is the 1995 NBA's Collective Bargaining Agreement (CBA) converted to markdown and disseminated in the format you are currently viewing this as.

The original version of the 1995 CBA can be found on my Github website \href{https://atlhawksfanatic.github.io/}{atlhawksfanatic.github.io} (\href{https://github.com/atlhawksfanatic/atlhawksfanatic.github.io/raw/master/research/CBA/1995-NBA-NBPA-Collective-Bargaining-Agreement.pdf}{pdf found here}) as a way to cross-reference any potential discrepancies.

The original CBA was converted to a text file with \href{https://github.com/ropensci/pdftools}{pdftools}, then broken up by Article and Exhibit through regular expressions into .Rmd files. Those .Rmd files serves as the basis for this bookdown site.

The purpose of this project is three-fold:

\begin{enumerate}
\def\labelenumi{\arabic{enumi}.}
\tightlist
\item
  to historically document collective bargaining agreements of the NBA;
\item
  to provide easier navigation of the CBA through a structured format of each Article and Section; and
\item
  for me to better understand the CBA through this exercise.
\end{enumerate}

I do not own any rights to the CBA and am simply redistributing it in a different format. This is not the official version of the CBA and I am not responsible for any errors that might be present in this document. If you believe you have found an error, please let me know and I will correct it.

Contact: \href{atlhawksfanatic@gmail.com}{via email}, \href{https://github.com/atlhawksfanatic}{through Github}, or \href{https://twitter.com/atlhawksfanatic}{on Twitter}

\hypertarget{definitions}{%
\chapter{DEFINITIONS}\label{definitions}}

\hypertarget{definitions.}{%
\section{Definitions.}\label{definitions.}}

As used in this Agreement, the following terms shall have the following meanings:

\begin{enumerate}
\def\labelenumi{\arabic{enumi}.}
\tightlist
\item
  ``Agreement'' means this Collective Bargaining Agreement made as of September 18, 1995 and entered into July 11, 1996.
\item
  ``Audit Report'' means the audit report prepared in accordance with Article VII, Section 8.
\item
  ``Average Player Salary'' means:

  \begin{enumerate}
  \def\labelenumii{\arabic{enumii}.}
  \tightlist
  \item
    for the 1995-96 and 1996-97 Seasons, total Team Salaries for all NBA teams other than the Expansion Teams, divided by 338; and
  \item
    for the 1997-98 through 2000-01 Seasons, total Team Salaries for all 29 Teams, divided by 363.
  \end{enumerate}
\item
  ``Averaged Contract'' means a Player Contract subject to the averaging rules set forth in Article VII, Section 5(e).
\item
  ``Banked Room'' means the difference, if any, between the Room available to sign a Player Contract and the sum of the Salary plus Unlikely Bonuses in the first Season of such Contract.
\item
  ``Base Year Compensation'' means an amount used to calculate the Exception that results from the assignment of certain Player Contracts, as determined in accordance with Article VII, Section 6(g)(4).
\item
  ``Basketball Related Income'' or ``BRI'' means basketball related income as defined in Article VII, Section 1(a).
\item
  ``Benefits'' means the sum of all amounts paid or to be paid on an accrual basis during any Salary Cap Year by the NBA or NBA Teams, other than Expansion Teams during their first two Seasons, for the specific benefits to be provided to players in accordance with Article IV.
\item
  ``Cash Compensation'' means the sum of Current Cash Compensation and Deferred Cash Compensation attributable to a particular Salary Cap Year.
\item
  ``Contract'' (see ``Uniform Player Contract'').
\item
  ``Commissioner'' means the Commissioner of the NBA.
\item
  ``Current Cash Compensation'' means the component of Current Compensation that is payable in cash or a cash equivalent (e.g., check, money order), excluding signing and performance bonuses.
\item
  ``Current Compensation'' means all components of Salary other than Deferred Compensation.
\item
  ``Deferred Cash Compensation'' means the component of Deferred Compensation that is payable in cash or a cash equivalent.
\item
  ``Deferred Compensation'' means the component of Salary payable to a player during the period commencing after the term covered by the Player Contract, in accordance with the rules set forth in Article VII. The determination of whether compensation is Deferred Compensation will be based upon the time set by the Player Contract for the player to receive the compensation, without regard to whether the obligation is funded currently or secured in any fashion.
\item
  ``Draft'' or ``NBA Draft'' means the NBA's annual draft of Rookie basketball players.
\item
  ``Early Qualifying Veteran Free Agent'' means a Veteran Free Agent who, prior to becoming a Veteran Free Agent, played under one or more Player Contracts covering some or all of each of the two preceding Seasons, and who: (i) either exclusively played with his Prior Team during such two Seasons, or, if he played for more than one Team during such period, changed Teams only (x) by means of assignment, or (y) by signing with his Prior Team during the first of the two Seasons; or (ii) becomes a Veteran Free Agent on either July 1, 1997 or July 1, 1998 and played with his Prior Team for some or all of each of the preceding two Seasons, and who did not change Teams during such two Seasons by signing with his Prior Team as a Veteran Free Agent.
\item
  ``Early Termination Option'' means an option in favor of a player to shorten the stated term of a Player Contract.
\item
  ``Effective Date'' means: (i) with respect to a Contract containing an Option, the June 30 following the Season which, if the Option were not exercised, would be the last Season of the Contract; and (ii) with respect to a Contract containing an Early Termination Option, the June 30 following the Season which, if the Early Termination Option were exercised, would be the last Season of the Contract.
\item
  ``Estimated Average Player Salary'' means, for a particular Season, 108\% of the prior Season's Average Player Salary.
\item
  ``Exception'' means an exception to the rule that a Team's Team Salary may not exceed the Salary Cap.
\item
  ``Expansion Teams'' means the Teams in Vancouver, British Columbia and Toronto, Ontario, Canada that commenced playing in the 1995-96 NBA Season.
\item
  ``Extension'' means an agreement to lengthen the term of a Player Contract, other than pursuant to the exercise of an Option.
\item
  ``First Round Pick'' means a player selected by a Team in the first round of the Draft.
\item
  ``Free Agent'' means: (i) a Veteran Free Agent; (ii) a Rookie Free Agent; or (iii) a Veteran whose Player Contract has been terminated in accordance with the NBA waiver procedure.
\item
  ``Likely Bonus'' means a bonus included in a player's Salary in accordance with Article VII, Section 3(d).
\item
  ``Member'' or ``Team'' means any team that is a member of the NBA.
\item
  ``Minimum Annual Salary'' means the minimum Salary that must be included in a Player Contract for a Season.
\item
  ``Minimum Team Salary'' means the minimum amount in Salary obligations to, or on behalf of, players with respect to an NBA Season that each Team must incur or pay.
\item
  ``Negotiate'' means, with respect to a player or his representatives on the one hand, and a Team or its representatives on the other hand, to engage in any written or oral communication relating to the possible employment, or terms of employment, of such player by such Team as a basketball player, regardless of who initiates such communication.
\item
  ``Non-Cash Compensation'' means the component of Salary that is not paid in cash or a cash equivalent (e.g., game tickets, automobiles, single hotel rooms).
\item
  ``Non-Qualifying Veteran Free Agent'' means a Veteran Free Agent who is not a Qualifying Veteran Free Agent or an Early Qualifying Veteran Free Agent.
\item
  ``Option'' means an option in a Player Contract in favor of a Team or player to extend such Contract beyond its stated term.
\item
  ``Option Buy-Out Amount'' means any amount payable to a player in connection with either the exercise of an Early Termination Option or the non-exercise of an Option.
\item
  ``Player Contract'' (see ``Uniform Player Contract'').
\item
  ``Prior Team'' means the Team for which a player was last under Contract prior to becoming a Qualifying Veteran Free Agent, Early Qualifying Veteran Free Agent or a Non-Qualifying Veteran Free Agent.
\item
  ``Qualifying Veteran Free Agent'' means a Veteran Free Agent who: (i) prior to becoming a Veteran Free Agent, played under one or more Player Contracts covering some or all of each of the three preceding Seasons and either played exclusively with his Prior Team during such three Seasons, or, if he played with more than one Team during such period, changed Teams only (x) by means of assignment, or (y) by signing with his Prior Team during the first of the three Seasons; (ii) became a Veteran Free Agent on July 1,1995; (iii) becomes a Veteran Free Agent on July 1, 1996 and either played exclusively with his Prior Team during some or all of each of the 1994-95 and 1995-96 Seasons, or, if he played with more than one Team during such two Seasons, changed Teams only (x) by means of assignment, or (y) by signing with his Prior Team during the first of the two Seasons; or (iv) becomes a Veteran Free Agent on July 1, 1996 and played with his Prior Team for some or all of each of the two preceding Seasons, or becomes a Veteran Free Agent on July 1, 1997 and played with his Prior Team for some or all of each of the three preceding Seasons, and who did not change Teams during such two Seasons or three Seasons, respectively, by signing with his Prior Team as a Veteran Free Agent.
\item
  ``Regular Salary'' means a player's Salary, less any component thereof that is a signing bonus (or deemed a signing bonus in accordance with Article VII) and any component thereof that is a performance bonus.
\item
  ``Regular Season'' means, with respect to any Season, the period beginning on the first day and ending on the last day of regularly scheduled (as opposed to exhibition or playoff) competition between NBA Teams.
\item
  ``Related Entity'' means any person or entity owning or controlling, or owned by or controlled by, an NBA Team.
\item
  ``Renegotiation'' means a Contract amendment that provides for changes in Salary and/or performance bonuses.
\item
  ``Replacement Player'' means, where appropriate, either a player who is acquired by a Team pursuant to the Assigned Player Exception, or a player who is signed or acquired by a Team pursuant to the Disabled Player Exception.
\item
  ``Required Tender'' means an offer of a Uniform Player Contract, signed by the Team, that: (i) is either personally delivered to the player or his representative or sent by prepaid certified, registered, or overnight mail to the last known address of the player or his representative; (ii) with respect to a First Round Pick, (A) provides the player with at least until the first day of the following Regular Season to accept, (B) has a stated term of three Seasons, (C) in each such Season, calls for at least 80\% of the Rookie Scale Amount then applicable to the player, and (D) in each such Season, provides for salary protection for lack of skill and insured or non-insured injury or illness of not less than 80\% of the Rookie Scale Amount then applicable to the player; and (iii) with respect to a Second Round Pick, (A) provides the player with at least 30 days to accept, (B) has a stated term of one Season, and (C) calls for at least the Minimum Annual Salary then applicable to the player.
\item
  ``Rookie'' means a person who has never signed a Player Contract with an NBA Team.

  \begin{enumerate}
  \def\labelenumii{\arabic{enumii}.}
  \tightlist
  \item
    ``Draft Rookie'' means a Rookie who is selected in the NBA Draft.
  \item
    ``Non-Draft Rookie'' means a Rookie who is not selected in the NBA Draft for which he is first eligible.
  \end{enumerate}
\item
  ``Rookie Free Agent'' means: (i) a Draft Rookie who, pursuant to the provisions of Article X, is no longer subject to the exclusive negotiating rights of any Team, and who may be signed by any Team; or (ii) a Non-Draft Rookie.
\item
  ``Rookie Scale Amounts'' means the amounts set forth in the tables annexed hereto as Exhibit B.
\item
  ``Rookie Scale Contract'' means the initial NBA contract entered into between a First Round Pick and the Team that holds his draft rights in accordance with Article VIII, Section 1 or 2.(av) ``Room'' means the extent to which: (i) a Team's then-current Team Salary is less than the Salary Cap; or (ii) a Team is entitled to use one of the Salary Cap Exceptions set forth in Article VII, Section 6(c), (d) and (g) (Disabled Player, \$1 Million and Assigned Player Exceptions).
\item
  ``Salary'' means, with respect to a Salary Cap Year, the compensation in money, property, investments, or anything else of value earned by, or paid or payable to, an NBA player (including players whose Player Contracts have been terminated in accordance with the NBA's waiver procedure) or to a person or entity designated by a player, in accordance with a Player Contract, plus any other amount that is deemed to constitute Salary in accordance with the terms of this Agreement, not including any benefits the player received in accordance with the terms of Article IV and any portion of the player's compensation that is attributable to another Salary Cap Year in accordance with this Agreement. Salary also includes any consideration received by a retired player that is deemed to constitute Salary in accordance with the terms of Article XIII.
\item
  ``Salary Cap'' means the maximum allowable Team Salary for each Team for a Salary Cap Year, subject to the rules and exceptions set forth in this Agreement.
\item
  ``Salary Cap Year'' means the period from July 1 to the following June 30.
\item
  ``Season'' or ``NBA Season'' means the period beginning on the first day of training camp and ending immediately after the last game of the NBA Finals.
\item
  ``Second Round Pick'' means a player selected by a Team in the second round of the Draft.
\item
  ``Team'' or ``NBA Team'' (see ``Member'').
\item
  ``Team Salary'' means, with respect to a Salary Cap Year, the sum of a Team's Salary obligations plus other amounts as computed in accordance with Article VII, less applicable credit amounts as computed in accordance with Article VII.
\item
  ``Total Salaries and Benefits'' means the total amount of Salaries and Benefits paid or payable by all NBA Teams for or with respect to a Salary Cap Year in accordance with this Agreement, other than the Salaries and Benefits paid by the Expansion Teams during their first two Seasons, as determined in accordance with Article VII. For purposes of this definition only, total Salaries shall include all performance bonuses excluded from Salaries in accordance with Article VII, Section 3(d) but actually earned by NBA players during such Salary Cap Year, and shall exclude all performance bonuses included in Salaries in accordance with Article VII, Section 3(d) but not actually earned by NBA players during such Salary Cap Year.
\item
  ``Traded Player'' means a player whose Player Contract is assigned by one Team to another Team other than by means of the NBA waiver procedure.
\item
  ``20\% Rule'' means the rule limiting permissible Salary increases set forth in Article VII, Section 5(c).
\item
  ``Unlikely Bonus'' means a bonus excluded from a player's Salary in accordance with Article VII, Section 3(d).
\item
  ``Uniform Player Contract'' or ``Player Contract'' or ``Contract'' means the standard form of written agreement between a person and a Member required for use in the NBA by Article II below, pursuant to which such person is employed by such Member as a professional basketball player.
\item
  ``Veteran'' or ``Veteran Player'' means a person who has signed at least one Player Contract with an NBA Team.
\item
  ``Veteran Free Agent'' means a Veteran who completed his Player Contract by rendering the playing services called for thereunder.
\end{enumerate}

\hypertarget{uniform-player-contract}{%
\chapter{UNIFORM PLAYER CONTRACT}\label{uniform-player-contract}}

\hypertarget{required-form.}{%
\section{Required Form.}\label{required-form.}}

The Player Contract to be entered into by each player and the Team by which he is employed shall be a Uniform Player Contract in the form annexed hereto as Exhibit A.

\hypertarget{limitation-on-amendments.}{%
\section{Limitation on Amendments.}\label{limitation-on-amendments.}}

\begin{enumerate}
\def\labelenumi{(\alph{enumi})}
\tightlist
\item
  Except as provided in Sections 3, 8, and 9 of this Article, no amendments to the form of Uniform Player Contract provided for by Section 1 of this Article shall be permitted.
\item
  If a Team and a player enter into (i) a Uniform Player Contract containing an amendment not specifically permitted by this Agreement or (ii) a subsequent amendment to an existing Player Contract where such amendment is not specifically permitted by this Agreement, then such Contract or subsequent amendment, as the case may be, shall be disapproved by the Commissioner and, consequently, rendered null and void.
\end{enumerate}

\hypertarget{allowable-amendments.}{%
\section{Allowable Amendments.}\label{allowable-amendments.}}

In their individual contract negotiations, a player and a Team may amend the provisions of a Uniform Player Contract, but only in the following respects:

\begin{enumerate}
\def\labelenumi{(\alph{enumi})}
\tightlist
\item
  By agreeing upon provisions (to be set forth in Exhibit 1 to a Uniform Player Contract) with respect to the Cash Compensation to be paid or amounts to be loaned to the player for rendering the services described in such Contract.
\item
  By agreeing upon provisions (to be set forth in Exhibit 1 to a Uniform Player Contract) with respect to any form of Non-Cash Compensation to be paid or provided to the player for rendering the services described in such Contract.
\item
  By agreeing upon provisions (to be set forth in Exhibit 1 to a Uniform Player Contract) with respect to bonuses, or increases or reductions in Cash Compensation, for (i) the player's execution of a Uniform Player Contract (a ``signing bonus''), (ii) the exercise or non-exercise of an option pursuant to Articles VII and XII, (iii) the player's achievement (in the case of a bonus or increase in Cash Compensation) of agreed-upon benchmarks relating to his performance as a player, (iv) the player's achievement (in the case of a bonus or increase in Cash Compensation) or non-achievement (in the case of a reduction in Cash Compensation) of agreed-upon benchmarks relating to his physical condition or academic achievement, or (v) the Team's performance during a particular NBA Season (in the case of a bonus or increase in Cash Compensation), subject to the limitations imposed by paragraph 3(c) of the Uniform Player Contract. Any amendment agreed upon pursuant to subsections (c)(iii), (iv), or (v) above must be structured so as to provide an incentive for positive achievement by the player (under (c) (iii) and (iv)) and the Team (under (c)(v)).
\item
  By agreeing upon a Salary payment schedule (to be set forth in Exhibit 1 to a Uniform Player Contract) different from that provided for by paragraph 3(a) of the prescribed form of Uniform Player Contract; provided, however, that, if the Salary to be paid for the Season with respect to which such payment schedule applies is not greater than the Minimum Annual Salary called for with respect to that Season pursuant to Article II, Section 6, such different payment schedule shall be more favorable to the player than that provided for by paragraph 3(a) of the prescribed form of Uniform Player Contract.
\item
  By agreeing upon provisions (to be set forth in Exhibit 2 to a Uniform Player Contract) that the Cash Compensation provided for by a Uniform Player Contract (as described in Exhibit 1 to such Contract) shall be, in whole or in part, and subject to any conditions or limitations, protected or insured (as provided for by, and in accordance with the definitions set forth in, Section 4 of this Article) in the event that such Contract is terminated by the Team by reason of the player's:

  \begin{enumerate}
  \def\labelenumii{(\roman{enumii})}
  \tightlist
  \item
    lack of skill;
  \item
    personal conduct;
  \item
    death not covered by an insurance policy procured by a Team for the player's benefit (``non-insured death'');
  \item
    death covered by an insurance policy procured by a Team for the player's benefit (``insured death'');
  \item
    disability or unfitness to play skilled basketball resulting from a basketball-related injury not covered by an insurance policy procured by a Team for the player's benefit (``non-insured basketball-related injury'');
  \item
    disability or unfitness to play skilled basketball resulting from any injury or illness not covered by an insurance policy procured by a Team for the player's benefit (``non-insured injury or illness'');
  \item
    disability or unfitness to play skilled basketball resulting from an injury or illness covered by an insurance policy procured by a Team for the player's benefit (``insured injury or illness'');
  \item
    mental disability not covered by an insurance policy procured by a Team for the player's benefit (``non-insured mental disability''); and/or
  \item
    mental disability covered by an insurance policy procured by a Team for the player's benefit (``insured mental disability'').(f) By agreeing upon a provision (to be set forth in Exhibit 3 to a Uniform Player Contract) limiting or eliminating the player's right to receive his Cash Compensation (in accordance with paragraphs 7(c) and 16(b) of the prescribed form of Uniform Player Contract) when the player's disability or unfitness to play skilled basketball is caused by the re-injury of an injury sustained prior to, or by the aggravation of a condition that existed prior to, the execution of the Uniform Player Contract providing for such Cash Compensation.
  \end{enumerate}
\item
  By agreeing upon a provision (to be set forth in Exhibit 4 to a Uniform Player Contract) entitling a player to receive Cash Compensation upon the sale, exchange, assignment, or transfer of such player's Uniform Player Contract, subject, however, to the provisions of Article XXIV.
\item
  By agreeing upon a provision (to be set forth in Exhibit 5 to a Uniform Player Contract) permitting the player to participate or engage in some or all of the activities otherwise prohibited by paragraph 12 of the prescribed form of Uniform Player Contract; provided, however, that paragraph 12 of the prescribed form of Uniform Player Contract may not be amended to permit a player to participate in any public game or public exhibition of basketball not approved in accordance with Article XXIII of this Agreement.
\item
  By agreeing upon a provision (to be set forth in Exhibit 6 to a Uniform Player Contract) that conditions the validity of the Contract on the player's ability to pass, in the sole discretion of a physician designated by the Team, a physical examination conducted within forty-eight (48) hours of the execution of the Contract.
\item
  By agreeing to delete clauses (b )(ii) and/or (b )(iii) of paragraph 5 of the prescribed form of Uniform Player Contract in their entirety.
\item
  By agreeing to delete paragraph 7(b) of the prescribed form of Uniform Player Contract in its entirety and substituting therefor the provision set forth in Exhibit 7 to a Uniform Player Contract.
\item
  By agreeing either (i) to delete paragraph 13(b) of the prescribed form of Uniform Player Contract in its entirety, or (ii) to delete the last sixteen words of paragraph 13(b) of such Contract.
\end{enumerate}

\hypertarget{salary-protection-or-insurance.}{%
\section{Salary Protection or Insurance.}\label{salary-protection-or-insurance.}}

\begin{enumerate}
\def\labelenumi{(\alph{enumi})}
\tightlist
\item
  \textbf{Lack of Skill.} When a Team agrees to protect, in whole or in part, the Cash Compensation provided for by a Uniform Player Contract in the event such Contract is terminated by the Team, pursuant to paragraph 16(a)(ii) thereof, by reason of the player's lack of skill, such agreement shall mean that, subject to any conditions or limitations set forth in Exhibit 2 to the Uniform Player Contract, notwithstanding the provisions of paragraphs 16(a)(ii), 16(d), 16(e), and 16(g) of such Contract, the termination of such Contract by the Team on account of the player's failure to exhibit sufficient skill or competitive ability shall in no way affect the player's right to receive the Cash Compensation payable pursuant to Exhibit 1 to such Contract in the amounts and at the times called for by such Exhibit.
\item
  \textbf{Personal Conduct.} When a Team agrees to protect, in whole or in part, the Cash Compensation provided for by a Uniform Player Contract in the event such Contract is terminated by the Team, pursuant to paragraph 16(a)(i) thereof, by reason of the player's personal conduct, such agreement shall mean that, subject to any conditions or limitations set forth in Exhibit 2 to the Uniform Player Contract, notwithstanding the provisions of paragraphs 16(a)(i), 16(d), 16(e), and 16(g) of such Contract, the termination of such Contract by the Team on account of the player's failure, refusal or neglect to conform his personal conduct to standards of good citizenship and/or the player's failure, refusal or neglect to conform his personal conduct to standards of good sportsmanship and/or the player's failure, refusal or neglect to obey the Team's training rules shall in no way affect the player's right to receive the Cash Compensation payable pursuant to Exhibit 1 to such Contract in the amounts and at the times called for by such Exhibit.
\item
  \textbf{Non-Insured Death.} When a Team agrees to protect, in whole or in part, the Cash Compensation provided for by a Uniform Player Contract in the event such Contract is terminated by the Team, pursuant to paragraph 16(a)(iii) thereof, by reason of the player's non-insured death, such agreement shall mean that, subject to any conditions or limitations set forth in Exhibit 2 to the Uniform Player Contract, notwithstanding the provisions of paragraphs 16(a), 16(b), 16(c), 16(d), 16(e), and 16(g) of such Contract, the termination of such Contract by the Team on account of the player's failure to render his services thereunder, if such failure has been caused by the player's death, shall in no way affect the player's (or his estate's or duly appointed beneficiary's) right to receive the Cash Compensation payable pursuant to Exhibit 1 to such Contract in the amounts and at the times called for by such Exhibit; provided, however, that (i) such death does not result from the player's participation in activities prohibited by paragraph 12 of the Uniform Player Contract (as such paragraph may be modified by Exhibit 5 to the Player Contract), suicide, the abuse of alcohol, or the use of any controlled substance; (ii) at the time of the player's failure to render playing services, the player is not in material breach of such Contract; (iii) if the Team, for its own benefit, seeks to procure an insurance policy covering the player's death, the player cooperates with the Team in procuring such an insurance policy; and (iv) if the Team, for its own benefit, has procured such an insurance policy, the player (and/or his estate and/or duly appointed beneficiary) cooperates with the Team and insurance company in the processing of the Team's claim under such policy.
\item
  \textbf{Insured Death.} When a Team agrees to protect, in whole or in part, the Cash Compensation provided for by a Uniform Player Contract in the event such Contract is terminated by the Team, pursuant to paragraph 16(a)(iii) thereof, by reason of the player's insured death, such agreement shall mean that, subject to any conditions set forth in Exhibit 2 and/or Exhibit 3 to the Uniform Player Contract, the Team has procured (or will procure forth-with) an insurance policy (specifically designated in Exhibit 2 to such Contract) for the benefit of the player Of his estate or beneficiary that, subject to the conditions and limitations contained in the policy, would pay a benefit in the event of the player's death in an amount equal to or less than the Cash Compensation remaining to be paid to the player under Exhibit 1 of his Player Contract at the time of his death; provided, however, that (i) such death does not result from the player's participation in activities prohibited by paragraph 12 of the Uniform Player Contract (as such paragraph may be modified by Exhibit 5 to the Player Contract), suicide, the abuse of alcohol, or the use of any controlled substance; and (ii) at the time of the player's failure to render playing services, the player is not in material breach of such Contract.
\item
  \textbf{Non-Insured Basketball-Related Injury.} When a Team agrees to protect, in whole or in part, the Cash Compensation provided for by a Uniform Player Contract in the event such Contract is terminated by the Team, pursuant to paragraphs 7(c), 16(b), and/or 16(c) thereof, by reason of the player's disability or unfitness to play skilled basketball resulting from a non-insured basketball-related injury, such agreement shall mean that, subject to any conditions or limitations set forth in Exhibit 2 and/or Exhibit 3 to the Uniform Player Contract, notwithstanding the provisions of paragraphs 7(b), 7(c), 16(a)(iii), 16(b), 16(c), 16(d), and 16(g) of such Contract, the termination of such Contract by the Team because the player has been disabled and/or is unfit to play skilled basketball as a direct result of an injury sustained while participating in any basketball practice or game played for the Team shall in no way affect the player's right to receive the Cash Compensation payable pursuant to Exhibit 1 to such Contract in the amounts and at the times called for by such Exhibit; provided, however, that (i) such injury does not result from an attempted suicide or the use of any controlled substance; (ii) at the time of the player's termination, the player is not in material breach of such Contract; (iii) if the Team, for its own benefit, seeks to procure an insurance policy covering the player's injury, the player cooperates with the Team in procuring such an insurance policy; and (iv) if the Team, for its own benefit, has procured such an insurance policy, the player cooperates with the Team and the insurance company in the processing of the Team's claim under such policy.
\item
  \textbf{Non-Insured Injury or Illness.} When a Team agrees to protect, in whole or in part, the Cash Compensation provided for by a Uniform Player Contract in the event such contract is terminated by the Team, pursuant to paragraphs 7 (c), 16(b) and/or 16(c) thereof, by reason of the player's disability or unfitness to play skilled basketball resulting from any non-insured injury or illness, such agreement shall mean that, subject to any conditions or limitations set forth in Exhibit 2 and/or Exhibit 3 to the Uniform Player Contract, notwithstanding the provisions of paragraphs 7(b), 7(c), 16(a)(iii), 16(b), 16(c), 16(d), and 16(g) of such Contract, the termination of such Contract by the Team on account of an injury, illness, or disability suffered or sustained by the player shall in no way affect the player's right to receive the Cash Compensation payable pursuant to Exhibit 1 to such Contract in the amounts and at the times called for by such Exhibit; provided, however, that (i) such injury, illness, or disability does not result from the player's participation in activities prohibited by paragraph 12 of the Uniform Player Contract (as such paragraph may be modified in Exhibit 5 to the Player Contract), attempted suicide, the abuse of alcohol, or the use of any controlled substance; (ii) at the time of such injury, illness, or disability the player is not in material breach of such Contract; (iii) if the Team, for its own benefit, seeks to procure an insurance policy covering the player's injury and/or illness, the player cooperates with the Team in procuring such an insurance policy; and (iv) if the Team, for its own benefit, has procured such an insurance policy, the player cooperates with the Team and insurance company in the processing of the Team's claim under such policy.
\item
  \textbf{Insured Injury or Illness.} When a Team agrees to protect, in whole or in part, the Cash Compensation provided for by a Uniform Player Contract in the event such Contract is terminated by the Team, pursuant to paragraphs 7(c), 16(b) and/or 16(c) thereof, by reason of the player's disability or unfitness to play skilled basketball resulting from an insured injury or illness, such agreement shall mean that, subject to any conditions or limitations set forth in Exhibit 2 and/or Exhibit 3 to the Uniform Player Contract, the Team has procured (or will procure forthwith) an insurance policy (specifically designated in Exhibit 2 to such Contract) for the benefit of the player or his estate or beneficiary that, subject to the conditions and limitations contained in the policy, would pay a benefit in the event of the player's disability or unfitness to play skilled basketball resulting from an injury or illness in an amount equal to or less than the Cash Compensation remaining to be paid to the player under Exhibit 1 of his Player Contract at the time of his termination; provided, however, that (i) such injury or illness does not result from the player's participation in activities prohibited by paragraph 12 of the Uniform Player Contract (as such paragraph may be modified by Exhibit 5 to the Player Contract), attempted suicide, the abuse of alcohol, or the use of any controlled substance; and (ii) at the time of the player's termination, the player is not in material breach of such Contract.
\item
  \textbf{Non-Insured Mental Disability.} When a Team agrees to protect, in whole or in part, the Cash Compensation provided for by a Uniform Player Contract in the event such Contract is terminated by the Team, pursuant to paragraph 16(a)(iii) thereof, by reason of the player's non-insured mental disability, such agreement shall mean that, subject to any conditions or limitations set forth in Exhibit 2 and/or Exhibit 3 to the Uniform Player Contract, notwithstanding the provisions of paragraphs 16(a), 16(b), 16(c), 16(d), 16(e), and 16(g) of such Contract, the termination of such Contract by the Team on account of the player's failure to render his services thereunder, if such failure has been caused by the player's mental disability, shall in no way affect the player's (or his duly appointed legal representative's) right to receive the Cash Compensation payable pursuant to Exhibit 1 to such Contract in the amounts and at the times called for by such Exhibit; provided, however, that (i) such mental disability does not result from the player's attempted suicide or the use of any controlled substance; (ii) at the time of the player's failure to render playing services, the player is not in material breach of such Contract; (iii) if the Team, for its own benefit, seeks to procure an insurance policy covering the player's mental disability, the player (and/or his duly appointed legal representative) cooperates with the Team in procuring such an insurance policy; and (iv) if the Team, for its own benefit, has procured an insurance policy covering the player's mental disability, the player (and/or his duly appointed legal representative) cooperates with the Team and insurance company in the processing of the Team's claim under such policy.
\item
  \textbf{Insured Mental Disability.} When a Team agrees to protect, in whole or in part, the Cash Compensation provided for by a Uniform Player Contract in the event such Contract is terminated by the Team, pursuant to paragraph 16(a)(iii) thereof, by reason of the player's insured mental disability, such agreement shall mean that, subject to any conditions set forth in Exhibit 2 and/or Exhibit 3 to the Uniform Player Contract, the Team has procured (or will procure forthwith) an insurance policy (specifically designated in Exhibit 2 to such Contract) for the benefit of the player or his estate or beneficiary that, subject to the conditions and limitations contained in the policy, would pay a benefit in the event of the player's mental disability in an amount equal to or less than the Cash Compensation remaining to be paid to the player under Exhibit 1 of his Player Contract at the time of his termination; provided, however, that (i) such mental disability does not result from the player's participation in activities prohibited by paragraph 12 of the Uniform Player Contract (as such paragraph may be modified by Exhibit 5 to the Player Contract), attempted suicide or the use of any controlled substance; and (ii) at the time of the player's termination, the player is not in material breach of such Contract.
\item
  No agreement by a Team to protect, in whole or in part, the Cash Compensation provided for by a Uniform Player Contract shall require (or be construed as requiring) such Team to continue the player on the Team, Active List, or Roster; nor shall any such agreement afford the player any right to continue, or to be deemed as having continued, on such Team, Active List, or Roster for any purpose.
\item
  When a Team agrees to protect, in whole or in part, the Cash Compensation provided for by a Uniform Player Contract, and such protection is contingent on the satisfaction of a condition expressly set forth in Exhibit 2 to that Contract, such protection shall be applicable and effective only if the Player Contract has not previously been terminated at the time such condition is satisfied.
\item
  When a Team agrees to protect, in whole or in part, the Cash Compensation provided for in any option year (in favor of the team or the player) included in a Uniform Player Contract, such protection shall be applicable and effective only if the option to extend the term provided for in the Contract is exercised.
\end{enumerate}

\hypertarget{conformity.}{%
\section{Conformity.}\label{conformity.}}

All currently effective Player Contracts, and all Player Contracts entered into following the execution of this Agreement that do not otherwise so provide, shall be deemed amended in such manner to require the parties to comply with all terms of this Agreement, including the terms of the Uniform Player Contract annexed hereto as Exhibit A. All Player Contracts shall be subject to the terms of this Agreement, which shall supersede the terms of any Player Contract inconsistent herewith. No Player Contract shall provide for the waiver by a player or a Team of any benefits or the sacrifice of any rights to which the player or the Team is entitled by virtue of a Uniform Player Contract or this Agreement.

\hypertarget{minimum-annual-salary.}{%
\section{Minimum Annual Salary.}\label{minimum-annual-salary.}}

\begin{enumerate}
\def\labelenumi{(\alph{enumi})}
\tightlist
\item
  Except with respect to Ten-Day Contracts provided for in Article II, Section 8, and Rest-of-Season Contracts provided for in Article II, Section 9, no Player Contract shall provide for a Salary of less than the following:

  \begin{enumerate}
  \def\labelenumii{(\roman{enumii})}
  \tightlist
  \item
    For the 1995-96 Season: \$225,000
  \item
    For the 1996-97 Season: \$247,500
  \item
    For the 1997-98 through 2000-01 Seasons:
    (A) Prior to the issuance of the Audit Report for the prior Season, the Minimum Annual Salary shall be the Minimum Annual Salary for the prior Season increased by ten percent (10\%).
    (B) Following the issuance of the Audit Report for the prior Season, the Minimum Annual Salary shall be the Minimum Annual Salary for the prior Season increased by the greater of ten percent (10\%) or the percentage by which BRI for the prior Season increased over BRI for the Season immediately preceding such prior Season. Once the Audit Report for the prior Season is issued, any Player Contract that provides for the Minimum Annual Salary calculated pursuant to subsection (A) above shall be deemed amended to provide for the Minimum Annual Salary calculated pursuant to this subsection (B).
  \end{enumerate}
\item
  Notwithstanding the provisions of Section 6(a) of this Article, no Player Contract between a Team and a player selected in the second round of the NBA Draft, or between a Team and a player not selected in the NBA Draft for which he is first eligible, shall provide, with respect to the player's first NBA Season, for a Salary of less than the following:

  \begin{enumerate}
  \def\labelenumii{(\roman{enumii})}
  \tightlist
  \item
    For the 1995-96 Season: \$200,000
  \item
    For the 1996-97 Season: \$220,000
  \item
    For the 1997-98 through 2000-01 Seasons:
    (A) Prior to the issuance of the Audit Report for the prior Season, the Minimum Annual Salary shall be the Minimum Annual Salary for the prior Season increased by ten percent (10\%).
    (B) Following the issuance of the Audit Report for the prior Season, the Minimum Annual Salary shall be the Minimum Annual Salary for the prior Season increased by the greater of ten percent (10\%) or the percentage by which BRI for the prior Season increased over BRI for the Season immediately preceding such prior Season. Once the Audit Report for the prior Season is issued, any Player Contract that provides for the Minimum Annual Salary calculated pursuant to subsection (A) above shall be deemed amended to provide for the Minimum Annual Salary calculated pursuant to this subsection (B).
  \end{enumerate}
\item
  In determining whether a Player Contract satisfies the Minimum Annual Salary requirements established by this Section, the allocation of signing bonuses, deemed signing bonuses (pursuant to Article VII, Section 3(b », and other bonuses not contingent upon a player's or a Team's attaining a particular level of performance shall be considered as part of the Salary provided for by a Player Contract, provided that such Player Contract makes clear that the Salary for each Season (including bonuses) equals or exceeds the Minimum Annual Salary for such Season.
\item
  Nothing in this Section shall alter the respective rights and liabilities of a player and a Team, as provided for in the Uniform Player Contract or in this Agreement, with respect to the termination of a Player Contract.
\end{enumerate}

\hypertarget{promotional-activities.}{%
\section{Promotional Activities.}\label{promotional-activities.}}

\begin{enumerate}
\def\labelenumi{(\alph{enumi})}
\tightlist
\item
  A player may not enter into any contract or other commercial arrangement to perform any of the promotional activities set forth in paragraph 13(b) of the Uniform Player Contract (regardless of whether, in connection with such activities, the player is in any way identified with the Team) on behalf of an entity or person that offers a product or service that competes in the same product or service category as the products or services of a Protected Team Sponsor (as defined in subsection (i) below). The designation of Protected Team Sponsors shall be governed by the following:

  \begin{enumerate}
  \def\labelenumii{(\roman{enumii})}
  \tightlist
  \item
    Prior to each NBA Season during the term of this Agreement, a Team may designate no more than two (2) sponsors of the Team that it seeks to protect against conflicting commercial or promotional activities in the Team's local marketing area involving that Team's players (``Protected Team Sponsors''). The designation of a sponsor as a Protected Team Sponsor shall last for one year, from September 1 to the next August 31.
  \item
    If a Team chooses to designate Protected Team Sponsor(s), the Team must notify its players of such designation, and of the product or service categories in which its Protected Team Sponsor(s) compete, on or before the September 1 immediately preceding the year in which the Protected Team Sponsor designation will apply.
  \item
    A Team may not designate a sponsor as a Protected Team Sponsor unless the sponsor conducts business in a product or service category set forth in Exhibit E to this Agreement. Local automobile dealerships and retail shoe outlets shall not be designated as Protected Team Sponsors. Nothing in this Section 7(a) shall prohibit a player from entering into a contract or other commercial arrangement to perform national promotional activities on behalf of an entity or person that conducts business on a nationwide basis.
  \end{enumerate}
\item
  A player's obligation (pursuant to paragraph 13(e) of a Uniform Player Contract) to participate, upon request, in all other reasonable promotional activities of the Team and the Association shall be deemed satisfied if, during each year of the period covered by such Contract, the Player makes six individual personal appearances and six group appearances for or on behalf of or at the request of the Team (or Team Affiliate) by which he is employed and/or the NBA. Up to two of these twelve appearances may be assigned by the Team and/or the NBA in any year to NBA Properties. The Player shall be reimbursed for the actual expenses incurred in connection with any such appearance, provided that such expenses result directly from the appearance and are ordinary and reasonable. The Player shall also receive compensation from the Team by which he is employed of at least \$1,000, in accordance with paragraph 13(e) of the Uniform Player Contract, for each promotional appearance he makes for a commercial sponsor of such Team. Any personal or group appearance required under this subsection (b) must:

  \begin{enumerate}
  \def\labelenumii{(\roman{enumii})}
  \tightlist
  \item
    take place during (A) the period from the first day of a Season through the day of the NBA Draft following such Season, or (B) during the off-season, provided that no player may be required to make more than one off-season appearance in any year covered by his Contract and no player may be required to make such an off-season appearance unless he resides in or is otherwise located in the area where the appearance is to take place;
  \item
    occur in the home city (or geographic vicinity thereof) of the player's Team (subject to subsection (b)(i)(B) above) or in a city (or geographic vicinity thereof) to which the player has traveled to play in a scheduled NBA game;
  \item
    not occur at a time that would interfere with a player's reasonable preparation to play on the day of a Team game;
  \item
    not exceed a reasonable period of time; and
  \item
    not require the player to sign autographs as the primary purpose of the appearance.
  \end{enumerate}
\end{enumerate}

\hypertarget{ten-day-contracts.}{%
\section{Ten-Day Contracts.}\label{ten-day-contracts.}}

\begin{enumerate}
\def\labelenumi{(\alph{enumi})}
\tightlist
\item
  Beginning on January 5 (if a business day or the next succeeding business day) of any NBA Season, and solely for the purpose of replacing an injured player, a Team may enter into a Player Contract with a player, which Contract may, notwithstanding the provisions of paragraphs 7(c), 16(b), and 16(c) of the Uniform Player Contract, provide that such player will be compensated only for the period actually spent in the service of such Team. The duration of such Contract (a ``lO-Day Contract'') shall be limited to ten (10) days or a period encompassing three (3) games played by such Team, whichever is longer, and no Team may enter into such a Contract with the same player more than twice during the course of anyone Season.
\item
  The Salary payable pursuant to a lO-Day Contract shall not be less than an amount calculated by multiplying the Minimum Annual Salary (as set forth in Article II, Sections 6(a) and (b)) for the NBA Season in which such Contract is executed by a fraction, the numerator of which is the number of days spent by such player in the service of such Team and the denominator of which is the total number of days of that NBA Regular Season.
\item
  Notwithstanding anything to the contrary contained in a Uniform Player Contract, a lO-Day Contract may be terminated by written notice to the player and payment of only such sums as required by such Contract.
\end{enumerate}

\hypertarget{rest-of-season-contracts.}{%
\section{Rest-of-Season Contracts.}\label{rest-of-season-contracts.}}

\begin{enumerate}
\def\labelenumi{(\alph{enumi})}
\tightlist
\item
  At any time after the start of an NBA Regular Season, a Team may enter into a Player Contract that may provide compensation to a player only for the remainder of that Season (a ``Rest-of-Season Contract'').
\item
  The Salary payable pursuant to a Rest-of-Season Contract shall not be less than an amount calculated by multiplying the Minimum Annual Salary (as set forth in Article II, Sections 6(a) and (b)) for the NBA Season in which such Contract is executed by a fraction, the numerator of which is the number of days remaining in the NBA Regular Season and the denominator of which is the total number of days of that NBA Regular Season.
\end{enumerate}

\hypertarget{general.}{%
\section{General.}\label{general.}}

\begin{enumerate}
\def\labelenumi{(\alph{enumi})}
\item
  \begin{enumerate}
  \def\labelenumii{(\roman{enumii})}
  \tightlist
  \item
    Any oral or written agreement between a player and a Team concerning terms and conditions of employment shall be reduced to writing in the form of a Uniform Player Contract or an amendment thereto as soon as practicable. Immediately upon the consummation of any such agreement, the Team shall notify the NBA by facsimile or e-mail and provide the NBA with all economic terms of such agreement. As soon as practicable, but no later than three (3) business days following receipt of such notice by the NBA, the NBA shall provide the same notice to the Players Association.
  \item
    Notwithstanding subsection (a)(i) above, neither the NBA nor the Players Association shall contend that any agreement concerning terms and conditions of employment is binding upon the Player or the Team until a Player Contract embodying such terms and conditions has been duly executed by the parties. Nothing herein is intended to affect (A) any authority of the Commissioner to approve or disapprove Player Contracts, or (B) the effect of the Commissioner's approval or disapproval on the validity of such Player Contracts.
  \item
    A violation of the first sentence of subsection (a) (i) above may be considered evidence of a violation of Article XIII.
  \end{enumerate}
\item
  No player shall attend the regular training camp of any Team, or participate in organized practices with the Team at any time, unless he is a party to a Player Contract then in effect. For purposes of this Section 10(b)), a player shall be considered to be a party to a Player Contract then in effect if such Contract has been extended in accordance with an option in favor of the Team or player permitted by this Agreement.
\item
  No Team shall make any direct or indirect payment of any money, property, investments, loans, or anything else of value for fees or otherwise to an agent, attorney, or representative of a player (for or in connection with such person's representation of such player); nor shall any Player Contract provide for such payment. The foregoing shall not, however, prevent a Team from sending a player's regular paycheck to a player's agent, attorney, or representative if so instructed in writing by the player.
\end{enumerate}

\hypertarget{player-expenses}{%
\chapter{PLAYER EXPENSES}\label{player-expenses}}

\hypertarget{moving-expenses.}{%
\section{Moving Expenses.}\label{moving-expenses.}}

\begin{enumerate}
\def\labelenumi{(\alph{enumi})}
\tightlist
\item
  A Team's obligation to reimburse a player for ``reasonable'' expenses related to the assignment of a Player Contract from one Member to another (in accordance with paragraph 10 of a Uniform Player Contract) shall extend to the reimbursement of the actual expenses incurred by such player in moving to the home territory of his new Team, provided that such expenses result directly from the assignment and are ordinary and reasonable, and provided further that, prior to his actually incurring such expenses, the player consults with the Team to which his Contract has been assigned (furnishing a written estimate of such proposed expenses, if requested by the Team), so as to afford such assignee-Team an opportunity to make reasonably comparable alternative arrangements for the move of the player. In the event that the assignee-Team requests an estimate of such proposed expenses, the player shall furnish such estimate to the Team within a reasonable time following the notice of the assignment of the Player Contract. Upon receipt of such estimate from the player, the Team shall, within ten (10) days, either agree to reimburse the player for the expenses set forth in such estimate or make alternative arrangements (at the Team's expense) for the move of the player.
\item
  A player whose Contract is assigned from one Team to another shall be reimbursed by the assignee-Team for the cost of a hotel room in a hotel (comparable to that in which such Team's players are lodged while ``on the road'') in the assignee-Team's home city for up to thirty (30) days following the assignment.
\item
  A Player whose Contract is assigned from one Team to another shall receive from the assignee-Team a sum equal to three months' rent on his living quarters in the city from which he is assigned; provided, however, that such payment shall be made only and to the extent that the player is legally obligated for such rent, and shall not exceed \$1,500 per month.
\item
  Prior to its reimbursing an assigned player as provided in this Section, an assignee-Team may require satisfactory proof that the player has paid the amounts for which he seeks reimbursement, and, in the case of rent reimbursements, satisfactory proof that the player is legally obligated to pay such rent and the amount thereof. Upon notice to the player, the assignee-Team may, as an alternative to reimbursement, pay the expenses incurred upon assignment (in accordance with the foregoing provisions of this Section) directly to the persons, firms, or corporations involved.
\item
  So as to minimize the potential liability of NBA Teams under this Section, a player who does not establish permanent or year-round residence in the home city (or geographic vicinity thereof) of the Team by which he is employed shall use his best efforts (i) to obtain a short-term lease on the living quarters he selects, and (ii) to procure lease provisions authorizing him to sublet such premises and/or granting such Team the option to take over such lease in the event the Contract of such player is assigned to another NBA Team.
\end{enumerate}

\hypertarget{meal-expense-allowance.}{%
\section{Meal Expense Allowance.}\label{meal-expense-allowance.}}

\begin{enumerate}
\def\labelenumi{(\alph{enumi})}
\tightlist
\item
  The meal expense allowance, provided for in paragraph 4 of a Uniform Player Contract, shall be as follows:
  For the 1995-96 Season: \$80 per day.
  For the 1996-97 through 2000-01 Seasons: \$80 plus a cost of living adjustment (which shall be calculated by applying to \$80 the percentage increase in the national Consumer Price Index between October 1995 and October of the then-current Season, and which shall be rounded off to the nearest whole dollar) per day.
\item
  When a Team is ``on the road'' for less than a full day, a partial meal expense shall be paid based upon the time of departure from or time of arrival in the Team's home city, in accordance with the following:

  \begin{enumerate}
  \def\labelenumii{(\roman{enumii})}
  \tightlist
  \item
    Departure-after 9:00 a.m. or arrival before 7:00 a.m., no meal expense allowance for breakfast.
  \item
    Departure after 1:00 p.m. or arrival before 11:30 a.m., no meal expense allowance for lunch.
  \item
    Departure after 7:00 p.m. or arrival before 5:30 p.m., no meal expense allowance for dinner.
    For purposes of this Section 2(b), the meal expense allowance for breakfast shall be deemed to be 18\% of the applicable daily meal expense allowance (rounded off to the nearest whole dollar); the meal expense allowance for lunch shall be deemed to be 28\% of the applicable daily meal expense allowance (rounded off to the nearest whole dollar); and the meal expense allowance for dinner shall be deemed to be 54\% of the applicable daily meal expense allowance (rounded off to the nearest whole dollar).
  \end{enumerate}
\item
  For purposes of this Agreement and paragraph 4 of the Uniform Player Contract, the ``home city'' of an NBA Team shall be deemed to include only the city in which the facility regularly used by the Team for home games is located and any other location at which such home games are played, provided that such other location(s) is not more than 75 miles from such city.
\end{enumerate}

\hypertarget{benefits}{%
\chapter{BENEFITS}\label{benefits}}

\hypertarget{player-benefits.}{%
\section{Player Benefits.}\label{player-benefits.}}

Except as set forth below, effective with the 1995-1996 NBA Season, and continuing for the term of this Agreement, the NBA shall provide the following benefits to NBA players and, in the case of Sections (a) and (f) below, former NBA players:

\begin{enumerate}
\def\labelenumi{(\alph{enumi})}
\item
  \begin{enumerate}
  \def\labelenumii{(\arabic{enumii})}
  \tightlist
  \item
    Subject to the provisions of Section (a)(3) below, League-wide pension benefits in accordance with the terms of the National Basketball Association Players' Pension Plan, as restated effective February 2, 1989 (the ``Plan''). Beginning with the 1996-97 Season, the Plan will, subject to the approval of the Internal Revenue Service, be amended to provide for the following changes which shall be effective only throughout the period of this Agreement:

    \begin{enumerate}
    \def\labelenumiii{(\roman{enumiii})}
    \tightlist
    \item
      The ``Normal Retirement Pension'' payable to a player under the Plan shall be increased to the maximum monthly amount permitted by the applicable benefit limitations under the Internal Revenue Code to be paid to the player at his ``Normal Retirement Date'' under the Plan (the ``Maximum Monthly Benefit''). For purposes of the preceding sentence, the applicable benefit limitations shall be the limitations in effect for the year in which this Agreement is executed. Notwithstanding the foregoing:

      \begin{enumerate}
      \def\labelenumiv{(\Alph{enumiv})}
      \tightlist
      \item
        The benefits payable under the Plan shall at all times be subject to the limitations on benefits under the Internal Revenue Code, as amended (the ``Code'').
      \item
        If all or any portion of the actuarially determined contributions to be made to the Plan will not be fully deductible under the Code when paid, the Maximum Monthly Benefit shall not exceed the amount which would result in all of such contributions being fully deductible when paid. The parties agree that the determinations described in the preceding sentence, including any actuarial assumptions and projections related thereto, shall be made by the current actuaries of the Plan and any such determinations shall be binding and conclusive.
      \item
        Except as otherwise provided herein, the Maximum Monthly Benefit shall, effective as of the beginning of a Plan Year of the Plan, be adjusted for increases in the cost of living in the same manner as the cost of living adjustment for the dollar limitation under section 415(b)(I)(A) of the Code. In no event, however, shall the adjusted Maximum Monthly Benefit for a Plan Year exceed an amount that would require the actuarially determined contributions to be made to the Plan to fund for such adjusted Benefit for the Plan Year to exceed the actuarially determined contributions made to the Plan to fund for the Maximum Monthly Benefit in effect for the immediately preceding Plan Year by more than five (5) percent. The parties agree that the determinations described in the preceding sentence, including any actuarial assumptions and projections related thereto, shall be made by the current actuaries of the Plan and any such determinations shall be binding and conclusive.
      \item
        The applicable provisions of this Section I(a) shall apply only to those players who had not yet begun to receive a benefit under the Plan as of July I, 1996 and to those players who were receiving monthly benefits under the Plan as of September I, 1996; provided, however, that in the case of those players who were receiving monthly benefits under the Plan as of September I, 1996, this section I(a) shall apply only with respect to benefit payments to be made on or after September 1, 1996 and shall not require the recalculation of benefit payments made prior to such date.
      \end{enumerate}
    \item
      The actuarial reduction for the ``Qualified Joint and Survivor Annuity'' (as defined under section 1.29 of the Plan) for married players and their spouses currently provided under section 3.10 of the Plan shall be eliminated. The preceding sentence shall apply only with respect to benefit payments commencing on or after September 1, 1996 in the form of a Qualified Joint and Survivor Annuity and, in the case of a player receiving monthly benefit payments under the Plan as of September 1, 1996, only if such benefit payments are currently being made in the form of a Qualified Joint and Survivor Annuity.
    \item
      The ``Normal Retirement Benefit'' payable to a Pre-1965 Player under Article XX of the Plan shall be increased to \$200 per month for each ``Year of Pre-1965 Credited Service''; provided, however, that the benefits payable under the Plan to Pre-1965 Players shall at all times be subject to the limitations on benefits under the Code. The benefit to be paid in accordance with the preceding sentence shall apply only with respect to benefit payments to be made on or after September 1, 1996 and shall not require the recalculation of benefit payments made prior to such date.
    \end{enumerate}
  \item
    Notwithstanding anything else in this Agreement: (i) if any change or amendment made to the Code, or the Employee Retirement Income Security Act of 1974, as amended (``ERISA''), or to any regulations (whether final, temporary or proposed regulations) or rulings issued thereunder; or (ii) if any interpretation, application or enforcement (or any proposed interpretation, application or enforcement), by a court of competent jurisdiction in the United States or by the Internal Revenue Service, of the Code, ERISA, or any regulations or rulings issued thereunder; or (iii) if any regulations (whether final, temporary or proposed regulations) or rulings issued by the Internal Revenue Service under the Code or ERISA; or (iv) if any provisions of this Agreement, including any of the amendments or benefit increases to be provided under the Plan pursuant to this Section, would result in the Plan no longer being a tax-qualified Plan under section 401(a) of the Code, or would require NBA Teams to incur costs over and above any costs required to be incurred to implement the provisions of this Agreement or any prior collective bargaining agreement in order for the Plan to maintain its tax-qualified status under section 401(a) of the Code (provided, however, that such additional costs are incurred solely in connection with the provision of pension benefits to their non-player employees or to non-player employees of affiliates (within the meaning of sections 414(b), (c) or (m) of the Code) of such Teams), then any obligation to maintain and/or make contributions to the Plan pursuant to this Agreement or pursuant to any prior collective bargaining agreement shall terminate; provided, however, that any such termination shall not impair the legally binding effect of any other provision of this Agreement or any prior collective bargaining agreement, nor shall it create any right (x) to unilaterally implement during the term of this Agreement any terms concerning the provisions of pension benefits to the players, (y) to lockout, or (z) to strike. In the event of such termination, the NBA Teams shall provide alternative benefits to the players, at an annual cost (as determined on an after-tax basis) to NBA Teams equal to the annual cost that such Teams would have incurred under the Plan commencing on the date of termination. The NBA and the Players Association shall agree upon the type(s) of alternative benefits to be provided.
  \item
    Players employed by Toronto and Vancouver (``Canadian players'') shall receive pension benefits of comparable value by means of the Plan and separate pension plans to be established and maintained by Toronto and Vancouver (``Separate Plans''); provided, however, that (i) if the provision of pension benefits under the Plan to the Canadian players would, at any time, result in the Plan being subject to Canadian Provincial Pension Legislation and/or Canadian Federal Tax Laws (to the extent that the application of such tax laws would result in adverse tax consequences to the Plan, the NBA Members and/or the Canadian players), and/or (ii) if the Separate Plans would not, upon their establishment or at any future time, either satisfy U.S. tax qualification requirements or be able to be registered under Canadian Provincial Pension Legislation and/or Canadian Federal Tax Laws, then any obligation to establish, maintain and/or make contributions to both the Plan with respect to Canadian players and the Separate Plans pursuant to this Agreement or pursuant to any prior collective bargaining agreement shall terminate. In the event of such termination, Toronto and Vancouver shall provide alternative benefits to the Canadian players at an annual cost (as determined on an after-tax basis) to Toronto and Vancouver equal to the annual cost that Toronto and Vancouver would have incurred under the Plan and the Separate Plans commencing on the date of termination. The NBA and the Players Association shall agree upon the type(s) of alternative benefits to be provided.
  \end{enumerate}
\item
  Life insurance and accidental death and dismemberment benefits, as set forth in the Prudential Insurance Company Policy No.~GRP31300 GEN AS5-102 (the ``Prudential Policy'') (which benefits were in effect for the 1995-96 Season).
\item
  Disability insurance benefits, as set forth in the Boston Mutual Life Insurance Co., Policy No.~G-LTD-12381 (which benefits were in effect for the 1995-96 Season).
\item
  Workers' compensation benefits, in accordance with applicable statutes.
\item
  Medical and Dental insurance benefits for the 1995-96 Season, in accordance with the terms of the Prudential Policy. Beginning with the 1996-97 Season and continuing for the term of this Agreement, the benefits provided under such policy shall be modified only to the following extent:

  \begin{enumerate}
  \def\labelenumii{(\arabic{enumii})}
  \tightlist
  \item
    Subject to deductibles, the Prudential Policy shall cover 80\% of the first \$5,000, and 100\% thereafter, of qualifying expenses (as defined in the Prudential Policy) for each player and his eligible dependents in each year; provided, however, that the maximum co-insurance obligation per family per year shall not exceed \$3,000.
  \item
    Each player shall pay an annual deductible of \$300 for himself and each family member; provided, however, that no further deductible obligation shall be required for any family member in any plan year in which a deductible of \$300 has been paid for each of three family members.
  \end{enumerate}
\item
  League-wide severance in the amount of up to \$775,000 for the 1995-96 Season and \$600,000 for each of the 1996-97 and 1997-98 Seasons, it being agreed and understood by the parties that the distribution and amounts of severance payments to individual players (or former players) shall remain the responsibility of the Players Association and that, regardless of the bases upon which the Players Association elects to make such distributions, the NBA's liability for severance shall not exceed the amounts set forth in this paragraph.
\item
  Funding for the activities of the Joint Labor Management Committee in the amount of \$1.2 million for the 1995-96 Season and \$1.5 million for the 1996-97 Season.
\item
  Funding for an HIV/ AIDS education program through the 1998-99 season in accordance with the terms of the agreement between Mosaic Health Inc.~and the National Basketball Players Association dated November 7, 1995 (the ``Mosaic Agreement''). For the 1999-2000 and 2000-01 Seasons, the parties shall agree upon a continued or new education program(s) for players, to be funded in the 1999-2000 and 2000-01 Seasons in aggregate amounts equal to 105\% and 110\%, respectively, of the average of the amounts paid under the Mosaic Agreement for the 1996-97 through 1998-99 seasons.
\item
  Funding for the annual Players Association High School Basketball Camp (or any substitute program mutually agreed upon by the parties) in the amount of \$175,000 for the 1995-96 Season, increasing by 10\% per year thereafter.
\item
  Player Playoff Pool amounts, as follows:

  \begin{longtable}[]{@{}lc@{}}
  \toprule()
  \endhead
  1995-96 Season & \$7.0 million \\
  1996-97 Season & \$7.0 million \\
  1997-98 Season & \$7.0 million \\
  1998-99 Season & \$7.5 million \\
  1999-2000 Season & \$7.5 million \\
  2000-2001 Season & \$7.5 million \\
  \bottomrule()
  \end{longtable}

  If the NBA increases the number of Teams participating in the playoffs, the Player Playoff Pool shall be increased by \$437,500 for each Team added with respect to the 1995-96 through 1997-98 Seasons and by \$468,750 for each Team added with respect to the 1998-99 through 2000-01 Seasons. The NBA will consult with the Players Association with respect to the method of allocation of the Player Playoff Pool.
\item
  The employer's portion of payroll taxes; and
\item
  The Players Association's one-half share of the payment of fees and expenses to the Accountants in connection with any audit conducted under this Agreement.
\end{enumerate}

\hypertarget{projected-benefits.}{%
\section{Projected Benefits.}\label{projected-benefits.}}

\begin{enumerate}
\def\labelenumi{(\alph{enumi})}
\tightlist
\item
  For purposes of computing the Salary Cap and Minimum Team Salary in accordance with Article VII, ``Projected Benefits'' shall mean the projected amounts to be paid or accrued by the NBA or the Teams, other than the Expansion Teams during their first two Seasons, for the upcoming Season with respect to the Benefits to be provided for such Season. In the event that the amount of any benefit for the upcoming Season is not reasonably calculable, then, for purposes of computing Projected Benefits, such amount shall be projected to be 108\% of the amount expended for the same benefit for the prior Season.
\item
  The determination of Projected Benefits shall be made by mutual agreement of the parties no later than August 1 preceding each Season of this Agreement, beginning with the 1996-97 Season. In the event that the parties are unable to agree upon Projected Benefits by such date, the determination shall be made by the Accountants, whose decision will be final and unappealable.
\end{enumerate}

\hypertarget{insurance-carriers.}{%
\section{Insurance Carriers.}\label{insurance-carriers.}}

At any time during the term of this Agreement, the NBA may change the carrier of any of the foregoing insurance programs, subject to the Players Association's prior written approval, which approval shall not be unreasonably withheld. In no event shall any change in insurance carrier result in a change in the types or levels of any of the Benefits provided for above.

\hypertarget{compensation-and-expenses-in-connection-with-military-duty}{%
\chapter{COMPENSATION AND EXPENSES IN CONNECTION WITH MILITARY DUTY}\label{compensation-and-expenses-in-connection-with-military-duty}}

\chaptermark{COMPENSATION AND EXPENSES \ldots}

\hypertarget{salary.}{%
\section{Salary.}\label{salary.}}

A player drafted into military service during the Season, or a player serving on active duty with a reserve unit during the Season, shall be compensated for so long as the player remains on the Active List of the Member in such amount as may be negotiated between the player and the Member by which he is employed, subject to the provisions of Article VII.

\hypertarget{travel-expenses.}{%
\section{Travel Expenses.}\label{travel-expenses.}}

\begin{enumerate}
\def\labelenumi{(\alph{enumi})}
\tightlist
\item
  A player serving on military weekend duty with a reserve unit during the Season shall be entitled to reimbursement for any net out-of-pocket expenses incurred by such player in traveling to and from his place of duty to enable him to join his Team for purposes of participating in a Regular Season game.
\item
  In the event that the Player Contract of a player who is required to serve on military weekend duty with a reserve unit is sold, exchanged, assigned or transferred to another Team, the player shall be entitled to reimbursement for any out-of-pocket expenses incurred by such player in traveling during the off-season to and from his home and his place of military weekend duty with a reserve unit; provided that (i) the player makes reasonable efforts to change his reserve unit location to one located reasonably close to his home and (ii) such obligation to reimburse the player shall cease six months from the date that such player's Contract is sold, exchanged, assigned, or transferred.
\end{enumerate}

\hypertarget{player-conduct}{%
\chapter{PLAYER CONDUCT}\label{player-conduct}}

\hypertarget{fines-and-suspensions.}{%
\section{Fines and Suspensions.}\label{fines-and-suspensions.}}

In addition to any other rights a Team may have by Contract (including but not limited to the rights set forth in paragraph 9 of the Uniform Player Contract) or by law, when a player, without proper and reasonable cause, fails or refuses to render the services required by a Player Contract and/or when a player is, for proper cause, suspended by his Team or the NBA in accordance with the terms of such Contract, the Salary payable to the player for the year of the Contract during which such refusal or failure and/or suspension occurs may be reduced (or, in the case of a suspension, shall be reduced) as follows:

\begin{enumerate}
\def\labelenumi{(\alph{enumi})}
\tightlist
\item
  By \$1,000 for each missed (training or Regular Season) day of practice (for the first two such missed practices), and by \$2,500 for each missed practice thereafter;
\item
  By \$4,000 for each missed Exhibition Game;
\item
  By 1/82nd of the player's Current Cash Compensation for each missed Regular Season or Playoff game;
\item
  By \$1,000 for each missed promotional appearance required in accordance with Article II, Section 7(b) and paragraph 13 of the Uniform Player Contract;
\item
  By \$10,000 for failing to attend the Rookie Transition Program; and
\item
  By \$5,000 for failing to attend any other programs designated as mandatory by either the NBA or the Players Association (e.g., HIV and substance abuse education programs), the designation of which by either party shall be subject to approval by the other, which approval shall not be unreasonably withheld.
\end{enumerate}

\hypertarget{charitable-contributions.}{%
\section{Charitable Contributions.}\label{charitable-contributions.}}

If, pursuant to the terms of this Agreement, (a) any fine or suspension is imposed on a player, (b) such fine or suspension-related salary amount is paid to the League, and (c) the fine or suspension is not grieved by the player under the provisions of Article XXXI, then, beginning with the 1996-97 Season, the NBA shall remit fifty percent (50\%) of the Salary withheld (or payment received) as a result of the fine or suspension to the National Basketball Players Association Foundation (the ``NBPA Foundation'') or such other charitable organization selected by the Players Association that qualifies for treatment under Section 501(c)(3) of the Internal Revenue Code of 1986, as now in effect or as it may hereafter be amended (a ``Section 501(c)(3) Organization''), and that is approved by the NBA (which approval shall not be unreasonably withheld) (both hereinafter, the ``NBPA-Selected Charitable Organization''). The NBA shall remit the remaining fifty percent (50\%) of the withheld Salary to a Section 501(c)(3) organization selected by the NBA and approved by the Players Association, which approval shall not be unreasonably withheld. If the player files a timely Grievance under Article XXXI, the Salary withheld (or the payment received), including fifty percent (50\%) of any accrued interest, shall be contributed to charity in the manner set forth above only to the extent the player does not prevail in the Grievance. The remittances made by the NBA pursuant to this Section shall be made annually, fifteen (15) days following the end of the NBA Season during which the fine or suspension-related salary amount is received. For purposes of this Article and all other provisions of this Agreement, any money remitted or paid to the National Basketball Players Association Foundation by the NBA shall be used for charitable purposes only, and not, for example, for any salaries of Foundation employees or administrative expenses.

\hypertarget{team-salary-salary-cap-and-minimum-team-salary}{%
\chapter{TEAM SALARY, SALARY CAP, AND MINIMUM TEAM SALARY}\label{team-salary-salary-cap-and-minimum-team-salary}}

\chaptermark{TEAM SALARY, SALARY CAP \ldots}

\hypertarget{basketball-related-income.}{%
\section{Basketball Related Income.}\label{basketball-related-income.}}

\begin{enumerate}
\def\labelenumi{(\alph{enumi})}
\tightlist
\item
  For purposes of this Agreement, the following terms shall have the meanings set forth below:

  \begin{enumerate}
  \def\labelenumii{(\arabic{enumii})}
  \tightlist
  \item
    ``Basketball Related Income'' (``BRI'') means the aggregate revenues (including the value of any property or services received in any barter transactions) received or to be received on an accrual basis, for or with respect to each Salary Cap Year during the term of this Agreement, by the NBA, NBA Properties, Inc., including any of its subsidiaries whether now in existence or created in the future (hereinafter ``Properties''), the Market Extension Partnership, and all NBA Teams other than Expansion Teams during their first two Seasons (but including the Expansion Teams' shares of national television, radio, cable and other broadcast revenues, and any other League-wide revenues shared by the Expansion Teams, provided such revenues are otherwise included in BRI), from all sources, whether known or unknown, whether now in existence or created in the future, derived from, relating to or arising directly or indirectly out of the performance of Players in NBA basketball games or in NBA-related activities. For purposes of this definition of BRI, ``Player'' means a person: who is under contract to an NBA Team; who completed the playing services called for under a contract with an NBA Team at the conclusion of the prior Season; or who was under contract with an NBA Team during (but not at the conclusion of) the prior playing Season, but only with respect to the period for which he was under such contract. BRI shall include, but not be limited to, the following:

    \begin{enumerate}
    \def\labelenumiii{(\roman{enumiii})}
    \tightlist
    \item
      Regular Season gate receipts, net of applicable taxes, (including, without limitation, gate receipts received or to be received on an accrual basis by any entity related to an NBA Team in accordance with subsection (a)(4)(i) below) including: (A) the value (determined on the basis of the price of the ticket) of all tickets traded by a Team for goods or services; and (B) the value (determined on the basis of the League-wide average ticket price for non-Season tickets) of all tickets for NBA Regular Season games provided by a Team on a complimentary basis, without monetary or other compensation to a Team, provided, however, that (x) the value of the first 1.2 million of such complimentary tickets for all NBA Regular Season games in an NBA Season shall be excluded from BRI, and (y) in addition, tickets provided as part of sponsorships and other transactions, where the proceeds from such transactions have been included in BRI, shall not be included in determining the number of complimentary tickets in any NBA Season;
    \item
      all proceeds of any kind from the broadcast or exhibition of, or the sale, license or other conveyance or exploitation of the right to broadcast or exhibit, NBA Pre-Season, Regular Season and Playoff games, highlights or portions of such games, and non-game NBA programming, on any and all forms of radio, television, telephone, internet, and any other communications media, forms of reproduction and other technologies, whether presently existing or not, anywhere in the world, whether live or on any form of delay, including, without limitation, network, local, cable, direct broadcast satellite and any form of pay television, and all other means of distribution and exploitation, whether presently existing or not and whether now known or hereafter developed, including, without limitation, such proceeds received or to be received on an accrual basis by any entity related to an NBA Team in accordance with subsection (a)(4)(i) below, net of reasonable and customary expenses related thereto, but not including the value of any broadcast time or cablecast time provided as part of any such transaction that is used solely to promote the NBA, its Teams and its players;
    \item
      all Exhibition game proceeds of any kind, net of applicable taxes and all customary and reasonable game, pre-Season and training camp expenses, including, without limitation, such proceeds received or to be received on an accrual basis by any entity related to an NBA Team in accordance with subsection (a)(4)(i) below;
    \item
      all playoff gate receipts of any kind, net of admission taxes, arena rentals to the extent reasonable and customary, and all other reasonable and customary expenses, except the player playoff pool, including, without limitation, such proceeds received or to be received on an accrual basis by any entity related to an NBA Team in accordance with subsection (a)(4)(i) below;
    \item
      all proceeds of any kind from in-arena sales of novelties and concessions, sales of novelties in team-identified stores within a 75-miles radius of the arena, NBA game parking and programs, Team sponsorships and promotions, temporary arena signage, and arena club revenues, in each case, to the extent that such proceeds are related to the performance of Players in NBA basketball games, including, without limitation, such proceeds received or to be received on an accrual basis by any entity related to an NBA Team in accordance with subsection (a)( 4 )(i) below, net of reasonable and customary expenses related thereto, subject to the provisions of subsection (a)(3) below;
    \item
      forty percent of the gross proceeds from fixed arena signage present in arenas in which an NBA Team plays more than one-half of its Regular Season home games, including, without limitation, such proceeds received or to be received on an accrual basis by any entity related to an NBA Team in accordance with subsection (a)(4)(i) below;
    \item
      forty percent of all the gross proceeds of any kind, net of all applicable taxes, from the sale, lease or licensing of luxury suites calculated on the basis of the actual proceeds received or to be received on an accrual basis by the entity, including, without limitation, proceeds received or to be received on an accrual basis by any entity related to an NBA Team in accordance with subsection (a)(4)(i) below, that sold, leased, or licensed such luxury suites; provided, however, that, other than the additional amounts paid by luxury suite holders to the Team for tickets, if any, this amount shall be the only amount included in BRI for the sale, lease or licensing of luxury suites and that, to the extent that the sale, lease or licensing of the luxury suite grants rights to the luxury suite for a period of more than one year, for purposes of calculating the amount includable in BRI for any NBA Season, the proceeds shall be determined on the basis of the annual fee or charge provided for in any such transaction and, if payments are made in addition to or in the absence of such an annual fee or charge, the value of such payments shall be amortized over the period of the sale, lease or license, unless such period exceeds twenty years, in which event an amortization period of twenty years shall be used;
    \item
      except as provided in subsection (a)(2) below, proceeds received by Properties, net of actual expenses that are directly attributable to the generation of such proceeds, as long as those expenses are consistent with the types and categories of expenses incurred by Properties reflected in the audited financial reports for Properties for the year ended July 31, 1994 (or, in the case of new sources of revenues, as long as the expenses are reasonable and customary, in the opinion of the Accountants), including proceeds derived from the following categories (defined in the same manner as was used in those audited financial reports): (A) international television; (B) sponsorships; (C) NBA-related revenues from NBA Entertainment; (D) the All-Star Game and McDonald's Championship; (E) other NBA special events; and (F) all other sources of revenue received by Properties other than those specifically excluded under subsection (a)(2) below; and
    \item
      proceeds from premium seat licenses (other than licenses of luxury suites, which are governed by subsection (a)(l)(vii) above) attributable to NBA-related events amortized over the period of the license (including, without limitation, such proceeds received or to be received on an accrual basis ,by any entity related to an NBA Team, in accordance with subsection (a)( 4 )(i) below), unless such period exceeds twenty years, in which event an amortization period of twenty years shall be used,
    \end{enumerate}
  \item
    It is understood that the following is a non-exclusive list of examples of revenues received by the NBA, Properties, NBA Teams and each of their related entities that are not derived from, and do not relate to or arise out of, the performance of Players in NBA basketball games or in NBA-related activities or are otherwise expressly excluded from the definition of BRI: proceeds from the assignment, sale or trade of Player Contracts, proceeds from the sale of any existing NBA franchise (or any interest therein) or the grant of NBA expansion franchises, dues or capital contributions received by the NBA, fines, revenue sharing (by reason of revenue transfers or otherwise) among Teams, interest income, insurance recoveries, sales of interest in real estate and other property, and proceeds received by Properties (i) pursuant to the Group License Agreement; (ii) pursuant to the Events Agreement; and (iii) relating to the following categories (defined in the same manner as was used in the audited financial reports for Properties for the year ended July 31, 1994): (A) licensing; and/or (B) the representation by Properties of third parties. For purposes of the foregoing sentence, ``third parties'' refers to persons or entities that are not owned or controlled by persons or entities that own or control an NBA team or, if such third party is ``an entity related to an NBA team'' (as defined below), NBA Properties' representation does not relate either to such entity's NBA ownership or NBA players.
  \item
    Subject to Article VII, Section 9 (Players Association Audit Rights), with respect to expenses incurred in connection with all proceeds coming within subsections (a)(l)(v) and (viii) above, all reported expenses shall be conclusively presumed to be reasonable and customary (other than expenses related to sources of revenues that were not reflected in the audited financial report for Properties for the year ended July 31, 1994), and such expenses shall not be the subject of the accounting procedures set forth in Article VII, Section 8 of this Agreement. Such expenses shall be disallowed, however, to the extent that they exceed the ratio of League-wide reported expenses to League-wide reported revenues (the ``Expense Ratio'') for that category of revenues set forth in Exhibit C hereto.
  \item
    It is acknowledged by the parties hereto that for purposes of determining BRI:

    \begin{enumerate}
    \def\labelenumiii{(\roman{enumiii})}
    \tightlist
    \item
      Some NBA Teams have engaged or may engage in transactions with third parties that are owned or controlled, directly or indirectly, by the persons or entities owning or controlling the NBA Team (such third parties are referred to in this Agreement as ``an entity related to an NBA Team''). As provided in subsections (a)(l)(i)-(vii) and (ix) above, the relevant proceeds received or to be received on an accrual basis by any entity related to an NBA Team that come within those paragraphs and that relate to such entity's related NBA Team shall be included in BRI. To the extent, however, that such net proceeds cannot reasonably be determined for transactions entered into with an entity related to an NBA Team after the date of this Agreement, the Accountants retained pursuant to the provisions of Article VII, Section 8 of this Agreement shall determine a reasonable value which shall be included in BRI for the first NBA Season to which the transaction pertains. For each Season thereafter in which the transaction remains in effect, the same amount - or an equivalent annualized amount if the transaction pertained to less than a full Season - will be included in BRI increased (or decreased, as the case may be) by the amount the League-wide per-Team average for such net proceeds increased (or decreased) over the previous Season.
    \item
      With respect to transactions with an entity related to an NBA Team pertaining to proceeds that come within subsections (a)(1)(ii), (a)(1)(v) and (a)(l)(vi) above that are in effect as of June 21, 1995, the parties agree that the projected amounts to be earned by the entity related to the NBA Team for the 1995-96 NBA Season as reflected in Exhibit D hereto are reasonable and that the methodology employed to determine these amounts, including all allocations of revenues and expenses, so long as the transaction between the NBA Team and the related third party remains unchanged, shall be conclusively presumed to be reasonable. This methodology will be used to determine the applicable amounts for NBA Seasons subsequent to the 1995-96 NBA Season.
    \item
      With respect to the transactions listed below in this subsection (a)(4)(iii) between an NBA Team and an entity related to the Team, the parties agree that, because the proceeds attributable to these transactions cannot be accurately ascertained, the following procedures shall be used for each NBA Season in which these transactions remain in effect:
      (A) New York Knicks transaction with MSG Network regarding the sale of local media rights (the ``MSG Network Transaction''): BRI for the Knicks for the 1995-96 NBA Season shall include an amount equal to the net proceeds included in BRI attributable to the Los Angeles Lakers' sale, license or other conveyance of all local media rights (including, but not limited to, broadcast and cable television and radio). In each subsequent Season covered by this Agreement, this amount shall be increased (or decreased, as the case may be), for each category of local media rights, by the average of the League-wide average percentage increase (or decrease) for each such category.
      (B) New York Knicks transactions with related parties involving signage: BRI for the Knicks for the 1995-96 NBA Season shall include \$3,000,000 for signage. In each subsequent Season covered by this Agreement, this amount shall be increased (or decreased, as the case may be) by the League-wide average percentage increase (or decrease) in signage as determined in such subsections (a)(1)(v) and (a)(1)(vi),
    \end{enumerate}
  \item
    Pursuant to the NBA's national broadcast and network cable television agreements with NBC and TBS in effect as of the execution date of this Agreement, NBA Teams might receive revenue sharing proceeds in the 1997-98 NBA Season that are attributable, at least in part, to NBA game telecasts in that and prior NBA Seasons (referred to hereinafter as ``Revenue Sharing Proceeds''), An amount up to \$36 million of Revenue Sharing Proceeds shall be included in BRI for the 1997-98 NBA Season, In addition, Revenue Sharing Proceeds in excess of \$36 million (the ``Excess Revenue Sharing Proceeds''), if any, shall be included in BRI for the 1997-98 NBA Season to the extent of (i) 50\% of the Excess Revenue Sharing Proceeds, plus (ii) the amount, if any, determined by the following calculation:

    \begin{itemize}
    \tightlist
    \item
      Step 1: add 50\% of the Excess Revenue Sharing Proceeds (the ``1996-97 Rollback Amount'') to 1996-97 BRI (the ``1996-97 Adjusted BRI'');
    \item
      Step 2: multiply 1996-97 Adjusted BRI by .5013;
    \item
      Step 3: subtract from the result in Step 2 Total Salaries and Benefits for the 1996-97 Season;
    \item
      Step 4: if the result in Step 3 is a positive amount, divide such amount by .5013, then include the result in 1997-98 BRI, up to the amount of the 1996-97 Rollback Amount.
    \end{itemize}

    With respect to revenue sharing proceeds or other contingent payments attributable to NBA Seasons during the term of this Agreement, if any, that are provided for in national broadcast or network cable television agreements that succeed the present NBA/NBC and NBA/TBS agreements (the ``Successor Agreements''), such proceeds or contingent payments, if any, shall be included in BRI in a manner to be determined by agreement of the parties, or if the parties do not reach an agreement, in a manner to be determined by the Accountants.
  \item
    ``Projected BRI'' means the sum of amounts determined in accordance with the following:

    \begin{enumerate}
    \def\labelenumiii{(\roman{enumiii})}
    \tightlist
    \item
      With respect to BRI sources other than national broadcast or network cable television contracts, Projected BRI shall include BRI for the preceding Salary Cap Year, increased by 8\%.
    \item
      With respect to national broadcast or network, cable television contracts, Projected BRI shall include (in addition to amounts determined in accordance with subsection (a)(6)(iii) and (iv) below) the following: (A) for the 1996-97 Season, \$200,000,000 shall be included in Projected BRI with respect to the NBA/NBC agreement, and \$90,000,000 shall be included in Projected BRI with respect to the NBA/TBS agreement; (B) for the 1997-98 Season, \$180,000,000 shall be included in Projected BRI with respect to the NBA/NBC agreement, and \$94,000,000 shall be included in Projected BRI with respect to the NBA/TBS agreement; and (C) for the 1998-99, 1999-2000 and 2000-01 Seasons, Projected BRI for a Season shall include an amount equal to the greater of: (i) 108\% of the prior Season's national broadcast and network cable television revenues; or (ii) the actual revenues stated in the Successor Agreements for such Season plus the Revenue Sharing Forecast (as defined in subsection (a)(6)(iv) below). Except with respect to clause (C)(ii) in the prior sentence, the foregoing shall exclude for all purposes any revenue sharing proceeds that are or may be received by the NBA pursuant to such contracts (including the Revenue Sharing Proceeds), which proceeds shall be treated for Projected BRI purposes solely in accordance with subsection (a)(6)(iii) and (iv) below.
    \item
      With respect to the Revenue Sharing Proceeds:
      (A) \$50 million shall be included in Projected BRI for the 1997-98 Salary Cap Year.
      (B) Additional amounts in respect of the Revenue Sharing Proceeds may be included in Projected BRI for the 1998-99 through 2000-01 Salary Cap Years to the extent determined by the following calculation (which shall be made no later than August 15, 1997):

      \begin{itemize}
      \tightlist
      \item
        Step 1: Subtract \$36 million from the Revenue Sharing Estimate (as defined below);
      \item
        Step 2: Multiply the result in Step 1 by 0.5 (the ``Adjusted Rollback Amount'');
      \item
        Step 3: Add the result in Step 2 to 1996-97 BRI;
      \item
        Step 4: Multiply the result in Step 3 by .5013;
      \item
        Step 5: Subtract from the result in Step 4 Total Salaries and Benefits for the 1996-97 Season;
      \item
        Step 6: If the result in Step 5 is a positive amount, divide such amount by .5013, then add the result, up to the amount of the Adjusted Rollback Amount, to the result in Step 2;
      \item
        Step 7: Subtract from the result in Step 6 \$14 million (the ``Adjusted Revenue Sharing Estimate'');
      \item
        Step 8: The Adjusted Revenue Sharing Estimate, if any, shall be included in Projected BRI for the 1998-99 through 2000-2001 Seasons as follows:

        \begin{enumerate}
        \def\labelenumiv{(\alph{enumiv})}
        \setcounter{enumiv}{23}
        \tightlist
        \item
          Any amount of the Adjusted Revenue Sharing Estimate, up to an amount that would result in a Salary Cap increase of \$250,000, shall be included in Projected BRI for the 1998-99 Season;
        \item
          Any amount of the Adjusted Revenue Sharing Estimate that remains after application of (x) above, up to an amount that would result in a Salary Cap increase of \$250,000, shall be included in Projected BRI for the 1999-2000 Season;
        \item
          Any amount of the Adjusted Revenue Sharing Estimate that remains after application of (x) and (y) above, up to an amount that would result in a Salary Cap increase of \$250,000, shall be included in Projected BRI for the 2000-01 Season.
        \end{enumerate}
      \end{itemize}

      For purposes of this subsection (a)(6)(iii), Revenue Sharing Estimate means an estimate of the total amount of Revenue Sharing Proceeds that the NBA Teams are entitled to receive pursuant to the NBA/NBC and NBA/TBS agreements assuming that the advertising revenues that will be received by NBC and TBS for NBA game telecasts during the 1997-98 Season will equal the average of the advertising revenues received by NBC and TBS, respectively, for NBA game telecasts during the 1994-95 through 1996-97 Seasons. The Revenue Sharing Estimate shall be mutually agreed upon by the NBA and the Players Association on or before August 1, 1997. In the event that the parties are unable to reach agreement on the Revenue Sharing Estimate, such amount shall be determined by the Accountants.
    \item
      With respect to any revenue sharing proceeds or other contingent payments that are provided for in a Successor Agreement with respect to the 1998-99, 1999-2000 and 2000-2001 Seasons, the parties shall mutually agree upon an estimate of the amount of such revenue sharing proceeds or contingent payments, if any, includable in Projected BRI for a Season on or before the August 1 prior to such Season (the ``Revenue Sharing Forecast''). In the event that the parties are unable to reach agreement on a Revenue Sharing Forecast, such amount, if any, shall be determined by the Accountants.
    \end{enumerate}
  \item
    ``Local Expansion Team BRI'' means the BRI of the Expansion Teams during their first two Seasons, but not including the Expansion Teams' share of League-wide revenues that are otherwise included in BRI (including, but not limited to, their share of national television, cable, radio and other broadcast revenues).
  \item
    ``Projected Local Expansion Team BRI'' means Local Expansion Team BRI for the immediately preceding Season, increased by 8\%.
  \item
    ``Interim Projected BRI'' means a projection of BRI for a Salary Cap Year based upon the Interim Audit Report provided for in Article VII, Section 8.
  \end{enumerate}
\item
  The NBA and each NBA Team shall in good faith act and use their best efforts so as to maximize Basketball Related Income for each Season during the term of this Agreement.
\end{enumerate}

\hypertarget{calculation-of-salary-cap-and-minimum-team-salary.}{%
\section{Calculation of Salary Cap and Minimum Team Salary.}\label{calculation-of-salary-cap-and-minimum-team-salary.}}

\begin{enumerate}
\def\labelenumi{(\alph{enumi})}
\tightlist
\item
  \textbf{Salary Cap.}

  \begin{enumerate}
  \def\labelenumii{(\arabic{enumii})}
  \tightlist
  \item
    For each Season during the term of this Agreement, there shall be a Salary Cap. The Salary Cap for each Season will equal the greater of:

    \begin{enumerate}
    \def\labelenumiii{(\roman{enumiii})}
    \item
      \begin{longtable}[]{@{}cc@{}}
      \toprule()
      \endhead
      1996-97: & \$24.3 million \\
      1997-98: & \$25.0 million \\
      1998-99: & \$26.0 million \\
      1999-00: & \$27.0 million \\
      2000-01: & \$28.0 million \\
      \bottomrule()
      \end{longtable}

      (the ``Guaranteed Minimum Salary Caps''); or
    \item
      48.04\% of Projected BRI, less Projected Benefits (as defined in Article IV, Section 2(a), plus or minus any Salary Cap adjustments (pursuant to subsection (d) below), divided by (A) 27 with respect to the 1996-1997 Season, and (B) 29 with respect to each remaining Season of the Agreement (the ``Calculated Salary Cap'').
    \end{enumerate}
  \item
    Notwithstanding subsection (1) above, in the event that Projected BRI for the 1996-97 Season, plus Projected Local Expansion Team BRI for the 1996-97 Season, less Projected Benefits (including for the Expansion Teams), plus or minus any Salary Cap adjustments (pursuant to subsection (d) below) divided by 29 (the ``Expansion Adjusted 1996-97 Salary Cap''), exceeds the Calculated Salary Cap for 1996-97, the Expansion Adjusted 1996-97 Salary Cap will apply for purposes of subsection (a)(1)(ii) above.
  \item
    The Salary Cap for the 1995-96 Season is \$23.0 million. In the event that, based upon the calculation of BRI for the 1995-96 Season, the Calculated Salary Cap for the 1995-96 Season would have exceeded \$23.0 million, the difference will be added to the Salary Cap for the 1996-97 Season. For purposes of the foregoing calculation only, the Calculated Salary Cap for the 1995-96 Season will equal 48.04\% of actual 1995-96 BRI, less Benefits, divided by 27.
  \item
    Commencing on the first day of each Salary Cap Year, the Salary Cap shall be the greater of (i) the Guaranteed Minimum Salary Cap for such Salary Cap Year, or (ii) the Salary Cap for the prior Salary Cap Year (the ``July 1 Salary Cap''). The July 1 Salary Cap will remain in effect until August 15, at which time the Salary Cap will be the amount calculated pursuant to subsections (a)(1)-(3) above.
  \item
    Notwithstanding subsection (4) above, in the event that the Audit Report for a Salary Cap Year, beginning with the 1996-97 Salary Cap Year, has not been completed as of the following August 15, then, beginning on August 16 and continuing until the Audit Report is completed, the Salary Cap shall be the greater of (i) the July 1 Salary Cap or (ii) an amount calculated pursuant to subsections (a)(1)- (3) above, except that Interim Projected BRI shall be utilized instead of Projected BRI.The Salary Cap calculated pursuant to this subsection (5) (the ``Interim Salary Cap'') will remain in effect until the Audit Report is completed, at which time the Salary Cap will be the amount calculated pursuant to subsections (a)(I)-(3) above.
  \end{enumerate}
\item
  \textbf{Minimum Team Salary.}

  \begin{enumerate}
  \def\labelenumii{(\arabic{enumii})}
  \tightlist
  \item
    For each Season during the term of this Agreement, there shall be a Minimum Team Salary equal to 75\% of the Salary Cap for such Season.
  \item
    In the event that, by the conclusion of the Salary Cap Year for a Season, a Team has failed to make aggregate Salary payments and/or incur aggregate Salary obligations equal to or greater than the applicable Minimum Team Salary for that Season, the NBA shall cause such Team to make such payments (to be disbursed to the players on such Team pro rata or in accordance with such other formula as may be reasonably determined by the Players Association).
  \item
    Nothing contained herein shall preclude a Team from having a Team Salary in excess of the Minimum Team Salary, provided that the Team's Team Salary does not exceed the Salary Cap plus any additional amounts authorized pursuant to the Exceptions set forth in this Article VII.
  \end{enumerate}
\item
  \textbf{Expansion Team Salary Caps and Minimum Team Salaries.} The Expansion Teams shall have the same Salary Caps and Minimum Team Salaries as the other 27 teams, except as follows:

  \begin{enumerate}
  \def\labelenumii{(\arabic{enumii})}
  \tightlist
  \item
    1995-96 Season:
    Salary Cap: 66 and 2/3\% of the Salary Cap determined in accordance with Section 2(a) above for the 1995-96 Season (``1995 Expansion Team Salary Cap'')
    Minimum Team Salary: 75\% of 1995 Expansion Team Salary Cap
  \item
    1996-97 Season:
    Salary Cap: 75\% of the Salary Cap determined in accordance with Section 2(a) above for the 1996-97 Season (``1996 Expansion Team Salary Cap'')
    Minimum Team Salary: 75\% of 1996 Expansion Team Salary Cap
  \end{enumerate}
\item
  \textbf{Adjustments to Salary Cap and Minimum Team Salary.}

  \begin{enumerate}
  \def\labelenumii{(\arabic{enumii})}
  \item
    \begin{enumerate}
    \def\labelenumiii{(\roman{enumiii})}
    \tightlist
    \item
      Beginning in the 1997-98 Season, in the event that Total Salaries and Benefits paid with respect to any Season is greater than 50.13\% of BRI for such Season, then for purposes of calculating the Calculated Salary Cap for the subsequent Season, the amount of such overage shall be deducted from 48.04\% of Projected BRI for such subsequent Season less Projected Benefits (the ``Overage Deduction''); provided, however, that in no event shall the Overage Deduction reduce the Calculated Salary Cap for the subsequent Season by more than \$500,000. In the event that an Overage Deduction for any Season exceeds \$500,000, there shall be no carry forward of any such excess to a subsequent Season.
    \item
      In the event that 48.04\% of actual BRI for a Season is less than 48.04\% of Projected BRI for that Season, then for purposes of calculating the Calculated Salary Cap for the subsequent Season, the difference shall be deducted from 48.04\% of Projected BRI for such subsequent Season less Projected Benefits. In the event that, in any Season, adjustments to 48.04\% of Projected BRI less Projected Benefits are called for pursuant to both this subsection (d)(1)(ii) and the preceding subsection (d)(1)(i), the maximum adjustment to 48.04\% of Projected BRI shall equal the greater of the adjustments called for under such subsections.
    \item
      In the event Total Salaries and Benefits paid with respect to any Season is less than 48.04\% of BRI for such Season (plus, if applicable, any shortfall against 48.04\% of BRI for the prior Season), then for purposes Of calculating the Calculated Salary Cap for the subsequent Season, the amount of such shortfall shall be added to 48.04\% of Projected BRI for the subsequent Season less Projected Benefits; provided, however, that in the event there is a shortfall with respect to the 2000-01 Season, the amount of such shortfall shall be paid by the NBA to the Players Association for distribution to all NBA players who were on an NBA roster during the 2000-01 Season no later than September 1, 2001 on such proportional basis as may be reasonably determined by the Players Association.
    \item
      In the event that actual Benefits for any Season exceeds Projected Benefits for such Season, the difference shall be added to Projected Benefits for the Subsequent Season.
    \item
      In the event that actual Benefits for any Season are less than Projected Benefits for such Season, the difference shall be deducted from Projected Benefits for the subsequent Season; provided, however, that in the event there is a shortfall with respect to the 2000-01 Season, the amount of such shortfall shall be paid by the NBA to the Players Association for distribution to all NBA players who were on an NBA roster during the 2000-01 Season no later than September 1, 2001 on such proportional basis as may be reasonably determined by the Players Association.
    \item
      In the event that the Interim Salary Cap for a Season exceeds the Salary Cap for such Season, the difference shall be deducted from the Salary Cap (and, when applicable, the Interim Salary Cap) for the following Season.
    \end{enumerate}
  \end{enumerate}
\end{enumerate}

\hypertarget{determination-of-salary.}{%
\section{Determination of Salary.}\label{determination-of-salary.}}

For the purposes of determining a player's Salary with respect to an NBA Season, the following rules shall apply:

\begin{enumerate}
\def\labelenumi{(\alph{enumi})}
\tightlist
\item
  \textbf{Deferred Compensation.}

  \begin{enumerate}
  \def\labelenumii{(\arabic{enumii})}
  \tightlist
  \item
    General Rules:

    \begin{enumerate}
    \def\labelenumiii{(\roman{enumiii})}
    \tightlist
    \item
      All Player Contracts entered into or extended after the date hereof shall specify the Season(s) in which any Deferred Compensation is earned. Deferred Compensation shall be included in a player's Salary in the Season in which such compensation is earned.
    \item
      Notwithstanding subsection (1)(i) above, for purposes of an annuity compensation arrangement included in a Player Contract in accordance with Article XXV, Section 3 of this Agreement only, Deferred Compensation shall include only the portion of the cost of the annuity instrument to be paid by the Team after the playing term covered by the Contract, if any, and shall not include any compensation that the player is scheduled to receive after the term of the Contract pursuant to such annuity compensation arrangement. The portion of the cost of the annuity paid by the Team while the player is required to render playing services under the Player Contract shall be included in Salary for the year in which such cost is paid.
    \end{enumerate}
  \item
    Over 35 Rule: Notwithstanding any provision in a Player Contract to the contrary, the following provisions are applicable to any Player Contract entered into or extended that, going forward, covers four or more NBA Seasons, including one or more Seasons commencing after such player will reach or has reached age 35 (an ``Over 35 Contract'');

    \begin{enumerate}
    \def\labelenumiii{(\roman{enumiii})}
    \tightlist
    \item
      Except as provided in subsections (ii)-(iv) below, the aggregate Salaries in an Over 35 Contract for Seasons commencing with the fourth Season or the first Season following the player's 35th birthday, whichever is later, shall be attributed to the prior Seasons pro rata on the basis of the Salaries for such prior Seasons.
    \item
      If a player who is age 32, 33 or 34 enters into an Over 35 Contract covering more than four NBA Seasons, the aggregate Salaries in such Over 35 Contract for Seasons commencing with the fifth Season shall be' attributed to the prior Seasons pro rata on the basis of the Salaries for such prior Seasons.
    \item
      If a player who has played for his current Team for at least ten consecutive Seasons enters into an Over 35 Contract that is an Extension and that, beginning with the date the Extension is signed, covers more than five NBA Seasons, the aggregate Salaries in such Over 35 Contract for Seasons commencing with the sixth Season shall be attributed to the prior Seasons pro rata on the basis of the Salaries for such prior Seasons.
    \item
      For each Season of an Over 35 Contract beginning with the third Season prior to the First Zero Year (as defined in subsection (vii) below), if the player plays in such Season, then the Salaries of the player for the subsequent three or fewer Seasons covered by the Contract (including any Zero Year) shall, on the July 1 of the first subsequent Season, be aggregated and attributed in equal shares to each of such three or fewer Seasons.
    \item
      Notwithstanding subsection (2)(i) above, there shall be no re-allocation of Salaries pursuant to this Section 3(a)(2) for:

      \begin{enumerate}
      \def\labelenumiv{(\Alph{enumiv})}
      \tightlist
      \item
        any Contract or Extension covering four or fewer NBA Seasons entered into by a player at age 32, 33 or 34; and
      \item
        any Extension that, beginning with the date the Extension is signed, covers five or fewer NBA Seasons and is entered into by a player who has played for his current Team for at least ten consecutive Seasons.
      \end{enumerate}
    \item
      For purposes of this Section 3(a)(2) (governing Over 35 Contracts) only, NBA Seasons shall be deemed to commence on October 1 and conclude on the last day of the Salary Cap Year.
    \item
      ``Zero Year'' means, with respect to an Over 35 Contract, any Season in which the Salary called for under the Contract has been attributed, in accordance with subsection (2)(i), (2)(ii) or (2)(iii) above, to prior Seasons of the Contract. ``First Zero Year'' means, with respect to an Over 35 Contract, the earliest Season in which the Salary called for under the Contract has been attributed, in accordance with subsection (2)(i), (2)(ii) or (2)(iii) above, to prior Seasons of the Contract.
    \end{enumerate}
  \end{enumerate}
\item
  \textbf{Signing Bonuses.}

  \begin{enumerate}
  \def\labelenumii{(\arabic{enumii})}
  \tightlist
  \item
    Amounts Treated as Signing Bonuses: For purposes of determining a player's Salary, the term ``signing bonus'' shall include:

    \begin{enumerate}
    \def\labelenumiii{(\roman{enumiii})}
    \tightlist
    \item
      any amount specifically described in a Player Contract as a signing bonus;
    \item
      any Option Buy-Out Amount;
    \item
      at the time of an assignment of a Player Contract, any amount other than increases in remaining payments to be made pursuant to paragraph 3 of the Contract that, under the terms of the Contract, is earned in the form of a bonus upon assignment of the Contract; and
    \item
      payments in excess of \$250,000 with respect to foreign players, in accordance with subsection (f) below.
    \end{enumerate}
  \item
    Proration: Any signing bonus contained in a Player Contract shall be allocated in equal parts over the number of Seasons (or remaining Seasons in the case of a signing bonus described in subsection (l)(iii) above) covered by such Contract that are fully protected for skill, provided, however, that no portion of a signing bonus contained in a Player Contract that provides for an Early Termination Option shall be allocated to any Season following the Effective Date of such option. In the event that no Season covered by a Player Contract is fully protected for skill, then the entire amount of the signing bonus shall be allocated to the first Season of the Contract or, in the case of a signing bonus described in subsection (l)(iii) above, the Season during which the player's Contract is assigned.
  \item
    Signing Bonus Credits: Upon the occurrence of an event that determines that a player shall not be entitled to receive an Option Buy-Out Amount (the ``non-payment determination''):

    \begin{enumerate}
    \def\labelenumiii{(\roman{enumiii})}
    \tightlist
    \item
      all amounts that were included in the player's Salary pursuant to subsection (l)(ii) above for Seasons up to and including the Season in which the non-payment determination is made (the ``unpaid amounts'') shall be deducted from the calculation of Total Salaries and Benefits for the Season in which the non-payment determination is made;
    \item
      all amounts that were included in the player's Salary pursuant to subsection (l)(ii) above for Seasons following the Season in which the non-payment determination is made shall be deducted from the player's Salary for such Seasons; and
    \item
      the unpaid amounts shall be deducted from the Team's Team Salary, in accordance with the following:
      (A) The total amount available to be deducted from the Team's Team Salary (the ``credit amount'') will equal the aggregate of the unpaid amounts less, for each Season in which a portion of the unpaid amounts was included in the player's Salary and in which his Team's Team Salary did not fall below the Salary Cap, the smallest amount by which his Team's Team Salary exceeded the Salary Cap during such Salary Cap Year.
      (B) The credit amount shall be allocated, in equal parts, over the same number of Seasons over which the unpaid amounts were allocated, beginning with the first Season following the non-payment determination, plus, for each Season following the first Season of such allocation, 10\% of the amount allocated to the first Season.
      (C) If, during the course of any Salary Cap Year in which a credit allocation has been made, the Team's Team Salary does not fall below the Salary Cap, the full credit allocation for such Salary Cap Year will be carried forward to a subsequent Salary Cap Year. If, during the course of a Salary Cap Year in which a credit allocation has been made, the Team's Team Salary does fall below the Salary Cap, the amount carried forward, if any, will equal the amount of the credit allocation for such Season less the largest amount by which the Team's Team Salary fell below the Salary Cap during such Salary Cap Year. In the event a credit allocation is carried forward pursuant to this subsection, such amount shall be deducted from Team Salary in the Season immediately following the last Season in which a portion of the credit amount is then currently being allocated, subject to the terms of this subsection (iii)(C).
    \end{enumerate}
  \item
    Extensions:

    \begin{enumerate}
    \def\labelenumiii{(\roman{enumiii})}
    \tightlist
    \item
      In the event that a Team with a Team Salary at or over the Salary Cap enters into an Extension that calls for or contains a signing bonus, such signing bonus shall be paid no sooner than the first day of the Salary Cap Year covered by the extended term and shall be allocated, in equal parts, over the number of Seasons covered by the extended term that are fully protected for skill. In the event that no Season in the extended term is fully protected for skill, then the entire amount of the signing bonus shall be allocated to the first Season of the extended term.
    \item
      A Team with a Team Salary below the Salary Cap may enter into an Extension that calls for or contains a signing bonus to be paid at any time during the Contract's original or extended term. In the event that a Team with a Team Salary below the Salary Cap enters into an Extension that calls for or contains a signing bonus to be paid no sooner than the first day of the Salary Cap Year covered by such extended term, the bonus shall be allocated in accordance with the proration rules set forth in subsection (4)(i) above. In the event a Team with a Team Salary below the Salary Cap enters into an Extension that calls for or contains a signing bonus to be paid prior to the first day of the Salary Cap Year covered by the extended term, the following rules shall apply:
      (A) The signing bonus shall be allocated in equal parts over the Seasons remaining under the original term of the Contract and the extended term that are fully protected for skill; and
      (B) The Extension shall be deemed a Renegotiation and shall be subject to the rules governing Renegotiations set forth in Section 7 below.
    \end{enumerate}
  \end{enumerate}
\item
  \textbf{Loans to Players.} The following rules shall apply to any loan made by any Team to or at the direction of a player:

  \begin{enumerate}
  \def\labelenumii{(\arabic{enumii})}
  \tightlist
  \item
    If any such loan bears no interest (or interest at an effective rate lower than 9\% per annum), then an amount equal to 9\% per annum of the outstanding balance (or an amount equal to the difference between 9\% per annum of the outstanding balance and the actual rate of interest to be paid by the player) shall be included in the player's Salary.
  \item
    No loan made to a player after July 1, 1996 may (along with other outstanding loans to the player) exceed the amount of the player's Salary for the then-current Season that is fully protected for skill. All loans must be repaid through deductions from the player's remaining compensation over the term of the Contract (prior to the Effective Date of any Early Termination Option) in equal annual amounts (the ``annual allocable repayment amounts''). If a loan is made at a time when the remaining Salary due for the then-current Season that is fully protected for skill is less than the annual allocable repayment amount that would be owed on a loan for the full amount of the player's Salary that is, fully protected for skill for the then-current Season (the ``maximum annual allocable repayment amount''), the maximum loan amount for that Season shall be reduced by the amount by which the maximum annual allocable repayment amount exceeds the amount of remaining Salary that is fully protected for skill. (For example, if a Player has \$1 million in Salary in the first Season of a five-year Contract, and a loan is made during that Season at a time when the Player has already received his Salary for that Season, the loan may not exceed \$800,000.)
  \item
    Any forgiveness by a Team of a loan to a player shall be deemed a Renegotiation in the Salary Cap Year of such forgiveness and shall be subject to the rules governing Renegotiations set forth in Section 7 below.
  \end{enumerate}
\item
  \textbf{Performance Bonuses.}

  \begin{enumerate}
  \def\labelenumii{(\arabic{enumii})}
  \tightlist
  \item
    For purposes of determining a player's Salary each Season, except as provided in subsections (2) through (4) below, any amounts that may be earned as a bonus based upon the performance of an individual player or a Team (provided such bonus may be included in a Player Contract in accordance with Section 5(f) below), shall be included in Salary only if such bonus would be earned if the Team's or player's performance were identical to the performance in the immediately preceding Season.
  \item
    Notwithstanding subsection (1) above, in the event that, with respect to the first Season covered by a Contract or Renegotiation, or the first Season of an extended term, the NBA or the Players Association believes that the performance of a player and/or his team during the immediately preceding Season does not fairly predict the likelihood of the player earning a performance bonus during the current Season, the NBA or the Players Association may request that a jointly selected basketball expert (``Expert'') determine whether (i) in the case of an NBA challenge, it is very likely that the bonus will be earned, or (ii) in the case of a Players Association challenge, it is very likely that the bonus will not be earned. The party initiating a proceeding before the Expert shall carry the burden of proof. The Expert shall render his determination within two weeks of the initiation of the proceeding. Notwithstanding anything to the contrary in this subsection (2), no party may, in connection with any proceeding before the Expert, refer to the facts that, absent a challenge pursuant to this subsection (2), a bonus would or would not be included in a player's Salary pursuant to subsection (1) above, or would be termed ``Likely'' or ``Unlikely'' pursuant to Article I, Section l(z) or (bg). If, following an NBA challenge, the Expert determines that a performance bonus is very likely to be earned, the bonus shall be included in the player's Salary. If, following a Players Association challenge, the Expert determines that a performance bonus is very likely not to be earned, the bonus shall be excluded from the Player's Salary. The Expert's determination that a bonus is very likely to be earned or very likely not to be earned shall be final, binding and unappealable. The fees and costs of the Expert in connection with any proceeding brought pursuant to this subsection (2) shall be borne equally by the parties.
  \item
    In the case of a Rookie or a Veteran who did not play during the immediately preceding Season, a performance bonus will be included in Salary if it is likely to be earned. In the event that the NBA and the Players Association cannot agree as to whether a bonus is likely to be earned, such dispute will be referred to the Expert, who will determine whether the bonus is likely to be earned or not likely to be earned. The Expert shall render his determination within two weeks of the initiation of the proceeding. The Expert's determination that a bonus is likely to be earned or not likely to be earned shall be final, binding and unappealable. The fees and costs of the Expert in connection with any proceeding brought pursuant to this subsection (3) shall be borne equally by the parties.
  \item
    In the event that either party initiates a proceeding pursuant to subsection (2) or (3) above, the player's Salary plus the full amount of any disputed bonuses shall be included in Team Salary during the pendency of the proceeding.
  \item
    In the event the NBA and the Players Association cannot agree on an Expert, any challenge pursuant to subsections (2) and (3) above may be filed with the Grievance Arbitrator in accordance with Article XXXI, Sections 2- 6.
  \item
    Notwithstanding anything to the contrary in this Section 3(d), bonuses and other compensation that are within the sole discretion of the player or unrelated to the performance of skilled basketball (such as weight bonuses, bonuses based upon academic achievement or compensation for off-Season workouts) shall be included in Salary.
  \end{enumerate}
\item
  \textbf{Averaging.} In accordance with the rules set forth in Section 5(e) below, a player's Salary for each of two or more Seasons shall be deemed in certain circumstances to be the average of the aggregate Salaries payable for such Seasons.
\item
  \textbf{Foreign Player Payments.}

  \begin{enumerate}
  \def\labelenumii{(\arabic{enumii})}
  \tightlist
  \item
    Any amount in excess of \$250,000 paid or to be paid by or at the direction of any NBA Team to (i) any basketball team other than an NBA Team, or (ii) any other entity, organization, representative or person, for the purpose of inducing a foreign player to enter into a Player Contract or in connection with securing the right to enter into a Player Contract with a foreign player (as defined in Article X, Section 6) shall be deemed Salary (in the form of a signing bonus) to the player.\\
  \item
    Subject to Article XIII, any payment of \$250,000 or less paid by or at the direction of any NBA Team pursuant to subsection (1) above (the ``\$250,000 exclusion''), shall not be deemed Salary to the player.
  \item
    The \$250,000 exclusion may be paid in a single installment or in multiple installments. The \$250,000 exclusion, whether used in whole or in part, may be used by an NBA Team whenever it signs a foreign player to a new Player Contract, except that the \$250,000 exclusion may not be used, in whole or in part, more than once in any three-Season period with respect to the same foreign player.
  \item
    The \$250,000 exclusion, or any part of it, shall be deemed to have been used as of the date of the Player Contract to which it applies, regardless of when it is actually paid. A schedule of payments relating to the \$250,000 exclusion, or any part of it, agreed upon at the time of the signing of the Player Contract to which it applies, shall not be deemed a multiple use of the \$250,000 exclusion.
  \item
    Notwithstanding subsection (1) above, no amount paid or to be paid pursuant to this subsection (f) shall be counted toward the Minimum Team Salary obligation of such Team in accordance with Section 2(b) or (c) above.
  \end{enumerate}
\end{enumerate}

\hypertarget{determination-of-team-salary.}{%
\section{Determination of Team Salary.}\label{determination-of-team-salary.}}

\begin{enumerate}
\def\labelenumi{(\alph{enumi})}
\tightlist
\item
  \textbf{Computation.} For purposes of computing Team Salary under this Agreement, all of the following amounts shall be included:

  \begin{enumerate}
  \def\labelenumii{(\arabic{enumii})}
  \tightlist
  \item
    Subject to the rules set forth in this Article VII, the aggregate Salaries of all active players (and former players to the extent provided by the terms of this Agreement) attributable to a particular Salary Cap Year, including, without limitation:

    \begin{enumerate}
    \def\labelenumiii{(\roman{enumiii})}
    \tightlist
    \item
      Salaries payable to players whose Player Contracts have been terminated pursuant to the NBA's waiver procedure (without regard to any revised payment schedule that might be provided for in the terminated Player Contracts).
    \item
      Any amount paid to a retired player in accordance with the player's Player Contract.
    \item
      Amounts paid pursuant to awards or judgments for, or settlements of, disputes between a Player and a Team concerning Salary obligations under a Player Contract (as allocated over time, where appropriate, in accordance with the terms of the judgment or settlement), except to the extent that such amounts were previously included in a player's Salary.
    \end{enumerate}
  \item
    The aggregate Free Agent Amounts (as defined in subsection (d) below) attributable to Veteran Free Agents that last played for the Team.
  \item
    An amount with respect to a Team's unsigned First Round Pick, if any, as determined in accordance with subsection (e) below.
  \item
    Value or consideration received by retired players that is determined to be includable in Team Salary in accordance with Article XIII, Section 5(b).
  \item
    The amount of any Salary Cap Exception that is deemed included in Team Salary in accordance with Section 6(j)(2) below.
  \end{enumerate}
\item
  \textbf{Expansion.} The Salary of any player selected by an Expansion Team in an expansion draft and terminated in accordance with the NBA waiver procedure before the first day of the Expansion Team's first Season shall not be included in the Expansion Team's Team Salary, except, to the extent such Salary is paid, for purposes of determining whether the Expansion Team has satisfied its Minimum Team Salary obligation for such Season.
\item
  \textbf{Assigned Contracts.} For purposes of calculating Team Salary, with respect to any Player Contract that is assigned, the assignee Team shall, upon assignment, be deemed to be paying the entire Salary for the then-current Season and all future Seasons.
\item
  \textbf{Free Agents.} Until a Team's Veteran Free Agent resigns with his Team, signs with another NBA Team, or is renounced, he will be included in his prior Team's Team Salary at one of the following amounts (``Free Agent Amounts''):

  \begin{enumerate}
  \def\labelenumii{(\arabic{enumii})}
  \tightlist
  \item
    A Qualifying Veteran Free Agent will be included at 150\% of his prior Salary if it was equal to or greater than the Estimated Average Player Salary, and 200\% of his prior Salary if it was less than the Estimated Average Player Salary.
  \item
    An Early Qualifying Veteran Free Agent will be included at 130\% of his prior Salary; provided, however, that the player's prior Team may, by written notice to the NBA, renounce its rights to sign the player pursuant to the Early Qualifying Veteran Free Agent Exception, in which case the player will be deemed a Non-Qualifying Veteran Free Agent for purposes of this Section 4(d) and Section 6(b) below.
  \item
    A Non-Qualifying Veteran Free Agent will be included at 120\% of his prior Salary.
  \item
    Notwithstanding subsections (1)-(3) above, if the player's prior Salary was equal to or less than the Minimum Annual Salary, he will be included at the then-current Season's Minimum Annual Salary.
  \item
    For purposes of this subsection (d) only, a player's ``prior Salary'' means his Regular Salary for the prior Season plus any signing bonus allocation and the amount of any performance bonuses actually earned for such Season.
  \item
    For purposes of this Section 4(d) only, in the event that a Veteran Free Agent's prior Contract provides for an increase or decrease in Salary between the second-to-last and last Seasons covered by the Contract of greater than \$4 million, such player's prior Salary shall be deemed to be equal to the average of the Salaries for the last two Seasons of the Contract.
  \end{enumerate}
\item
  \textbf{First Round Picks.}

  \begin{enumerate}
  \def\labelenumii{(\arabic{enumii})}
  \tightlist
  \item
    Beginning with the 1995 Draft, a First Round Pick, immediately upon selection in the Draft, shall be included in the Team Salary of the Team that holds his draft rights at 100\% of his applicable Rookie Scale Amount, and, subject to subsection (2) below, shall continue to be included in the Team Salary of any Team that holds his draft rights (including any Team to which the player's draft rights are assigned) until such time as the player signs with such Team or until the Team loses or assigns its exclusive draft rights to the player.
  \item
    In the event that a First Round Pick signs with a non-NBA team, the player's applicable Rookie Scale Amount shall be excluded from the Team Salary of the Team that holds his draft rights, beginning on the date he signs such non-NBA contract or the first day of the Regular Season, whichever is later, and shall be included again in his Team's Team Salary at the applicable Rookie Scale Amount on the following July 1 or the date the player's contract ends (or the player is released from his non-NBA contractual obligations), whichever is earlier, unless the Team relinquishes its exclusive rights to the player in accordance with Article X, Section 3(f). If, after such following July 1, or any subsequent July 1, the player signs another, or remains under, contract with a non-NBA team, the player's applicable Rookie Scale Amount will again be excluded from Team Salary beginning on the date of the contract signing or the first day of the Regular Season commencing after such July 1, whichever is later, and will again be included in Team Salary at the applicable Rookie Scale Amount on the following July I or the date the player's contract ends (or the player is released from his non-NBA contractual obligations), whichever is earlier, unless the Team relinquishes its exclusive rights to the player in accordance with Article X, Section 3(f).
  \item
    For purposes of this Section 4(e), the ``applicable Rookie Scale Amount'' for a First Round Pick means, with respect to any Salary Cap Year, the Rookie Scale Amount that would apply if the player were drafted in the Draft immediately preceding such Salary Cap Year at the same draft position at which he was actually selected.
  \end{enumerate}
\item
  \textbf{Renouncing.} To renounce a Veteran Free Agent, a Team can, at any time, provide the NBA with an express, written statement renouncing its right to re-sign the player, effective no earlier than the July 1 following the last Season covered by the Contract. (The NBA shall notify the Players Association of any such renunciation by fax within two business days following receipt of notice of such renunciation.) In the event of a renunciation, the player's Prior Team (i) may not resign such player for 56 days after the first game of the following Season (or for 56 days after the first game of the then-current Season in the event such renunciation occurs after the start of the Season, but prior to the 56th day after the first game of such Season), and (ii) thereafter may re-sign such player only to the extent of the Team's then-current Room (i.e., the Team cannot sign such player pursuant to Section 6(b) below).
\item
  \textbf{Long-Term Injuries.} Any player who suffers a career-ending injury or illness, and whose contract is terminated by the Team in accordance with the NBA waiver procedure, will be excluded from his team's Team Salary as follows:

  \begin{enumerate}
  \def\labelenumii{(\arabic{enumii})}
  \tightlist
  \item
    If the injury or illness occurs on or after July 1, but prior to January 1 of any Season, then, beginning on the second July 1 following the injury or illness, the Team may apply to the NBA to have the player's Salary for each remaining Season of the Contract excluded from Team Salary. (For example, if the career-ending injury or illness occurs on August 1, 1996, the Team may apply to have the player's Salary excluded from Team Salary beginning on July 1, 1998.)
  \item
    If the injury or illness occurs on or after January 1 but prior to July 1 of any Season, then, beginning on the second anniversary of the injury or illness, the Team may apply to the NBA to have the player's Salary for each remaining Season of the Contract excluded from Team Salary.
  \item
    The determination of whether a player has suffered a career-ending injury or illness shall be made by a physician selected jointly by the NBA and the Players Association.
  \item
    Notwithstanding subsections (1) through (3) above, a player's Salary shall not be excluded from Team Salary if, after the date on which a career-ending injury or illness is alleged to have occurred but before his Salary is excluded from Team Salary, the player played in more than ten NBA games in anyone Season or in a total of 15 games over two Seasons.
  \item
    Notwithstanding subsections (1) through (3) above, if, after a player's Salary is excluded from Team Salary in accordance with this Section 4(g), the player plays in 10 NBA games in anyone Season, the excluded Salary for that Season and each subsequent Season shall thereupon be included in Team Salary. If, after a player's Salary is excluded from Team Salary in accordance with this Section 4(g), the player plays in 15 or more NBA games over two Seasons but did not play in ten games in the first of such two Seasons, the excluded Salary for the second Season and each subsequent Season shall thereupon be included in Team Salary. After a player's Salary for one or more Seasons has been included in Team Salary in accordance with this subsection (5), the player's Team shall be permitted at the appropriate time to re-apply to have the player's Salary (for each Season remaining at the time of the re-application) excluded from Team Salary in accordance with the rules set forth in this subsection (g).
  \end{enumerate}
\item
  \textbf{Summer Contracts.}

  \begin{enumerate}
  \def\labelenumii{(\arabic{enumii})}
  \tightlist
  \item
    Except as provided in subsection (2) below, from the day following the last day of a Season until the day prior to the first day of the next Regular Season, a Team may enter into Player Contracts that will not be included in computing Team Salary, provided that such Contracts satisfy the requirements of this Section 4(h) (a ``Summer Contract''). Except as set forth in the following sentence, no Summer Contract may provide for (i) any compensation of any kind, including a signing bonus, performance bonus or salary advance, that is or may be paid or earned prior to the first day of the next Regular Season, or (ii) salary protection or insurance of any kind. The only consideration that may be provided to a player signed to a Summer Contract, prior to the start of the Regular Season, is per diem, lodging, transportation, compensation in accordance with paragraph 3(b) of the Uniform Player Contract, and a disability insurance policy covering disabilities incurred while such player participates in summer leagues or rookie camps for the Team. No later than the day prior to the first day of a Regular Season, a Team that has entered into one or more Summer Contracts must terminate such Contracts except to the extent the Team has Room for such Contracts.
  \item
    A Team may not enter into a Summer Contract with a Veteran Free Agent who last played for the Team unless the Contract is for one Season only and provides for a Salary of no more than the Minimum Annual Salary applicable to such Veteran Free Agent.
  \end{enumerate}
\item
  \textbf{Team Salary Summaries.}

  \begin{enumerate}
  \def\labelenumii{(\arabic{enumii})}
  \tightlist
  \item
    The NBA shall provide the Players Association with Team Salary summaries and a list of current Exceptions and Base Year Compensations once a month during the Regular Season and once every two weeks during the off-season.
  \item
    In the event that the NBA fails to provide the Players Association with any Team Salary summary or list of Exceptions or Base Year Compensations as provided for in subsection (1) above, the Players Association shall notify the NBA of such failure, and the NBA, upon receipt of such notice, shall, as soon as reasonably possible, but in no event later than two business days following receipt of such notice, provide the Players Association with any such summary or list that should have been provided pursuant to subsection (1) above.
  \end{enumerate}
\end{enumerate}

\hypertarget{operation-of-salary-cap}{%
\section{Operation of Salary Cap}\label{operation-of-salary-cap}}

\begin{enumerate}
\def\labelenumi{(\alph{enumi})}
\tightlist
\item
  \textbf{Basic Rule.} A Team's Team Salary may not exceed the Salary Cap at any time unless the Team is using one of the Exceptions set forth in Section 6 below.
\item
  \textbf{Room.} Subject to the provisions of this Section 5, any Team with Room may enter into a Player Contract that calls for a Salary in the first Season of such Contract that would not exceed the Team's then-current Room.
\item
  \textbf{20\% Rule.}

  \begin{enumerate}
  \def\labelenumii{(\arabic{enumii})}
  \tightlist
  \item
    Except as provided in Sections 5(d) (Banked Room), 6(b)(1) (Qualifying Veteran Free Agent Exception), and 7(b) (Rookie Scale Extensions) below, for each Season of a Player Contract after the first Season, and for each Season of an Extension after the first Season of the extended term (including, in either case, any Season following the Effective Date of an Option or an Early Termination Option), the player's Salary, excluding performance bonuses, may increase over the previous Season's Salary, excluding performance bonuses, by no more than (i) 20\% of the Regular Salary for the first Season of the Contract, or (ii) in the case of an Extension, 20\% of the Regular Salary for the last Season of the original term of the Contract.
  \item
    In the event that the first Season of a Contract provides for performance bonuses, the total amount of Likely Bonuses in each subsequent Season of the Contract may increase by up to 20\% of the amount of Likely Bonuses in the first Season, and the total amount of Unlikely Bonuses in each subsequent Season may increase by up to 20\% of the amount of Unlikely Bonuses in the first Season. With respect to an Extension, in the event that the last Season of the original term of the Contract provides for performance bonuses, the amount of Likely Bonuses and Unlikely Bonuses in each Season of the Extension after the first Season of the extended term may increase by up to 20\% of the amount of Likely Bonuses and Unlikely Bonuses, respectively, in the last Season of the original term.
  \end{enumerate}
\item
  \textbf{Banked Room.} In the event that the first Season of a Player Contract results in Banked Room, the unused portion of such Banked Room may be used in any subsequent Season to provide for an increase in Salary and Unlikely Bonuses over the previous Season's Salary in excess of that permitted by the 20\% Rule.
\item
  \textbf{Averaging.}

  \begin{enumerate}
  \def\labelenumii{(\arabic{enumii})}
  \tightlist
  \item
    If, in accordance with Article VII, Section 5(d) (Banked Room), 6(b)(1) (Qualifying Veteran Free Agent Exception), or 7(b) (Rookie Scale Extensions), the Salary in any Season of a Player Contract after the first Season (or, in any Season of an Extension, after the first Season of the extended term), including any Season following the Effective Date of an Option or an Early Termination Option, increases or decreases over the previous Season's Salary by more than 20\% of Regular Salary in the first Season of the Contract (or, in the case of an Extension, by more than 20\% of the Regular Salary in the last Season of the original term of the Contract), the player shall be deemed to have a Salary for each Season of the Contract (or extended term) equal to the average of the aggregate Salaries for each such Season.
  \item
    If, in accordance with subsection (1) above, a Contract (or Extension) with an Early Termination Option would be subject to averaging but the average of the Salaries in the Seasons preceding the Effective Date of the Early Termination Option exceed the average of the Salaries in the Seasons following the Effective Date of the Early Termination Option, then, notwithstanding subsection (1) above, only the Seasons preceding the Effective Date of the Early Termination Option shall be averaged; provided, however, that if the player fails to exercise the Early Termination Option and his Salary increases or decreases in any Season following the Effective Date of the Early Termination Option by more than 20\% of the Regular Salary in the first Season of the Contract, the player shall be deemed to have a Salary for each Season following the Effective Date of the Early Termination Option equal to the average of the aggregate Salaries for the Seasons following the Effective Date of the Early Termination Option.
  \item
    In the event a Rookie Scale Contract is extended pursuant to Section 7(b) below, and the Contract is assigned to another Team prior to the subsequent July 1, the player subject to such Contract shall be deemed to have a Salary for the last Season of the original term of the Contract and the extended term equal to the average of the aggregate Salaries for such Seasons.
  \item
    A Contract that is not subject to averaging at the time it is entered into shall not later be subject to averaging, other than as a result of a Renegotiation or Extension. An Averaged Contract shall not later be de-averaged notwithstanding any subsequent renegotiation of such Contract.
  \end{enumerate}
\item
  \textbf{Performance Bonuses.}

  \begin{enumerate}
  \def\labelenumii{(\arabic{enumii})}
  \tightlist
  \item
    No Team may enter into a Player Contract containing a performance bonus for the first Season of the Contract that, if earned or paid during the first Season covered by such Contract, would result in the Team's Team Salary exceeding the Room under which it is signing the Contract. For the sole purpose of determining whether a Team has Room for a new Unlikely Bonus, the Team's Room shall be deemed reduced by all Unlikely Bonuses in Contracts approved by the Commissioner that may be paid to all of the Team's players that entered into Player Contracts (including Renegotiations) during that Salary Cap Year.
  \item
    The following provisions shall apply to any Averaged Contract containing a performance bonus:

    \begin{enumerate}
    \def\labelenumiii{(\roman{enumiii})}
    \tightlist
    \item
      In the event that at the end of any Season, a performance bonus that is included in a player's Salary for a subsequent Season is determined to be no longer includable in Salary for that subsequent Season, the player's Salary for such subsequent Season shall be reduced by the amount of such bonus.
    \item
      In the event that, at the end of any Season, a performance bonus that is not included in a player's Salary for a subsequent Season is determined to be includable in Salary for that subsequent Season, the player's Salary for such subsequent Season shall be increased by the amount of such bonus.
    \end{enumerate}
  \end{enumerate}
\item
  \textbf{No Futures Contracts.} A Team may only enter into a Player Contract with a player that covers at least the then-current Season, unless such Team and player already are parties to a Player Contract covering the then-current Season.
\end{enumerate}

\hypertarget{exceptions-to-the-salary-cap}{%
\section{Exceptions to the Salary Cap}\label{exceptions-to-the-salary-cap}}

There shall be the following exceptions to the rule that a Team's Team Salary may not exceed the Salary Cap:

\begin{enumerate}
\def\labelenumi{(\alph{enumi})}
\tightlist
\item
  \textbf{Existing Contracts.} A Team may exceed the Salary Cap to the extent of its current contractual commitments, provided that such contracts satisfied the provisions of this Article VII when entered into or were entered into prior to the execution date of this Agreement in accordance with the rules then in effect.
\item
  \textbf{Veteran Free Agent Exception.} Beginning on the July 1 following the last Season covered by a Veteran Free Agent's Player Contract, such player may enter into a new Player Contract with his Prior Team (or, in the case of a player selected in an Expansion Draft that year, with the Team that selected such player in an Expansion Draft) as follows:

  \begin{enumerate}
  \def\labelenumii{(\arabic{enumii})}
  \tightlist
  \item
    If the player is a Qualifying Veteran Free Agent, the new Player Contract may provide for Salary of any amount for any Season of the Contract.
  \item
    If the player is a Non-Qualifying Veteran Free Agent, the new Player Contract may provide for a Salary for the first Season of up to the greater of (i) 120\% of the Regular Salary for the final Season of the player's prior Contract, plus 120\% of any Likely Bonuses and Unlikely Bonuses, respectively, called for in the final Season covered by the player's prior Contract, or (ii) 120\% of the Minimum Annual Salary for the then-current Season. Salary increases and increases in Unlikely Bonuses in subsequent Seasons shall be governed by the 20\% Rule.
  \item
    If the player is an Early Qualifying Veteran Free Agent, the new Player Contract must cover at least two Seasons and may provide for a Salary for the first Season of up to the greater of (i) 175\% of the Regular Salary for the final Season covered by his prior Contract, plus 175\% of any Likely Bonuses and Unlikely Bonuses, respectively, called for in the final Season covered by the player's prior Contract, or (ii) 108\% of the Average Player Salary for the prior Season (or if the prior Season's Average Player Salary has not been determined, 108\% of the Estimated Average Player Salary for the prior Season). Salary increases and increases in Unlikely Bonuses in subsequent Seasons shall be governed by the 20\% Rule.
  \end{enumerate}
\item
  \textbf{Disabled Player Exception.}

  \begin{enumerate}
  \def\labelenumii{(\arabic{enumii})}
  \tightlist
  \item
    Subject to the rules set forth in subsection (j) below, a Team may, in accordance with the rules set forth in this subsection (c), sign or acquire one Replacement Player to replace a player who, as a result of a Disabling Injury or Illness (as defined below), is unable to render playing services (the ``Disabled Player''). Such Replacement Player's Contract may provide a Salary for the first Season of up to the lesser of (i) 50\% of the Disabled Player's Salary at the time the Disabling Injury or Illness occurred, or (ii) 108\% of the Average Player Salary for the prior Season (or, if the prior Season's Average Player Salary has not been determined, 108\% of the Estimated Average Player Salary for the prior Season). Salary increases and increases in Unlikely Bonuses for subsequent Seasons shall be governed by the 20\% Rule.
  \item
    For purposes of this subsection (c), Disabling Injury or Illness means:

    \begin{enumerate}
    \def\labelenumiii{(\roman{enumiii})}
    \tightlist
    \item
      for the period July 1 through the immediately following November 30, any injury or illness that will render a player unable to play all (or the remainder) of the then-current (or upcoming) Season; and
    \item
      for the period December 1 through the immediately following June 30, any injury or illness that will render a player unable to play all of the following Season.
    \end{enumerate}
  \item
    The Exception for a Disabling Injury or Illness that occurs during the period July 1 through the immediately following November 30 shall arise on the date the Team knew or reasonably should have known that the injury or illness would cause the player to miss the then-current (or upcoming) Season, and shall expire 45 days from the date the Exception arises.
  \item
    The Exception for a Disabling Injury or Illness that occurs during the period December 1 through the immediately following June 30 shall arise on the date the Team knew or reasonably should have known that the injury or illness would cause the player to miss all of the following Season; provided, however, that if the Team knew or reasonably should have known prior to the July 1 immediately following the injury or illness that the injury or illness would cause the player to miss all of the following Season, and if the Team does not use the Exception prior to such July 1, then the Exception shall be deemed to arise on July 1. The Exception for a Disabling Injury or Illness that occurs during the period December 1 through the immediately following June 30 shall expire-on the October 1 immediately following the date on which the Exception arises.
  \item
    The determination of whether a player has suffered a Disabling Injury or Illness shall be made by a physician designated by the NBA. The NBA shall advise the Players Association of the determination of its physician within one business day of such determination. In the event the Players Association disputes the NBA physician's determination, the parties will immediately refer the matter to a neutral physician (to be selected by the parties at the commencement of each Salary Cap Year) to review the relevant medical information and, if requested, examine the player and render, within three business days of his receipt of such information, a final determination, which will be final, binding and unappealable. The cost of the NBA physician will be borne by the NBA. The cost of the neutral physician will be borne jointly by the NBA and the Players Association.
  \item
    Notwithstanding a determination by a physician designated by the NBA that a player has suffered a Disabling Injury or Illness, such player, upon recovering from his injury or illness, may be restored to his Team's active list, without affecting any right the Team may have to sign a Replacement Player.
  \item
    In no event may a Team enter into a Contract with a Replacement Player pursuant to subsection (c)(4), unless the Disabled Player's Contract covers the Season following the Season in which the Disabling Injury or Illness occurs.
  \item
    The Disabled Player Exception is available only to the Team with which the player was under Contract at the time his Disabling Injury or Illness occurred.
  \end{enumerate}
\item
  \textbf{\$1 Million Exception.} Subject to the rules set forth in subsection (j) below, a Team may sign one or more Player Contracts not to exceed two Seasons (including partial Seasons) in length that, in the aggregate, provide for first-year Salaries and Unlikely Bonuses totalling up to \$1 million; in accordance with, and subject to, the following:

  \begin{enumerate}
  \def\labelenumii{(\arabic{enumii})}
  \tightlist
  \item
    A Team may use all or any portion of the \$1 Million Exception during no more than three separate Salary Cap Years during the term of this Agreement; provided, however, that the \$1 Million Exception or any portion thereof may not be used in any two consecutive Salary Cap Years.
  \item
    Player Contracts signed pursuant to the \$1 Million Exception may provide for an increase in Regular Salary for the second Season of up to 15\% of the Regular Salary provided for in the first Season, and an increase in Likely Bonuses and Unlikely Bonuses of up to 15\% of the Likely Bonuses and Unlikely Bonuses, respectively, provided for in the first Season.
  \item
    A Team may not use the \$1 Million Exception to sign its own Free Agent.
  \item
    The \$1 Million Exception, if applicable, arises on July 1 of each Salary Cap Year.
  \end{enumerate}
\item
  \textbf{Rookie Exception.} A Team may enter into a Rookie Scale Contract in accordance with Article VIII.
\item
  \textbf{Minimum Annual Salary Exception.} A Team may sign a player to a one-year Player Contract at the Minimum Annual Salary applicable to that player (or, if the Contract is signed during the Regular Season, a prorated portion of such Minimum Annual Salary based on the number of remaining Regular Season days covered by the Contract).
\item
  \textbf{Assigned Player Exception.}

  \begin{enumerate}
  \def\labelenumii{(\arabic{enumii})}
  \item
    Subject to the rules set forth in subsection (j) below, a Team may, for a period of one year following the date of the assignment of a Player Contract to another Team, replace the Traded Player with one or more players acquired by assignment as follows:

    \begin{enumerate}
    \def\labelenumiii{(\roman{enumiii})}
    \tightlist
    \item
      a Team may replace a Traded Player with one or more Replacement Players whose Player Contracts are acquired simultaneously and whose post-assignment Salaries for the then-current Season, in the aggregate, are no more than an amount equal to 115\% of the pre-assignment Salary (or Base Year Compensation, if applicable) of the Traded Player, plus \$100,000.
    \item
      If a Team's assignment of a Traded Player and acquisition of one or more Replacement Players do not occur simultaneously, then the post-assignment Salary or aggregate Salaries of the Replacement Player(s) for the Salary Cap Year in which the Replacement Player(s) are acquired may not exceed 100\% of the pre-assignment Salary (or Base Year Compensation, if applicable) of the Traded Player at the time the Traded Player's Contract was assigned, plus \$100,000.
    \item
      A Team may aggregate the pre-assignment Salaries in two or more Player Contracts for the purpose of acquiring in a simultaneous trade one or more Replacement Players whose post-assignment Salaries, in the aggregate, are no more than an amount equal to 115\% of the pre-assignment aggregated Salaries (or Base Year Compensations, if applicable) of the Traded Players, plus \$100,000. Notwithstanding the preceding sentence, no Player Contract signed or acquired pursuant to an Exception, other than the Veteran Free Agent Exception set forth in Section 6(b) above, may give rise to an aggregated trade exception for a period of two months from the date the Player Contract is signed or acquired.
    \end{enumerate}
  \item
    Except as provided in subsection (3) below, and notwithstanding subsection (j) below, a Team with a Team Salary below the Salary Cap may acquire one or more players by assignment whose post-assignment Salaries, in the aggregate, are no more than an amount equal to the Team's Room plus \$100,000.
  \item
    In lieu of conducting a trade in accordance with subsection (2) above, and notwithstanding subsection (j) below, a Team with a Team Salary below the Salary Cap may (i) replace a Traded Player with one or more Replacement Players whose Player Contracts are acquired simultaneously and whose post-assignment Salaries for the then-current Season, in the aggregate, are no more than an amount equal to 115\% of the pre-assignment Salary of the Traded Player, plus \$100,000, or (ii) aggregate the pre-assignment Salaries in two or more Player Contracts for the purpose of acquiring in a simultaneous trade one or more Replacement Players whose post-assignment Salaries, in the aggregate, are no more than an amount equal to 115\% of the pre-assignment aggregated Salaries of the Traded Players, plus \$100,000. Notwithstanding the preceding sentence, no Player Contract signed or acquired pursuant to an Exception, other than the Veteran Free Agent Exception set forth in Section 6(b) above, may be assigned by a Team in accordance with this subsection (3) for a period of two months from the date the Player Contract is signed or acquired.
  \item
    \begin{enumerate}
    \def\labelenumiii{(\roman{enumiii})}
    \tightlist
    \item
      For purposes of the Assigned Player Exception, a player shall be subject to a Base Year Compensation in the event that the Team Salary of the player's Team is at or above the Salary Cap and the player:
    \end{enumerate}

    \begin{enumerate}
    \def\labelenumiii{(\Alph{enumiii})}
    \tightlist
    \item
      is a Qualifying Veteran Free Agent or Early Qualifying Veteran Free Agent who, in accordance with subsection (b) above, enters into a new Player Contract with his prior Team that provides for a Salary for the first Season of such new Contract greater than 120\% of the Salary for the last Season of the player's immediately prior Contract;
    \item
      is a First Round Pick who, in accordance with Section 7(b), enters into an extension of his Contract between July 1 and October 1 following his second Season that provides for a Salary for the first year of the extended term greater than 120\% of the Salary for the last Season of the original term of the Contract;
    \item
      is subject to a Base Year Compensation on the effective date of this Agreement that has not been extinguished pursuant to subsection (4)(ii)(D) below; or
    \item
      will be subject to a Base Year Compensation on some future date based upon an Extension entered into prior to the effective date of this Agreement.
    \end{enumerate}
  \end{enumerate}

  \begin{enumerate}
  \def\labelenumii{(\roman{enumii})}
  \setcounter{enumii}{1}
  \tightlist
  \item
    A player's Base Year Compensation shall be computed as follows:
    (A) During the first 365 days from the date a player's Base Year Compensation goes into effect (``Year One''), his Base Year Compensation will equal the greater of (1) the Salary for the last Season of his preceding Contract or, in the case of an Extension, the last Season of the original term of the Contract (the preceding amount hereinafter referred to as the ``Base Year Salary''), or (2) one-third of the Salary for Year One of his new Contract (or extended term, if applicable).
    (B) During the second 365 days from the date the player's Base Year Compensation goes into effect (``Year Two''), his Base Year Compensation will equal the greater of (1) 120\% of his Base Year Salary, or (2) two-thirds of the Salary for Year Two of his new Contract (or extended term, if applicable).
    (C) A player's Base Year Compensation will expire and be of no further effect on the 731st day of his new Contract (or extended term, if applicable).
    (D) A player subject to a Base Year Compensation as of the effective date of this Agreement in accordance with subsection (4)(i)(C) above shall be treated as if the phase-out rule set forth above was in effect when such Base Year Compensation came into effect.
  \item
    In the event a player who is subject to a Base Year Compensation during the last Season of his Contract signs an Extension or a new Player Contract with his Prior Team, the player shall continue to be subject to a Base Year Compensation in his new Contract or Extension through (A) in the case of a new Contract, the 730th day from the date the player's Base Year Compensation went into effect or the 365th day from the date the new Contract is signed, whichever is later, and (B) in the case of an Extension, the 730th day from the date the player's Base Year Compensation went into effect; provided, however, that in the event the player is a Qualifying Veteran Free Agent or an Early Qualifying Veteran Free Agent, and the new Contract would itself subject the player to a Base Year Compensation in accordance with subsection (4)(i)(A) above, then the player shall be subject to a new Base Year Compensation for a period of 730 days from the date the new Contract is signed in accordance with subsection (4 )(ii) above. For purposes of computing such Base Year Compensation pursuant to subsection (4)(ii), the player's Base Year Salary shall equal the Base Year Compensation applicable in the last Season of his prior Contract or, in the case of an Extension, the last Season of the original term of the Contract.
  \item
    A player's Base Year Compensation shall be extinguished upon any of the following:
    (A) The Team Salary of the player's Team falls below the Salary Cap, unless this occurs prior to the beginning of an extended term that gives rise to the Base Year described in subsection (4 )(i)(D) above;
    (B) The player signs a Contract with a Team other than his prior Team; or
    (C) The player is traded, unless the trade occurs prior to the beginning of an extended term described in subsection 4(i)(D) above.
  \end{enumerate}
\item
  \textbf{Reinstatement.} If a player who has been disqualified from further association with the NBA in accordance with Section 1 of the NBA/NBPA Anti-Drug Agreement is later reinstated pursuant to Section 9 of that Agreement, the Team for which the player last played may enter into a Player Contract, in accordance with the rules set forth in Section 9 of the NBA/NBPA Anti-Drug Agreement, with such player that provides for a Salary for the first Season of up to the player's Salary for the Salary Cap Year in which he was disqualified, even if the Team has a Team Salary at or above the Salary Cap or such Player Contract causes the Team to have a Team Salary above the Salary Cap. If, in accordance with the preceding sentence, a Team and a player enter into a Player Contract and such Contract covers more than one Season, increases in Salary for Seasons following the first Season shall be governed by the 20\% Rule.
\item
  \textbf{Non-Aggregation.} Other than in accordance with subsection (g) above, a Team may not aggregate or combine any of the Exceptions set forth above in order to sign one or more players at Salaries greater than that permitted by anyone of the Exceptions. If a Team has more than one Exception available at the same time, the Team shall have the right to choose which Exception it wishes to use to sign a player.
\item
  \textbf{Other Rules.}

  \begin{enumerate}
  \def\labelenumii{(\arabic{enumii})}
  \tightlist
  \item
    A Team shall be entitled to use the Disabled Player, \$1 Million, and Assigned Player Exceptions set forth in subsections (c), (d) and (g) above, respectively, except as set forth in subsections (g)(2) and (3) above, only if, at the time any such Exception arises and at all times until it is used, (i) the Team's Team Salary is at or above the Salary Cap, or (ii) the amount by which the Team's Team Salary is below the Salary Cap is less than the amount that would be available to the Team in accordance with the Exception.
  \item
    In the event that, when a Disabled Player Exception, \$1 Million Exception or Assigned Player Exception arises, the Team's Team Salary is below the Salary Cap (or in the event that, prior to the expiration of any such Exception, the Team's Team Salary falls below the Salary Cap) by less than the amount of such Exception, then (i) the Team's Team Salary shall include, until the Exception is actually used or until the Team no longer is entitled to use the Exception, the amount of the Exception (or any unused portion of the Exception), and (ii) the amount by which the Team's Team Salary is less than the Salary Cap shall thereby be extinguished. When the Disabled Player Exception is used to sign or acquire a player, the Replacement Player's Salary for the first Season of his Contract, instead of the amount of the Exception, shall be included in Team Salary. When a \$1 Million Exception or an Assigned Player Exception is used to sign or acquire a player, respectively, the Salary for the first Season of the signed or acquired Contract plus any unused portion of the Exception, instead of the full amount of the Exception, shall be included in Team Salary. A Team may at any time renounce its rights to use an Exception, in which case the Exception (or any unused portion of the Exception) no longer will be included in Team Salary.
  \end{enumerate}
\end{enumerate}

\hypertarget{renegotiations-and-extensions.}{%
\section{Renegotiations and Extensions.}\label{renegotiations-and-extensions.}}

\begin{enumerate}
\def\labelenumi{(\alph{enumi})}
\tightlist
\item
  \textbf{Veteran Extensions.} No Player Contract, other than a Rookie Scale Contract, may be extended except in accordance with the following:

  \begin{enumerate}
  \def\labelenumii{(\arabic{enumii})}
  \tightlist
  \item
    Subject to the rules set forth in subsection (2) below, a Player Contract covering a term of six or seven Seasons may be extended beginning on the fourth anniversary of the signing of the Contract, and a Player Contract with a term of four or five years may be extended beginning on the third anniversary of the signing of the Contract.
  \item
    A Player Contract that has been extended, or that has been renegotiated to provide for an increase in Salary or performance bonuses in any Season of the Contract of more than 10\%, may not subsequently be extended until the third anniversary of such Extension or Renegotiation. Any Player Contract in effect on the effective date of this Agreement may be extended beginning on the third anniversary of the later of:

    \begin{enumerate}
    \def\labelenumiii{(\roman{enumiii})}
    \tightlist
    \item
      the signing of the Contract, or (ii) the signing of any Extension, or of any Renegotiation providing for an increase in Salary or performance bonuses in any Season of more than 10\%, entered into prior to the date of this Agreement.
    \end{enumerate}
  \item
    A Player Contract extended in accordance with subsection (1) or (2) above may, in the first Season of the extended term, provide for a Salary of up to 120\% of the Regular Salary in the last Season of the original term of the Contract. In the event that the last Season of the original term of the Contract provides for performance bonuses, the first Season of the extended term may provide for Likely Bonuses and Unlikely Bonuses of up to 120\% of the Likely Bonuses and Unlikely Bonuses, respectively, in the last year of the original term. The permissible increases in Salary and performance bonuses in the subsequent Seasons of the extended term shall be governed by the 20\% Rule.
  \item
    Notwithstanding subsection (3) above:

    \begin{enumerate}
    \def\labelenumiii{(\roman{enumiii})}
    \tightlist
    \item
      any Player Contract that, pursuant to the Salary Cap rules, has been averaged as of the effective date of this Agreement (including Contracts that begin with the 1995-1996 Season) may, in the first Season of an extended term, provide for a Salary of up to the greater of (A) 120\% of the averaged Regular Salary in the last Season of the original term of the Contract, or (B) 120\% of what the Regular Salary would have been in the last Season of the original term had the Contract not been averaged, plus, in either case, 120\% of any performance bonuses provided for in the last Season of such Contract in accordance with subsection (3) above; and
    \item
      any Player Contract of a player who has played for his current Team for at least ten Seasons and whose Salary in the last Season of the original term of the Contract is less than the Salary in the second-to-last Season of such Contract may, in the first Season of an extended term, provide for a Salary equal to 120\% of the greater of (1) the average of the Regular Salaries for each Season covered by the original Contract beginning with the Season in which such 'Contract was entered into, or previously extended, as the case may be, or (2) the Regular Salary in the last Season covered by his original Contract, plus, in either case and subject to the foregoing limitations, 120\% of any performance bonuses called for in the last Season of such Contract in accordance with subsection (3) above.
    \end{enumerate}
  \end{enumerate}
\item
  \textbf{Rookie Scale Extensions.} A First Round Pick may enter into an extension of a Rookie Scale Contract during the period July 1 to October 1 following the second Season of such Contract. Such an Extension may provide for any amount of Salary in each Season of the extended term.
\item
  \textbf{Renegotiations.} No Player Contract may be renegotiated except in accordance with the following:

  \begin{enumerate}
  \def\labelenumii{(\arabic{enumii})}
  \tightlist
  \item
    Subject to subsections (2) and (3) below, a Player Contract covering a term of four or more Seasons may be renegotiated beginning on the third anniversary of the signing of the Contract.
  \item
    Subject to subsection (3) below, any Player Contract that has been renegotiated in accordance with subsection (1) above to provide for an increase in Salary or performance bonuses in any Season of the Contract of more than 10\%, or extended in accordance with subsections (a) and (b) above, may not subsequently be renegotiated until the third anniversary of such Extension or Renegotiation.
  \item
    Assuming subsections (1) or (2) above are satisfied, a Team with a Team Salary below the Salary Cap may renegotiate a Player Contract by increasing the player's Regular Salary or providing Likely Bonuses or Unlikely Bonuses in an amount not to exceed, in the aggregate, the Team's Room in the then-current Season and, for each subsequent Season, increasing the player's Regular Salary, Likely Bonuses and Unlikely Bonuses by the amount of the increase in Salary, Likely Bonuses and Unlikely Bonuses, respectively, for the then-current Season, plus, for each such subsequent Season, 20\% of the increase in Regular Salary, Likely Bonuses and Unlikely Bonuses, respectively, in the then-current Season.
  \item
    In no event may a Team with a Team Salary at or above the Salary Cap renegotiate a Player Contract.
  \end{enumerate}
\item
  \textbf{Other.}

  \begin{enumerate}
  \def\labelenumii{(\arabic{enumii})}
  \tightlist
  \item
    In no event shall a Team and player negotiate a decrease in Salary for the then-current Season or for any remaining Season of a Player Contract.
  \item
    A Player Contract that is extended pursuant to subsection (a) above may be renegotiated simultaneously, but only in accordance with the rules set forth in subsection (c) above.
  \item
    For the sole purpose of enabling an assignee Team to acquire a Player Contract by trade, the player and the assignor Team may agree to waive all or any portion of an assignment bonus, but only to the extent necessary to make the trade permissible in accordance with the rules set forth in Section 6(g) above. In the event that, in connection with a trade, a player's Contract is amended in accordance with this subsection (3), such Contract may not be subsequently extended or renegotiated until the later of (i) six months from the date of the assignment, or (ii) the first date on which the Contract could otherwise be extended or renegotiated pursuant to this Section 7.
  \item
    A Team and a player may at any time agree to amend a Player Contract to provide that, in exchange for reducing the amount of (or eliminating) the player's salary protection, the Team will terminate the Contract in accordance with the NBA waiver procedure.
  \item
    In no event shall a Team and player amend a Contract for the purpose of terminating or shortening the term of the Contract, except in accordance with the NBA waiver procedure or Article XII, Section 2.
  \item
    Except to the extent a Team's Team Salary is below the Salary Cap, no Salary for any Season covered by the extended term of a Contract may be paid during the original term, including, without limitation, signing bonuses.
  \item
    For purposes of this Section 7, if a Player Contract is signed after the beginning of a Season, the Season in which the Contract is signed shall be counted as one full Season covered by the Contract.
  \end{enumerate}
\end{enumerate}

\hypertarget{accounting-procedures.}{%
\section{Accounting Procedures.}\label{accounting-procedures.}}

\begin{enumerate}
\def\labelenumi{(\alph{enumi})}
\item
  \begin{enumerate}
  \def\labelenumii{(\arabic{enumii})}
  \tightlist
  \item
    The NBA and the Players Association shall jointly engage an independent auditor (the ``Accountants'') to provide the parties with an ``Audit Report'' setting forth BRI, Team Salary and Benefits of each NBA team for the immediately preceding Season. The Audit Report is to be prepared in accordance with the provisions and definitions contained in this Agreement. The engagement of the Accountants shall be deemed to be renewed annually unless they are discharged by either party during the period from the submission of an Audit Report up to January 1 of the following year. The parties agree to share equally the costs incurred by the Accountants in preparing the Audit Report.
  \item
    The Accountants shall submit a draft Audit Report to the NBA and the Players Association, along with relevant supporting documentation, on or before the July 31 following the conclusion of each Salary Cap Year. The final Audit Report shall be submitted by the Accountants to the parties on or before the following August 15. The NBA, the Players Association and the Teams shall use their best efforts to facilitate the Accountants' timely completion of the Audit Report. Beginning with the Audit Report for the 1996-97 Salary Cap Year, in the event that, for any reason, the Accountants fail to submit to the parties a final Audit Report by August 15, the Accountants shall prepare an interim Audit Report (the ``Interim Audit Report'') by such date setting forth the Accountants' best estimate of BRI and Total Salaries and Benefits for such Salary Cap Year. Such Interim Audit Report shall include:

    \begin{enumerate}
    \def\labelenumiii{(\roman{enumiii})}
    \tightlist
    \item
      All amounts of BRI and Total Salaries and Benefits for such Salary Cap Year as to which the Accountants have completed their review and, by written agreement of the Players Association and the NBA (waiving their respective rights to dispute such amounts), are not in dispute.
    \item
      With respect to any amounts that are in dispute (whether such dispute would be for the Accountants or the System Arbitrator to decide under this Agreement), the NBA's good faith proposal as to the proper amount that should be included in the Audit Report.
    \item
      With respect to any amounts as to which the Accountants have not yet completed their review, the portion of such amount, if any, as to which the Accountants have completed their review and, by written agreement of the Players Association and the NBA (waiving their respective rights to challenge such portion), are not in dispute.
    \item
      A projected amount for national broadcast and network cable television revenues, as determined in accordance with Article VII, Section l(a)(6).\\
      As soon as practicable after the Interim Audit Report is submitted to the parties, the Accountants shall submit the final Audit Report, including a description of the differences, if any, from the Interim Audit Report.
    \end{enumerate}
  \end{enumerate}
\item
  For purposes of determining BRI and Total Salaries and Benefits, the Accountants shall perform at least such review procedures as shall be agreed upon by the parties. In connection with the preparation of Audit Reports for each Salary Cap Year, each Team and the NBA shall submit a report to the Accountants, the NBA and the Players Association setting forth BRI, Team Salaries and Benefits information for such Salary Cap Year on forms agreed upon by the NBA, the Players Association and the Accountants (the ``BRI Reports'').
\item
  The Accountants shall review the reasonableness of any estimates of revenues or expenses for a Season included in any Team's and the NBA's BRI Reports for such Season and may make such adjustments in such estimates as they deem appropriate. To the extent the actual amounts of revenues received or expenses incurred for a Season differ from such estimates, adjustments shall be made in BRI for the following Salary Cap Year in accordance with the provisions of subsection (g) below.
\item
  With respect to expenses deducted by the NBA or the Teams, the NBA and the Teams shall report in BRI Reports only those expenses that are reasonable and customary in accordance with the provisions of Article VII, Section l(a)(l). Subject to the terms of Article VII, Section 1(a)(3) (Expense Ratios) and Section 9 (Players Association Audit Rights), all categories of expenses deducted in a BRI Report completed by the NBA or a Team shall be reviewed by the Accountants, but such categories shall be presumed to be reasonable and customary and the amount of the expenses deducted by the NBA or a Team that come within such expense categories shall also be presumed to be reasonable and customary, unless such categories or amounts are found by the Accountants to be either unrelated to the revenues involved or grossly excessive.
\item
  The Accountants shall notify designated representatives of the NBA and the Players Association: (i) if the Accountants have any questions concerning the amounts of revenues or expenses reported by the Teams and the NBA or any other information contained in the BRI Reports; or (ii) if the Accountants propose that any adjustments be made to any revenue or expense item or any other information contained in the BRI Reports.
\item
  In the event of any dispute concerning the amounts (as opposed to includability or the interpretation, validity or application of this Agreement) of any revenues or expenses to be included in the BRI Reports that cannot be resolved among the parties (hereinafter referred to as ``Disputed Adjustments''), such dispute shall be resolved by the Accountants after consulting and meeting with representatives of both parties. Notwithstanding the foregoing, either party shall have the right to contest, by commencing a proceeding before the System Arbitrator pursuant to the provisions of Article XXXII of this Agreement, any Disputed Adjustments made by the Accountants whenever such Disputed Adjustments for all Teams are adverse to the party commencing the proceeding in an aggregate amount of \$5 million or more for any Season covered by this Agreement. If the Disputed Adjustments for all Teams are adverse to the party commencing the proceeding in an aggregate amount of \$5 million or more but less than \$10 million for any Season of the Agreement, the parties agree that: (i) the hearing will take place on an expedited basis and will not last longer than one full day, provided, however, that if, despite the reasonable efforts of the parties, the hearing cannot be completed in one day, the hearing shall continue, unless the parties otherwise agree, day-to-day until concluded; and (ii) if the party that brings the proceeding does not prevail after the hearing, then that party shall pay the reasonable costs and expenses, including attorneys' fees, of the other party for its defense of the proceeding. The immediately preceding sentence shall have no application to proceedings in which the Disputed Adjustments for all Teams adverse to the party bringing the proceeding equal or exceed \$10 million in the aggregate. All other disputes among the parties as to the interpretation, validity, or application of this Agreement, or with respect to any Salary or Benefits amount included in a BRI Report, shall be resolved exclusively pursuant to the dispute resolution procedures of Article XXXII.
\item
  The Accountants shall indicate which amounts included in BRI for a Season, if any, represent estimates of revenues. With respect to any such estimated revenues, the Accountants shall, in preparing the Audit Report for the immediately succeeding Season (``Subsequent Audit Report''), determine the actual revenues received for the prior Season and include in such Subsequent Audit Report the amount of the aggregate difference, if any, between all such estimated revenues for the prior Season and the actual revenues received for such Season (the ``Estimated Revenue Adjustment'').
\end{enumerate}

\hypertarget{players-association-audit-rights.}{%
\section{Players Association Audit Rights.}\label{players-association-audit-rights.}}

\begin{enumerate}
\def\labelenumi{(\alph{enumi})}
\tightlist
\item
  \textbf{Team Audits.} The Players Association shall have the right as part of the annual review of BRI Reports to retain its own accountants (the ``Players Association's Accountants''), at its own expense, after the submission of each Audit Report under this Agreement (the ``First Audit''), to audit the books and records of at least five NBA teams (of its choosing) and shall also have the right to review the books and records of the NBA League Office, provided, however, that such review shall be limited to (i) revenue items, and (ii) expense items that appear in the BRI Reports. In the event that, in the opinion of the Players Association's Accountants, such audit indicates misallocations or miscategorizations of revenues or expenses (other than with respect to matters that constituted Disputed Adjustments in connection with the prior Audit Report) resulting in an understatement of BRI in excess of \$1.5 million, they shall submit to the NBA proposed adjustments to BRI consistent with their findings. In the event that the NBA disputes such proposed adjustments, such proposed adjustments shall be deemed to be ``Disputed Adjustments'' and shall be resolved in accordance with the procedures of Section 8(f) above. In addition, in the event that First Audit Disputed Adjustments in excess of \$1.5 million are resolved in favor of the Players Association, the Players Association shall then have the right, that Season, to have the Players Association's Accountants audit an additional five NBA teams, in accordance with the foregoing procedures (the ``Second Audit''). If, as a result of the Second Audit, additional Disputed Adjustments in excess of \$1.5 million are resolved in favor of the Players Association, the Players Association shall then have the right, that Season, to have the Players Association's Accountants audit all remaining NBA Teams. The amount of any and all Disputed Adjustments that are ultimately resolved in favor of the Players Association in accordance with this Section 9(a) shall be added to BRI in the Season in which such resolution is reached.
\item
  \textbf{Expense Audit.} The Players Association shall have the right to retain the Players Association's Accountants to conduct one audit, at its own expense, of the expenses incurred in connection with the proceeds that come within Article VII, Section l(a)(l)(viii) regardless of whether such expenses exceed the applicable Expense Ratios set forth in Exhibit C. In the event that in the opinion of the Players Association's Accountants, such audit indicates a misallocation or miscategorization of expenses resulting in an understatement of BRI, they shall submit proposed adjustments to the NBA consistent with their findings. In the event the NBA disputes such proposed adjustments, such proposed adjustments shall be deemed to be Disputed Adjustments and resolved in accordance with the procedures of Section 8(f) above. The amount of any and all such Disputed Adjustments that are resolved in the Players Association's favor shall be included in BRI in the year in which such resolution is reached. In addition, in the event that any such Disputed Adjustments are resolved in the Players Association's favor, the Accountants shall be directed to correct such expense misallocations and/or miscategorizations in the remaining Seasons of the Agreement.
\item
  \textbf{System Arbitrator Review.} Notwithstanding the foregoing procedures, in the event that any of the Disputed Adjustments set forth in this Section 9 concern the includability or the interpretation, validity or application of this Agreement, such Disputed Adjustments shall be resolved by the System Arbitrator.
\end{enumerate}

\hypertarget{rookie-scale}{%
\chapter{ROOKIE SCALE}\label{rookie-scale}}

\hypertarget{rookie-scale-contracts.}{%
\section{Rookie Scale Contracts.}\label{rookie-scale-contracts.}}

Except as provided in Sections 2 and 3 below, beginning with the 1995 NBA Draft, the following rules shall apply to every Rookie Scale Contract:

\begin{enumerate}
\def\labelenumi{(\alph{enumi})}
\tightlist
\item
  Each Rookie Scale Contract shall cover a period of three Seasons.
\item
  A Rookie Scale Contract shall provide in each of the three Seasons covered by the Contract at least 80\% of the applicable Rookie Scale Amount in Current Cash Compensation. Components of Salary in excess of 80\%, if any, are subject to individual negotiation, except that (i) in no event may Salary plus Unlikely Bonuses in any Season exceed 120\% of the applicable Rookie Scale Amount, and (ii) a Rookie Scale Contract may not provide for a signing bonus (except for ``foreign player payments'' in excess of \$250,000 made in accordance with Article VII, Section 3(f)) or a loan. A Rookie Scale Contract may provide for a payment schedule in any Season that is more favorable to the player than that called for under paragraph 3 of the Contract, provided that no payments for any Season are made prior to the July 1 preceding such Season.
\item
  A First Round Pick who does not sign with the Team that holds his draft rights for any portion of the Season immediately following the Draft in which he was selected shall be treated, for purposes of determining the applicable Rookie Scale Amounts at such time as he enters into a Rookie Scale Contract, as if he were drafted in the Draft immediately preceding the first Season of such Contract at the same draft position at which he was actually selected.
\item
  A Rookie Scale Contract must provide for salary protection for skill and non-insured injury or illness in each Season to the extent of not less than 80\% of the applicable Rookie Scale Amounts.
\end{enumerate}

\hypertarget{rookie-scale-contracts-for-later-signed-picks.}{%
\section{Rookie Scale Contracts for Later-Signed Picks.}\label{rookie-scale-contracts-for-later-signed-picks.}}

Except as provided in Section 3 below, a First Round Pick who does not sign with the Team that holds his draft rights for any portion of the three Seasons following the NBA Draft in which he was selected (and who did not play intercollegiate basketball during such period) may enter into either (i) a Rookie Scale Contract in accordance with Section 1 above, or (ii) if the Team has Room in excess of the applicable first-year Rookie Scale Amount, a Contract covering no fewer than three Seasons that provides for Salary in the first Season up to the amount of the Team's Room and increases in Salary in subsequent Seasons in accordance with Article VII, Section 5(c) (the 20\% Rule).

\hypertarget{loss-of-draft-rights.}{%
\section{Loss of Draft Rights.}\label{loss-of-draft-rights.}}

If for any reason a Team fails to make a Required Tender to a First Round Pick in accordance with Article X, withdraws a Required Tender in accordance with Article X, or renounces a First Round Pick in accordance with Article X, or if a First Round Pick selected in a Subsequent Draft does not sign a Contract for a period of one year following such Subsequent Draft in accordance with Article X, then the rules set forth in Sections 1 and 2 above shall not apply, and such First Round Pick shall become a Rookie Free Agent. In addition, any Team that fails to make a Required Tender to a First Round Pick, withdraws a Required Tender, renounces a First Round Pick, or fails to sign within one year a First Round Pick selected in a Subsequent Draft shall be prohibited from signing such player until after he has signed a Player Contract with another NBA Team, and either (i) he completes the playing services called for under the Contract, or (ii) the Contract is terminated in accordance with the NBA waiver procedure.

\hypertarget{length-of-player-contracts}{%
\chapter{LENGTH OF PLAYER CONTRACTS}\label{length-of-player-contracts}}

\hypertarget{maximum-term.}{%
\section{Maximum Term.}\label{maximum-term.}}

Except as provided in Article VII, Section 6(d) and (f) (\$1 Million Exception and Minimum Annual Salary Exception) and Article VIII, Sections 1 and 2 (Rookie Scale Contracts), a Player Contract may cover, in the aggregate, up to but no more than seven Seasons from the date such Contract is signed.

\hypertarget{computation-of-time.}{%
\section{Computation of Time.}\label{computation-of-time.}}

For purposes of Section 1 above, if a Player Contract is signed after the beginning of a Season, the Season in which the Contract is signed shall be counted as one full Season covered by the Contract.

\hypertarget{nba-draft}{%
\chapter{NBA DRAFT}\label{nba-draft}}

\hypertarget{term-and-timing-of-draft-provisions.}{%
\section{Term and Timing of Draft Provisions.}\label{term-and-timing-of-draft-provisions.}}

An NBA Draft will be held prior to the commencement of each NBA Season covered by the term of this Agreement and, despite the expiration of the other terms of this Agreement pursuant to Article XXXVIII, prior to the commencement of the 2001-02 NBA Season. Each such Draft will be held prior to the July 15 preceding the commencement of the NBA Season on a date to be designated by the Commissioner.

\hypertarget{number-of-choices.}{%
\section{Number of Choices.}\label{number-of-choices.}}

The NBA Draft shall consist of two (2) rounds, with each round consisting of the same number of selections as there will be Teams in the NBA the following Season.

\hypertarget{negotiating-rights-to-draft-rookies.}{%
\section{Negotiating Rights to Draft Rookies.}\label{negotiating-rights-to-draft-rookies.}}

\begin{enumerate}
\def\labelenumi{(\alph{enumi})}
\tightlist
\item
  A Team that drafts a player shall, during the period from the date of such NBA Draft (hereinafter the ``Initial Draft'') to the date of the next Draft (hereinafter the ``Subsequent Draft''), be the only Team with which such player may negotiate or sign a Player Contract, provided that, on or before the July 15 immediately following the Initial Draft (for a First Round Pick) or on or in the two (2) weeks before the September 5 immediately following the Initial Draft (for a Second Round Pick), such Team has made a Required Tender to such player. If a Team has made a Required Tender to such a player and the player has not signed a Player Contract within the period between the Initial Draft and the Subsequent Draft, the Team that drafts the player loses its exclusive right to negotiate with the player and the player will then be eligible for selection in the Subsequent Draft.
\item
  A Team that, in the Subsequent Draft, drafts a player who (i) was drafted in the Initial Draft, (ii) received a Required Tender from the Team that drafted him in the Initial Draft, and (iii) did not sign a Player Contract with such first Team prior to the Subsequent Draft, shall, during the period from the date of the Subsequent Draft to the date of the NBA Draft held in the following year, be the only Team with which such player may negotiate or sign a Player Contract, provided such Team has made a Required Tender. If such player has not signed a Player Contract within the period between the Subsequent Draft and the next NBA Draft with the Team that drafted him in the Subsequent Draft, that Team loses its exclusive right, which it obtained in the Subsequent Draft, to negotiate with the player, and the player will become a Rookie Free Agent as of the date of the next NBA Draft.
\item
  If a player is drafted in an Initial Draft and (i) receives a Required Tender, (ii) does not sign a Player Contract with a Team prior to the Subsequent Draft, and (iii) is not drafted by any Team in such Subsequent Draft, the player will immediately become a Rookie Free Agent.
\item
  If a player is drafted by a Team in either an Initial or Subsequent Draft and that Team does not make a Required Tender to such player, the player will become a Rookie Free Agent on the July 16 following such Draft (for a First Round Pick) or on the September 6 following such Draft (for a Second Round Pick).
\item
  A Team may at any time withdraw a Required Tender it has made to a player, provided that the player agrees in writing to the withdrawal. In the event that a Required Tender is withdrawn, the player shall thereupon become a Rookie Free Agent.
\item
  A Team that holds the exclusive rights to negotiate with and sign a drafted player may at any time renounce such exclusive rights, except that, if the Team has made a Required Tender to the player, a renunciation shall not be permitted during the time the player has been given to accept the Required Tender. In order to renounce its exclusive rights with respect to a drafted player, a Team shall provide the NBA with an express, written statement renouncing such exclusive rights. The NBA shall provide a copy of such statement to the Players Association within three (3) business days following its receipt thereof.
\end{enumerate}

\hypertarget{effect-of-contracts-with-other-professional-teams.}{%
\section{Effect of Contracts with Other Professional Teams.}\label{effect-of-contracts-with-other-professional-teams.}}

If a player is drafted by a Team in either an Initial or Subsequent Draft and, during a period in which he may negotiate and sign a Player Contract with only the Team that drafted him, either (i) is a party to a previously existing player contract with a professional basketball team not in the NBA that covers all or any part of the NBA Season immediately following said Initial or Subsequent Draft, or (ii) signs such a player contract, then the following rules will apply:

\begin{enumerate}
\def\labelenumi{(\alph{enumi})}
\tightlist
\item
  Subject to subsection (b) below, the Team that drafts the player shall retain the exclusive NBA rights to negotiate with and sign him for the period ending one year from the earlier of the following two dates: (i) the date the player notifies such Team that he is available to sign a Player Contract with such Team immediately, provided that such notice will not be effective until the player is under no contractual or other legal impediment to sign with such Team; or (ii) the date of the NBA Draft occurring in the twelve-month period from September 1 to August 30 in which the player notifies such Team of his availability and intention to play in the NBA during the Season immediately following said twelve-month period, provided that such notice will not be effective until the player is under no contractual or other legal impediment to play with such Team for said Season.
\item
  If the player notifies the Team that drafts him by July 1 of any year that by September 1 of such year he will, immediately thereafter or for any future season, be under no contractual or other legal impediment to sign and play with such Team, and provided that on September 1 the player is in fact under no such contractual or other legal impediment, then, in order to retain the exclusive NBA rights to negotiate with and sign the player as provided in subsection (a), such Team must make a Required Tender to the player by September 5 of such year.
\item
  If the player gives the required notice by July 1 of any year, and the Team that drafted him fails to make a Required Tender by September 5 of such year, the player shall thereupon become a Rookie Free Agent.
\item
  If, during the one-year period of exclusive NBA negotiating rights set forth in subsection (a) above, the player signs a player contract with a professional basketball team not in the NBA and (i) the player has not made a bona fide effort to negotiate a Player Contract with the Team possessing his exclusive NBA rights or (ii) such bona fide effort is made and such Team makes a Required Tender to such player in accordance with subsection (b) above, then such Team shall retain the exclusive NBA rights to negotiate with and sign the player for additional one-year periods as measured in and in accordance with the provisions of subsection (a).
\item
  If, during the one-year period of exclusive NBA negotiating rights set forth in subsection (a) above, the player signs a player contract with a professional basketball team not in the NBA and (i) the player has made a bona fide effort to negotiate a Player Contract with the Team possessing his exclusive NBA rights, and (ii) such Team fails to make a Required Tender to such player in accordance with Section (b) above, then in no event shall said exclusive NBA rights be retained.
\item
  If, during the one-year period of exclusive NBA negotiating rights set forth in subsection (a) above, the Team makes or has made a Required Tender to the player and the player does not sign a player contract with any professional basketball team, then (i) in the case of a player who was previously drafted in an Initial Draft, the next NBA Draft following such one-year period shall be deemed the Subsequent Draft as to such player, and the rules applicable to a player who is subject to a Subsequent Draft will apply, or (ii) in the case of a player who was previously drafted in a subsequent Draft, such player shall become a Rookie Free Agent at the end of such one-year period.
\item
  Notice under this Section 4 shall be provided in writing by personal delivery or prepaid certified, registered, or overnight mail sent to the Team's principal address or principal office (as then listed in the NBA's records), to the attention of the Team's general manager. For purposes of this Section 4, a ``professional basketball team'' shall mean any team in any country that pays money or compensation of any kind (in excess of a stipend for living expenses) to a basketball player for rendering services for such team.
\end{enumerate}

\hypertarget{application-to-players-with-remaining-intercollegiate-eligibility.}{%
\section{Application to Players with Remaining Intercollegiate Eligibility.}\label{application-to-players-with-remaining-intercollegiate-eligibility.}}

\begin{enumerate}
\def\labelenumi{(\alph{enumi})}
\tightlist
\item
  A person residing within the United States whose high school class has graduated shall become eligible to be selected in an NBA Draft if he renounces his intercollegiate basketball eligibility by written notice to the NBA at least forty-five (45) days prior to such Draft. If such person is selected in such Draft by a Team, the following rules apply:

  \begin{enumerate}
  \def\labelenumii{(\roman{enumii})}
  \tightlist
  \item
    If the player does not thereafter play intercollegiate basketball, then the Team that drafted him shall, during the period from the date of such Draft to the date of the Draft in which the player would, absent renunciation of intercollegiate eligibility, first have been eligible to be selected, be the only Team with which the player may negotiate or sign a Player Contract, provided that such Team makes a Required Tender to the player each year. For purposes hereof, the Draft in which such player would, absent renunciation of such intercollegiate eligibility, first have been eligible to be selected, will be deemed the ``Subsequent Draft'' as to that player, and the rules applicable to a player who has been drafted in a Subsequent Draft will apply. If the player, having been selected in a Draft for which he was eligible by virtue of renunciation of intercollegiate eligibility, has not signed a Player Contract with the Team that drafted him in such Draft following a Required Tender by that Team and is not drafted in the Subsequent Draft (as defined in the previous sentence), he shall become a Rookie Free Agent.
  \item
    If the player does thereafter play intercollegiate basketball, then the Team that drafted him shall retain the exclusive NBA rights to negotiate with and sign the player for the period ending one year from the date of the Draft in which the player would, absent renunciation of intercollegiate eligibility, first have been eligible to be selected, provided that such Team makes a Required Tender to the player each year. For purposes hereof, the Draft in which such player would, absent renunciation of intercollegiate eligibility, first have been eligible to be selected, will be deemed the ``Initial Draft'' as to that player. The next NBA Draft shall be deemed the ``Subsequent Draft'' as to that player, and the rules applicable to a player who has been drafted in a Subsequent Draft will apply.
  \end{enumerate}
\item
  A person residing within the United States whose high school class has graduated, who is not yet eligible to be selected in an NBA Draft, and who signs a player contract with a professional basketball team not in the NBA, shall thereupon become eligible to be selected in the next NBA Draft, and if so selected, shall be treated as though he were a player referred to in Section 4 above. For purposes of this subsection, a ``professional basketball team'' shall mean any team in any country that pays money or compensation of any kind (in excess of a stipend for living expenses) to a basketball player for rendering services to such team.
\end{enumerate}

\hypertarget{application-to-foreign-players.}{%
\section{Application to Foreign Players.}\label{application-to-foreign-players.}}

\begin{enumerate}
\def\labelenumi{(\alph{enumi})}
\tightlist
\item
  For purposes of this Section, a ``foreign player'' shall mean any person residing outside of the United States who participates in the game of basketball as an amateur or as a professional.
\item
  A foreign player is eligible to be selected in an NBA Draft held during the calendar year in which such player has his twenty-second (22nd) birthday. Any foreign player who is older than twenty-two (22), and who was not selected in the NBA Draft held during the calendar year of his twenty-second (22nd) birthday, is a Rookie Free Agent.
\item
  Notwithstanding subsection (b) above, a foreign player who is at least eighteen (18) years old and who has not exercised intercollegiate basketball eligibility in the United States shall become eligible to be selected in an NBA Draft held prior to the calendar year in which he has his twenty-second (22nd) birthday if he expresses his desire to become eligible to be selected in the next NBA Draft by written notice to the NBA at least forty-five (45) days prior to such Draft.
\item
  A foreign player who exercises intercollegiate basketball eligibility in the United States during the season prior to an NBA Draft shall be subject to the rules regarding completion or renunciation of collegiate eligibility, as set forth in Section 5 above.
\end{enumerate}

\hypertarget{assignment-of-draft-rights.}{%
\section{Assignment of Draft Rights.}\label{assignment-of-draft-rights.}}

In the event that the exclusive right to negotiate with a player obtained in any NBA Draft is assigned by a Team to another Team, in accordance with NBA procedures, the Team to which such right has been assigned shall have the same, but no greater, right to negotiate with and sign such player as possessed by the Team assigning such right, and such player shall have the same, but no greater, obligation to the Team to which such right has been assigned as he had to the Team assigning such right.

\hypertarget{general.-1}{%
\section{General.}\label{general.-1}}

\begin{enumerate}
\def\labelenumi{(\alph{enumi})}
\tightlist
\item
  The placement of a Rookie on the Armed Services List, or on any of the other lists described in the NBA By-Laws, or on any other list created by the NBA, shall not extend the period of exclusive negotiating rights which a Team has to any Draft Rookie beyond the period specified in this Agreement.
\item
  Nothing contained herein shall prevent the NBA, in accordance with the applicable provisions of the NBA Constitution and By-Laws, from prohibiting or otherwise responding to violations by Teams of the exclusive NBA rights obtained in any NBA Draft, as set forth or referred to in this Article. Other than as specifically agreed to herein, nothing contained in this Agreement shall be deemed to be an agreement by the Players Association to any provision of the NBA Constitution and By-Laws.
\item
  A person who has renounced his intercollegiate eligibility and expressed his desire to become eligible to be selected in the next NBA Draft pursuant to Section 5 or Section 6 above shall be entitled to withdraw from such Draft by providing written notice that is received by the NBA seven days prior to such Draft.
\end{enumerate}

\hypertarget{free-agency}{%
\chapter{FREE AGENCY}\label{free-agency}}

\hypertarget{general-rules.}{%
\section{General Rules.}\label{general-rules.}}

Subject to the provisions of Article VII, including, but not limited to, Article VII, Section 6(b): (a) a Free Agent (other than a Veteran Free Agent) is free at any time to negotiate and enter into a Player Contract with any Team; (b) a Veteran Free Agent is free at any time to negotiate and enter into a Player Contract with his Prior Team, and is free on or after the July 1 following the last Season covered by his Player Contract to negotiate and enter into a new Player Contract with any Team; (c) any Team may negotiate and sign a Player Contract with a Free Agent without any penalty or restriction. No compensation obligation of any kind to another Team, or right of first refusal of any kind, shall be applicable to any Free Agent.

\hypertarget{no-individually-negotiated-right-of-first-refusal.}{%
\section{No Individually-Negotiated Right of First Refusal.}\label{no-individually-negotiated-right-of-first-refusal.}}

\begin{enumerate}
\def\labelenumi{(\alph{enumi})}
\tightlist
\item
  No Player Contract, or any Renegotiation, Extension, or amendment of a Player Contract, executed after the date of this Agreement, may include any individually negotiated right of first refusal or other limitation on player movement following the last Salary Cap Year covered by his Player Contract.
\item
  No right of first refusal rule, practice, policy, regulation or agreement providing for a right of first refusal shall be applied to any player as a result of that player's entry into a player contract with or the playing with any team in any professional basketball league other than the NBA.
\end{enumerate}

\hypertarget{withholding-services.}{%
\section{Withholding Services.}\label{withholding-services.}}

A player who withholds playing services called for by a Player Contract for more than thirty (30) days after the start of the last Season covered by his Player Contract shall be deemed not to have ``complet{[}ed{]} his Player Contract by rendering the playing services called for thereunder.'' Accordingly, such a player shall not be a Veteran Free Agent and shall not be entitled to negotiate or sign a Player Contract with any other professional basketball team unless and until the Team for which the player last previously played expressly agrees otherwise.

\hypertarget{option-clauses}{%
\chapter{OPTION CLAUSES}\label{option-clauses}}

\hypertarget{team-options.}{%
\section{Team Options.}\label{team-options.}}

A Player Contract shall not contain any option in favor of the Team, except an Option (as defined in Article I, Section 1(ag)) that: (i) is specifically negotiated between a Veteran or a Rookie (other than a First Round Pick) and a Team; (ii) authorizes the extension of such Contract for no more than one year beyond the stated term; (iii) is exercisable only once; and (iv) provides that the Salary payable with respect to the option year is no less than 100\% of the Salary payable with respect to the last year of the stated term of such Contract and that all other non-monetary terms applicable in the last year of the stated term of such Contract shall be applicable in the option year.

\hypertarget{player-options.}{%
\section{Player Options.}\label{player-options.}}

A Player Contract shall not contain any option in favor of the player, except:

\begin{enumerate}
\def\labelenumi{(\alph{enumi})}
\tightlist
\item
  an Option that: (i) is specifically negotiated between a Veteran or a Rookie (other than a First Round Pick) and a Team; (ii) authorizes the extension of such Contract for no more than one year beyond the stated term; (iii) is exercisable only once; and (iv) provides that the Salary payable with respect to the option year is no less than 100\% of the Salary payable with respect to the last year of the stated term of such Contract and that all other nonmonetary terms applicable in the last year of the stated term of such Contract shall be applicable in the option year; and/or
\item
  an Early Termination Option (or ``ETO'') (as defined in Article I, Section l(r)), provided that such ETO is exercisable only once and takes effect no earlier than the end of the third year of the Contract. A Contract that does not provide for an ETO when signed may not be amended to provide for an ETO during the original term of the Contract. Notwithstanding the preceding sentence, a Team and a player may enter into an Extension that contains an ETO, provided that such ETO takes effect no earlier than three years from the date the Extension is signed or the conclusion of the original term of the Contract, whichever is later. A Contract (including an Extension) that contains an ETO must specify either the Effective Date of the ETO or that the Effective Date is contingent; provided, however, that the only allowable contingency shall be whether the player or Team meets performance benchmarks designated at the time the Contract is signed.
\end{enumerate}

\hypertarget{exercise-period.}{%
\section{Exercise Period.}\label{exercise-period.}}

Any Option or Early Termination Option must be exercised prior to the July 1 immediately prior to the Season covered by the option.

\hypertarget{option-buy-outs.}{%
\section{Option Buy-Outs.}\label{option-buy-outs.}}

Subject to the rules set forth in Article VII, a Player Contract that contains an Option or an Early Termination Option may provide for an Option Buy-Out Amount; provided, however, that in no event may an Option Buy-Out Amount exceed, in the case of an Option, 50\% of the Salary called for in the option year or, in the case of an ETO, 50\% of the Salary in the first Season following the Effective Date of the ETO.

\hypertarget{circumvention}{%
\chapter{CIRCUMVENTION}\label{circumvention}}

\hypertarget{general-prohibition.}{%
\section{General Prohibition.}\label{general-prohibition.}}

\begin{enumerate}
\def\labelenumi{(\alph{enumi})}
\tightlist
\item
  It is the intention of the parties that the provisions agreed to herein, including, without limitation, the rules relating to the Salary Cap, the Exceptions to the Salary Cap, and the free agency provisions, be interpreted so as to preserve the essential benefits achieved by both parties to this Agreement. Neither the parties hereto, nor any Team (or Team Affiliate) or player (or person acting with authority on behalf of such player), shall enter into any agreement, including, without limitation, any Player Contract (including any Renegotiation, Extension, or amendment of a Player Contract), or undertake any action or transaction, including, without limitation, the assignment or termination of a Player Contract, which includes any terms that are designed to serve the purpose of defeating or circumventing the intention of the parties as reflected by all of the provisions of this Agreement.
\item
  It shall constitute a violation of subsection (a) above for a Team to enter into an agreement or understanding with any sponsor or business partner or third party under which such sponsor, business partner or third party pays compensation for basketball services (even if such compensation is ostensibly designated as being for non-basketball services) to a player under Contract to the Team. Such an agreement with a sponsor or business partner may be inferred where: (i) such compensation from the sponsor or business partner is substantially in excess of the fair market value of any services to be rendered by the player for such sponsor or business partner; and (ii) the compensation in the Player Contract between the player and the Team is substantially below the fair market value of such Contract.
\end{enumerate}

\hypertarget{no-undisclosed-agreements.}{%
\section{No Undisclosed Agreements.}\label{no-undisclosed-agreements.}}

\begin{enumerate}
\def\labelenumi{(\alph{enumi})}
\tightlist
\item
  At no time shall there be any undisclosed agreements of any kind, express or implied, oral or written, or promises, undertakings, representations, commitments, inducements, assurances of intent, or understandings of any kind, between a player (or any person acting with authority on behalf of such player) and any Team (or Team Affiliate):

  \begin{enumerate}
  \def\labelenumii{(\roman{enumii})}
  \tightlist
  \item
    involving consideration of any kind to be paid, furnished or made available to the player, or any person or entity controlled by or related to the player, by the Team or Team Affiliate either during the term of the Player Contract or thereafter; or
  \item
    concerning any future Renegotiation, Extension, or amendment of an existing Player Contract, or entry into a new Player Contract.
  \end{enumerate}
\item
  At the time of the assignment of any Player Contract, there shall be no undisclosed agreements of any kind, express or implied, oral or written, or promises, undertakings, representations, commitments, inducements, assurances of intent, or understandings of any kind,between the player whose Player Contract has been assigned (or any person acting with authority on behalf of such player) and any Team (or Team Affiliate), concerning any future Renegotiation, Extension, or amendment of the Player Contract that has been assigned or the entry into any new Player Contract.
\end{enumerate}

\hypertarget{penalties.}{%
\section{Penalties.}\label{penalties.}}

\begin{enumerate}
\def\labelenumi{(\alph{enumi})}
\tightlist
\item
  Upon a finding of a violation of Section 1 above by the System Arbitrator, but only following the conclusion of any appeal to the Appeals Panel, the Commissioner shall be authorized to:

  \begin{enumerate}
  \def\labelenumii{(\roman{enumii})}
  \tightlist
  \item
    impose a fine of up to \$2,000,000 (50\% of which shall be payable to the NBA, and 50\% of which shall be payable to the NBPA-Selected Charitable Organization) on any Team found to have committed such violation;
  \item
    direct the forfeiture of one first round draft pick; and/or
  \item
    void any Player Contract, or any Renegotiation, Extension, or amendment of a Player Contract, between any player and any Team that are both found to have committed such violation.
  \end{enumerate}
\item
  Upon a finding of a violation of Section 2 above by the System Arbitrator, but only following the conclusion of any appeal to the Appeals Panel, the Commissioner shall be authorized to:

  \begin{enumerate}
  \def\labelenumii{(\roman{enumii})}
  \tightlist
  \item
    impose a fine of up to \$5,000,000 (50\% of which shall be payable to the NBA, and 50\% of which shall be payable to the NBPA-Selected Charitable Organization) on any Team found to have committed such violation;
  \item
    direct the forfeiture of draft picks;
  \item
    void any Player Contract, or any Renegotiation, Extension, or amendment of a Player Contract, between any player and any Team that are both found to have committed such violation; and/or
  \item
    suspend for up to one year any Team personnel found to have willfully engaged in such violation.
  \end{enumerate}
\item
  If the System Arbitrator finds that a player agent has willfully committed a violation of Section 1 or 2 above, and such finding, if appealed, is affirmed by the Appeals Panel, such finding shall be referred to the Players Association's Committee on Agent Regulation for such disciplinary action as the Committee deems appropriate. The Committee shall accept as binding and conclusive the findings of the System Arbitrator (or, in the case of an appeal, the Appeals Panel) that a violation of Section 1 or 2 has occurred. In addition, the NBA reserves any and all rights it may have to take such action and/or assert such claims as may be available to it against a player agent found to have violated Section 1 or 2. The Players Association reserves any and all rights it may have to oppose any action taken or claims asserted by the NBA against a player agent found to have violated Section 1 or 2, including any and all rights it may have to contest the authority of the NBA to take such action.
\end{enumerate}

\hypertarget{production-of-tax-materials.}{%
\section{Production of Tax Materials.}\label{production-of-tax-materials.}}

In any proceeding to enforce Section 1 or 2 above, the System Arbitrator shall have the authority, upon good cause shown, to direct any Team, Team Affiliate, or player to produce any tax returns or other relevant tax materials disclosing income figures for the player (non-income figures may be redacted), or disclosing expense figures by the Team or Team Affiliate (non-expense figures may be redacted), which materials shall not be released to the general public or the media and shall be treated as strictly confidential by all parties.

\hypertarget{valuation-procedures.}{%
\section{Valuation Procedures.}\label{valuation-procedures.}}

\begin{enumerate}
\def\labelenumi{(\alph{enumi})}
\item
  Any transaction (other than a player appearance called for by the Uniform Player Contract or Group License Agreement) between a player and a Team and/or Team Affiliate after the date of this Agreement that (i) is described as not involving playing services, and (ii) in which the player receives compensation or is being provided with an investment opportunity, shall be disclosed in writing to the League Office and the Players Association prior to or within five business days after the entering into of the transaction. The NBA shall have ten days after such disclosure in which to challenge the transaction, pursuant to the procedures set forth in subsection (c) below, on the ground that: (i) the compensation to the player is greater than a reasonable approximation of the fair market value of the non-basketball services or other consideration provided by the player in the transaction; (ii) the amount of the player's investment is not commercially reasonable, given the relative risks and rewards of such investment; or (iii) the consideration paid to the player for performing basketball services represents less than a reasonable approximation of the fair market value of such player's basketball services.
\item
  If a Team or Team Affiliate enters into a transaction after the date of this Agreement with a retired player who played for the Team within the past five years, in which the retired player is being compensated in excess of \$10,000 or is being provided with an investment opportunity, and if the compensation the retired player received from the Team when he was a player was substantially below the then fair market value for his services, then the NBA may challenge the transaction, pursuant to the procedures set forth in subsection (c) below, on the ground that: (i) the compensation to the player substantially exceeds the fair market value of the services or other consideration provided by the retired player in the business transaction; or (ii) the amount of the player's investment is not commercially reasonable, given the relative risks and rewards of such investment.
\item
  \begin{enumerate}
  \def\labelenumii{(\roman{enumii})}
  \tightlist
  \item
    Any challenge under this Section 5 shall be filed in writing with a business valuation expert jointly selected by the NBA and the Players Association, who shall render a decision within fifteen days after the filing of the challenge. The business valuation expert shall conduct a hearing in which the player or retired player, the Team and/or Team Affiliate, the Players Association, and the NBA are afforded the opportunity to appear and participate. The NBA shall have the burden of proof in the proceeding. The business valuation expert may permit discovery of relevant documents necessary to undertake the valuation. Within ten days of any decision by the business valuation expert, any of the parties may file an appeal with the System Arbitrator, who shall conduct a hearing and render a decision within twenty days of the filing of the appeal. There shall be no right of further appeal to the Appeals Panel.
  \item
    If the NBA prevails in its challenge under this Section 5, the player or retired player and the Team and/or Team Affiliate shall have fifteen days after the date of such determination (or the date of the conclusion of any appeal) in which to renegotiate or terminate: (x) the business transaction, if all parties to the transaction so agree; and (y) in the case of a challenge under Section 5(a), any Player Contract entered into contemporaneously with such transaction, without regard to any time limitations in this Agreement applicable to Renegotiations. If the player or retired player and the Team and/or Team Affiliate do not renegotiate or terminate the business transaction or Player Contract by the conclusion of such fifteen-day period, then, at that time: (xx) in the case of a challenge under Section 5(a), the consideration received by the player, or the value of the investment opportunity (net of any contribution by the player), in each case as determined by the business valuation expert or the System Arbitrator, as the case may be, shall be included in the player's Salary, subject to the Team's Room and other Salary Cap rules, and further subject to any allocation over time that the business valuation expert or System Arbitrator determines is appropriate; and (yy) in the case of a challenge under Section 5(b), the difference between (A) the compensation received by the retired player, or the value of the investment opportunity received by the retired player (net of any contribution by the retired player), and (B) a reasonable estimate of the fair market value of the services or other consideration provided by the retired player, or a reasonable estimate of the fair market value of the investment opportunity, in each case as determined by the business valuation expert or the System Arbitrator, as the case may be, shall be included in the Team's Team Salary, subject to the Team's Room and other Salary Cap rules, and further subject to any allocation over time that the business valuation expert or System Arbitrator determines is appropriate.
  \item
    If the NBA prevails in its challenge under this Section 5, and the player or retired player and the Team and/or Team Affiliate renegotiate or terminate the business transaction or Player Contract, any revised terms of the transaction or Player Contract shall be promptly disclosed to the NBA and the Players Association, and may, at the request of the NBA, be re-subjected to the procedures of this subsection (c).
  \end{enumerate}
\item
  Any information disclosed to the League Office and the Players Association pursuant to the procedures of this Section 5 shall be treated strictly confidential, and shall not be released to the general public or the media.
\end{enumerate}

\hypertarget{other-undertakings.}{%
\section{Other Undertakings.}\label{other-undertakings.}}

\begin{enumerate}
\def\labelenumi{(\alph{enumi})}
\tightlist
\item
  No Team shall have a financial arrangement with or offer a financial inducement to any player (not including retired players) not signed to a current Player Contract.
\item
  Prior to the assignment of any Player Contract, the Team from which such Player Contract is to be assigned and the player whose Player Contract is to be assigned shall be required to divest themselves, on terms mutually agreeable to the player and the Team, of any pre-existing financial arrangements between such Team and such player. The foregoing shall not apply to Deferred Compensation obligations and loans.
\item
  Nothing contained in subsections (a) and (b) above shall interfere with a Team's obligation to pay a player Deferred Compensation earned under a prior Player Contract.
\end{enumerate}

\hypertarget{anti-collusion-provisions}{%
\chapter{ANTI-COLLUSION PROVISIONS}\label{anti-collusion-provisions}}

\hypertarget{no-collusion.}{%
\section{No Collusion.}\label{no-collusion.}}

Subject to Section 2 below, no NBA Team, its employees or agents, will enter into any contracts, combinations or conspiracies, express or implied, with the NBA or any other NBA Team, their employees or agents: (a) to negotiate or not to negotiate with any Veteran or Rookie; (b) to offer or not to offer a Player Contract to any Free Agent; or (c) concerning the terms or conditions of employment offered to any Veteran or Rookie.

\hypertarget{non-collusive-conduct.}{%
\section{Non-Collusive Conduct.}\label{non-collusive-conduct.}}

The following conduct shall not be a violation of Section 1 above:

\begin{enumerate}
\def\labelenumi{(\alph{enumi})}
\tightlist
\item
  the formulation and negotiation of collective bargaining proposals;
\item
  agreements between NBA Teams necessary to the assignment of a Player Contract of a Veteran or the assignment of the exclusive negotiating rights to a Draft Rookie, where such assignment is contingent upon the signing by the Veteran or Draft Rookie of a new or amended Player Contract or Extension; provided, however, that if such contingency is fulfilled by the player entering into a new or amended Player Contract or Extension, this subsection shall only apply if the assignment is actually consummated;
\item
  an agreement between NBA Teams concerning the signing of a new Player Contract by a Veteran Free Agent with his Prior Team, where such agreement is necessary for the subsequent assignment of the new Player Contract between the agreeing Teams; provided, however, that this subsection (c) shall apply only if the subsequent assignment is consummated, and only to agreements entered into after the execution date of this Agreement;
\item
  the conduct authorized by the terms and conditions of the NBA Draft (as set forth in Article X above); and
\item
  any action taken by the NBA League Office to exclude from the League, suspend or discipline any player for reasons involving gambling, drugs, or the commission of a crime. (This subsection, however, shall not affect any other rights of any player or the Players Association to contest such action.)
\end{enumerate}

\hypertarget{individual-negotiations.}{%
\section{Individual Negotiations.}\label{individual-negotiations.}}

No NBA Team shall fail or refuse to negotiate with, or enter into a Player Contract with, any player who is free to negotiate and sign a Player Contract with any NBA Team, on any of the following grounds:

\begin{enumerate}
\def\labelenumi{(\alph{enumi})}
\tightlist
\item
  that the player has previously been subject to the exclusive negotiating rights obtained by another NBA Team in an NBA Draft; or
\item
  that the player has previously refused or failed to enter into a Player Contract containing an option; or
\item
  that the player has become a Free Agent.\\
  The fact that a Team has not negotiated with, made any offers to, or entered into any Player Contracts with players who are free to negotiate and sign Player Contracts with any Team, shall not, by itself, be deemed proof that such Team failed or refused to negotiate with, make any offers to, or enter into any Player Contracts with any players on any of the prohibited grounds referred to in this Section 3.
\end{enumerate}

\hypertarget{league-disclosures.}{%
\section{League Disclosures.}\label{league-disclosures.}}

The NBA League Office shall not knowingly communicate or disclose, directly or indirectly, to any NBA Team that another NBA Team has negotiated with or is negotiating with any Free Agent, prior to the execution of a Player Contract with that player.

\hypertarget{meet-and-confer.}{%
\section{Meet and Confer.}\label{meet-and-confer.}}

\begin{enumerate}
\def\labelenumi{(\alph{enumi})}
\tightlist
\item
  During the period from July 1 through the first day of the Regular Season of each year covered by this Agreement, the General Counsel of the NBA (or his designee) shall meet once a week with the General Counsel of the Players Association (or his designee) for the purpose of reviewing (i) each Team's Team Salary summary and the list of Exceptions then currently available to each Team and (ii) any advice rendered during the previous week by either the NBA or the Players Association regarding the interpretation of the Salary Cap.
\item
  During the period from the first day of the Regular Season through June 30 of each year covered by this Agreement:

  \begin{enumerate}
  \def\labelenumii{(\roman{enumii})}
  \tightlist
  \item
    the General Counsel of the NBA (or his designee) and the General Counsel of the Players Association (or his designee) shall meet at reasonable intervals upon the request of the Players Association for the purpose of reviewing (x) each Team's Team Salary summary and the list of Exceptions then currently available to each Team and (y) any advice regarding the interpretation of the Salary Cap rendered since the last such meeting; and
  \item
    if the NBA informs a Team that a specifically proposed assignment or other player transaction would be inconsistent with or in violation of the terms of this Agreement and/or the limitations of the Salary Cap as interpreted by the NBA, the NBA shall promptly notify the Players Association that such an interpretation has been communicated and the basis for such interpretation. The NBA shall provide such notice to the Players Association within two business days following the communication of such an interpretation to a Team.
  \end{enumerate}
\item
  The substance of any communications under this Section 5 may be referred to or used by the NBA or the Players Association in any proceeding.
\item
  By agreeing to the meeting and notification requirement of subsections (a) and (b) of this Section 5, neither the NBA nor the Players Association intends to waive nor shall be deemed to have waived any attorney-client or other privilege with respect to any communications.
\item
  The provisions of this Agreement are not intended to create any substantive rights in any party, other than as provided for herein. This Agreement may be enforced, and any alleged violations may be remedied, only as provided for herein.
\end{enumerate}

\hypertarget{enforcement-of-anti-collusion-provisions.}{%
\section{Enforcement of Anti-Collusion Provisions.}\label{enforcement-of-anti-collusion-provisions.}}

Any individual player, or the Players Association acting in that player's or any number of players' behalf, may bring an action before the System Arbitrator alleging a violation of Article XIV, Section 1 of this Agreement. Issues of relief and liability shall be determined in the same proceeding (including the amount of damages, pursuant to Section 10 below, if any). The complaining party will bear the burden of demonstrating by a clear preponderance of the evidence that the challenged conduct was in violation of Article XIV, Section 1 of this Agreement and caused economic injury to such player(s).

\hypertarget{satisfaction-of-burden-of-proof.}{%
\section{Satisfaction of Burden of Proof.}\label{satisfaction-of-burden-of-proof.}}

The failure by a Team or Teams to make offers or sign Contracts for the playing services of Free Agents shall not, by itself or in combination only with evidence about the playing skills of the player(s) not receiving such offers or contracts, satisfy the burden of proof set forth in Section 6 above. However, such evidence may support a finding of a violation of Section 1 above, but only in combination with other evidence that either by itself or in combination with the evidence referred to in the immediately preceding sentence indicates that the challenged conduct was in violation of Section 1 and caused economic injury to such player(s).

\hypertarget{summary-judgment.}{%
\section{Summary Judgment.}\label{summary-judgment.}}

The System Arbitrator may, at any time following the conclusion of any permitted discovery, determine whether or not the complainant's evidence is sufficient to raise a genuine issue of material fact capable of satisfying the standards imposed by Sections 6 and 7 above. If the System Arbitrator determines that complainant's evidence is not so sufficient, he shall dismiss the action.

\hypertarget{remedies.}{%
\section{Remedies.}\label{remedies.}}

In the event that an individual player or players, or the Players Association acting on his, or their, behalf, successfully proves a violation of Section 1 above, the player or players determined by the System Arbitrator to have suffered economic injury as a result of the violation will have the right:

\begin{enumerate}
\def\labelenumi{(\alph{enumi})}
\tightlist
\item
  to terminate his (or their) existing Player Contract(s) at his (or their) option (however, such termination shall not take effect until the conclusion of a then ongoing NBA Season, if any). Such right of termination shall not arise until the recommendation of the System Arbitrator finding a violation is no longer subject to further appeal and must be exercised by the player within thirty days therefrom. Such player shall immediately become a Free Agent upon such termination; however, any such player may choose to reinstate his Player Contract at any time up until September 15 of that year; and
\item
  to recover damages as described in Section 10 below. However, if the player terminates his Player Contract under subsection (a) above and does not reinstate it pursuant thereto, he may not recover damages for the period after such termination takes effect. A player who does not terminate his contract or who reinstates it pursuant to subsection (a) above, may recover damages for the entire period of his injury.
\end{enumerate}

\hypertarget{calculation-of-damages.}{%
\section{Calculation of Damages.}\label{calculation-of-damages.}}

Upon any finding of a violation of Section 1 above, compensatory damages (i.e., the amount by which any player has been injured as a result of such violation) and non-compensatory damages (i.e., the amount exceeding compensatory damages) shall be awarded as follows:

\begin{enumerate}
\def\labelenumi{(\alph{enumi})}
\tightlist
\item
  Two (2) times the amount of compensatory damages, in the event that all of the Teams found to have violated Section 1 have committed such a violation for the first time. Any Team found to have committed such a violation for the first time shall be jointly and severally liable for two (2) times the amount of compensatory damages.
\item
  Three (3) times the amount of compensatory damages, in the event that any of the Teams found to have violated Section 1 have committed such a violation for the second time. In the event that damages are awarded pursuant to this subsection (b): (i) any Team found to have committed such a violation for the first time shall be jointly and severally liable for two (2) times the amount of compensatory damages; and (ii) any Team found to have committed such a violation for the second time shall be jointly and severally liable for three (3) times the amount of compensatory damages.
\item
  Three (3) times the amount of compensatory damages, plus, for each Team found to have violated Section 1 for at least the third time, two million five hundred thousand dollars (\$2,500,000), in the event that any of the Teams found to have violated Section 1 have committed such violation for at least the third time. In the event that damages are awarded pursuant to this subsection (c): (i) any Team found to have committed such a violation for the first time shall be jointly and severally liable for two (2) times the amount of compensatory damages; (ii) any Team found to have committed such a violation for at least the second time shall be jointly and severally liable for three (3) times the amount of compensatory damages; and (iii) any Team found to have committed such a violation for at least the third time shall, in addition, pay a fine of two million five hundred thousand dollars (\$2,500,000).
\end{enumerate}

\hypertarget{payment-of-damages.}{%
\section{Payment of Damages.}\label{payment-of-damages.}}

In the event damages are awarded pursuant to Section 10 above, the amount of compensatory damages shall be paid to the injured player or players. The amount of non-compensatory damages, including any fines, shall be paid to the Players Association, which may use it for any purpose other than to pay it to any player who has received compensatory damages, except that any such player may receive some portion of a non-compensatory damage award as part of a proportional distribution to Players Association members.

\hypertarget{effect-of-damages-on-salary-cap.}{%
\section{Effect of Damages on Salary Cap.}\label{effect-of-damages-on-salary-cap.}}

In the event damages are awarded pursuant to Section 10 above, the amount of non-compensatory damages, including any fines, will not be included in any of the computations described in Article VII above. The amount of compensatory damages awarded will be included in such computations.

\hypertarget{contribution.}{%
\section{Contribution.}\label{contribution.}}

Any Team found liable under Section 1 above shall have the right to seek contribution from any other Team found liable for the same violation in a proceeding before the Commissioner who shall determine what contribution, if any, is fair and equitable. The Commissioner's determination with regard to contribution shall be final and binding upon and unappealable by any Team. A contribution determination by the Commissioner may be appealed by the Players Association to the System Arbitrator, except that if such a determination involves fewer than four (4) Teams found to have committed a violation of Section 1 above and allocates damages equally among the Teams found liable, there shall be no appeal to the System Arbitrator. In the event of a contribution determination by the Commissioner, the NBA shall provide the Players Association with the data and information that the Commissioner used or relied upon in making his determination. Any contribution determination appealed by the Players Association to the System Arbitrator shall be upheld unless it is clearly erroneous.

\hypertarget{no-reimbursement.}{%
\section{No Reimbursement.}\label{no-reimbursement.}}

Any damages awarded pursuant to Section 10 above must be paid by the individual Teams found liable and those Teams may not be reimbursed or indemnified by any other Team or the NBA, except to the extent of any award of contribution made pursuant to Section 13 above.

\hypertarget{costs.}{%
\section{Costs.}\label{costs.}}

In any action brought for an alleged violation of Section 1 above, the System Arbitrator shall order the payment of reasonable attorneys' fees by any party found to have brought such an action or to have asserted a defense to such an action without any reasonable basis for asserting such a claim or defense.

\hypertarget{termination-of-agreement.}{%
\section{Termination of Agreement.}\label{termination-of-agreement.}}

The Players Association shall have the right to terminate this Agreement (pursuant to the procedure set forth in Article XXXVIII of this Agreement), under the following circumstances:

\begin{enumerate}
\def\labelenumi{(\alph{enumi})}
\tightlist
\item
  Where there has been a finding or findings of one or more instances of a violation of Section 1 above with respect to anyone NBA season which, either individually or in total, involved five (5) or more Teams and caused injury to five (5) or more players; or
\item
  Where there has been a finding or findings of one or more instances of a violation of Section 1 above with respect to any two consecutive NBA Seasons which, either individually or in total, involved seven (7) or more Teams and caused economic injury to seven (7) or more players. For purposes of this Section 16(b), a player found to have been injured by a violation of Section I in each of two consecutive Seasons shall be counted as an additional player injured by such a violation for each such NBA Season; or
\item
  Where, in a proceeding brought by the Players Association, it is shown by clear and convincing evidence that ten (10) or more Teams have engaged in a violation or violations of Section 1 above, causing economic injury to one or more NBA players. In order to terminate this Agreement pursuant to this subsection (c) and Article XXXVIII of this Agreement:

  \begin{enumerate}
  \def\labelenumii{(\roman{enumii})}
  \tightlist
  \item
    the proceeding must be brought by the Players Association; and
  \item
    the NBA and the System Arbitrator must be informed at the outset of any such proceeding that the Players Association is proceeding under this subsection (c) for the purpose of establishing its entitlement to terminate this Agreement.
  \end{enumerate}
\end{enumerate}

\hypertarget{discovery.}{%
\section{Discovery.}\label{discovery.}}

\begin{enumerate}
\def\labelenumi{(\alph{enumi})}
\tightlist
\item
  In any of the actions described in this Article XIV, the System Arbitrator shall grant reasonable and expedited discovery upon the application of any party where, and to the extent, he or she determines it is reasonable to do so. Such discovery may include the production of documents and the taking of depositions.
\item
  Notwithstanding Section 17(a) above, the Players Association and the NBA shall each have the right to obtain discovery upon request in any three proceedings brought during the term of this Agreement. The scope and extent of such discovery shall be determined by the System Arbitrator.
\end{enumerate}

\hypertarget{time-limits.}{%
\section{Time Limits.}\label{time-limits.}}

Any action under Section 1 above must be brought within 90 days of the time when the player knows or reasonably should have known that he had a claim, or within 90 days of the start of the NBA season in which a violation of Section 1 is claimed, whichever is later. In the absence of a System Arbitrator, the complaining party shall file such claim for breach of this Agreement pursuant to Section 301 of the Labor Management Relations Act in either the U.S. District Court for the Southern District of New York or the U.S. District Court for the District of New Jersey. Any party alleged to have violated Section 1 shall have the right, prior to any proceedings on the merits, to make an initial motion to dismiss any complaint that does not comply with the timeliness requirement of this Section 18.

\hypertarget{certifications}{%
\chapter{CERTIFICATIONS}\label{certifications}}

\hypertarget{contract-certification.}{%
\section{Contract Certification.}\label{contract-certification.}}

\begin{enumerate}
\def\labelenumi{(\alph{enumi})}
\tightlist
\item
  Every Player Contract, or any Renegotiation, Extension, or amendment of a Player Contract, entered into during the term of this Agreement shall be accompanied by a certification, sworn to separately by (i) the person who executed the Player Contract on behalf of the Team, (ii) the player, and (iii) any player agent who negotiated the Contract on behalf of the player, under penalties of perjury, that the Player Contract, Renegotiation, Extension, or amendment sets forth all components of a player's Salary from the Team or Team Affiliates and that there are no undisclosed agreements of any kind, express or implied, oral or written, and that there are no promises, undertakings, representations, commitments, inducements, assurances of intent, or understandings of any kind that have not been disclosed to the NBA:

  \begin{enumerate}
  \def\labelenumii{(\roman{enumii})}
  \tightlist
  \item
    involving consideration to be paid, furnished or made available to the player, or any person or entity controlled by or related to the player, by the Team or Team Affiliates either during the term of the Player Contract or thereafter; or
  \item
    concerning any future Renegotiation, Extension, or amendment of the Player Contract or the entry,into any new Player Contract.
  \end{enumerate}
\item
  Prior to the assignment of any Player Contract of a player who is in the last Salary Cap Year of the Contract (or the last Salary Cap Year before the player has the right to terminate the Contract), the player, the player's agent, and the Team to which such Contract is to be assigned shall each submit to the NBA a certification, sworn to under penalties of perjury, that there are no undisclosed agreements of any kind, express or implied, oral or written, and that there are no promises, undertakings, representations, commitments, inducements, assurances of intent, or understandings of any kind that have not been disclosed to the NBA, concerning any new Player Contract, any future Renegotiation, Extension, or amendment of the Player Contract that has been assigned, or any matters of compensation (including future compensation) between the player (or the player's agent) and the Team to which the Player Contract has been assigned.
\item
  If a player, within two years after the assignment of such player's Player Contract, enters into a new Player Contract, or any Renegotiation, Extension, or amendment of the Player Contract that had been assigned, the Team, the player, and the player's agent shall each submit to the NBA a certification, sworn to under penalties of perjury, that, at the time of the assignment, there were no agreements of any kind, express or implied, oral or written, or promises, undertakings, representations, commitments, inducements, assurances of intent, or understandings of any kind, between the player (or the player's agent) and the Team to which the Contract has been assigned that have not been disclosed to the NBA, concerning any new Player Contract, or any future Renegotiation, Extension, or amendment of the Player Contract that had been assigned.
\end{enumerate}

\hypertarget{end-of-season-certification.}{%
\section{End of Season Certification.}\label{end-of-season-certification.}}

\begin{enumerate}
\def\labelenumi{(\alph{enumi})}
\tightlist
\item
  At the conclusion of each NBA Season, a Governor (or Alternate Governor) and the executive primarily responsible for basketball operations on behalf of the Team shall each submit to the NBA a certification, sworn to under penalties of perjury, that the Team has not, to the extent of their knowledge after reasonable inquiry, violated the terms of Article XIV, Section 1, nor received from the NBA League Office any communication disclosing that an NBA Team has negotiated with any Free Agent prior to the execution of a Player Contract with that player. Upon receipt of each such certification, the NBA shall forward a copy of the certification to the Players Association.
\item
  A violation of this Section 2 may be deemed evidence of a violation of Article XIV, Section 1.
\end{enumerate}

\hypertarget{false-certification.}{%
\section{False Certification.}\label{false-certification.}}

Any criminal complaint of perjury filed by the NBA or any Team pursuant to Section 1 above shall be against the player, the player's agent, and the Team official making such certification alleged to have committed such perjury.

\hypertarget{mutual-reservation-of-rights}{%
\chapter{MUTUAL RESERVATION OF RIGHTS}\label{mutual-reservation-of-rights}}

Upon the expiration or termination of this Agreement, no person shall be deemed to have waived, by reason of the entry into or effectuation of this Agreement, any other collective bargaining agreement, or any Player Contract, or any of the terms of any of them, or by reason of any practice or course of dealing, their respective rights under law with respect to any issue or their ability to advance any legal argument.

\hypertarget{procedure-with-respect-to-playing-conditions-at-various-facilities}{%
\chapter{PROCEDURE WITH RESPECT TO PLAYING CONDITIONS AT VARIOUS FACILITIES}\label{procedure-with-respect-to-playing-conditions-at-various-facilities}}

\chaptermark{PROCEDURE WITH RESPECT \ldots}

When a new franchise is granted or when an existing franchise moves to another city, the Players Association shall, upon request and within a reasonable period of time, have the right to inspect the facility to be used by such franchise. Similarly, the Players Association shall, upon reasonable notice to the Team(s) involved and the NBA, have the right to inspect the training camp and practice facilities used by such Team(s). If, following such inspection, the Players Association is of the opinion that the playing conditions at such facility will endanger the health and safety of NBA players, it shall promptly notify the Commissioner in writing. Promptly following such notice, representatives of the Players Association and of the Team(s) involved, and the Commissioner or his designee shall meet in an effort to resolve the matter. It is agreed that the failure of the parties to resolve the matter shall not impair the legally binding effect of this Agreement or create any right to (a) unilaterally implement during the term of this Agreement any provision concerning such unresolved matter, (b) lock out, or (c) strike. If no resolution satisfactory to the Players Association, the Team(s) involved and the Commissioner is reached, the issue whether the playing conditions at the facility in question will endanger the health and safety of NBA players will, without interruption of the schedule or training game or practice activities, immediately be submitted to and determined by the Grievance Arbitrator in accordance with the provisions of Article XXXI; provided, however, that the Grievance Arbitrator need not render an award within 24 hours of the conclusion of the hearing, but shall issue his award as expeditiously as possible under the circumstances.

\hypertarget{travel-accommodations-locker-room-facilities-and-parking}{%
\chapter{TRAVEL ACCOMMODATIONS, LOCKER ROOM FACILITIES AND PARKING}\label{travel-accommodations-locker-room-facilities-and-parking}}

\chaptermark{TRAVEL ACCOMMODATIONS, LOCKER \ldots}

\hypertarget{hotel-arrangements.}{%
\section{Hotel Arrangements.}\label{hotel-arrangements.}}

Each Team agrees to use its best efforts to make the following arrangements for its players while they are ``on the road'':

\begin{enumerate}
\def\labelenumi{(\alph{enumi})}
\tightlist
\item
  To have their baggage picked up by porters.
\item
  To have them stay in first class hotels.
\item
  To have extra-long beds available to them in each hotel.
\end{enumerate}

If there is a finding that a Team has committed a willful violation of this provision, the NBA shall impose a \$1,000 fine on such Team.

\hypertarget{first-class-travel.}{%
\section{First Class Travel.}\label{first-class-travel.}}

\begin{enumerate}
\def\labelenumi{(\alph{enumi})}
\tightlist
\item
  Each Team shall provide first class travel accommodations on all trips in excess of one hour, except when such accommodations are not available; provided, however, that a Team's head coach may fly first class in place of a player when eight or more first class seats are provided to players. In the event a Team's head coach flies first class in place of a player, one player, designated by the Players Association, shall be paid the difference between the amount paid by such Team for a first class seat on the flight involved and the cost of the seat purchased for such designated player on that flight.
\item
  If there is a finding that a Team has committed a willful violation of this Section, the NBA shall impose a \$1,000 fine on such Team.
\end{enumerate}

\hypertarget{locker-room-facilities.}{%
\section{Locker Room Facilities.}\label{locker-room-facilities.}}

Each Team agrees to use its best efforts to provide suitable locker room facilities and to stabilize the temperature in locker rooms to make it consistent with the temperature on playing courts.

\hypertarget{parking-facilities.}{%
\section{Parking Facilities.}\label{parking-facilities.}}

Each Team agrees to make parking facilities available to its players without charge in connection with games and practices conducted at the facility regularly used by such Team for home games and/or practices.

\hypertarget{union-security-dues-and-check-off}{%
\chapter{UNION SECURITY, DUES AND CHECK-OFF}\label{union-security-dues-and-check-off}}

\hypertarget{membership.}{%
\section{Membership.}\label{membership.}}

As a condition of employment commencing with the execution of this Agreement, for the duration of this Agreement only, and wherever legal: (a) any active player who is or later becomes a member in good standing of the Players Association must maintain his membership in good standing in the Players Association; and (b) any active player (including a player in the future) who is not a member in good standing of the Players Association must, on the 30th day following the beginning of his employment or the 30th day following the execution of this Agreement, whichever is later, pay, pursuant to Section 2 below or otherwise, to the Players Association an annual service fee in the same amount as any initiation fee and dues required of members of the Players Association.

\hypertarget{check-off.}{%
\section{Check-off.}\label{check-off.}}

Commencing with the execution of this Agreement and for the duration of this Agreement only, each Team will check-off the initiation fee and annual dues, assessments or service fees, as the case may be, in equal weekly installments from the first four regular season pay checks for each player for whom a current check-off authorization has been provided to the Team. The Team will forward the check-off monies to the Players Association within fourteen (14) days of each check-off.

\hypertarget{enforcement.}{%
\section{Enforcement.}\label{enforcement.}}

\begin{enumerate}
\def\labelenumi{(\alph{enumi})}
\tightlist
\item
  Upon written notification to the NBA by the Players Association that a player has not paid any initiation fee, dues or the equivalent service fee in violation of Section 1 of this Article XIX, the NBA will raise the matter for discussion with the player and his Team. If there is no resolution of the matter within seven days, then the Team will, upon notification of the Players Association, suspend the player without pay, wherever legal. Such suspension will continue until the Players Association has notified the Team in writing that the suspended player has satisfied his obligation as contained in Section 1 of this Article XIX. The parties hereby agree that suspension without pay is adopted as a substitute for and in lieu of discharge as the penalty for a violation of the union security clause of the Agreement and that no player will be discharged for a violation of that clause. A copy of all notices required by this Section will be simultaneously mailed to the player involved and the
  NBA.
\item
  The term ``member in good standing'' as used in this Article XIX applies only to the payment of dues or any initiation fee and not to any other factors involved in union discipline.
\item
  Other than pursuant to Section 2 above, no Team shall pay any initiation fees, dues, or equivalent service fee on behalf of any player.
\end{enumerate}

\hypertarget{no-liability.}{%
\section{No Liability.}\label{no-liability.}}

Neither the NBA nor any Team shall be liable for any salary, bonus, or other monetary claims that result from a player being suspended pursuant to the terms of Section 3 above.

\hypertarget{scheduling}{%
\chapter{SCHEDULING}\label{scheduling}}

\hypertarget{number-of-regular-season-games.}{%
\section{Number of Regular Season Games.}\label{number-of-regular-season-games.}}

Each Team agrees that in no event will it play more than 82 Regular Season games.

\hypertarget{location-of-games.}{%
\section{Location of Games.}\label{location-of-games.}}

During the Regular Season schedule, games between NBA Teams may be played at any location, within or outside the United States. The NBA shall supervise the arrangements made with respect to games played outside the United States and the accommodations provided to players participating in such games.

\hypertarget{holidays.}{%
\section{Holidays.}\label{holidays.}}

\begin{enumerate}
\def\labelenumi{(\alph{enumi})}
\tightlist
\item
  No Team will be required to play a game on December 25, unless such game is to be telecast or cablecast nationally.
\item
  Games scheduled to be played on January 1 and Good Friday shall not commence prior to 6 p.m. (local time), unless the Players Association consents thereto, which consent shall not be unreasonably withheld. The Players Association will, upon request, consent to the earlier commencement of two games on each of such dates if such games are to be broadcast or cablecast nationally, and provided that the Teams involved are in the same time zone or otherwise in close geographic proximity.
\item
  Teams at home on December 25 and January 1 (each, a ``Holiday'') may, but shall not be required to, conduct a practice on either (or both) of such Holidays, provided: (i) the Team's players have requested that they practice on the Holiday, as communicated to the Team by the Team's player representative; and (ii) within seven (7) days before or after the Holiday, the Team's players are provided with a ``day off'' - i.e., the Team will not conduct any practice, including any optional practice, on such date, and the Team will not have a scheduled game on such date.
\item
  Teams shall not depart for an away game or series of away games prior to 3 p.m. (local time) on December 25 or January 1, unless reasonable transportation arrangements for such game or games cannot be made at or after 3 p.m. (local time).
\end{enumerate}

\hypertarget{travel.}{%
\section{Travel.}\label{travel.}}

The NBA and its Teams shall use their best efforts to devise reasonable travel schedules when Regular Season games are played outside the United States and Canada. No Team shall be required to play a scheduled game on the same day that such Team has traveled across two time zones, except in unusual circumstances and unless the Players Association consents thereto, which consent shall not be unreasonably withheld.

\hypertarget{training-camp.}{%
\section{Training Camp.}\label{training-camp.}}

\begin{enumerate}
\def\labelenumi{(\alph{enumi})}
\tightlist
\item
  When players (other than Rookies) are required to report to training camp pursuant to paragraph 2 of the Uniform Player Contract on the twenty-ninth day prior to the first game of an NBA Regular Season, they may only be required to attend a team dinner and team meetings, to participate in photograph sessions, and to submit to a physical examination on that day.
\item
  Notwithstanding the foregoing, if a Veteran Player is under contract to a Team that is scheduled during a particular NBA Season to participate outside of the United States in a Pre-Season Exhibition game or a Regular Season game during the first week of the Regular Season (except a game involving either the Toronto or Vancouver franchise and another NBA team played in the home territory of either Toronto or Vancouver), the player will not be required to attend training camp during that particular NBA Season earlier than 4 p.m. (local time) on the thirty-second day prior to the first game of an NBA Regular Season. Rookies may be required to attend training camp at an earlier date, but no earlier than ten (10) days prior to the date that Veterans are required to attend.
\end{enumerate}

\hypertarget{further-discussions.}{%
\section{Further Discussions.}\label{further-discussions.}}

The parties hereto shall arrange for representatives of the Players Association to meet with such representatives of the NBA and its Teams as may be designated by the Commissioner to discuss the subject of scheduling.

\hypertarget{nba-all-star-game}{%
\chapter{NBA ALL-STAR GAME}\label{nba-all-star-game}}

\hypertarget{awards.}{%
\section{Awards.}\label{awards.}}

\begin{enumerate}
\def\labelenumi{(\alph{enumi})}
\tightlist
\item
  For their participation in the 1996, 1997, or 1998 NBA All-Star Game, players on the winning team shall each receive \$12,000, and players on the losing team shall each receive \$6,000. For their participation in the 1999, 2000, or 2001 NBA All-Star Game, players on the winning team shall each receive \$15,000, and players on the losing team shall each receive \$7,500.
\item
  Commencing with the 1996 Rookie All-Star Game, players on the winning team shall each receive \$4,000 and players on the losing team shall each receive \$3,000.
\end{enumerate}

\hypertarget{player-guests.}{%
\section{Player Guests.}\label{player-guests.}}

Each player who participates in the NBA All-Star Game, NBA Rookie All-Star Game, or any NBA All-Star Skills Competition may invite a guest, who shall be reimbursed for the cost of round-trip first-class air transportation between the home city of the Team by which such player is employed and the site of the All-Star Game or Rookie All-Star Game.

\hypertarget{players-not-participating-in-all-star-activities.}{%
\section{Players Not Participating in All-Star Activities.}\label{players-not-participating-in-all-star-activities.}}

Players not invited to participate in the NBA All-Star Game or Rookie All-Star Game shall have three days off during the All-Star Game break.

\hypertarget{medical-treatment-of-players-and-release-of-medical-information}{%
\chapter{MEDICAL TREATMENT OF PLAYERS AND RELEASE OF MEDICAL INFORMATION}\label{medical-treatment-of-players-and-release-of-medical-information}}

\chaptermark{MEDICAL TREATMENT OF \ldots}

\hypertarget{one-surgeon.}{%
\section{One Surgeon.}\label{one-surgeon.}}

Each Team agrees that a player requiring the care and treatment of an orthopedic surgeon will, so far as practicable, be referred to and treated by one orthopedic surgeon (rather than several).

\hypertarget{committee-of-team-physicians.}{%
\section{Committee of Team Physicians.}\label{committee-of-team-physicians.}}

Representatives designated by the Players Association shall participate in meetings of the committee of Team physicians appointed by the NBA for the purpose of discussing matters related to the medical care and treatment of players.

\hypertarget{public-release-of-medical-information.}{%
\section{Public Release of Medical Information.}\label{public-release-of-medical-information.}}

\begin{enumerate}
\def\labelenumi{(\alph{enumi})}
\tightlist
\item
  Subject to subsection (b) below, each Team may make public medical information relating to the players in its employ, provided that such information relates solely to the reasons why any such player has not been or is not rendering services as a player.
\item
  A player or his immediate family (where appropriate) shall have the right to approve the terms and timing of any public release of medical information relating to any injuries or illnesses suffered by that player that are potentially life- or career-threatening, or that do not arise from the player's participation in NBA games or practices.
\end{enumerate}

\hypertarget{non-team-physicians.}{%
\section{Non-Team Physicians.}\label{non-team-physicians.}}

A player who consults a physician other than such player's Team physician shall give notice of such consultation to his Team's physician and shall authorize and direct such other physician to provide his Team with all information it may request concerningany condition that, in the judgment of the Team's physician, may affect such player's ability to play skilled basketball.

\hypertarget{arbitration-proceedings.}{%
\section{Arbitration Proceedings.}\label{arbitration-proceedings.}}

During the course of any arbitration proceeding, the Grievance Arbitrator may, by appropriate process, require any person (including, but not limited to, a Team and a Team physician, and a player and any physician consulted by such player) to provide to the player or that player's Team, as the case may be, all medical information in the possession of any person relating to the subject matter of the arbitration.

\hypertarget{draftees.}{%
\section{Draftees.}\label{draftees.}}

Prior to any NBA Draft, the NBA and/or its Teams acting jointly may request that persons eligible for such Draft voluntarily submit to the administration of standardized medical or laboratory tests (other than tests for controlled substances), the results of which shall be made available to any Team upon request, but which shall be kept confidential from the public and the media. Any person who submits to the administration of such tests may, prior to such Draft, be requested to submit voluntarily to an examination by the physician(s) for an NBA Team(s), but shall not be requested to undergo any medical or laboratory test administered at the request of the NBA and/or its Teams.

\hypertarget{exhibition-games-off-season-games-and-events}{%
\chapter{EXHIBITION GAMES, OFF-SEASON GAMES, AND EVENTS}\label{exhibition-games-off-season-games-and-events}}

\chaptermark{EXHIBITION GAMES, OFF-SEASON \ldots}

\hypertarget{exhibition-games.}{%
\section{Exhibition Games.}\label{exhibition-games.}}

Subject to the provisions of paragraph 2 of the Uniform Player Contract, players shall be required to participate in Exhibition games between an NBA Team and a non-member of the NBA at any location, within or outside the United States, subject to the following conditions:

\begin{enumerate}
\def\labelenumi{(\alph{enumi})}
\tightlist
\item
  The NBA shall supervise the arrangements made with respect to tournaments or series conducted outside the United States and the accommodations provided to NBA players participating in such foreign tournaments or series.
\item
  The NBA shall use its best efforts to establish an Exhibition game schedule pursuant to which excessive travel will be avoided and reasonable periods of time between games will be allotted.
\item
  In any year in which it is played, the annual Basketball Hall of Fame Exhibition game shall be considered as one of the eight exhibition games prior to the Regular Season referred to in paragraph 2 of the Uniform Player Contract.
\end{enumerate}

\hypertarget{inter-squad-scrimmage.}{%
\section{Inter-Squad Scrimmage.}\label{inter-squad-scrimmage.}}

In addition to the Exhibition games provided for by paragraph 2 of the Uniform Player Contract, and during each of the playoff series conducted during the term of this Agreement, any Team that qualifies for the playoffs but is not required to participate in the first round thereof, may arrange and require its players to participate in one inter-squad game or scrimmage with another similarly situated Team, provided that such game or scrimmage is not open to members of the general public.

\hypertarget{off-season-basketball-events.}{%
\section{Off-Season Basketball Events.}\label{off-season-basketball-events.}}

No player may play in any public off-season basketball game, summer league (e.g., Southern California Pro League or the Doral Arrowwood Summer League), or public exhibition or competition of basketball skills (e.g., a slam dunk contest or a ``tour'' organized by an NBA business partner) (each a ``Basketball Event'') unless such Basketball Event is approved in writing by the NBA. The NBA will consider an off-season Basketball Event for approval only if a request for such approval is submitted in writing to the NBA prior to the April 1 immediately preceding the off-season in which such Basketball Event is to be conducted, and only if the arrangements made with respect to any such off-season Basketball Event are confirmed in writing by the NBA and satisfy the following requirements:

\begin{enumerate}
\def\labelenumi{(\alph{enumi})}
\tightlist
\item
  General Requirements.

  \begin{enumerate}
  \def\labelenumii{(\roman{enumii})}
  \tightlist
  \item
    The Basketball Event takes place on or after July 1, but in no event later than September 15 (or, in the case of a summer league, September 1);
  \item
    Prior to the Basketball Event, each participating player receives the express written consent of his Team to participate in the Basketball Event;
  \item
    The person(s) organizing the Basketball Event obtains disability insurance covering each participating player, in an amount acceptable to the NBA;
  \item
    The Basketball Event is not televised, broadcast on radio, or exploited in any media, live or on tape; and
  \item
    The names and logos of the NBA and/or any NBA Team are not used or referred to in connection with the Basketball Event.
  \end{enumerate}
\item
  Additional Charitable Game Requirements. The NBA will consider an off-season charitable game for approval only if, in addition to the general requirements set forth in subsection (a)144 above, the arrangements made with respect to such charitable game also satisfy the following:

  \begin{enumerate}
  \def\labelenumii{(\roman{enumii})}
  \tightlist
  \item
    The Players Association approves the game in writing (which approval shall not be unreasonably withheld);
  \item
    All proceeds from the sale of tickets to the game are used for charitable purposes;
  \item
    The game is officiated by NBA referees;
  \item
    There is at least one (1) NBA Team trainer and at least one (1) physician present at the game;
  \item
    The name or likeness of an NBA player is not used, or referred to, in advertisements or promotions for or related to the game, except that if the organizer of the game is an NBA player, such organizer-player's name or likeness may be used, or referred to, in such advertisements or promotions;
  \item
    Participating players are not paid or compensated (in excess of per diem and actual expenses incurred in traveling to and participating in the game);
  \item
    The organizer guarantees that the game will produce at least \$100,000 for charity, and, if directed by the NBA and the Players Association, posts security for such amount in a form satisfactory to the NBA and the Players Association which grants the NBA and/or the Players Association the right to sue to recover such amount for the benefit of the charity;
  \item
    The game is played in the United States or Canada; and
  \item
    The organizer agrees to provide the NBA and the Players Association with an audited statement of revenues and expenses within thirty (30) days following the game.
  \end{enumerate}
\item
  Additional Summer League Requirements. The NBA will consider an off-season summer league for approval only if, in addition to the general requirements set forth in subsection (a) above, the arrangements made with respect to such league also satisfy the following:

  \begin{enumerate}
  \def\labelenumii{(\roman{enumii})}
  \tightlist
  \item
    Each league game is officiated by an official approved by the NBA;
  \item
    Participating players are not paid or compensated (except as provided under Section 4(c) below);
  \item
    No league game is accompanied by an exhibition of basketball skills (e.g., a slam dunk contest), unless such exhibition has been separately approved in writing by the NBA;
  \item
    There is at least one (1) trainer and at least one (1) physician present at each league game; and
  \item
    Each league game is played in the United States or Canada.
  \end{enumerate}
\item
  Notwithstanding any other terms of this Section 3, and without limiting the right of the NBA to approve all arrangements of a proposed Basketball Event, the NBA may, in its sole discretion, require, as a condition of its approval of a Basketball Event (other than a charitable game or summer league), that the Basketball Event organizer pay an appropriate fee to the NBA prior to the commencement of the Basketball Event.
\item
  For purposes of this Section 3, any game or competition in which an NBA player participates on behalf of USA Basketball or any other national basketball federation will not be considered a Basketball Event; provided, however, that the NBA shall, after reasonable consultation with the Players Association, determine the requirements that shall govern player participation in any such game or competition.
\end{enumerate}

\hypertarget{summer-leagues.}{%
\section{Summer Leagues.}\label{summer-leagues.}}

\begin{enumerate}
\def\labelenumi{(\alph{enumi})}
\tightlist
\item
  No NBA Team may simultaneously enroll more than three (3) Veterans in any summer basketball league during an off-season. For purposes of this subsection, the following players are not considered Veterans:

  \begin{enumerate}
  \def\labelenumii{(\roman{enumii})}
  \tightlist
  \item
    a player who has never signed a Player Contract or whose first Player Contract begins with the Season immediately following the off-season in which such summer league is to be conducted;
  \item
    a player not under contract to an NBA Team at the time he enrolls in such summer league;
  \item
    a player under contract to an NBA Team but who missed twenty-five (25) or more of the Team's games during the Regular Season immediately preceding such off-season; and
  \item
    a player who played for a team in the Continental Basketball Association during all, or any portion, of the Regular Season immediately preceding such off-season.
  \end{enumerate}
\item
  Any player enrolled in a summer basketball league shall be furnished and requested to sign a ``Notice to Veteran Players'' in the form attached hereto as Exhibit G, and/or a ``Form Regarding Summer League Participation'' as attached hereto as Exhibit H. Each Team shall promptly forward to the NBA League Office a copy of each signed form, a copy of which shall promptly be forwarded by the NBA League Office to the Players Association.
\item
  The only compensation that may be paid by a Team or any person or entity affiliated with a Team to a player participating in a summer basketball league is a reasonable expense allowance for (i) meals, but no greater than that set forth in Article III, Section 2; (ii) lodging; and (iii) transportation to and from the player's home to the site of the summer league, and to and from the site of the player's lodging during the summer league to the site of summer-league-related activities. In addition, the Team may purchase a disability insurance policy for the player covering the term of the applicable summer league.
\item
  No Team shall schedule, and no player shall participate in, a summer basketball league that is scheduled to extend, or does in fact extend, past September 1 of any calendar year.
\end{enumerate}

\hypertarget{prohibition-of-no-trade-contracts}{%
\chapter{PROHIBITION OF NO-TRADE CONTRACTS}\label{prohibition-of-no-trade-contracts}}

\hypertarget{general-limitation.}{%
\section{General Limitation.}\label{general-limitation.}}

No Player Contract may contain any prohibition or limitation of an NBA Team's right to assign such Contract to another NBA Team.

\hypertarget{exceptions-to-general-limitation.}{%
\section{Exceptions to General Limitation.}\label{exceptions-to-general-limitation.}}

Notwithstanding the provisions of Section 1 of this Article XXIV:

\begin{enumerate}
\def\labelenumi{(\alph{enumi})}
\tightlist
\item
  A Player Contract may contain (in Exhibit 4 to such Player Contract) a provision entitling a Player to receive, on a one-time basis, upon the first assignment of that Player Contract, a sum of money that does not exceed, or property or investments with a fair market value that does not exceed, 15\% of the compensation remaining to be earned by the Player pursuant to that Player Contract at the time of such first assignment.
\item
  A Player Contract entered into by a player who has eight (8) or more years of service in the NBA and who has rendered four (4) or more years of service for the Team entering into such Contract may contain a prohibition or limitation of such Team's right to assign such contract to another NBA Team. For the purposes of this Section 2(b), a player shall be credited with one year of NBA service for each NBA Season for which he is signed to play, whether or not his services are retained by that NBA Team for the start, or for any portion, of the Season for which he is signed; provided, however, that if a player signs a Player Contract with a stated term of more than one Season, he shall receive credit for the first Season of such Contract, and thereafter will receive credit only for such Seasons in which his services are retained for any portion of such Season (including regular training camp). Notwithstanding the above, a player will not receive credit for a year of service for an NBA Season in which he: (i) withholds playing services called for by a Player Contract for more than thirty days after the NBA Season begins; or (ii) is signed only to one or more lO-day Player Contracts with more than 10 days remaining in that NBA season.
\end{enumerate}

\hypertarget{limitation-on-deferred-compensation}{%
\chapter{LIMITATION ON DEFERRED COMPENSATION}\label{limitation-on-deferred-compensation}}

\hypertarget{general-limitation.-1}{%
\section{General Limitation.}\label{general-limitation.-1}}

No NBA Team may sign a Player Contract with any player under which more than 30\% of Salary is Deferred Compensation. For purposes of this provision only, Deferred Compensation shall mean Deferred Compensation during the period commencing more than two years after the playing term covered by a Player Contract.

\hypertarget{attribution.}{%
\section{Attribution.}\label{attribution.}}

All Player Contracts shall specify the Season(s) to which any Deferred Compensation is attributable.

\hypertarget{rabbi-trusts.}{%
\section{Rabbi Trusts.}\label{rabbi-trusts.}}

\begin{enumerate}
\def\labelenumi{(\alph{enumi})}
\tightlist
\item
  Notwithstanding Section 1, a Player Contract may provide for an annuity to be purchased by the Team that will pay the Player (or his designees) an amount of Deferred Compensation in excess of 30\% of Salary, provided that:

  \begin{enumerate}
  \def\labelenumii{(\roman{enumii})}
  \tightlist
  \item
    The Team and the Player agree with respect to the form and terms of the annuity instrument and the institution from which it is purchased;
  \item
    Ownership of the annuity and all related aspects are structured in a manner that qualifies the arrangement as a tax deferred (``rabbi'') trust, in the opinion of the NBA's tax advisor; and
  \item
    The total cost of the annuity and the schedule of payment of such costs are specified in the Player Contract.
  \end{enumerate}
\item
  Notwithstanding anything to the contrary contained in subsection (a) above:(i) If the institution obligated to make payment under the annuity fails to do so for any reason (other than non-compliance by the Team with the provisions of the annuity contract), the Team shall thereupon become obligated to pay to the Player as Deferred Compensation an amount, if any, equal to the unpaid portion of the purchase price of the annuity for which the Team remains obligated; and (ii) If the creditors of the Team and not the Player receive payments under the annuity, the Team shall thereupon become obligated to pay to the Player as Deferred Compensation an amount equal to the full purchase price of the annuity.
\end{enumerate}

\hypertarget{team-rules}{%
\chapter{TEAM RULES}\label{team-rules}}

\hypertarget{standard-team-rules.}{%
\section{Standard Team Rules.}\label{standard-team-rules.}}

Subject to Section 2 below, each Team may require its players to abide by the rules set forth in the ``Standard Club Rules'' annexed as Exhibit G to the 1988 NBA/NBPA Collective Bargaining Agreement. Until the NBA and the Players Association agree on a set of Uniform Team Rules, the NBA and the Players Association agree that such Standard Club Rules constitute ``reasonable rules'' within the meaning of paragraph 5 of a Uniform Player Contract.

\hypertarget{uniform-team-rules.}{%
\section{Uniform Team Rules.}\label{uniform-team-rules.}}

\begin{enumerate}
\def\labelenumi{(\alph{enumi})}
\tightlist
\item
  The parties agree that, within 90 days from the execution of this Agreement, they will negotiate for the purpose of establishing a set of Uniform Team Rules that will apply to all Teams, except as provided below.
\item
  Any Team may establish a rule or rules supplementing the Uniform Team Rules with respect to matters not covered by the Uniform Team Rules, provided such supplemental rule(s) are in writing and are made available for review by the Players Association no later than the July 31st preceding the training camp period during which they are to become effective. If no timely Grievance objecting to any such supplemental rule(s) is filed by the Players Association with the NBA and the Team establishing such rule(s), such rule(s) shall be deemed to constitute ``reasonable rules'' within the meaning of paragraph 5 of a Uniform Player Contract. If a timely Grievance with respect to any such supplemental rule(s) is filed by the Players Association, such rule(s) shall not become effective unless and until they are found to constitute reasonable rule(s) by the Grievance Arbitrator.
\item
  If a Team seeks to establish a rule or rules that deviate from (rather than supplement) the Uniform 'Team Rules, it must make such non-conforming rule(s) available for review by the Players Association no later than the July 31 st preceding the training camp period during which such rule(s) are to become effective. If no timely Grievance objecting to such non-conforming rule(s) is filed by the Players Association with the NBA and the Team establishing such rule(s), such rule(s) shall be deemed to constitute ``reasonable rules'' within the meaning of paragraph 5 of a Uniform Player Contract. If a timely Grievance with respect to such non-conforming rule(s) is filed by the Players Association, such rule(s) shall not become effective unless and until they are found to constitute reasonable rule(s) by the Grievance Arbitrator. In any such Grievance, the Team shall have the burden of establishing that its non-conforming rule(s) are reasonable and justified by particular circumstances warranting a departure from the Uniform Team Rules.
\end{enumerate}

\hypertarget{right-of-set-off}{%
\chapter{RIGHT OF SET-OFF}\label{right-of-set-off}}

\hypertarget{set-off-calculation.}{%
\section{Set-Off Calculation.}\label{set-off-calculation.}}

\begin{enumerate}
\def\labelenumi{(\alph{enumi})}
\tightlist
\item
  When a Team terminates a Player Contract in circumstances where such Team, following the termination, continues to be liable for the compensation called for by such Contract (including any Deferred Compensation), the Team's liability for such compensation shall be reduced pro rata by any amounts earned by the player (for services as a player) from any professional basketball team during the period covered by the terminated Contract (including, but not limited to, amounts earned but not paid during such period); provided, however, that such reduction in liability shall not be more than 50\% of that portion of the player's compensation from the new team which is in excess of the Minimum Player Salary applicable at the time of termination.
\item
  For the purposes of this Article, a ``professional basketball team'' shall mean any team in any country that pays money or compensation of any kind (in excess of a stipend for living expenses) to a basketball player for rendering services to such team.
\end{enumerate}

\hypertarget{successive-terminations.}{%
\section{Successive Terminations.}\label{successive-terminations.}}

In the event of successive terminations by NBA Teams of Player Contracts involving the same Player, the Team first to terminate shall be entitled to the right of set-off provided for by this Article until its compensation liability has been eliminated in its entirety, and the right of set-off shall then pass in order to the Team(s) terminating any subsequent Contract(s).

\hypertarget{deferred-compensation.}{%
\section{Deferred Compensation.}\label{deferred-compensation.}}

In calculating the amount of set-off to which a Team may be entitled pursuant to this Article, Deferred Compensation payable to a player for or with respect to a period covered by the terminated Contract shall be discounted on an annual basis by a percentage equal to the prime rate as set by Citibank, N.A. and in effect at the time the agreement providing for such Deferred Compensation was made.

\hypertarget{broadcast-or-telecast-rights}{%
\chapter{BROADCAST OR TELECAST RIGHTS}\label{broadcast-or-telecast-rights}}

\hypertarget{league-rights.}{%
\section{League Rights.}\label{league-rights.}}

During the term of this Agreement, the Players Association agrees that the NBA and its Teams have the right to use, distribute, or license any performance by the players, under this Agreement or the Uniform Player Contract, for any form of broadcast or telecast, including over-the-air television, cable television, pay television, direct broadcast satellite television, and any form of cassette, cartridge, or disk system, or other means of distribution known or unknown.

\hypertarget{no-suit.}{%
\section{No Suit.}\label{no-suit.}}

The Players Association, for itself and present and future NBA players, covenants not to sue (or finance any suit against) the NBA, any of its Teams, or their agents, successors, assigns, or licensees, with respect to the use, distribution, or license, for any form of broadcast or telecast, including over-the-air television, cable television, pay television, or direct broadcast satellite television, and any form of cassette, cartridge, or disk system, or other means of distribution known or unknown, of any performances by any player rendered under this Agreement or prior collective bargaining agreements, or under Player Contracts made pursuant thereto, during any period up to and including the day following the last Playoff game of the 2000-01 NBA Season.

\hypertarget{reservation-of-rights.}{%
\section{Reservation of Rights.}\label{reservation-of-rights.}}

The Players Association expressly reserves its rights to bargain collectively on the subject described in Section 1 of this Article XXVIII at the expiration of this Agreement. Such reservation shall not, however, preclude the NBA from contending that the subject described in Section 1 of this Article XXVIII is not a mandatory subject of collective bargaining.

\hypertarget{miscellaneous}{%
\chapter{MISCELLANEOUS}\label{miscellaneous}}

\hypertarget{active-roster-size.}{%
\section{Active Roster Size.}\label{active-roster-size.}}

Each Team agrees to have twelve players on its Active List and to have a minimum of eight players on the bench for all Regular Season games. Notwithstanding the foregoing, any Team may from time to time as appropriate, but for no more than two consecutive weeks at a time during the Regular Season, have eleven players on its Active List.

\hypertarget{playing-rules-and-officiating.}{%
\section{Playing Rules and Officiating.}\label{playing-rules-and-officiating.}}

\begin{enumerate}
\def\labelenumi{(\alph{enumi})}
\tightlist
\item
  One representative of the Executive Board of the Players Association shall be permitted to attend the meetings of and have a vote on the NBA Competition Committee with respect to issues relating to the Official Playing Rules and Officiating.
\item
  The Players Association may on behalf of the players annually submit to the Commissioner one written critique of referees, without reference to any individual referee.
\end{enumerate}

\hypertarget{playoffs.}{%
\section{Playoffs.}\label{playoffs.}}

The number of Teams participating in the playoffs shall equal sixteen. Notwithstanding the foregoing, the NBA shall have the right to increase the number of Teams participating in the playoffs.

\hypertarget{implementation-of-agreement.}{%
\section{Implementation of Agreement.}\label{implementation-of-agreement.}}

\begin{enumerate}
\def\labelenumi{(\alph{enumi})}
\tightlist
\item
  The NBA and the Players Association will use their respective best efforts to have NBA Teams and NBA players comply with the terms and provisions of this Agreement.
\item
  The NBA and the Players Association shall use their respective best efforts and take all reasonable steps to cooperate to defend the enforceability of this Agreement against any challenge thereto.
\end{enumerate}

\hypertarget{release-for-fighting.}{%
\section{Release for Fighting.}\label{release-for-fighting.}}

Each NBA Team (hereinafter ``such Team'') hereby releases and waives every claim it may have against any player employed by other NBA Teams for injuries sustained by any player in the employ of such Team which arise out of or in connection with any fighting or other form of violent and/or unsportsmanlike conduct during the course of any Exhibition, Regular Season, and/or Playoff game.

\hypertarget{game-tickets-for-retired-players.}{%
\section{Game Tickets for Retired Players.}\label{game-tickets-for-retired-players.}}

Each Team agrees to provide retired players with three or more years of NBA service with the opportunity to purchase two tickets at box office prices to its NBA home games, and to hold such tickets for such players, provided tickets are available and the retired players provide the Team with 48 hours advance notice of their desire for such tickets.

\hypertarget{limitation-on-player-ownership.}{%
\section{Limitation on Player Ownership.}\label{limitation-on-player-ownership.}}

During the term of this Agreement, no NBA player may acquire or hold a direct or indirect interest in the ownership of any NBA Team, provided, however, that any player may own shares of any publicly traded company that directly or indirectly owns an NBA Team.

\hypertarget{nondisclosure.}{%
\section{Nondisclosure.}\label{nondisclosure.}}

The parties agree that (a) the economic terms of any individual Uniform Player Contract entered into by a Team and a player and (b) any information contained in or disclosed to the Players Association in connection with the Audit Reports shall not be disclosed to the media by (i) the NBA, its teams, or their respective employees, or (ii) the Players Association, NBA players, or their respective employees, agents, or representatives.

\hypertarget{game-tickets.}{%
\section{Game Tickets.}\label{game-tickets.}}

\begin{enumerate}
\def\labelenumi{(\alph{enumi})}
\tightlist
\item
  In the event that a Team provides home-game tickets to its players, seat locations must be allocated to players based on seniority, with the most senior players (based on years of NBA service) receiving the most favorable seat locations.
\item
  NBA Teams shall use best efforts to provide four (4) tickets to authorized representatives of the Players Association to any home game at box office prices, provided notice of such request is given at least forty-eight (48) hours before the game.
\end{enumerate}

\hypertarget{additional-canadian-provisions.}{%
\section{Additional Canadian Provisions.}\label{additional-canadian-provisions.}}

\begin{enumerate}
\def\labelenumi{(\alph{enumi})}
\item
  The bases upon which a player may be disciplined or discharged or a Player Contract terminated, as set forth in this Agreement and/or in the Uniform Player Contract, shall constitute just and reasonable cause within the meaning of any applicable Canadian statute (federal or provincial).
\item
  During the term of this Agreement, the NBA and Players Association shall consult regularly about issues relating to the workplace which affect the parties or any player bound by this Agreement.
\item
  \begin{enumerate}
  \def\labelenumii{(\roman{enumii})}
  \tightlist
  \item
    If and to the extent Sections 45 and/or 46 of the Ontario Labour Relations Act are or may be found applicable to this Agreement, the parties agree that the provisions thereof shall apply only to disputes between the Toronto Raptors and players employed by the Toronto Raptors.
  \item
    If and to the extent Section 84(2) of the British Columbia Labour Relations Code is or may be found applicable to this Agreement, the parties agree that the provisions thereof shall apply only to disputes between the Vancouver Grizzlies and players employed by the Vancouver Grizzlies.
  \end{enumerate}
\item
  The parties acknowledge and agree that a player employed by an NBA Team pursuant to the provisions of a Uniform Player Contract, a lO-Day Contract, or a Rest-of-Season Contract is and/or shall be deemed to be an ``employee employed for a definite term or task'' within the meaning of Section 57(10)(a) of the Ontario Employment Standards Act and an ``employee employed for a definite term'' within the meaning of Sections 43(c)(1) of the British Columbia Employment Standards Act, so as to render inapplicable to NBA players the provisions of Section 57 of the Ontario Employment Standards Act and Section 42 of the British Columbia Employment Standards Act.
\item
  The parties acknowledge and agree that the severance benefits provided to players pursuant to this Agreement (including the provisions of Player Contracts that provide, in certain circumstances, for the continued payment of Salary to a player following the termination of a Player Contract) constitute and/or shall be deemed to constitute a ``settlement of all severance pay claims'' within the meaning of Section 58(18) of the Ontario Employment Standards Act and/or ``a contractual severance pay scheme under which payments for loss of employment based upon length of service are provided'' within the meaning of Section 58(7)(b) of the Ontario Employment Standards Act, so as to render inapplicable to NBA players the provisions of such Section 58 of such Act.
\item
  Upon the NBA's request, the Players Association shall cooperate with the NBA in a reasonable manner in connection with any effort the NBA may make to seek an exemption from any Canadian (federal or provincial) law or regulation affecting the employment relationship that is inconsistent with the provisions of this Agreement or any other agreement between the Players Association and the NBA (or NBA Properties) or between any player and any NBA Team.
\item
  The parties hereby specifically exclude the operation of subsections (2) and (3) of Section 50 of the British Columbia Labour Relations Code.
\item
  All players employed by NBA Teams shall be paid in U.S. dollars, regardless of where such Teams are located.
\end{enumerate}

\hypertarget{no-strike-and-no-lockout-provisions-and-other-undertakings}{%
\chapter{NO-STRIKE AND NO-LOCKOUT PROVISIONS AND OTHER UNDERTAKINGS}\label{no-strike-and-no-lockout-provisions-and-other-undertakings}}

\chaptermark{NO-STRIKE AND NO-LOCKOUT \ldots}

\hypertarget{no-strike.}{%
\section{No Strike.}\label{no-strike.}}

During the term of this Agreement, neither the Players Association nor its members shall engage in any strikes, cessations or stoppages of work, or any other similar interference with the operations of the NBA or any of its Teams.

\hypertarget{no-lockout.}{%
\section{No Lockout.}\label{no-lockout.}}

During the term of this Agreement, neither the NBA nor its Teams shall engage in any lockouts, cessations or stoppages of work or any other similar interference with the employment of NBA players by NBA Teams.

\hypertarget{no-breach-of-player-contracts.}{%
\section{No Breach of Player Contracts.}\label{no-breach-of-player-contracts.}}

The Players Association agrees that it will not engage in any concerted activities to breach, induce the breach of, or threaten to breach or induce the breach of, any Player Contract.

\hypertarget{best-efforts-of-players-association.}{%
\section{Best Efforts of Players Association.}\label{best-efforts-of-players-association.}}

The Players Association will use its best efforts to prevent each player from rendering, or threatening to render, services as a professional basketball player for another professional basketball team during the term of a Player Contract between such player and the Team for which he plays (except as said Player Contract may be assigned, sold, or transferred in accordance with the provisions thereof); to prevent each player from refusing, or threatening to refuse, to participate in any scheduled Exhibition game, Regular Season game, All-Star Game, Rookie All-Star Game, All-Star Skills Competition, or Playoff game; to prevent each player from other-wise breaching, or threatening to breach, such Player Contract; and to prevent each player from making any demand upon the NBA or any of its Teams, including, but not limited to, a demand (accompanied by threats that the player will render services as a professional basketball player for another professional basketball team during the term of such Player Contract) that such Player Contract be renegotiated during the term thereof; provided, however, that this provision is not intended to prevent any player from entering into negotiations with a Team, in accordance with Article VII, with respect to the compensation to be paid to said player for the Season(s) following the last playing Season covered by any Player Contract, or renewal or extension thereof.

\hypertarget{players-threat-to-withhold-services.}{%
\section{Player's Threat to Withhold Services.}\label{players-threat-to-withhold-services.}}

The NBA and the Players Association agree that a player who publicly demands a renegotiation of his Player Contract, and who threatens to withhold the services he has agreed to render under such Player Contract or to perform at a level below his full capabilities unless such renegotiation takes place, shall be considered to have engaged in conduct impairing the faithful and thorough discharge of the duties incumbent upon the player within the meaning of paragraph 5 of the Uniform Player Contract.

\hypertarget{no-negotiations-with-other-teams-while-under-contract.}{%
\section{No Negotiations with Other Teams While Under Contract.}\label{no-negotiations-with-other-teams-while-under-contract.}}

Except as permitted in accordance with Article XI (Veteran Free Agent), no player who is a party to a Player Contract with a Team shall, during the term of such Contract (including any permissible option year), enter into negotiations with another Team.

\hypertarget{grievance-and-arbitration-procedure}{%
\chapter{GRIEVANCE AND ARBITRATION PROCEDURE}\label{grievance-and-arbitration-procedure}}

\hypertarget{scope.}{%
\section{Scope.}\label{scope.}}

\begin{enumerate}
\def\labelenumi{(\alph{enumi})}
\tightlist
\item
  Any dispute (such dispute hereinafter being referred to as a ``Grievance'') involving the interpretation or application of, or compliance with, the provisions of this Agreement or the provisions of a Player Contract (except as provided in paragraph 9 of a Uniform Player Contract), including a dispute concerning the validity of a Player Contract, shall be resolved exclusively by the Grievance Arbitrator in accordance with the procedures set forth in this Article; provided, however, that disputes arising under Articles VII, VIII, X, XI, XII, XIII, XIV, XV, XVI, XXXVI, XXXVIII, and XXXIX shall be determined by the System Arbitrator provided for in Article XXXII.
\item
  The Grievance Arbitrator shall also have jurisdiction over disputes involving player discipline, to the extent set forth in Section 8 below.
\end{enumerate}

\hypertarget{initiation.}{%
\section{Initiation.}\label{initiation.}}

\begin{enumerate}
\def\labelenumi{(\alph{enumi})}
\tightlist
\item
  Grievances may be initiated, as set forth below, by a player, a Team, the NBA, or the Players Association, except that the Players Association may not initiate a Grievance involving player discipline without the approval of the player(s) concerned.
\item
  No party may initiate a Grievance until and unless it has first discussed the matter with the party or parties against whom the Grievance is to be initiated in an attempt to settle it.
\item
  A Grievance must be initiated within twenty (20) days from the date of the occurrence upon which the Grievance is based, or within twenty (20) days from the date upon which the facts of the matter became known or reasonably should have become known to the party initiating the Grievance, whichever is later.
\item
  Subject to the provisions of Sections 2(a)-(c) above, (i) a player or the Players Association may initiate a Grievance by filing written notice thereof with a Team and furnishing a copy of such notice to the NBA; (ii) a Team may initiate a Grievance by filing written notice thereof with the Players Association and furnishing copies of such notice to the player(s) involved and to the NBA; and (iii) the NBA may initiate a Grievance by filing written notice thereof with the Players Association and furnishing copies of such notice to the player(s) and Team(s) involved.
\end{enumerate}

\hypertarget{hearings.}{%
\section{Hearings.}\label{hearings.}}

\begin{enumerate}
\def\labelenumi{(\alph{enumi})}
\tightlist
\item
  Within thirty (30) days of the date on which the Grievance Arbitrator is appointed, and on each successive anniversary date thereafter, the Arbitrator shall designate two (2) hearing dates per month (for the next twelve (12) months) for use by the parties to this Agreement. In addition, the Grievance Arbitrator shall be available for expedited hearings when necessary.
\item
  Either the NBA or the Players Association, upon at least thirty (30) days' written notice to the other side and to the Grievance Arbitrator, may arrange to have a hearing scheduled on one of the specific dates that the Arbitrator has reserved. If the other party is not prepared to go forward on that date, it may request a postponement of the hearing to a subsequent date certain, but no later than 30 days from the date on which the hearing was originally scheduled; postponement may be opposed by the party who originally scheduled the Grievance, in which case the request will be referred to the Arbitrator for a decision.
\item
  If a Grievance is set for hearing and the hearing date is then postponed by a party, the postponement fee (if any) of the Arbitrator will be borne by the postponing party unless that party objects and the Arbitrator finds that the postponement was for good cause. Should good cause be found, the parties will share any postponement costs equally.
\item
  No party may request or be granted more than two postponements of any previously-scheduled hearing. If, after requesting and receiving two hearing postponements, a party fails to attend a third scheduled hearing, the Grievance shall be resolved against that party.
\item
  If the Grieving party does not request a hearing that is scheduled to take place within one year of the filing of the Grievance, or, in the case of a postponement of the first hearing date, if the Grieving party does not request a second hearing that is scheduled to occur within two years of the filing of the Grievance, the Grievance shall be dismissed with prejudice. Any Grievance filed prior to the execution date of this Agreement shall be dismissed with prejudice unless the Grieving Party schedules a hearing to take place within one year of the execution of this Agreement. For purposes of computing time under this subsection, the time shall be tolled during any period when there is no Grievance Arbitrator or when the Grieving party has been unable to schedule a hearing (after making efforts to do so) because the Grievance Arbitrator is unavailable.
\item
  Hearings before the Grievance Arbitrator shall be held in New York (alternating between the NBA and Players Association offices). All such hearings shall be conducted in accordance with the Voluntary Labor Arbitration Rules of the American Arbitration Association.
\end{enumerate}

\hypertarget{procedure.}{%
\section{Procedure.}\label{procedure.}}

\begin{enumerate}
\def\labelenumi{(\alph{enumi})}
\tightlist
\item
  Not later than seven (7) days prior to the hearing, the parties shall submit to the Grievance Arbitrator a joint statement of the issue(s) in dispute. If the parties cannot agree on such a joint statement, they shall each submit separate statements setting forth the disputed issue(s).
\item
  Not later than three (3) business days prior to the hearing, the parties shall exchange witness lists, relevant documents, and citations of legal authorities that the parties intend to rely on Absent a showing of good cause, no party may proffer the testimony of a witness to the Grievance Arbitrator that has not been identified to the other side as required by this subsection.
\item
  The parties may agree to file pre-hearing or post-hearing briefs in any case.
\end{enumerate}

\hypertarget{arbitrators-decision-and-award.}{%
\section{Arbitrator's Decision and Award.}\label{arbitrators-decision-and-award.}}

\begin{enumerate}
\def\labelenumi{(\alph{enumi})}
\tightlist
\item
  Except as set forth in Section 10 below, the Grievance Arbitrator shall render an Award as soon as practicable, but in no event more than thirty (30) days following the conclusion of a Grievance hearing or the submission of post-hearing briefs where applicable. That Award may be accompanied by a written opinion, or the written opinion may follow within a reasonable time thereafter. The Award shall constitute full, final and complete disposition of the Grievance, and shall be binding upon the player(s) and Team(s) involved and the parties to this Agreement.
\item
  The Grievance Arbitrator shall have jurisdiction and authority only to: (i) interpret, apply, or determine compliance with the provisions of this Agreement; (ii) interpret, apply or determine compliance with the provisions of Player Contracts; (iii) determine the validity of Player Contracts pursuant to Section 1 of this Article; (iv) award damages in connection with a proceeding provided for in Section 11 below; (v) award declaratory relief in connection with a proceeding initiated by a Team to determine whether such Team may properly terminate a Player Contract pursuant to paragraphs 16(a)(i) or 16(a)(iii) of such Contract, and what, if any, liability such Team would incur as a result of such termination; and (vi) resolve disputes arising under Article XXII, Section 5 and Article XXVI of this Agreement and the NBA/NBPA Anti-Drug Agreement in the manner set forth therein. The Grievance Arbitrator shall not have jurisdiction or authority to add to, detract from, or alter in any way the provisions of this Agreement or any Player Contract.
\item
  In any Grievance that involves an action taken by the Commissioner (or his designee) concerning (i) the preservation of the integrity of, or the maintenance of public confidence in, the game of basketball, and (ii) a fine and/or suspension that results in a financial impact to the player of more than \$25,000, the Grievance Arbitrator shall apply an ``arbitrary and capricious'' standard of review.
\end{enumerate}

\hypertarget{grievance-arbitrator.}{%
\section{Grievance Arbitrator.}\label{grievance-arbitrator.}}

The parties to this Agreement shall agree upon the appointment of a new Grievance Arbitrator, who shall serve for the duration of this Agreement; provided, however, that as of September 1, 1997, and as of each successive September 1, either of the parties to this Agreement may discharge the Grievance Arbitrator by serving thirty (30) days' prior written notice upon him and upon the other party to this Agreement. The parties shall thereupon either agree upon a successor Grievance Arbitrator or select a successor from an American Arbitration Association list of prominent professional arbitrators, alternately striking names from such list until only one remains. The Grievance Arbitrator so discharged shall continue to serve until his successor is agreed upon or selected.

\hypertarget{injury-grievances.}{%
\section{Injury Grievances.}\label{injury-grievances.}}

\begin{enumerate}
\def\labelenumi{(\alph{enumi})}
\tightlist
\item
  Disputes arising under paragraphs 7, 16(b), or 16(c) of a Uniform Player Contract as to (i) whether a player was in sufficiently good condition to play skilled basketball, (ii) whether the player was injured as a direct result of participating in any basketball practice or game played for the Team, and/or (iii) whether such injury disabled the player and/or rendered him unfit to play skilled basketball, shall be processed and determined in the same manner as a Grievance under Sections 2-6 of this Article XXXI, except that if a party to such Grievance so elects, a physician designated by the President of the American College of Orthopedic Surgeons (or such other similar organization as the parties agree may be most appropriate to the issues in dispute) and who has no relationship with any party covered by this Agreement shall conduct a physical examination of the player and shall perform the functions of the Grievance Arbitrator. The physician so designated shall render a written decision which shall constitute full, final and complete disposition of the dispute, and shall be binding upon the player(s) and Team(s) involved and the parties to this Agreement. Any fees or costs associated with the physician's determination will be borne equally by both sides.
\item
  All other disputes arising under paragraphs 7, 16(b), or 16(c) of a Uniform Player Contract (including, but not limited to, a dispute as to whether the suspension of a player or the termination of a Player Contract was by reason of a disability resulting from a re-injury to or aggravation of a previous injury or preexisting condition) shall not be subject to the special procedure set forth in Section 7(a) above, but rather shall be processed and determined in the same manner as any other Grievance under Sections 2-6 of this Article.
\end{enumerate}

\hypertarget{special-procedures-with-respect-to-player-discipline.}{%
\section{Special Procedures with Respect to Player Discipline.}\label{special-procedures-with-respect-to-player-discipline.}}

\begin{enumerate}
\def\labelenumi{(\alph{enumi})}
\tightlist
\item
  Any dispute involving (i) a fine or suspension imposed upon a player by the Commissioner (or his designee) for conduct on the playing court, or (ii) action taken by the Commissioner (or his designee) concerning the preservation of the integrity of, or the maintenance of public confidence in, the game of basketball resulting in a financial impact to the player of \$25,000 or less, shall be processed exclusively as follows:

  \begin{enumerate}
  \def\labelenumii{(\roman{enumii})}
  \tightlist
  \item
    Within twenty (20) days following written notification of the action taken by the Commissioner (or his designee), a player affected thereby or the Players Association may appeal in writing to the Commissioner.
  \item
    The Commissioner shall designate a time and place for hearing, which shall be commenced within 10 days following his receipt of the notice of appeal.
  \item
    As soon as practicable following the conclusion of such hearing, the Commissioner shall render a written decision, which decision shall constitute full, final and complete disposition of the dispute, and shall be binding upon the Player(s) and Club(s) involved and the parties to this Agreement.
  \item
    In the event such appeal involves a fine or suspension imposed by the Commissioner's designee, the Commissioner, as a consequence of such appeal and hearing, shall have authority only to affirm or reduce such fine or suspension, and shall not have authority to increase such fine or suspension.
  \end{enumerate}
\item
  In the event a matter filed as a Grievance in accordance with the provisions of Section 2(d) gives rise to issues involving the integrity of, or public confidence in, the game of basketball, and the financial impact to the player of the action being grieved is \$25,000 or less, the Commissioner may, at any stage of its processing, order that the matter be withdrawn from such processing and thereafter be processed in accordance with the procedure provided in Section 8(a).
\end{enumerate}

\hypertarget{escrow-procedure.}{%
\section{Escrow Procedure.}\label{escrow-procedure.}}

In the event that a Grievance challenging a Commissioner or Team-imposed fine and/or suspension is filed in accordance with this Article, the amount of any fine or salary lost by virtue of the suspension shall be deposited with an Escrow Agent selected by the NBA and subsequently disbursed in accordance with the provisions of the NBA-NBPA Grievance Escrow Agreement, which is attached and made part of this Agreement as Exhibit F.

\hypertarget{disputes-with-respect-to-the-terms-of-a-player-contract.}{%
\section{Disputes with Respect to the Terms of a Player Contract.}\label{disputes-with-respect-to-the-terms-of-a-player-contract.}}

\begin{enumerate}
\def\labelenumi{(\alph{enumi})}
\tightlist
\item
  If either the NBA or the Players Association asserts that a term or provision of a Player Contract is not permitted by this Agreement, either may have the dispute involving such Contract term or provision resolved by initiating a Grievance. If such a Grievance is initiated by the NBA, the 20-day time period referred to in Section 2( c) of this Article XXXI shall commence with the date upon which the NBA received the Player Contract (or amendment thereto) containing the disputed term or provision. If such a Grievance is initiated by the Players Association, the 20-day time period referred to in Section 2( c) of this Article XXXI shall commence with the date upon which the Player Contract (or amendment thereto) containing the disputed term or provision was first made available for inspection by the Players Association.
\item
  If, as a result of the Grievance and Arbitration procedure, a Player Contract is found to contain a term or provision that is not permitted by this Agreement, then (i) such term or provision shall be deleted from the Player Contract and have no force or effect, and the Player Contract shall in all other respects remain valid and binding upon the parties thereto, and (ii) if the Team and the Player agree to reform or revise the Player Contract within 30 days of the Grievance Arbitrator's decision, such reformation or revision shall be exempted from the rules governing Renegotiations contained in Article VII, Section 7(c).
\item
  Nothing set forth above shall affect in any manner the Commissioner's authority with respect to the approval or disapproval of Player Contracts pursuant to paragraph 11 of the Uniform Player Contract; and the fact that the Commissioner has approved or not disapproved a Player Contract containing a term or provision not permitted by this Agreement shall not be referred to in the course of the Grievance and Arbitration procedure and shall not be considered in any manner or for any purpose by the Grievance Arbitrator in connection with a dispute concerning that Player Contract.
\end{enumerate}

\hypertarget{disputes-with-respect-to-players-under-contract-who-withhold-playing-services.}{%
\section{Disputes with Respect to Players Under Contract Who Withhold Playing Services.}\label{disputes-with-respect-to-players-under-contract-who-withhold-playing-services.}}

In addition to any other rights a Member may have under contract or law, including those under paragraph 9 of a Uniform Player Contract, a Member may recover damages in a proceeding before the Grievance Arbitrator when a player who is party to a currently effective Player Contract fails or refuses to render the services called for under the Player Contract. In any such proceeding, where the Grievance Arbitrator determines that damages are continuing to accrue at the time of the hearing, the Arbitrator shall award such damages (if any) as the Member has by then sustained, and the hearing shall remain open to enable the submission of proof on the issue of continuing damages.

\hypertarget{expedited-procedure.}{%
\section{Expedited Procedure.}\label{expedited-procedure.}}

\begin{enumerate}
\def\labelenumi{(\alph{enumi})}
\tightlist
\item
  Subject to Article XXXII, Section 1, notwithstanding the foregoing, in the event of a dispute arising under Article XVII, Article XXX, or Article XXXI, Section 11 of this Agreement, or under paragraph 15 of a Uniform Player Contract (but only insofar as such paragraph provides), or in the event of an alleged breach by a player of paragraph 9 of a Uniform Player Contract, the NBA, or the Players Association, or a Team may request that such dispute or alleged breach be referred immediately to the Grievance Arbitrator. In any such case, the dispute or alleged breach shall be asserted by notice in writing or by facsimile given to the other party or parties, the NBA, and the Grievance Arbitrator.
\item
  In addition, any disputes or questions that under Articles VII, X, and XI are to be arbitrated pursuant to the Expedited Procedure shall, except with respect to notice, be arbitrated in the manner set forth in this Section 12.
\item
  The Grievance Arbitrator shall convene a hearing with respect to such dispute or alleged breach at the earliest possible time, but in no event later than 24 hours following his receipt of such notice. If the Grievance Arbitrator is not immediately available and the parties are unable to agree upon another grievance arbitrator, the American Arbitration Association shall appoint such other grievance arbitrator.
\item
  The award, which shall be issued not later than 24 hours after the conclusion of the hearing, shall be in writing and may be issued with or without opinion. If any party desires an opinion, one shall be issued but its issuance shall not delay compliance with the enforcement of the award. The award shall constitute full, final and complete disposition of the dispute or alleged breach, and shall be binding upon the player(s) and Team(s) involved and the parties to this Agreement.
\item
  The failure of any party to attend the hearing as scheduled shall not delay the hearing, and the Grievance Arbitrator is authorized to proceed to take evidence and issue an award as though such party were present.
\end{enumerate}

\hypertarget{threshold-amounts-for-certain-grievances.}{%
\section{Threshold Amounts for Certain Grievances.}\label{threshold-amounts-for-certain-grievances.}}

A fine or suspension imposed by a Team shall be appealable to the Grievance Arbitrator only if it results in a financial impact on the player of more than \$2,000. A fine or suspension imposed by the Commissioner shall be appealable to the Grievance Arbitrator only if it results in a financial impact on the player of more than \$10,000.

\hypertarget{miscellaneous.}{%
\section{Miscellaneous.}\label{miscellaneous.}}

\begin{enumerate}
\def\labelenumi{(\alph{enumi})}
\tightlist
\item
  Each of the time limits set forth herein may be extended by mutual agreement of the parties involved.
\item
  In any meeting or hearing provided for herein, a player may be accompanied by a representative of the Players Association who may participate in such meeting or hearing and represent the player. In any such meeting or hearing, the NBA and any other party may attend and be accompanied by a representative who may participate in such meeting or hearing and represent the NBA and any such party.
\item
  The parties recognize that a player may be subjected to disciplinary action for just cause by his Team or by the Commissioner (or his designee). Therefore, in Grievances regarding discipline, the issue to be resolved shall be whether there has been just cause for the penalty imposed.
\item
  Nothing contained herein shall excuse a player from prompt compliance with any discipline imposed upon him. If discipline imposed upon a player is determined to be improper by a final disposition under this Article XXXI, the player shall promptly be made whole.
\item
  Nothing contained in this Article XXXI shall be deemed to limit or impair the right of the NBA or any Team to impose discipline upon a player(s) or to take any other action not inconsistent with the provisions of a Player Contract or this Agreement.
\item
  Subject to Section 3(c) above, all costs of arbitration, including the fees and expenses of the Grievance Arbitrator, shall be borne equally by the parties thereto; but each party shall bear the cost of its own witnesses, counsel, and the like.
\item
  A Team shall not be required to terminate a Player Contract under the NBA waiver procedure as a condition precedent to the filing of a Grievance with respect to such Player Contract. To the extent that the decision of the Impartial Arbitrator in In re Otis Birdsong, Dec.~No.~87-2, May 14, 1987, is inconsistent with the foregoing, it is hereby overruled.
\item
  In a proceeding involving the interpretation of a Player Contract, no Player Contract other than the Player Contract that is the subject of dispute shall be admissible in evidence.
\end{enumerate}

\hypertarget{system-arbitrator}{%
\chapter{SYSTEM ARBITRATOR}\label{system-arbitrator}}

\hypertarget{jurisdiction-and-authority.}{%
\section{Jurisdiction and Authority.}\label{jurisdiction-and-authority.}}

The NBA and the Players Association shall agree upon a System Arbitrator, who shall have exclusive jurisdiction to determine any and all disputes arising under Articles VII, VIII, X, XI, XII, XIII, XIV, XV, XVI, XXXVI, XXXVIII, and XXXIX of this Agreement; provided, however, that any claim or dispute arising under such Articles involving the NBA, any Team, the Players Association or any player, which is specifically under the terms of this Agreement to be determined in accordance with the Expedited Procedures described in Article XXXI, Section 12 above, shall be determined by the Grievance Arbitrator provided for in Article XXXI. The System Arbitrator shall hold hearings on alleged violations of the foregoing Articles, subject to review by the Appeals Panel, in the manner set forth below.

\begin{enumerate}
\def\labelenumi{(\alph{enumi})}
\tightlist
\item
  The System Arbitrator shall make findings of fact and award appropriate relief including, without limitation, damages and specific performance.
\item
  The System Arbitrator shall have authority to order the production of documents, the conduct of pre-hearing depositions, and attendance of witnesses at the hearing with respect to any party to this Agreement, and/or any player or Team. The System Arbitrator shall have the authority to compel the attendance of witnesses and the production of documents at any hearing within the jurisdiction of the System Arbitrator in accordance with the New York C.P.L.R.
\item
  Rulings of the System Arbitrator shall upon their issuance constitute full, final and complete disposition of the dispute, shall be binding upon the parties to this Agreement and upon any player(s) or Team(s) involved, and shall be followed by them unless a notice of appeal is served by the appealing party upon the responding party and filed with the System Arbitrator within 10 days of the date of the decision of the System Arbitrator appealed from. If and when a decision of the System Arbitrator is reversed or modified by the Appeals Panel, the effect of such reversals or modification shall be deemed by the parties to be retroactive to the time of issuance of the ruling of the System Arbitrator. The parties may seek appropriate relief to effectuate and enforce this provision.
\item
  The System Arbitrator shall not have jurisdiction or authority to add to, detract from, or alter in any way the provisions of this Agreement or any Player Contract.
\end{enumerate}

\hypertarget{costs-relating-to-system-arbitration.}{%
\section{Costs Relating to System Arbitration.}\label{costs-relating-to-system-arbitration.}}

The compensation of the System Arbitrator and the costs and expenses incurred in connection with any proceeding brought before the System Arbitrator shall be borne equally by the parties to this Agreement; provided, however, that each party to such proceeding shall bear its own attorneys' fees and litigation costs.

\hypertarget{procedure-for-system-arbitration.}{%
\section{Procedure for System Arbitration.}\label{procedure-for-system-arbitration.}}

All matters before the System Arbitrator shall be heard and determined in an expedited manner. An enforcement proceeding may be commenced upon 72 hours' written notice (or upon shorter notice if ordered by the System Arbitrator) served upon the party against whom the enforcement proceeding is brought and filed with the System Arbitrator. All such notices and all orders and notices issued and directed by the System Arbitrator shall be served on the NBA, counsel for the NBA, the Players Association, and counsel for the Players Association, in addition to any counsel appearing for individual NBA players or individual NBA Teams. The Players Association and the NBA shall have the right to participate in all such enforcement proceedings, and the Players Association may appear in any enforcement proceedings on behalf of any NBA player.

\hypertarget{selection-of-system-arbitrator.}{%
\section{Selection of System Arbitrator.}\label{selection-of-system-arbitrator.}}

In the event that the Players Association and the NBA cannot agree on the identity of a System Arbitrator, the parties agree to submit the issue to the Center for Public Resources (or such other organization as the parties may agree) which shall submit to the parties a list of eleven (11) attorneys (none of whom shall have nor whose firm shall have represented within the past five years players, player agents, labor organizations representing athletes, sports leagues or governing bodies, sports teams, team affiliates, or owners in any professional sport). If the parties cannot within seven (7) days of receipt of such list agree to the identity of the System Arbitrator from among the names on such list, they shall return said list, with up to five names deleted therefrom by each party, to the Center for Public Resources (or such other organization as the parties may agree), which shall choose from the remaining names on the list the identity of the System Arbitrator. The first System Arbitrator selected under the provisions of this Agreement shall serve for a two-year term. Thereafter the System Arbitrator shall serve for two-year renewal terms unless notice of termination is given either by the NBA or by the Players Association. Notice of termination of the System Arbitrator shall be given to the other party, and to the System Arbitrator at least forty-five (45) days preceding the end of any two-year term. Following the giving of such notice, a new System Arbitrator shall be selected in accordance with the procedures set forth in this paragraph. The System Arbitrator whose term has ended shall continue to hear all disputes filed prior to the date of the appointment of a new System Arbitrator.

\hypertarget{selection-of-appeals-panel.}{%
\section{Selection of Appeals Panel.}\label{selection-of-appeals-panel.}}

There shall be a three member Appeals Panel for each appeal noticed from a decision of the System Arbitrator. To select such a panel, the parties will ask the Center for Public Resources (or such other organization as the parties may agree) to submit to the parties a list of fifteen (15) attorneys (none of whom shall have nor whose firm shall have represented within the past five years players, player agents, labor organizations representing athletes, sports leagues or governing bodies, sports teams, team affiliates or owners in any professional sport). If the parties cannot within 7 days of receipt of such list agree to the identity of the Appeals Panel from among the names on such list, they shall meet and alternate striking one name at a time from the list until three names on the list remain. The three remaining names on the list shall comprise the Appeals Panel for that particular appeal. The compensation of the members of the Appeals Panel and the costs of proceedings before the Appeals Panel shall be borne equally by the parties to this Agreement; provided, however, that each party shall bear its own attorneys' fees and litigation costs.

\hypertarget{procedure-relating-to-appeals-of-determination-by-the-system-arbitrator.}{%
\section{Procedure Relating to Appeals of Determination by the System Arbitrator.}\label{procedure-relating-to-appeals-of-determination-by-the-system-arbitrator.}}

A party seeking to appeal a determination of the System Arbitrator must serve on the other party and file with the System Arbitrator a notice of appeal, within ten (10) days of the date of the determination of the System Arbitrator appealed from. The timely service and filing of a notice of appeal shall automatically stay the determination of the System Arbitrator pending resolution by the Appeals Panel. The party seeking the appeal must serve on the opposing party and file with the Appeals Panel its brief in support thereof within twenty-five (25) days of the date of the System Arbitrator's decision from which the appeal is taken. The responding party shall serve the appealing party and file with the Appeals Panel its responding brief within fifty (50) days of the date of the decision of the System Arbitrator from which the appeal is taken. The Appeals Panel shall schedule oral argument on the appeal within sixty (60) days of the date of the decision of the System Arbitrator from which the appeal has been taken and shall issue its decision within 90 days from the date of the System Arbitrator's decision.The Appeals Panel shall review the findings of fact and conclusions of law made by the System Arbitrator using the standards of review employed by the U.S. Court of Appeals for the Second Circuit. The decision of the Appeals Panel shall constitute full, final, and complete disposition of the dispute, and shall be binding upon the parties to this Agreement and upon any player(s) or Team(s) involved.

\hypertarget{recognition-clause}{%
\chapter{RECOGNITION CLAUSE}\label{recognition-clause}}

The NBA recognizes the Players Association as the exclusive collective bargaining representative of persons who are employed by NBA Members as professional basketball players (and/or who may become so employed during the term of this Agreement or any extension thereof); and the Players Association warrants that it is duly empowered to enter into this Agreement for and on behalf of such persons. The NBA and the Players Association agree that, notwithstanding the foregoing, such persons and NBA Members may, on an individual basis, bargain with respect to and agree upon the provisions of Player Contracts, but only as and to the extent permitted by this Agreement.

\hypertarget{savings-clause}{%
\chapter{SAVINGS CLAUSE}\label{savings-clause}}

In the event that any provision hereof is found to be inconsistent with the Internal Revenue Code (or the rules and regulations issued thereunder), the National Labor Relations Act, any other federal, state, provincial, or local statute or ordinance, or the rules and regulations of any other government agency, or is determined to have an adverse effect upon the right of the NBA (or any successor entity) to a tax exemption under Section 501(c)(6) of the Internal Revenue Code of 1954 (or any successor section of like import), then the parties hereto agree to make such changes as are necessary to avoid such inconsistency or to obtain or maintain such exemption retaining, to the extent possible, the intention of such provision.

\hypertarget{player-agents}{%
\chapter{PLAYER AGENTS}\label{player-agents}}

\hypertarget{approval-of-player-contracts.}{%
\section{Approval of Player Contracts.}\label{approval-of-player-contracts.}}

The NBA shall not approve any Player Contract between a player and a Team unless such player: (i) is represented in the negotiations with respect to such Player Contract by an agent or representative duly certified by the Players Association in accordance with the Players Association's Agent Regulation Program and authorized to represent him; or (ii) acts on his own behalf in negotiating such Player Contract.

\hypertarget{fines.}{%
\section{Fines.}\label{fines.}}

The NBA shall impose a fine of \$10,000 upon any Team that negotiates a Player Contract with an agent or representative not certified by the Players Association in accordance with the Players Association's Agent Regulation Program if, at the time of such negotiations, such Team either (i) knows that such agent or representative has not been so certified or (ii) fails to make reasonable inquiry of the NBA as to whether such agent or representative has been so certified. Notwithstanding the preceding sentence, in no event shall any Team be subject to a fine if the Team negotiates a Player Contract with the agent designated as the player's authorized agent on the then-current agent list provided by the Players Association to the NBA in accordance with Section 4 below.

\hypertarget{indemnity.}{%
\section{Indemnity.}\label{indemnity.}}

The Players Association agrees to indemnify and hold harmless the NBA and each of its Teams and their successors, agents, attorneys, heirs and executors from any and all claims of any kind arising from or relating to (i) the Players Association 's Agent Regulation Program, and (ii) the provisions of this Article, including, without limitation, any judgments, costs and settlements, provided that the Players Association is immediately notified of such claim in writing (and, in no event later than five days from the receipt thereof), is given the opportunity to assume the defense thereof, and the NBA and/or its Teams (whichever is sued) use their best efforts to defend such claim, and do not admit liability with respect to and do not settle such claim without the prior written consent of the Players Association.

\hypertarget{agent-lists.}{%
\section{Agent Lists.}\label{agent-lists.}}

The Players Association agrees to provide the NBA League Office with a list of (i) all agents certified under the Players Association's Agent Regulation Program, and (ii) the players represented by each such agent. Such list shall be updated once every two weeks from the day after the NBA Finals to the first day of the next succeeding Regular Season and shall be updated once every month at all other times.

\hypertarget{confirmation-by-the-players-association.}{%
\section{Confirmation by the Players Association.}\label{confirmation-by-the-players-association.}}

If the NBA has reason to believe that the agent representing a player in Contract negotiations is not a certified agent or is not the agent authorized to represent the player, the NBA may, at its election, request in writing from the Players Association confirmation as to whether the agent that represented the player in the Contract negotiations is in fact the player's certified representative. If within three business days of the date the Players Association receives such written request, the NBA does not receive a written response from the Players Association stating that the agent that represented the player is not the player's certified representative, then the NBA shall be free to act as if the agent was certified.

\hypertarget{group-licensing-rights}{%
\chapter{GROUP LICENSING RIGHTS}\label{group-licensing-rights}}

\hypertarget{rights-granted.}{%
\section{Rights Granted.}\label{rights-granted.}}

The Players Association, on behalf of present and future NBA players, agrees that NBA Properties, Inc., has the exclusive right to use the ``Player's Attributes'' of each NBA player as such term is defined and for such group licensing purposes as are set forth in the Agreement between NBA Properties, Inc., and the National Basketball Players Association, dated as of September 18, 1995 (the ``Group License Agreement'').

\hypertarget{player-appearances.}{%
\section{Player Appearances.}\label{player-appearances.}}

A player may, during each Contract Year covered by a Player Contract to which he is a party, be required (a) to make up to four appearances at the request of and in connection with licensing arrangements made by NBA Properties, Inc., in accordance with the terms of the Group License Agreement, and (b) to make up to two additional appearances at the request of NBA Properties in accordance with paragraph 13(e) of a Uniform Player Contract and Article II, Section 7(b). When a player makes an appearance in accordance with Article II, Section 7(b), he shall be paid at least \$1,000. When a player fails to appear or reasonably to cooperate during an appearance at any of the licensing appearances referred to in this Section, he may be fined for each failure in an amount up to \$1,000.

\hypertarget{uniform.}{%
\section{Uniform.}\label{uniform.}}

During any NBA game or practice, including warm-up periods and going to and from the locker room to the playing floor, a player shall wear only the Uniform as supplied by his Team. For purposes of the preceding sentence only, ``Uniform'' means all clothing and other items (such as kneepads, wristbands and headbands, but not including Sneakers) worn by a player during an NBA game or practice. ``Sneakers'' means athletic shoes of the type worn by players while playing an NBA game.

\hypertarget{integration-entire-agreement-choice-of-law}{%
\chapter{INTEGRATION, ENTIRE AGREEMENT, CHOICE OF LAW}\label{integration-entire-agreement-choice-of-law}}

\chaptermark{INTEGRATION, ENTIRE AGREEMENT \ldots}

\hypertarget{integration-entire-agreement.}{%
\section{Integration, Entire Agreement.}\label{integration-entire-agreement.}}

This Agreement, together with the exhibits hereto, constitutes the entire understanding between the parties and all understandings, conversations and communications, proposals, and counter proposals, oral and written (including any draft of this Agreement) between the Members of the NBA and the Players Association, or on behalf of them, are merged into and superseded by this Agreement and shall be of no force or effect, except as expressly provided herein. No such understandings, conversations, communications, proposals, counter proposals or drafts shall be referred to in any proceeding by the parties. Further, no understanding contained in this Agreement shall be modified, altered or amended, except by a writing signed by the party against whom enforcement is sought.

\hypertarget{choice-of-law.}{%
\section{Choice of Law.}\label{choice-of-law.}}

This Agreement is made under and shall be governed by the internal law of the State of New York, except where federal law may govern.

\hypertarget{term-of-agreement}{%
\chapter{TERM OF AGREEMENT}\label{term-of-agreement}}

\hypertarget{expiration-date.}{%
\section{Expiration Date.}\label{expiration-date.}}

Except as provided in Section 2 of this Article XXXVIII and otherwise expressly provided herein, this Agreement shall be effective from September 18, 1995 and shall continue in full force and effect through June 30, 2001.

\hypertarget{termination-by-nba.}{%
\section{Termination by NBA.}\label{termination-by-nba.}}

\begin{enumerate}
\def\labelenumi{(\alph{enumi})}
\tightlist
\item
  If with respect to the 1997-98 NBA season, or any Season thereafter during the term of this Agreement, it is determined (as provided for in Section 2(b) of this Article XXXVIII) that the NBA has paid or will be obligated to pay Total Salaries and Benefits in an amount that exceeds 51.8\% of Projected BRI for such Season (i.e., the BRI previously projected for purposes of calculating the Salary Cap for such Season), the NBA shall have the right to terminate this Agreement (effective as of the June 30 following such Season) by giving written notice of such termination to the Players Association. For the purposes of this Section 2, Projected BRI shall include the amounts determined in accordance with Article VII, Sections l(a)(6)(i) and (ii) and, for the 1997-98 Salary Cap Year, an additional \$36 million, and shall not include (for any Salary Cap Year) the amounts determined in accordance with Article VII, Section 1(a)(6)(iii).
\item
  The determination provided for by Section 2(a) shall be made as follows:

  \begin{enumerate}
  \def\labelenumii{(\roman{enumii})}
  \tightlist
  \item
    On or before April 1 of the 1997-98 NBA season, and on or before April 1 of any Season that may thereafter be covered by this Agreement, the NBA will provide the Accountants with the best data then available as to the amounts the NBA has paid or estimates it will be obligated to pay in Salaries and Benefits with respect to such Season.
  \item
    In accordance with such procedures as it determines are reasonable under the circumstances, the Accountants shall review the data provided by the NBA for the purpose of determining whether the NBA has paid or will be obligated to pay Total Salaries and Benefits in an amount that exceeds 51.8\% of Projected BRI for the Season in question; and the NBA, and NBA Teams shall cooperate in all reasonable respects with the Accountants' conduct of such audit.
  \item
    At the earliest possible time, but in no event later than the April 30 immediately following its receipt of the information provided by the NBA, the Accountants shall notify the NBA and the Players Association in writing whether or not the NBA has paid or will be obligated to pay Total Salaries and Benefits in an amount that exceeds 51.8\% of Projected BRI for the Season in question and shall provide the NBA and the Players Association with the information and calculations on which that determination is based.
  \item
    The NBA shall have the right to terminate this Agreement by giving the notice provided for by Section 2(a):
    (A) If the Accountants determine that the NBA has paid or will pay Total Salaries and Benefits in an amount that exceeds 51.8\% of Projected BRI for the Season in question; or
    (B) If the Accountants fail to furnish the notice provided for by Section 2(b )(iii) by the date provided for by that Section and the NBA reasonably believes that it has paid or will be obligated to pay Total Salaries and Benefits in an amount that exceeds 51. 8 \% of Projected BRI for the Season in question.
  \end{enumerate}
\item
  If the Players Association elects to dispute the determination made by the Accountants or the NBA that the NBA has paid or will be obligated to pay Total Salaries and Benefits in an amount that exceeds 51.8 \% of Projected BRI for the Season in question, it shall so notify the NBA in writing within two business days from the date upon which the NBA has given the notice provided for by Section 2(a).
\item
  Any and all disputes concerning the determination made by the Accountants or the NBA shall be resolved exclusively by the System Arbitrator (subject to review by the Appeals Panel) in the manner provided for an enforcement proceeding pursuant to Article XXXII.
\item
  If the System Arbitrator or the Appeals Panel, as the case may be, concludes that the NBA has not paid or will not be obligated to pay Total Salaries and Benefits in an amount that exceeds 51.8\% of Projected BRI for the Season in question, the only relief to which the Players Association shall be entitled shall be (i) a declaration that the NBA did not have the right to terminate this Agreement and/or that its termination thereof was not effective and (ii) an order directing the NBA to rescind its notice of termination and resume performance pursuant to this Agreement. In no event shall the System Arbitrator or the Appeals Panel, in a proceeding brought pursuant to this Article, have the jurisdiction or authority to issue any other form of declaratory or equitable or injunctive relief or to award damages of any kind.
\end{enumerate}

\hypertarget{termination-by-players-association.}{%
\section{Termination by Players Association.}\label{termination-by-players-association.}}

In the event the conditions of Article XIV, Section 16 are satisfied, the Players Association shall have the right to terminate this Agreement. To execute such a termination, the Players Association may serve upon the NBA written notice of termination within thirty (30) days after the System Arbitrator's report finding the requisite conditions (pursuant to Article XIV, Section 16) becomes final and any appeals therefrom have been exhausted. In the absence of a System Arbitrator, the Players Association shall have the option to execute such a termination by serving upon the NBA written notice of such termination within thirty (30) days after any decision by a court finding the requisite conditions (pursuant to Article XIV, Section 16). In the latter situation, if the finding of the court is reversed on appeal, the Agreement shall be immediately reinstated and both parties reserve their rights with respect to any conduct by the other party during the period from the termination notice to the date upon which the Agreement was reinstated.

\hypertarget{mutual-right-of-termination.}{%
\section{Mutual Right of Termination.}\label{mutual-right-of-termination.}}

If at any time during the term of this Agreement any provision contained in Article VII, X, XI and XIV of this Agreement is enjoined, vacated, declared null and void or is rendered unenforceable by any court of competent jurisdiction, then either the NBA or the Players Association shall have the right to terminate this Agreement by serving upon the other party written notice of termination within thirty (30) days.

\hypertarget{no-waiver.}{%
\section{No Waiver.}\label{no-waiver.}}

The failure of the NBA or the Players Association to exercise its right to terminate this Agreement with respect to any playing season in accordance with this Article shall not be deemed a waiver of or in any way impair or prejudice the NBA or the Players Association's right, if any, to terminate this Agreement in accordance with this Article with respect to any succeeding season.

\hypertarget{expansion}{%
\chapter{EXPANSION}\label{expansion}}

\hypertarget{veteran-allocation.}{%
\section{Veteran Allocation.}\label{veteran-allocation.}}

The NBA may determine during the term of this Agreement to expand the number of Teams (each such Team a ``New Expansion Team'') and to have existing Teams make available for assignment to any such New Expansion Teams the Player Contracts of a certain number of Veterans under the same terms and in the same manner that Player Contracts were made available to the Expansion Teams.

\hypertarget{phase-in-of-new-expansion-teams.}{%
\section{Phase-In of New Expansion Teams.}\label{phase-in-of-new-expansion-teams.}}

Notwithstanding anything to the contrary set forth in this Agreement, the provisions of this Agreement that govern an Expansion Team during the Expansion Team's first two Seasons shall govern any New Expansion Team during the New Expansion Team's first two Seasons. Without limiting the generality of the preceding sentence, the parties agree that (a) the provisions of Article VII, Section 1(a)(1), (7), and (8) and Article VII, Section 2(a)(1) and (2) that relate to the BRI of an Expansion Team during its first two Seasons shall also apply to the BRI of any New Expansion Team during its first two Seasons, (b) any New Expansion Team shall not be included in the determination of Average Player Salary for the New Expansion Team's first Two Seasons but, thereafter, an amount equal to the Team Salary for the New Expansion Team shall be added to the numerator, and 12.5 for each New Expansion Team shall be added to the denominator of the calculation of Average Player Salary for a Season, and (c) the Salary Cap and Minimum Team Salary of a New Expansion Team during its first two Seasons shall be as follows:

\begin{enumerate}
\def\labelenumi{(\alph{enumi})}
\tightlist
\item
  New Expansion Team's first Season: (i) the Salary Cap shall equal 66 and 2/3\% of the Salary Cap for that Season determined in accordance with Article VII, Section 2(a); and (ii) the Minimum Team Salary shall equal 75\% of the amount of the New Expansion Team's Salary Cap as determined in accordance with the preceding clause; and
\item
  New Expansion Team's second Season: (i) the Salary Cap shall equal 75\% of the Salary Cap for that Season determined in accordance with Article VII, Section 2(a); and (ii) the Minimum Team Salary shall equal 75\% of the amount of the New Expansion Team's Salary Cap as determined in accordance with the preceding clause.
\end{enumerate}

\hypertarget{other}{%
\chapter{OTHER}\label{other}}

\hypertarget{headings-and-organization.}{%
\section{Headings and Organization.}\label{headings-and-organization.}}

The headings and organization of this Agreement are solely for the convenience of the parties, and shall not be deemed part of, or considered in construing or interpreting, this Agreement.

\hypertarget{time-periods.}{%
\section{Time Periods.}\label{time-periods.}}

Unless specifically stated otherwise, the specification of any time period in this Agreement shall include any non-business days within such period, except that any deadline falling on a Saturday, Sunday, or Federal Holiday shall be deemed to fall on the following business day.

\hypertarget{exhibits.}{%
\section{Exhibits.}\label{exhibits.}}

All of the Exhibits hereto are an integral part of this Agreement and of the agreement of the parties thereto.

NATIONAL BASKETBALL ASSOCIATION\\
By: /s/ David J. Stern\\
David J. Stern\\
Commissioner

NATIONAL BASKETBALL PLAYERS ASSOCIATION\\
By: /s/ Charles ``Buck'' Williams\\
Charles ``Buck'' Williams\\
President

\hypertarget{appendix-appendix}{%
\appendix}


\hypertarget{national-basketball-association-uniform-player-contract}{%
\chapter{NATIONAL BASKETBALL ASSOCIATION UNIFORM PLAYER CONTRACT}\label{national-basketball-association-uniform-player-contract}}

\hypertarget{section}{%
\section{}\label{section}}

THIS AGREEMENT made this \_\_\_ day of \_\_\_\_\_\_\_\_ 19 \_ by and between (hereinafter called the ``Team''), a member of the National Basketball Association (hereinafter called the ``NBA'' or ``League'') and \_\_\_\_\_\_\_\_\_\_\_ whose address is shown below (hereinafter called the ``Player'').

WITNESSETH:

In consideration of the mutual promises hereinafter contained, the parties hereto promise and agree as follows:

\begin{enumerate}
\def\labelenumi{\arabic{enumi}.}
\tightlist
\item
  \textbf{TERM.} The Team hereby employs the Player as a skilled basketball player for a term of \_\_ year(s) from the 1st day of September 19 \_\_.
\item
  \textbf{SERVICES.}

  \begin{enumerate}
  \def\labelenumii{(\alph{enumii})}
  \tightlist
  \item
    The services to be rendered by the Player pursuant to this Contract shall include: (i) attendance at any training camp, (ii) attendance at any practices and meetings conducted by the Team during any Regular Season or Playoffs, (iii) playing the games scheduled for the Team during any Regular Season, (iv) playing all Exhibition games scheduled by the Team or the League during and prior to any Regular Season, (v) playing (if invited to participate) in any of the NBA's All-Star Games (including the Rookie Game) and attending every event conducted in association with such All-Star Games (including, but not limited to, a reasonable number of media sessions and any event that is part of an All-Star Skills Competition if the Player had previously agreed to participate in that Competition), (vi) playing the Playoff games subsequent to) any Regular Season, and (vii) participating in promotional activities of the Team and the League as set forth in paragraph 13 herein.
  \item
    If the Player is a Veteran, the Player will not be required to attend training camp earlier than 4 p.m. (local time) on the twenty-ninth day prior to the first game of any Regular Season. Notwithstanding the foregoing, if the Team is scheduled during a particular NBA Season to participate outside of the United States or Canada in a Pre-Season Exhibition game or a Regular Season game during the first week of the Regular Season, such Veteran Player may be required to attend the training camp conducted in advance of that Regular Season by 4 p.m. (local time) on the thirty-second day prior to the first game of the Regular Season. Rookies may be required to attend training camp at an earlier date, but no earlier than ten (10) days prior to the date that Veterans are required to attend.
  \item
    Exhibition games shall not be played on the three (3) days prior to the opening of the Team's Regular Season schedule, nor on the day prior to a Regular Season game, nor on the day prior to and the day following the All-Star Game. Exhibition games prior to any Regular Season shall not exceed eight (including intra-squad games for which admission is charged), and Exhibition games during any Regular Season shall not exceed three.
  \end{enumerate}
\item
  \textbf{COMPENSATION.}

  \begin{enumerate}
  \def\labelenumii{(\alph{enumii})}
  \tightlist
  \item
    Subject to paragraph 3(b) below, the Team agrees to pay the Player for rendering the services described herein the compensation described in Exhibit 1 hereto (less all amounts required to be withheld by federal, state, and local authorities, and exclusive of any amount(s) which the Player shall be entitled to receive from the Player Playoff Pool). Unless otherwise provided in Exhibit 1, such compensation shall be paid in twelve (12) equal semi-monthly payments beginning with the first of said payments on November 15th of each year covered by the Contract and continuing with such payments on the first and fifteenth of each month until said compensation is paid in full; provided, however, if the Team does not qualify for the Playoffs, the payments for the year involved which would otherwise be due subsequent to the conclusion of the Regular Season shall become due and payable immediately after the conclusion of the Regular Season.
  \item
    The Team agrees to pay the Player \$1,200 per week, pro rata, less all amounts required to be withheld by federal, state, and local authorities, for each week (up to a maximum of four (4) weeks for veterans and up to a maximum of five (5) weeks for Rookies) prior to the Team's first Regular Season game that the Player is in attendance at training camp or Exhibition games; provided, however, that no such payments shall be made if, prior to the date on which he is required to attend training camp, the Player has been paid \$10,000 or more in compensation with respect to the NBA Season scheduled to commence immediately following such training camp. Any compensation paid by the Team pursuant to this subparagraph shall be considered an advance against any compensation owed to the Player pursuant to paragraph 3(a) above, and the first scheduled payment of such compensation (or such subsequent payments, if the first scheduled payment is not sufficient) shall be reduced by the amount of such advance.
  \item
    The Team will not pay and the Player will not accept any bonus or anything of value on account of the Team's winning any particular NBA game or series of games or attaining a certain position in the standings of the League as of a certain date, other than the final standing of the Team.
  \end{enumerate}
\item
  \textbf{EXPENSES.} The Team agrees to pay all proper and necessary expenses of the Player, including the reasonable board and lodging expenses of the Player while playing for the Team ``on the road'' and during the training camp period (but only until the Team breaks training camp) for as long as the Player is not then living at home. The Player, while ``on the road'' (and during the training camp period, only if the Team does not pay for meals directly), shall be paid a meal expense allowance as set forth in the Collective Bargaining Agreement currently in effect between the NBA and the National Basketball Players Association (hereinafter ``the NBA/NBPA Collective Bargaining Agreement''). No deductions from such meal expense allowance shall be made for meals served on an airplane. During the training camp period (and if the Team does not pay for meals directly), the meal expense allowance shall be paid in weekly installments commencing with the first week of training camp. For the purposes of this paragraph, the Player shall be considered to be ``on the road'' from the time the Team leaves its home city until the time the Team arrives back at its home city.
\item
  \textbf{CONDUCT.}

  \begin{enumerate}
  \def\labelenumii{(\alph{enumii})}
  \tightlist
  \item
    The Player agrees to observe and comply with all Team rules, as maintained or promulgated in accordance with the NBA/NBPA Collective Bargaining Agreement, at all times whether on or off the playing floor. Subject to the provisions of the NBA/NBPA Collective Bargaining Agreement, such rules shall be part of this Contract as fully as if herein written and shall be binding upon the Player.
  \item
    The Player agrees (i) to give his best services, as well as his loyalty, to the Team, and to play basketball only for the Team and its assignees; (ii) to be neatly and fully attired in public; (iii) to conduct himself on and off the court according to the highest standards of honesty, citizenship, and sportsmanship; and (iv) not to do anything that is materially detrimental or materially prejudicial to the best interests of the Team or of the League.
  \item
    For any violation of Team rules, any breach of any provision of this Contract, or for any conduct impairing the faithful and thorough discharge of the duties incumbent upon the Player, the Team may reasonably impose fines and/or suspensions on the Player in accordance with the terms of the NBA/NBPA Collective Bargaining Agreement.
  \item
    The Player acknowledges that he has read and is familiar with Article 35 of the NBA Constitution, a copy of which, as in effect on the date of this Contract, is attached hereto. Such Article provides that the Commissioner is empowered to impose fines upon and/or suspend the Player for causes and in the manner provided in such Article.
  \item
    The Player agrees that if the Commissioner, in his sole judgment, shall find that the Player has bet, or has offered or attempted to bet, money or anything of value on the outcome of any game participated in by any team which is a member of the NBA, the Commissioner shall have the power in his sole discretion to suspend the Player indefinitely or to expel him as a player for any member of the NBA, and the Commissioner's finding and decision shall be final, binding, conclusive, and unappealable.
  \item
    The Player agrees that he will not, during the term of this Contract, directly or indirectly, entice, induce, or persuade, or attempt to entice, induce, or persuade, any player or coach who is under contract to any NBA team to enter into negotiations for or relating to his services as a basketball player or coach, nor shall he negotiate for or contract for such services, except with the prior written consent of such team. Breach of this subparagraph, in addition to the remedies available to the Team, shall be punishable by fine and/or suspension to be imposed by the Commissioner.
  \item
    When the Player is fined and/or suspended by the Team or the NBA, he shall be given notice in writing (with a copy to the Players Association), stating the amount of the fine or the duration of the suspension and the reasons therefor.
  \end{enumerate}
\item
  \textbf{WITHHOLDING AND ESCROW.}

  \begin{enumerate}
  \def\labelenumii{(\alph{enumii})}
  \tightlist
  \item
    In the event the Player is fined and/or suspended by the Team or the NBA pursuant to any provision of paragraph 5 above, the Team shall withhold the amount of the fine or, in the case of a suspension, the amount provided in Article VI of the NBA/NBPA Collective Bargaining Agreement from any Current Cash Compensation due or to become due to the Player with respect to the NBA Season in which the conduct resulting in the fine and/or the suspension occurred (or a subsequent Season if the Player has received all Current Cash Compensation due to him for the then current Season). If, at the time the Player is fined and/or suspended, the Current Cash Compensation remaining to be paid to the Player under this Contract is not sufficient to cover such fine and/or suspension, then the Player agrees promptly to pay the amount directly to the Team. In no case shall the Player permit any such fine and/or suspension to be paid on his behalf by anyone other than himself.
  \item
    Any Current Cash Compensation withheld from or paid by the Player pursuant to this paragraph 6 shall be retained by the Team or the League, as the case may be, unless the Player contests the fine and/or suspension by initiating a timely Grievance in accordance with the provisions of the NBA/NBPA Collective Bargaining Agreement. If such Grievance is initiated and it satisfies Article XXXI, Section 13 of the NBA/NBPA Collective Bargaining Agreement, the amount withheld from the Player shall be placed in escrow, pursuant to Article XXXI, Section 9 of such Agreement, pending the resolution of the Grievance.
  \end{enumerate}
\item
  \textbf{PHYSICAL CONDITION.}

  \begin{enumerate}
  \def\labelenumii{(\alph{enumii})}
  \tightlist
  \item
    The Player agrees to report at the time and place fixed by the Team in good physical condition and to keep himself throughout each NBA Season in good physical condition.
  \item
    If the Player, in the judgment of the Team's physician, is not in good physical condition at the date of his first scheduled game for the Team, or if, at the beginning of or during any Season, he fails to remain in good physical condition (unless such condition results directly from an injury sustained by the Player as a direct result of participating in any basketball practice or game played for the Team during such Season), so as to render the Player, in the judgment of the Team's physician, unfit to play skilled basketball, the Team shall have the right to suspend such Player until such time as, in the judgment of the Team's physician, the Player is in sufficiently good physical condition to play skilled basketball. In the event of such suspension, the Regular Salary payable to the Player for any Season during such suspension shall be reduced in the same proportion as the length of the period during which, in the judgment of the Team's physician, the Player is unfit to play skilled basketball, bears to the length of such Season.
  \item
    If, during the term of this Contract, the Player is injured as a direct result of participating in any basketball practice or game played for the Team, the Team will pay the Player's reasonable hospitalization and medical expenses (including doctor's bills), provided that the hospital and doctor are selected by the Team, and provided further that the Team shall be obligated to pay only those expenses incurred as a direct result of medical treatment caused solely by and relating directly to the injury sustained by the Player. Subject to the provisions set forth in Exhibit 3, if in the judgment of the Team's physician, the Player's injuries resulted directly from playing for the Team and render him unfit to play skilled basketball, then, so long as such unfitness continues, but in no event after the Player has received his full compensation for the Season in which the injury was sustained, the Team shall pay to the Player the compensation prescribed in Exhibit 1 to this Contract for such Season. The Team's obligations hereunder shall be reduced by (i) any workers' compensation benefits, which, to the extent permitted by law, the Player hereby assigns to the Team and (ii) any insurance provided for by the Team whether paid or payable to the Player.
  \item
    The Player agrees to give to the Team's coach, trainer, or physician prompt notice of any injury or illness suffered by him that is likely to affect the Player's ability to render the services required under this Contract, including the time, place, cause, and nature of such injury or illness.
  \item
    Should the Player suffer an injury or illness as provided in this paragraph 7, he will submit himself to a medical examination and treatment by a physician designated by the Team. Such examination when made at the request of the Team shall be at its expense, unless made necessary by some act or conduct of the Player contrary to the terms of this Contract.
  \end{enumerate}
\item
  \textbf{PROHIBITED SUBSTANCES.} The Player acknowledges that, in the event he is found, in accordance with Section 1 of the NBA/NBPA Anti-Drug Agreement, to have engaged in the use, possession, or distribution of a ``Prohibited Substance'' as defined therein, it will result in the termination of this Contract and the Player's immediate dismissal and disqualification from any employment by the NBA and any of its teams. Notwithstanding any terms or provisions of this Contract (including any amendments hereto), in the event of such termination, all obligations of the Team, including obligations to pay compensation, shall cease, except the obligation of the Team to pay the Player's earned compensation (whether current or deferred) to the date of termination.
\item
  \textbf{UNIQUE SKILLS.} The Player represents and agrees that he has extraordinary and unique skill and ability as a basketball player, that the services to be rendered by him hereunder cannot be replaced or the loss thereof adequately compensated for in money damages, and that any breach by the Player of this Contract will cause irreparable injury to the Team, and to its assignees. Therefore, it is agreed that in the event it is alleged by the Team that the Player is playing, attempting or threatening to play, or negotiating for the purpose of playing, during the term of this Contract, for any other person, firm, corporation, or organization, the Team and its assignees (in addition to any other remedies that may be available to them judicially or by way of arbitration) shall have the right to obtain from any court or arbitrator having jurisdiction such equitable relief as may be appropriate, including a decree enjoining the Player from any further such breach of this Contract, and enjoining the Player from playing basketball for any other person, firm, corporation, or organization during the term of this Contract. The Player agrees that the Team may at any time assign such right to the NBA for the enforcement thereof. In any suit, action, or arbitration proceeding brought to obtain such equitable relief, the Player does hereby waive his right, if any, to trial by jury, and does hereby waive his right, if any, to interpose any counterclaim or set-off for any cause whatever.
\item
  \textbf{ASSIGNMENT.}

  \begin{enumerate}
  \def\labelenumii{(\alph{enumii})}
  \tightlist
  \item
    The Team shall have the right to sell, exchange, assign, or transfer this Contract to any other NBA team and the Player agrees to accept such sale, exchange, assignment, or transfer and to faithfully perform and carry out this contract with the same force and effect as if it had been entered into by the Player with the assignee team instead of with the Team. The Player further agrees that, should the Team contemplate the sale, exchange, assignment, or transfer of this Contract to one or more NBA teams, the Team's physician may furnish to the physicians and officials of such other team or teams all relevant medical information relating to the Player.
  \item
    In the event that this Contract is sold, exchanged, assigned or transferred to any other NBA team, all reasonable expenses incurred by the Player in moving himself and his family to the home city of the team to which such sale, exchange, assignment or transfer is made, as a result thereof, shall be paid by the assignee team. Such assignee team hereby agrees that its acceptance of the assignment of this Contract constitutes agreement on its part to make such payment.
  \item
    In the event that this Contract is assigned to another NBA team, the Player (and the Players Association) shall forthwith be notified orally or by a notice in writing, delivered to the Player personally or delivered or mailed to his last known address, and the Player shall report to the assignee team within forty-eight (48) hours after said notice has been received or within such longer time for reporting as may be specified in said notice. If the Player does not report to the team to which this Contract has been assigned within the aforesaid time, or if a proposed assignment is voided as a result of the Player's failure to report, the Player may be suspended by the assignee team or assignor Team, as the case may be, and he shall lose the sums which would otherwise be payable to him as long as the suspension lasts.
  \end{enumerate}
\item
  \textbf{VALIDITY AND FILING.} This Contract shall be valid and binding upon the Team and the Player immediately upon its execution. The Team agrees to file a copy of this Contract, and/or any amendment(s) thereto, with the Commissioner of the NBA as soon as practicable by facsimile and overnight mail, but in no event may such filing be made more than forty-eight (48) hours after the execution of this Contract and/or amendment(s). If pursuant to the NBA Constitution and By-Laws or the NBA/NBPA Collective Bargaining Agreement, the Commissioner disapproves this Contract (or amendment) within ten (10) days after the receipt thereof in his office by overnight mail, this Contract (or amendment) shall thereupon terminate and be of no further force or effect and the Team and the Player shall thereupon be relieved of their respective rights and liabilities thereunder. If the Commissioner's disapproval is subsequently overturned in any proceeding brought under the arbitration provisions of the NBA/NBPA Collective Bargaining Agreement (including any appeals), the Contract shall again be valid and binding upon the Team and the Player, and the Commissioner shall be afforded another ten-day period to disapprove the Contract (based on the Team's Room at the time the Commissioner's disapproval is overturned) as set forth in the foregoing sentence. The NBA will promptly inform the Players Association if the Commissioner disapproves this Contract.
\item
  \textbf{OTHER ATHLETIC ACTIVITIES.} The Player and the Team acknowledge and agree that (i) the Player's participation in other sports may impair or destroy his ability and skill as a basketball player, and (ii) the Player's participation in basketball out of season may result in injury to him. Accordingly, the Player agrees that he will not, without the written consent of the Team, engage in (x) sports endangering his health or safety (including, but not limited to, professional boxing or wrestling, motorcycling, moped-riding, auto racing, sky-diving, and hang gliding), or (y) any game or exhibition of basketball, football, baseball, hockey, lacrosse, or other athletic sport, under penalty of such fine and/or suspension as may be imposed by the Team and/or the Commissioner of the NBA. Nothing contained herein shall be intended to require the Player to obtain the written consent of the Team in order to enable the Player to participate in, as an amateur, the sport of golf, tennis, handball, swimming, hiking, softball, or volleyball.
\item
  \textbf{PROMOTIONAL ACTIVITIES.}

  \begin{enumerate}
  \def\labelenumii{(\alph{enumii})}
  \tightlist
  \item
    The Player agrees to allow the Team or the League to take pictures of the Player, alone or together with others, for still photographs, motion pictures, or television, at such times as the Team or the League may designate. No matter by whom taken, such pictures may be used in any manner desired by either the Team or the League for publicity or promotional purposes. The rights in any such pictures taken by the Team or by the League shall belong to the Team or to the League, as their interests may appear.
  \item
    The Player agrees that, during any year of this Contract, he will not make public appearances, participate in radio or television programs, permit his picture to be taken, write or sponsor newspaper or magazine articles, or sponsor commercial products without the written consent of the Team, which shall not be withheld except in the reasonable interests of the Team or the NBA.
  \item
    The Player agrees that he will comply with the Team's designation of Protected Team Sponsors, to the extent such designations are in compliance with Article II, Section 7 of the NBA/NBPA Collective Bargaining Agreement.
  \item
    Upon request, the Player shall consent to and make himself available for interviews by representatives of the media conducted at reasonable times.
  \item
    In addition to the foregoing, and subject to the conditions and limitations set forth in Article II, Section 7 of the NBA/NBPA.Collective Bargaining Agreement, the Player agrees to participate, upon request, in all other reasonable promotional activities of the Team and the NBA. For each such promotional appearance made on behalf of a commercial sponsor of the Team, the Team agrees to pay the Player \$1,000 or, if the Team agrees, such higher amount that is consistent with the Team's past practice and not otherwise unreasonable.
  \end{enumerate}
\item
  \textbf{GROUP LICENSE.}

  \begin{enumerate}
  \def\labelenumii{(\alph{enumii})}
  \tightlist
  \item
    The Player hereby grants to NBA Properties, Inc.~the exclusive rights to use the Player's Player Attributes as such term is defined and for such group licensing purposes as are set forth in the Agreement between NBA Properties, Inc.~and the National Basketball Players Association, made as of September 18, 1995 (the ``Group License''), a copy of which will, upon his request, be furnished the Player; and the Player agrees to make the appearances called for by such Agreement.
  \item
    Notwithstanding anything to the contrary contained in the Group License or this Contract, NBA Properties may use, in connection with League Promotions, the Player's (i) name or nickname and/or (ii) the Player's Player Attributes (as defined in the Group License) as such Player Attributes may be captured in game action footage or photographs. NBA Properties shall be entitled to use the Player's Player Attributes individually pursuant to the preceding sentence and shall not be required to use the Player's Player Attributes in a group or as one of multiple players. As used herein, League Promotion shall mean any advertising, marketing, or collateral materials or marketing programs conducted by the NBA, NBA Properties (or any subsidiary of NBA Properties) or any NBA team that is intended to promote (x) any game in which an NBA team participates or game telecast or broadcast (including Pre-Season, Exhibition, Regular Season, and Playoff games), (y) the NBA, its teams, or its players, or (z) the sport of basketball.
  \end{enumerate}
\item
  \textbf{TEAM DEFAULT.} In the event of an alleged default by the Team in the payments to the Player provided for by this Contract, or in the event of an alleged failure by the Team to perform any other material obligation that it has agreed to perform hereunder, the Player shall notify both the Team arid the League in-writing of the facts constituting such alleged default or alleged failure. If neither the Team nor the League shall cause such alleged default or alleged failure to be remedied within five (5) days after receipt of such written notice, the National Basketball Players Association shall, on behalf of the Player, have the right to request that the dispute concerning such alleged default or alleged failure be referred immediately to the Grievance Arbitrator in accordance with the provisions of the NBA/NBPA Collective Bargaining Agreement. If, as a result of such arbitration, an award issues in favor of the Player, and if neither the Team nor the League complies with such award within ten (10) days after the service thereof, the Player shall have the right, by a further written notice to the Team and the League, to terminate this Contract.
\item
  \textbf{TERMINATION.}

  \begin{enumerate}
  \def\labelenumii{(\alph{enumii})}
  \tightlist
  \item
    The Team may terminate this Contract upon written notice to the Player (but only after complying with the waiver procedure provided for in subparagraph (f) of this paragraph 16) if the Player shall do any of the following:

    \begin{enumerate}
    \def\labelenumiii{(\roman{enumiii})}
    \tightlist
    \item
      at any time, fail, refuse, or neglect to conform his personal conduct to standards of good citizenship, good moral character (defined here to mean not engaging in acts of moral turpitude, whether or not such acts would constitute a crime) and good sportsmanship, to keep himself in first class physical condition or to obey the Team's training rules; or
    \item
      at any time, fail, in the sole opinion of the Team's management, to exhibit sufficient skill or competitive ability to qualify to continue as a member of the Team; provided, however, (x) that if this Contract is terminated by the Team, in accordance with the provisions of this subparagraph, prior to January 10 of any Regular Season, and the Player, at the time of such termination, is unfit to play skilled basketball as the result of an injury resulting directly from his playing for the Team, the. Player shall continue to receive his full salary, less all workers' compensation benefits (which, to the extent permitted by law, and if not deducted from the Player's salary by the Team, the Player hereby assigns to the Team) and any insurance provided for by the Team paid or payable to the Player by reason of said injury, until such time as the Player is fit to play skilled basketball, but not beyond the Season during which such termination occurred; and provided, further, (y) that if this Contract is terminated by the Team, in accordance with the provisions of this subparagraph, during the period from the January 10 of any Regular Season through the end of such Regular Season, the Player shall be entitled to receive his full salary for said Season; or
    \item
      at any time, fail, refuse, or neglect to render his services hereunder or in any other manner materially breach this Contract.
    \end{enumerate}
  \item
    If this Contract is terminated by the Team by reason of the Player's failure to render his services hereunder due to disability caused by an injury to the Player resulting directly from his playing for the Team and rendering him unfit to play skilled basketball, and notice of such injury is given by the Player as provided herein, the Player shall be entitled to receive his full salary for the Season in which the injury was sustained, less all workers' compensation benefits (which, to the extent permitted by law, and if not deducted from the Player's salary by the Team, the Player hereby assigns to the Team) and any insurance provided for by the Team paid or payable to the Player by reason of said injury.
  \item
    Notwithstanding the provisions of subparagraph 16(b) above, if this Contract is terminated by the Team prior to the first game of a Regular Season by reason of the Player's failure to render his services hereunder due to an injury or condition sustained or suffered during a preceding Season, or after such Season but prior to the Player's participation in any basketball practice or game played for the Team, payment of the Player's board, lodging, and expense allowance during the training camp period, payment of the reasonable traveling expenses of the Player to his home city, and the expert training and coaching provided by the Team to the Player during the training season shall be full payment to the Player.
  \item
    If this Contract is terminated by the Team during the period designated by the Team for attendance at training camp, payment by the Team of any compensation earned through the date of termination under paragraph 3(b) above, payment of the Player's board, lodging, and expense allowance during such period to the date of termination, payment of the reasonable traveling expenses of the Player to his home city, and the expert training and coaching provided by the Team to the Player during the training season shall be full payment to the Player.
  \item
    If this Contract is terminated by the Team during any Season, except in the case provided for in subparagraphs (a)(ii) and (b) of this paragraph 16, the Player shall be entitled to receive as full payment hereunder a sum of money which, when added to the salary which he has already received during such Season, will represent the same proportionate amount of the annual sum set forth in Exhibit 1 hereto as the number of days of such Regular Season then past bears to the total number of days of such Regular Season, plus the reasonable traveling expenses of the Player to his home.(f) If the Team proposes to terminate this Contract in accordance with subparagraph (a) of this paragraph 16, the applicable waiver procedure shall be as follows:

    \begin{enumerate}
    \def\labelenumiii{(\roman{enumiii})}
    \tightlist
    \item
      The Team shall request the NBA Commissioner to request waivers from all other clubs. Such waiver request must state that it is for the purpose of terminating this Contract and it may not be withdrawn.
    \item
      Upon receipt of the waiver request, any other team may claim assignment of this Contract at such waiver price as may be fixed by the League, the priority of claims to be determined in accordance with the NBA Constitution and By-Laws.
    \item
      If this Contract is so claimed, the Team agrees that it shall, upon the assignment of this Contract to the claiming team, notify the Player of such assignment as provided in paragraph 1O(c) hereof, and the Player agrees he shall report to the assignee team as provided in said paragraph 1O(c).
    \item
      If the Contract is not claimed, the Team shall promptly deliver written notice of termination to the Player at the expiration of the waiver period.
    \item
      The NBA shall promptly notify the Players Association of the disposition of any waiver request.
    \item
      To the extent not inconsistent with the foregoing provisions of this subparagraph (f), the waiver procedures set forth in the NBA Constitution and By-Laws, a copy of which, as in effect on the date of this Contract, is attached hereto, shall govern.
    \end{enumerate}
  \item
    Upon any termination of this Contract by the Player, all obligations of the Team to pay compensation shall cease on the date of termination, except the obligation of the Team to pay the Player's compensation to said date.
  \end{enumerate}
\item
  \textbf{DISPUTES.} In the event of any dispute arising between the Player and the Team relating to any matter arising under this Contract or concerning the performance or interpretation thereof (except for a dispute arising under paragraph 9 hereof), such dispute shall be resolved in accordance with the Grievance and Arbitration Procedure set forth in the NBA/NBPA Collective Bargaining Agreement.
\item
  \textbf{PLAYER NOT A MEMBER.} Nothing contained in this Contract or in any provision of the NBA Constitution and By-Laws shall be construed to constitute the Player a member of the NBA or to confer upon him any of the rights or privileges of a member thereof.
\item
  \textbf{RELEASE.} The Player hereby releases and waives every claim he may have against the NBA and its related entities and every member of the NBA, and against every director, officer, owner, stockholder, trustee, partner, and employee of the NBA and its related entities and/or any member of the NBA and their related entities (excluding persons employed as players by any such member), and against any person retained by the NBA and/or the Players Association in connection with the NBA/NBPA Anti-Drug Agreement, the Grievance Arbitrator, and any other arbitrator or expert retained by the NBA and/or the Players Association under the terms of the NBA/NBPA Collective Bargaining Agreement, arising out of or in connection with (i) any injury that is subject to the provisions of paragraph 7, (ii) any fighting or other form of violent and/or, unsportsmanlike conduct occurring during the course of any practice and/or any Exhibition, Regular Season, and/or Playoff game (on or adjacent to the playing floor or any facility used for practices or games), (iii) the testing procedures or the imposition of any penalties set forth in paragraph 8 hereof and in the NBA/NBPA Anti-Drug Agreement, or (iv) any injury suffered in the course of his employment as to which he has or would have a claim for workers' compensation benefits. The foregoing shall not apply to any claim of medical malpractice against a Team-affiliated physician or other medical personnel.
\item
  \textbf{ENTIRE AGREEMENT.} This Contract (including any Exhibits hereto) contains the entire agreement between the parties and sets forth all components of the Player's compensation from the Team or any person or entity affiliated with, related to, or controlled by the Team, or any person or entity owning an interest in the Team, and there are no undisclosed agreements of any kind, express or implied, oral or written, promises, undertakings, representations, commitments, inducements, assurances of intent, or understandings of any kind that have not been disclosed to the NBA (a) involving consideration of any kind to be paid, furnished, or made available to the Player, or any person or entity controlled by or related to the Player, by the Team or any person or entity affiliated with, related to, or controlled by the Team, or any person or entity owning an interest in the Team, either during the term of this Contract or thereafter, or (b) concerning any future renegotiation, extension, or amendment of this Contract or the entry into any new Player Contract.
\end{enumerate}

\newpage

\textbf{\emph{EXAMINE THIS CONTRACT CAREFULLY BEFORE SIGNING IT.}}

THIS CONTRACT INCLUDES EXHIBITS \_\_\_\_\_\_\_\_\_\_, WHICH ARE ATTACHED HERETO AND MADE A PART HEREOF.

IN WITNESS WHEREOF the Player has hereunto signed his name and the Team has caused this Contract to be executed by its duly authorized officer.

WITNESSES:

\begin{longtable}[]{@{}ll@{}}
\toprule()
\endhead
Dated: \_\_\_\_\_\_\_\_\_\_\_\_\_\_\_\_\_\_\_\_\_ & By: \_\_\_\_\_\_\_\_\_\_\_\_\_\_\_\_\_\_\_\_\_\_\_\_\_\_\_\_ \\
& Title: \_\_\_\_\_\_\_\_\_\_\_\_\_\_\_\_\_\_\_\_\_\_\_\_\_\_\_\_ \\
& Team: \_\_\_\_\_\_\_\_\_\_\_\_\_\_\_\_\_\_\_\_\_\_\_\_\_\_\_\_ \\
& \\
Dated: \_\_\_\_\_\_\_\_\_\_\_\_\_\_\_\_\_\_\_\_\_ & By: \_\_\_\_\_\_\_\_\_\_\_\_\_\_\_\_\_\_\_\_\_\_\_\_\_\_\_\_ \\
& Player: \_\_\_\_\_\_\_\_\_\_\_\_\_\_\_\_\_\_\_\_\_\_\_\_\_\_\_\_ \\
& Player's Address: \\
& \_\_\_\_\_\_\_\_\_\_\_\_\_\_\_\_\_\_\_\_\_\_\_\_\_\_\_\_\_\_\_\_\_\_\_\_ \\
& \_\_\_\_\_\_\_\_\_\_\_\_\_\_\_\_\_\_\_\_\_\_\_\_\_\_\_\_\_\_\_\_\_\_\_\_ \\
\bottomrule()
\end{longtable}

\begin{longtable}[]{@{}l@{}}
\toprule()
\endhead
State of \_\_\_\_\_\_\_\_\_\_\_\_\_\_\_\_\_\_\_\_\_\_ \\
County of \_\_\_\_\_\_\_\_\_\_\_\_\_\_\_\_\_\_\_\_\_ \\
\bottomrule()
\end{longtable}

On \_\_\_\_\_\_\_\_\_\_\_\_\_\_\_\_\_\_\_\_\_, before me personally came\_\_\_\_\_\_\_\_\_\_\_\_\_\_\_\_\_\_\_\_\_ (referred to above as the ``Player''), and acknowledged to me that he had executed the foregoing Uniform Player Contract (including Exhibits thereto):

\begin{longtable}[]{@{}l@{}}
\toprule()
\endhead
Notary Public: \_\_\_\_\_\_\_\_\_\_\_\_\_\_\_\_\_\_\_\_\_ \\
\bottomrule()
\end{longtable}

\begin{longtable}[]{@{}l@{}}
\toprule()
\endhead
State of \_\_\_\_\_\_\_\_\_\_\_\_\_\_\_\_\_\_\_\_\_\_ \\
County of \_\_\_\_\_\_\_\_\_\_\_\_\_\_\_\_\_\_\_\_\_ \\
\bottomrule()
\end{longtable}

On \_\_\_\_\_\_\_\_\_\_\_\_\_\_\_\_\_\_\_\_\_, before me personally came \_\_\_\_\_\_\_\_\_\_\_\_\_\_\_\_\_\_\_\_\_ (a duly authorized officer of the ``Team''), and acknowledged to me that he/she had executed the foregoing Uniform Player Contract (including Exhibits thereto):

\begin{longtable}[]{@{}l@{}}
\toprule()
\endhead
Notary Public: \_\_\_\_\_\_\_\_\_\_\_\_\_\_\_\_\_\_\_\_\_ \\
\bottomrule()
\end{longtable}

\newpage

\hypertarget{excerpt-from-nba-constitution}{%
\subsection{EXCERPT FROM NBA CONSTITUTION}\label{excerpt-from-nba-constitution}}

\hypertarget{misconduct}{%
\subsubsection{MISCONDUCT}\label{misconduct}}

\begin{enumerate}
\def\labelenumi{\arabic{enumi}.}
\setcounter{enumi}{34}
\tightlist
\item
  The provisions of this Article 35 shall govern all Players in the Association, hereinafter referred to as ``Players.''

  \begin{enumerate}
  \def\labelenumii{(\alph{enumii})}
  \tightlist
  \item
    Each Member shall provide and require in every contract with any of its Players that they shall be bound and governed by the provisions of this Article. Each Member, at the direction of the Board of Governors or the Commissioner, as the case may be, shall take such action as the Board or the Commissioner may direct in order to effectuate the purposes of this Article.
  \item
    The Commissioner shall direct the dismissal and perpetual disqualification from any further association with the Association or any of its Members, of any Player found by the Commissioner after a hearing to have been guilty of offering, agreeing, conspiring, aiding or attempting to cause any game of basketball to result otherwise than on its merits.
  \item
    Any Player who gives, makes, issues, authorizes or endorses any statement having, or designed to have, an effect prejudicial or detrimental to the best interests of basketball or of the Association or of a Member or its Team, shall be liable to a fine not exceeding \$25,000, to be imposed by the Commissioner. The Member whose Player has been so fined shall pay the amount of the fine should such Player fail to do so within ten (10) days of its imposition.
  \item
    If in the opinion of the Commissioner any other act or conduct of a Player at or during an Exhibition, Regular Season, or Playoff game has been prejudicial to or against the best interests of the Association or the game of basketball, the Commissioner shall impose upon such Player a fine not exceeding \$25,000, or may order for a time the suspension of any such Player from any connection or duties with Exhibition, Regular Season, or Playoff games, or he may order both such fine and suspension.
  \item
    The Commissioner shall have the power to suspend for a definite or indefinite period, or to impose a fine not exceeding \$25,000, or inflict both such suspension and fine upon any Player who, in his opinion, shall have been gUilty of conduct that does not conform to standards of morality or fair play, that does not comply at all times with all federal, state, and local laws, or that is prejudicial or detrimental to the Association.
  \item
    Any Player who, directly or indirectly, entices, induces, persuades or attempts to entice, induce, or persuade any Player, Coach, Trainer, General Manager or any other person who is under contract to any other Member of the Association to enter into negotiations for or relating to his services or negotiates or contracts for such services shall, on being charged with such tampering, be given an opportunity to answer such charges after due notice and the Commissioner shall have the power to decide whether or not the charges have been sustained; in the event his decision is that the charges have been sustained, then the Commissioner shall have the power to suspend such Player for a definite or indefinite period, or to impose a fine not exceeding \$25,000, or inflict both such suspension and fine upon any such Player.
  \item
    Any Player who, directly or indirectly, wagers money or anything of value on the outcome of any game played by a Team in the league operated by the Association shall, on being charged with such wagering, be given an opportunity to answer such charges after due notice, and the decision of the Commissioner shall be final, binding and conclusive and unappealable. The penalty for such offense shall be within the absolute and sole discretion of the Commissioner and may include a fine, suspension, expUlsion and/or perpetual disqualification from further association with the Association or any of its Members.
  \item
    Except for a penalty imposed under Paragraph (g) of this Article 35, the decisions and acts of the Commissioner pursuant to Article 35 shall be appealable to the Board of Governors who shall determine such appeals in accordance with such rules and regulations as may be adopted by the Board in its absolute and sole discretion.
  \end{enumerate}
\end{enumerate}

\newpage

\hypertarget{excerpt-from-by-laws-of-the-association}{%
\subsection{EXCERPT FROM BY-LAWS OF THE ASSOCIATION}\label{excerpt-from-by-laws-of-the-association}}

5.01. \emph{Waiver Right.} Except for sales and trading between Members in accordance with these By-Laws, no Member shall sell, option or otherwise assign the contract with, right to the services of, or right to negotiate with, a Player without complying with the waiver procedure prescribed by these By-Laws.

5.02. \emph{Waiver Price.} The waiver price shall be \$1,000 per Player.

5.03. \emph{Waiver Procedure.} A Member desiring to secure waivers on a Player shall notify the Commissioner or the Commissioner's designee, who shall, on behalf of such Member, immediately notify all other Members of the waiver request. Such Player shall be assumed to have been waived unless a Member shall timely notify the Commissioner or the Commissioner's designee by facsimile or electronic mail, and telephone, of a claim to the rights of such Player. Once a Member has notified the Commissioner or the Commissioner's designee to attempt to secure waivers on a Player, such notice may not be withdrawn. A Player remains the financial responsibility of the Member placing him on waivers until the waiver period set by the Commissioner or the Commissioner's designee has expired.

5.04. \emph{Waiver Period.} If the Commissioner or the Commissioner's designee distributes notice of request for waiver at any time between August 15 and the end of the next Season, any Members wishing to claim rights to the Player shall do so by giving notice by telephone and electronic mail of such claim to the Commissioner or the Commissioner's designee within forty-eight (48) hours after the time of such notice. If the Commissioner or the Commissioner's designee distributes notice of request for waiver at any other time, any Member wishing to claim rights to the Player shall do so by providing notice of such claim to the Commissioner within ten (10) days after the date of such notice, A Team may not withdraw a claim to the rights to a Player on waivers.

5.05. \emph{Waiver Preferences.}
(a) In the event that more than one (1) Member shall have claimed rights to a Player placed on waivers, the claiming Member with the lowest team standing at the time the waiver was requested shall be entitled to acquire the rights to such Player. If the request for waiver shall occur after the last day of the Season and before 11 :59 p.m. eastern time on the following November 30, the standings at the close of the previous Season shall govern.
(b) If the winning percentages of two (2) claiming Teams are the same, then the tie shall be determined, if possible, on the basis of the Regular Season Games between the two (2) Teams, during the Season or during the preceding Season, as the case may be. If still tied, a toss of a coin shall determine priority. For the purpose of determining standings, both Conferences of the Association shall be deemed merged and a consolidated standing shall control.

5.06. \emph{Players Acquired Through Waivers.} A Member who has acquired the rights and title to the contract of a Player through the waiver procedure may not sell or trade such rights for a period of thirty (30) days after the acquisition thereof; provided, however, that if the rights to such Player were acquired between Seasons, the 30-day period described herein shall begin on the first day of the next succeeding Season.

5.07. \emph{Additional Waiver Rules.} The Commissioner or the Board of Governors may from time to time adopt additional rules (supplementary to these By-Laws) with respect to the operation of the waiver procedures. Such rules shall not be inconsistent with these By-Laws and shall apply to but shall not be limited to the mechanics of notice, inadvertent omission of notification to a Member and rules of construction as to time.

\newpage

\hypertarget{agent-certification}{%
\subsection{AGENT CERTIFICATION}\label{agent-certification}}

(To be completed only if Player was represented by an agent who negotiated the terms of this Contract.)

I, the undersigned, having negotiated this Contract on behalf of \_\_\_\_\_\_\_\_\_\_\_\_\_\_\_\_\_, do hereby swear and certify, under penalties of perjury, that the terms of Paragraph 20 of this Contract (``Entire Agreement'') are true and correct to the best of my knowledge and belief.

\begin{longtable}[]{@{}l@{}}
\toprule()
\endhead
Player Representative: \_\_\_\_\_\_\_\_\_\_\_\_\_\_\_\_\_ \\
State of \_\_\_\_\_\_\_\_\_\_\_\_\_\_\_\_\_\_\_\_\_\_\_\_\_\_\_\_\_\_\_ \\
County of \_\_\_\_\_\_\_\_\_\_\_\_\_\_\_\_\_\_\_\_\_\_\_\_\_\_\_\_\_\_ \\
\bottomrule()
\end{longtable}

On \_\_\_\_\_\_\_\_\_\_\_\_\_\_\_\_\_\_\_\_\_, before me personally came \_\_\_\_\_\_\_\_\_\_\_\_\_\_\_\_\_\_\_\_\_ and acknowledged to me that he/she had executed the foregoing Agent Certification.

\begin{longtable}[]{@{}l@{}}
\toprule()
\endhead
Notary Public: \_\_\_\_\_\_\_\_\_\_\_\_\_\_\_\_\_\_\_\_\_\_\_\_ \\
\bottomrule()
\end{longtable}

\newpage

\hypertarget{uniform-player-contract-1}{%
\section{UNIFORM PLAYER CONTRACT}\label{uniform-player-contract-1}}

\hypertarget{exhibit-1---compensation}{%
\subsection{Exhibit 1 - Compensation}\label{exhibit-1---compensation}}

\begin{longtable}[]{@{}l@{}}
\toprule()
\endhead
Player: \_\_\_\_\_\_\_\_\_\_\_\_\_\_\_\_\_\_\_\_\_\_\_\_\_\_\_\_\_\_\_\_ \\
Team: \_\_\_\_\_\_\_\_\_\_\_\_\_\_\_\_\_\_\_\_\_\_\_\_\_\_\_\_\_\_\_\_\_\_ \\
Date: \_\_\_\_\_\_\_\_\_\_\_\_\_\_\_\_\_\_\_\_\_\_\_\_\_\_\_\_\_\_\_\_\_\_ \\
\bottomrule()
\end{longtable}

\begin{longtable}[]{@{}
  >{\centering\arraybackslash}p{(\columnwidth - 4\tabcolsep) * \real{0.1071}}
  >{\centering\arraybackslash}p{(\columnwidth - 4\tabcolsep) * \real{0.4464}}
  >{\centering\arraybackslash}p{(\columnwidth - 4\tabcolsep) * \real{0.4464}}@{}}
\toprule()
\begin{minipage}[b]{\linewidth}\centering
Season
\end{minipage} & \begin{minipage}[b]{\linewidth}\centering
Current Base Compensation
\end{minipage} & \begin{minipage}[b]{\linewidth}\centering
Deferred Base Compensation
\end{minipage} \\
\midrule()
\endhead
\_\_\_\_\_\_\_\_ & \_\_\_\_\_\_\_\_\_\_\_\_\_\_\_\_\_\_\_\_\_\_\_ & \_\_\_\_\_\_\_\_\_\_\_\_\_\_\_\_\_\_\_\_\_\_\_\_\_ \\
\_\_\_\_\_\_\_\_ & \_\_\_\_\_\_\_\_\_\_\_\_\_\_\_\_\_\_\_\_\_\_\_ & \_\_\_\_\_\_\_\_\_\_\_\_\_\_\_\_\_\_\_\_\_\_\_\_\_ \\
\_\_\_\_\_\_\_\_ & \_\_\_\_\_\_\_\_\_\_\_\_\_\_\_\_\_\_\_\_\_\_\_ & \_\_\_\_\_\_\_\_\_\_\_\_\_\_\_\_\_\_\_\_\_\_\_\_\_ \\
\_\_\_\_\_\_\_\_ & \_\_\_\_\_\_\_\_\_\_\_\_\_\_\_\_\_\_\_\_\_\_\_ & \_\_\_\_\_\_\_\_\_\_\_\_\_\_\_\_\_\_\_\_\_\_\_\_\_ \\
\_\_\_\_\_\_\_\_ & \_\_\_\_\_\_\_\_\_\_\_\_\_\_\_\_\_\_\_\_\_\_\_ & \_\_\_\_\_\_\_\_\_\_\_\_\_\_\_\_\_\_\_\_\_\_\_\_\_ \\
\bottomrule()
\end{longtable}

\textbf{Payment Schedule} (if different from paragraph 3):

Current Base:

Deferred Base:

\textbf{Bonuses} (include dates of payment):

\textbf{Other Cash and Non-Cash Compensation Arrangements:}

\begin{longtable}[]{@{}ll@{}}
\toprule()
Initialed: & \\
\midrule()
\endhead
\_\_\_\_\_\_\_\_\_\_\_\_\_\_ & \_\_\_\_\_\_\_\_\_\_\_\_\_\_ \\
Player & Team \\
\bottomrule()
\end{longtable}

\newpage

\hypertarget{exhibit-2---salary-protection-or-insurance}{%
\subsection{Exhibit 2 - Salary Protection or Insurance}\label{exhibit-2---salary-protection-or-insurance}}

\begin{longtable}[]{@{}l@{}}
\toprule()
\endhead
Player: \_\_\_\_\_\_\_\_\_\_\_\_\_\_\_\_\_\_\_\_\_\_\_\_\_\_\_\_\_\_\_\_ \\
Team: \_\_\_\_\_\_\_\_\_\_\_\_\_\_\_\_\_\_\_\_\_\_\_\_\_\_\_\_\_\_\_\_\_\_ \\
Date: \_\_\_\_\_\_\_\_\_\_\_\_\_\_\_\_\_\_\_\_\_\_\_\_\_\_\_\_\_\_\_\_\_\_ \\
\bottomrule()
\end{longtable}

\begin{longtable}[]{@{}
  >{\centering\arraybackslash}p{(\columnwidth - 6\tabcolsep) * \real{0.0806}}
  >{\centering\arraybackslash}p{(\columnwidth - 6\tabcolsep) * \real{0.2419}}
  >{\centering\arraybackslash}p{(\columnwidth - 6\tabcolsep) * \real{0.2742}}
  >{\centering\arraybackslash}p{(\columnwidth - 6\tabcolsep) * \real{0.4032}}@{}}
\toprule()
\begin{minipage}[b]{\linewidth}\centering
Season
\end{minipage} & \begin{minipage}[b]{\linewidth}\centering
Type of Protection
\end{minipage} & \begin{minipage}[b]{\linewidth}\centering
Amount of Protection or Insurance
\end{minipage} & \begin{minipage}[b]{\linewidth}\centering
Conditions or Limitations
\end{minipage} \\
\midrule()
\endhead
\_\_\_\_\_ & \_\_\_\_\_\_\_\_\_\_\_ & \_\_\_\_\_\_\_\_\_\_\_\_\_\_\_ & \_\_\_\_\_\_\_\_\_\_\_\_\_\_\_\_\_\_ \\
\_\_\_\_\_ & \_\_\_\_\_\_\_\_\_\_\_ & \_\_\_\_\_\_\_\_\_\_\_\_\_\_\_ & \_\_\_\_\_\_\_\_\_\_\_\_\_\_\_\_\_\_ \\
\_\_\_\_\_ & \_\_\_\_\_\_\_\_\_\_\_ & \_\_\_\_\_\_\_\_\_\_\_\_\_\_\_ & \_\_\_\_\_\_\_\_\_\_\_\_\_\_\_\_\_\_ \\
\_\_\_\_\_ & \_\_\_\_\_\_\_\_\_\_\_ & \_\_\_\_\_\_\_\_\_\_\_\_\_\_\_ & \_\_\_\_\_\_\_\_\_\_\_\_\_\_\_\_\_\_ \\
\_\_\_\_\_ & \_\_\_\_\_\_\_\_\_\_\_ & \_\_\_\_\_\_\_\_\_\_\_\_\_\_\_ & \_\_\_\_\_\_\_\_\_\_\_\_\_\_\_\_\_\_ \\
\bottomrule()
\end{longtable}

\begin{longtable}[]{@{}ll@{}}
\toprule()
Initialed: & \\
\midrule()
\endhead
\_\_\_\_\_\_\_\_\_\_\_\_\_\_ & \_\_\_\_\_\_\_\_\_\_\_\_\_\_ \\
Player & Team \\
\bottomrule()
\end{longtable}

\newpage

\hypertarget{exhibit-3---prior-injury-exclusion}{%
\subsection{Exhibit 3 - Prior Injury Exclusion}\label{exhibit-3---prior-injury-exclusion}}

\begin{longtable}[]{@{}l@{}}
\toprule()
\endhead
Player: \_\_\_\_\_\_\_\_\_\_\_\_\_\_\_\_\_\_\_\_\_\_\_\_\_\_\_\_\_\_\_\_ \\
Team: \_\_\_\_\_\_\_\_\_\_\_\_\_\_\_\_\_\_\_\_\_\_\_\_\_\_\_\_\_\_\_\_\_\_ \\
Date: \_\_\_\_\_\_\_\_\_\_\_\_\_\_\_\_\_\_\_\_\_\_\_\_\_\_\_\_\_\_\_\_\_\_ \\
\bottomrule()
\end{longtable}

The Player's right to receive his Compensation as set forth in paragraphs 7(c), 16(a)(iii), 16(b) of this Contract, or otherwise is limited or eliminated with respect to the following reinjury of the injury or aggravation of the condition set forth below:

\begin{longtable}[]{@{}l@{}}
\toprule()
Describe injury or condition: \\
\midrule()
\endhead
\_\_\_\_\_\_\_\_\_\_\_\_\_\_\_\_\_\_\_\_\_\_\_\_\_\_\_\_\_\_\_\_\_\_\_\_\_\_\_\_\_\_\_\_\_\_\_\_\_\_\_\_\_\_\_\_\_\_\_\_\_ \\
\_\_\_\_\_\_\_\_\_\_\_\_\_\_\_\_\_\_\_\_\_\_\_\_\_\_\_\_\_\_\_\_\_\_\_\_\_\_\_\_\_\_\_\_\_\_\_\_\_\_\_\_\_\_\_\_\_\_\_\_\_ \\
\_\_\_\_\_\_\_\_\_\_\_\_\_\_\_\_\_\_\_\_\_\_\_\_\_\_\_\_\_\_\_\_\_\_\_\_\_\_\_\_\_\_\_\_\_\_\_\_\_\_\_\_\_\_\_\_\_\_\_\_\_ \\
\_\_\_\_\_\_\_\_\_\_\_\_\_\_\_\_\_\_\_\_\_\_\_\_\_\_\_\_\_\_\_\_\_\_\_\_\_\_\_\_\_\_\_\_\_\_\_\_\_\_\_\_\_\_\_\_\_\_\_\_\_ \\
\_\_\_\_\_\_\_\_\_\_\_\_\_\_\_\_\_\_\_\_\_\_\_\_\_\_\_\_\_\_\_\_\_\_\_\_\_\_\_\_\_\_\_\_\_\_\_\_\_\_\_\_\_\_\_\_\_\_\_\_\_ \\
\_\_\_\_\_\_\_\_\_\_\_\_\_\_\_\_\_\_\_\_\_\_\_\_\_\_\_\_\_\_\_\_\_\_\_\_\_\_\_\_\_\_\_\_\_\_\_\_\_\_\_\_\_\_\_\_\_\_\_\_\_ \\
\_\_\_\_\_\_\_\_\_\_\_\_\_\_\_\_\_\_\_\_\_\_\_\_\_\_\_\_\_\_\_\_\_\_\_\_\_\_\_\_\_\_\_\_\_\_\_\_\_\_\_\_\_\_\_\_\_\_\_\_\_ \\
\_\_\_\_\_\_\_\_\_\_\_\_\_\_\_\_\_\_\_\_\_\_\_\_\_\_\_\_\_\_\_\_\_\_\_\_\_\_\_\_\_\_\_\_\_\_\_\_\_\_\_\_\_\_\_\_\_\_\_\_\_ \\
\_\_\_\_\_\_\_\_\_\_\_\_\_\_\_\_\_\_\_\_\_\_\_\_\_\_\_\_\_\_\_\_\_\_\_\_\_\_\_\_\_\_\_\_\_\_\_\_\_\_\_\_\_\_\_\_\_\_\_\_\_ \\
\_\_\_\_\_\_\_\_\_\_\_\_\_\_\_\_\_\_\_\_\_\_\_\_\_\_\_\_\_\_\_\_\_\_\_\_\_\_\_\_\_\_\_\_\_\_\_\_\_\_\_\_\_\_\_\_\_\_\_\_\_ \\
\_\_\_\_\_\_\_\_\_\_\_\_\_\_\_\_\_\_\_\_\_\_\_\_\_\_\_\_\_\_\_\_\_\_\_\_\_\_\_\_\_\_\_\_\_\_\_\_\_\_\_\_\_\_\_\_\_\_\_\_\_ \\
\bottomrule()
\end{longtable}

\begin{longtable}[]{@{}
  >{\raggedright\arraybackslash}p{(\columnwidth - 0\tabcolsep) * \real{1.0000}}@{}}
\toprule()
\begin{minipage}[b]{\linewidth}\raggedright
Describe the extent to which liability for Compensation is limited or eliminated:
\end{minipage} \\
\midrule()
\endhead
\_\_\_\_\_\_\_\_\_\_\_\_\_\_\_\_\_\_\_\_\_\_\_\_\_\_\_\_\_\_\_\_\_\_\_\_\_\_\_\_\_\_\_\_\_\_\_\_\_\_\_\_\_\_\_\_\_\_\_\_\_ \\
\_\_\_\_\_\_\_\_\_\_\_\_\_\_\_\_\_\_\_\_\_\_\_\_\_\_\_\_\_\_\_\_\_\_\_\_\_\_\_\_\_\_\_\_\_\_\_\_\_\_\_\_\_\_\_\_\_\_\_\_\_ \\
\_\_\_\_\_\_\_\_\_\_\_\_\_\_\_\_\_\_\_\_\_\_\_\_\_\_\_\_\_\_\_\_\_\_\_\_\_\_\_\_\_\_\_\_\_\_\_\_\_\_\_\_\_\_\_\_\_\_\_\_\_ \\
\_\_\_\_\_\_\_\_\_\_\_\_\_\_\_\_\_\_\_\_\_\_\_\_\_\_\_\_\_\_\_\_\_\_\_\_\_\_\_\_\_\_\_\_\_\_\_\_\_\_\_\_\_\_\_\_\_\_\_\_\_ \\
\_\_\_\_\_\_\_\_\_\_\_\_\_\_\_\_\_\_\_\_\_\_\_\_\_\_\_\_\_\_\_\_\_\_\_\_\_\_\_\_\_\_\_\_\_\_\_\_\_\_\_\_\_\_\_\_\_\_\_\_\_ \\
\_\_\_\_\_\_\_\_\_\_\_\_\_\_\_\_\_\_\_\_\_\_\_\_\_\_\_\_\_\_\_\_\_\_\_\_\_\_\_\_\_\_\_\_\_\_\_\_\_\_\_\_\_\_\_\_\_\_\_\_\_ \\
\_\_\_\_\_\_\_\_\_\_\_\_\_\_\_\_\_\_\_\_\_\_\_\_\_\_\_\_\_\_\_\_\_\_\_\_\_\_\_\_\_\_\_\_\_\_\_\_\_\_\_\_\_\_\_\_\_\_\_\_\_ \\
\_\_\_\_\_\_\_\_\_\_\_\_\_\_\_\_\_\_\_\_\_\_\_\_\_\_\_\_\_\_\_\_\_\_\_\_\_\_\_\_\_\_\_\_\_\_\_\_\_\_\_\_\_\_\_\_\_\_\_\_\_ \\
\_\_\_\_\_\_\_\_\_\_\_\_\_\_\_\_\_\_\_\_\_\_\_\_\_\_\_\_\_\_\_\_\_\_\_\_\_\_\_\_\_\_\_\_\_\_\_\_\_\_\_\_\_\_\_\_\_\_\_\_\_ \\
\_\_\_\_\_\_\_\_\_\_\_\_\_\_\_\_\_\_\_\_\_\_\_\_\_\_\_\_\_\_\_\_\_\_\_\_\_\_\_\_\_\_\_\_\_\_\_\_\_\_\_\_\_\_\_\_\_\_\_\_\_ \\
\_\_\_\_\_\_\_\_\_\_\_\_\_\_\_\_\_\_\_\_\_\_\_\_\_\_\_\_\_\_\_\_\_\_\_\_\_\_\_\_\_\_\_\_\_\_\_\_\_\_\_\_\_\_\_\_\_\_\_\_\_ \\
\bottomrule()
\end{longtable}

\begin{longtable}[]{@{}ll@{}}
\toprule()
Initialed: & \\
\midrule()
\endhead
\_\_\_\_\_\_\_\_\_\_\_\_\_\_ & \_\_\_\_\_\_\_\_\_\_\_\_\_\_ \\
Player & Team \\
\bottomrule()
\end{longtable}

\newpage

\hypertarget{exhibit-4---assignment-payments}{%
\subsection{Exhibit 4 - Assignment Payments}\label{exhibit-4---assignment-payments}}

\begin{longtable}[]{@{}l@{}}
\toprule()
\endhead
Player: \_\_\_\_\_\_\_\_\_\_\_\_\_\_\_\_\_\_\_\_\_\_\_\_\_\_\_\_\_\_\_\_ \\
Team: \_\_\_\_\_\_\_\_\_\_\_\_\_\_\_\_\_\_\_\_\_\_\_\_\_\_\_\_\_\_\_\_\_\_ \\
Date: \_\_\_\_\_\_\_\_\_\_\_\_\_\_\_\_\_\_\_\_\_\_\_\_\_\_\_\_\_\_\_\_\_\_ \\
\bottomrule()
\end{longtable}

In the event this Contract is traded by the Team executing the Contract to another NBA Team, the Player shall be entitled to receive from the assignor Team, within thirty (30) days of the date of such trade, the following payment:

\begin{longtable}[]{@{}l@{}}
\toprule()
\endhead
\_\_\_\_\_\_\_\_\_\_\_\_\_\_\_\_\_\_\_\_\_\_\_\_\_\_\_\_\_\_\_\_\_\_\_\_\_\_\_\_\_\_\_\_\_\_\_\_\_\_\_\_\_\_\_\_\_\_\_\_\_ \\
\_\_\_\_\_\_\_\_\_\_\_\_\_\_\_\_\_\_\_\_\_\_\_\_\_\_\_\_\_\_\_\_\_\_\_\_\_\_\_\_\_\_\_\_\_\_\_\_\_\_\_\_\_\_\_\_\_\_\_\_\_ \\
\_\_\_\_\_\_\_\_\_\_\_\_\_\_\_\_\_\_\_\_\_\_\_\_\_\_\_\_\_\_\_\_\_\_\_\_\_\_\_\_\_\_\_\_\_\_\_\_\_\_\_\_\_\_\_\_\_\_\_\_\_ \\
\_\_\_\_\_\_\_\_\_\_\_\_\_\_\_\_\_\_\_\_\_\_\_\_\_\_\_\_\_\_\_\_\_\_\_\_\_\_\_\_\_\_\_\_\_\_\_\_\_\_\_\_\_\_\_\_\_\_\_\_\_ \\
\_\_\_\_\_\_\_\_\_\_\_\_\_\_\_\_\_\_\_\_\_\_\_\_\_\_\_\_\_\_\_\_\_\_\_\_\_\_\_\_\_\_\_\_\_\_\_\_\_\_\_\_\_\_\_\_\_\_\_\_\_ \\
\_\_\_\_\_\_\_\_\_\_\_\_\_\_\_\_\_\_\_\_\_\_\_\_\_\_\_\_\_\_\_\_\_\_\_\_\_\_\_\_\_\_\_\_\_\_\_\_\_\_\_\_\_\_\_\_\_\_\_\_\_ \\
\_\_\_\_\_\_\_\_\_\_\_\_\_\_\_\_\_\_\_\_\_\_\_\_\_\_\_\_\_\_\_\_\_\_\_\_\_\_\_\_\_\_\_\_\_\_\_\_\_\_\_\_\_\_\_\_\_\_\_\_\_ \\
\_\_\_\_\_\_\_\_\_\_\_\_\_\_\_\_\_\_\_\_\_\_\_\_\_\_\_\_\_\_\_\_\_\_\_\_\_\_\_\_\_\_\_\_\_\_\_\_\_\_\_\_\_\_\_\_\_\_\_\_\_ \\
\_\_\_\_\_\_\_\_\_\_\_\_\_\_\_\_\_\_\_\_\_\_\_\_\_\_\_\_\_\_\_\_\_\_\_\_\_\_\_\_\_\_\_\_\_\_\_\_\_\_\_\_\_\_\_\_\_\_\_\_\_ \\
\_\_\_\_\_\_\_\_\_\_\_\_\_\_\_\_\_\_\_\_\_\_\_\_\_\_\_\_\_\_\_\_\_\_\_\_\_\_\_\_\_\_\_\_\_\_\_\_\_\_\_\_\_\_\_\_\_\_\_\_\_ \\
\_\_\_\_\_\_\_\_\_\_\_\_\_\_\_\_\_\_\_\_\_\_\_\_\_\_\_\_\_\_\_\_\_\_\_\_\_\_\_\_\_\_\_\_\_\_\_\_\_\_\_\_\_\_\_\_\_\_\_\_\_ \\
\_\_\_\_\_\_\_\_\_\_\_\_\_\_\_\_\_\_\_\_\_\_\_\_\_\_\_\_\_\_\_\_\_\_\_\_\_\_\_\_\_\_\_\_\_\_\_\_\_\_\_\_\_\_\_\_\_\_\_\_\_ \\
\_\_\_\_\_\_\_\_\_\_\_\_\_\_\_\_\_\_\_\_\_\_\_\_\_\_\_\_\_\_\_\_\_\_\_\_\_\_\_\_\_\_\_\_\_\_\_\_\_\_\_\_\_\_\_\_\_\_\_\_\_ \\
\_\_\_\_\_\_\_\_\_\_\_\_\_\_\_\_\_\_\_\_\_\_\_\_\_\_\_\_\_\_\_\_\_\_\_\_\_\_\_\_\_\_\_\_\_\_\_\_\_\_\_\_\_\_\_\_\_\_\_\_\_ \\
\_\_\_\_\_\_\_\_\_\_\_\_\_\_\_\_\_\_\_\_\_\_\_\_\_\_\_\_\_\_\_\_\_\_\_\_\_\_\_\_\_\_\_\_\_\_\_\_\_\_\_\_\_\_\_\_\_\_\_\_\_ \\
\bottomrule()
\end{longtable}

\begin{longtable}[]{@{}ll@{}}
\toprule()
Initialed: & \\
\midrule()
\endhead
\_\_\_\_\_\_\_\_\_\_\_\_\_\_ & \_\_\_\_\_\_\_\_\_\_\_\_\_\_ \\
Player & Team \\
\bottomrule()
\end{longtable}

\newpage

\hypertarget{exhibit-5---other-sports-activities}{%
\subsection{Exhibit 5 - Other Sports Activities}\label{exhibit-5---other-sports-activities}}

\begin{longtable}[]{@{}l@{}}
\toprule()
\endhead
Player: \_\_\_\_\_\_\_\_\_\_\_\_\_\_\_\_\_\_\_\_\_\_\_\_\_\_\_\_\_\_\_\_ \\
Team: \_\_\_\_\_\_\_\_\_\_\_\_\_\_\_\_\_\_\_\_\_\_\_\_\_\_\_\_\_\_\_\_\_\_ \\
Date: \_\_\_\_\_\_\_\_\_\_\_\_\_\_\_\_\_\_\_\_\_\_\_\_\_\_\_\_\_\_\_\_\_\_ \\
\bottomrule()
\end{longtable}

Notwithstanding the provisions of paragraph 12 of this Contract, the Player and the Team agree that the Player need not obtain the consent of the Team in order to engage in the activities set forth below:

\begin{longtable}[]{@{}l@{}}
\toprule()
\endhead
\_\_\_\_\_\_\_\_\_\_\_\_\_\_\_\_\_\_\_\_\_\_\_\_\_\_\_\_\_\_\_\_\_\_\_\_\_\_\_\_\_\_\_\_\_\_\_\_\_\_\_\_\_\_\_\_\_\_\_\_\_ \\
\_\_\_\_\_\_\_\_\_\_\_\_\_\_\_\_\_\_\_\_\_\_\_\_\_\_\_\_\_\_\_\_\_\_\_\_\_\_\_\_\_\_\_\_\_\_\_\_\_\_\_\_\_\_\_\_\_\_\_\_\_ \\
\_\_\_\_\_\_\_\_\_\_\_\_\_\_\_\_\_\_\_\_\_\_\_\_\_\_\_\_\_\_\_\_\_\_\_\_\_\_\_\_\_\_\_\_\_\_\_\_\_\_\_\_\_\_\_\_\_\_\_\_\_ \\
\_\_\_\_\_\_\_\_\_\_\_\_\_\_\_\_\_\_\_\_\_\_\_\_\_\_\_\_\_\_\_\_\_\_\_\_\_\_\_\_\_\_\_\_\_\_\_\_\_\_\_\_\_\_\_\_\_\_\_\_\_ \\
\_\_\_\_\_\_\_\_\_\_\_\_\_\_\_\_\_\_\_\_\_\_\_\_\_\_\_\_\_\_\_\_\_\_\_\_\_\_\_\_\_\_\_\_\_\_\_\_\_\_\_\_\_\_\_\_\_\_\_\_\_ \\
\_\_\_\_\_\_\_\_\_\_\_\_\_\_\_\_\_\_\_\_\_\_\_\_\_\_\_\_\_\_\_\_\_\_\_\_\_\_\_\_\_\_\_\_\_\_\_\_\_\_\_\_\_\_\_\_\_\_\_\_\_ \\
\_\_\_\_\_\_\_\_\_\_\_\_\_\_\_\_\_\_\_\_\_\_\_\_\_\_\_\_\_\_\_\_\_\_\_\_\_\_\_\_\_\_\_\_\_\_\_\_\_\_\_\_\_\_\_\_\_\_\_\_\_ \\
\_\_\_\_\_\_\_\_\_\_\_\_\_\_\_\_\_\_\_\_\_\_\_\_\_\_\_\_\_\_\_\_\_\_\_\_\_\_\_\_\_\_\_\_\_\_\_\_\_\_\_\_\_\_\_\_\_\_\_\_\_ \\
\_\_\_\_\_\_\_\_\_\_\_\_\_\_\_\_\_\_\_\_\_\_\_\_\_\_\_\_\_\_\_\_\_\_\_\_\_\_\_\_\_\_\_\_\_\_\_\_\_\_\_\_\_\_\_\_\_\_\_\_\_ \\
\_\_\_\_\_\_\_\_\_\_\_\_\_\_\_\_\_\_\_\_\_\_\_\_\_\_\_\_\_\_\_\_\_\_\_\_\_\_\_\_\_\_\_\_\_\_\_\_\_\_\_\_\_\_\_\_\_\_\_\_\_ \\
\_\_\_\_\_\_\_\_\_\_\_\_\_\_\_\_\_\_\_\_\_\_\_\_\_\_\_\_\_\_\_\_\_\_\_\_\_\_\_\_\_\_\_\_\_\_\_\_\_\_\_\_\_\_\_\_\_\_\_\_\_ \\
\_\_\_\_\_\_\_\_\_\_\_\_\_\_\_\_\_\_\_\_\_\_\_\_\_\_\_\_\_\_\_\_\_\_\_\_\_\_\_\_\_\_\_\_\_\_\_\_\_\_\_\_\_\_\_\_\_\_\_\_\_ \\
\_\_\_\_\_\_\_\_\_\_\_\_\_\_\_\_\_\_\_\_\_\_\_\_\_\_\_\_\_\_\_\_\_\_\_\_\_\_\_\_\_\_\_\_\_\_\_\_\_\_\_\_\_\_\_\_\_\_\_\_\_ \\
\_\_\_\_\_\_\_\_\_\_\_\_\_\_\_\_\_\_\_\_\_\_\_\_\_\_\_\_\_\_\_\_\_\_\_\_\_\_\_\_\_\_\_\_\_\_\_\_\_\_\_\_\_\_\_\_\_\_\_\_\_ \\
\_\_\_\_\_\_\_\_\_\_\_\_\_\_\_\_\_\_\_\_\_\_\_\_\_\_\_\_\_\_\_\_\_\_\_\_\_\_\_\_\_\_\_\_\_\_\_\_\_\_\_\_\_\_\_\_\_\_\_\_\_ \\
\bottomrule()
\end{longtable}

\begin{longtable}[]{@{}ll@{}}
\toprule()
Initialed: & \\
\midrule()
\endhead
\_\_\_\_\_\_\_\_\_\_\_\_\_\_ & \_\_\_\_\_\_\_\_\_\_\_\_\_\_ \\
Player & Team \\
\bottomrule()
\end{longtable}

\newpage

\hypertarget{exhibit-6---physical-exam}{%
\subsection{Exhibit 6 - Physical Exam}\label{exhibit-6---physical-exam}}

\begin{longtable}[]{@{}l@{}}
\toprule()
\endhead
Player: \_\_\_\_\_\_\_\_\_\_\_\_\_\_\_\_\_\_\_\_\_\_\_\_\_\_\_\_\_\_\_\_ \\
Team: \_\_\_\_\_\_\_\_\_\_\_\_\_\_\_\_\_\_\_\_\_\_\_\_\_\_\_\_\_\_\_\_\_\_ \\
Date: \_\_\_\_\_\_\_\_\_\_\_\_\_\_\_\_\_\_\_\_\_\_\_\_\_\_\_\_\_\_\_\_\_\_ \\
\bottomrule()
\end{longtable}

The Player and the Team agree that this Contract will be invalid and of no further force and effect unless the Player passes, in the sole discretion of a physician designated by the Team, a physical examination conducted within forty-eight hours of the execution of this Contract.

\begin{longtable}[]{@{}ll@{}}
\toprule()
Initialed: & \\
\midrule()
\endhead
\_\_\_\_\_\_\_\_\_\_\_\_\_\_ & \_\_\_\_\_\_\_\_\_\_\_\_\_\_ \\
Player & Team \\
\bottomrule()
\end{longtable}

\newpage

\hypertarget{exhibit-7---substitute-for-upc-paragraph-7b}{%
\subsection{Exhibit 7 - Substitute for UPC paragraph 7(b)}\label{exhibit-7---substitute-for-upc-paragraph-7b}}

\begin{longtable}[]{@{}l@{}}
\toprule()
\endhead
Player: \_\_\_\_\_\_\_\_\_\_\_\_\_\_\_\_\_\_\_\_\_\_\_\_\_\_\_\_\_\_\_\_ \\
Team: \_\_\_\_\_\_\_\_\_\_\_\_\_\_\_\_\_\_\_\_\_\_\_\_\_\_\_\_\_\_\_\_\_\_ \\
Date: \_\_\_\_\_\_\_\_\_\_\_\_\_\_\_\_\_\_\_\_\_\_\_\_\_\_\_\_\_\_\_\_\_\_ \\
\bottomrule()
\end{longtable}

Paragraph 7(b) is hereby deleted and the following shall be substituted in place and instead thereof:

``7. (b) The Player agrees, notwithstanding any other provision of this Contract, that he will to the best of his ability maintain himself in physical condition sufficient to play skilled basketball at all times. If the Player, in the reasonable judgment of the physician designated for that purpose by the Team, is not in good physical condition at the date of his first scheduled game for the Team, of if, at the beginning of or during any Season, he fails to remain in good physical condition, in either event so as to render the Player unfit in the reasonable judgment of said physician to play skilled basketball, the Team shall have the right to suspend the Player for a period of one week. At the end of such one-week period, should the Team notify the Player, orally or in writing, that in its reasonable judgment it believes the Player is still not in good physical condition, the Team shall have the right to suspend the Player for successive one-week periods until the Player, in the reasonable judgment of the Team's physician, is in good physical condition; provided, however, that at the end of each such one-week period of suspension, the Team provides the notice specified above, and provided further that, at the end of any such one-week period and if the player so requests, then the Player shall be examined by a physician or physicians designated for such purpose by the President, or any Vice President if the President is not available, of the American Society of Orthopedic Physicians, or equivalent organization (the''Reviewing Physician''), whose sole judgment concerning the physical condition of the Player to play skilled basketball shall be binding upon the Team and the Player for purposes of this paragraph. The suspension of.the Player shall be terminated promptly upon the failure of the Team to give the Player the notice required at the end of the one-week period or upon the finding of said Reviewing Physician that the Player is in physical condition sufficient to play skilled basketball. In the event of a suspension permitted hereunder, the annual sum payable to the Player under this Contract shall be reduced in the same proportion as the length of the period of disability so determined bears to the length of the season.''

\begin{longtable}[]{@{}ll@{}}
\toprule()
Initialed: & \\
\midrule()
\endhead
\_\_\_\_\_\_\_\_\_\_\_\_\_\_ & \_\_\_\_\_\_\_\_\_\_\_\_\_\_ \\
Player & Team \\
\bottomrule()
\end{longtable}

\hypertarget{rookie-scales}{%
\chapter{ROOKIE SCALES}\label{rookie-scales}}

\newpage

\hypertarget{nba-rookie-scale-000s}{%
\section{1995-96 NBA Rookie Scale (\$000's)}\label{nba-rookie-scale-000s}}

\begin{longtable}[]{@{}clll@{}}
\toprule()
Pick & 1st Year Salary & 2nd Year Salary & 3rd Year Salary \\
\midrule()
\endhead
1 & 2,061.0 & 2,370.2 & 2,679.3 \\
2 & 1,844.0 & 2,120.6 & 2,397.2 \\
3 & 1,656.0 & 1,904.4 & 2,152.8 \\
4 & 1,493.0 & 1,717.0 & 1,940.9 \\
5 & 1,352.0 & 1,554.8 & 1,757.6 \\
6 & 1,228.0 & 1,412.2 & 1,596.4 \\
7 & 1,121.0 & 1,289.2 & 1,457.3 \\
8 & 1,027.0 & 1,181.1 & 1,335.1 \\
9 & 944.0 & 1,085.6 & 1,227.2 \\
10 & 896.8 & 1,031.3 & 1,165.8 \\
11 & 852.0 & 979.8 & 1,107.5 \\
12 & 809.4 & 930.8 & 1,052.2 \\
13 & 768.9 & 884.2 & 999.6 \\
14 & 730.4 & 840.0 & 949.6 \\
15 & 693.9 & 798.0 & 902.1 \\
16 & 659.2 & 758.1 & 857.0 \\
17 & 626.3 & 720.2 & 814.1 \\
18 & 595.0 & 684.2 & 773.4 \\
19 & 568.2 & 653.4 & 738.6 \\
20 & 545.5 & 627.3 & 709.1 \\
21 & 523.6 & 602.2 & 680.7 \\
22 & 502.7 & 578.1 & 653.5 \\
23 & 482.6 & 555.0 & 627.4 \\
24 & 463.3 & 532.8 & 602.3 \\
25 & 444.7 & 511.5 & 578.2 \\
26 & 430.0 & 494.5 & 559.0 \\
27 & 417.6 & 480.2 & 542.9 \\
28 & 415.0 & 477.3 & 539.5 \\
29 & 412.0 & 473.8 & 535.6 \\
\bottomrule()
\end{longtable}

\newpage

\hypertarget{nba-rookie-scale-000s-1}{%
\section{1996-97 NBA Rookie Scale (\$000's)}\label{nba-rookie-scale-000s-1}}

\begin{longtable}[]{@{}clll@{}}
\toprule()
Pick & 1st Year Salary & 2nd Year Salary & 3rd Year Salary \\
\midrule()
\endhead
1 & 2,267.1 & 2,607.2 & 2,947.2 \\
2 & 2,028.4 & 2,332.7 & 2,636.9 \\
3 & 1,821.6 & 2,094.8 & 2,368.1 \\
4 & 1,642.3 & 1,888.6 & 2,135.0 \\
5 & 1,487.3 & 1,710.3 & 1,933.4 \\
6 & 1,350.8 & 1,553.4 & 1,756.0 \\
7 & 1,233.1 & 1,418.1 & 1,603.0 \\
8 & 1,129.7 & 1,299.2 & 1,468.6 \\
9 & 1,038.4 & 1,194.2 & 1,349.9 \\
10 & 986.5 & 1,134.5 & 1,282.4 \\
11 & 937.2 & 1,077.7 & 1,218.3 \\
12 & 890.3 & 1,023.8 & 1,157.4 \\
13 & 845.8 & 972.7 & 1,099.5 \\
14 & 803.5 & 924.0 & 1,044.5 \\
15 & 763.3 & 877.8 & 992.3 \\
16 & 725.2 & 833.9 & 942.7 \\
17 & 688.9 & 792.2 & 895.6 \\
18 & 654.5 & 752.6 & 850.8 \\
19 & 625.0 & 718.8 & 812.5 \\
20 & 600.0 & 690.0 & 780.0 \\
21 & 576.0 & 662.4 & 748.8 \\
22 & 553.0 & 635.9 & 718.8 \\
23 & 530.8 & 610.5 & 690.1 \\
24 & 509.6 & 586.0 & 662.5 \\
25 & 489.2 & 562.6 & 636.0 \\
26 & 473.0 & 544.0 & 614.9 \\
27 & 459.4 & 528.3 & 597.2 \\
28 & 456.5 & 525.0 & 593.5 \\
29 & 453.2 & 521.2 & 589.2 \\
\bottomrule()
\end{longtable}

\newpage

\hypertarget{nba-rookie-scale-000s-2}{%
\section{1997-98 NBA Rookie Scale (\$000's)}\label{nba-rookie-scale-000s-2}}

\begin{longtable}[]{@{}clll@{}}
\toprule()
Pick & 1st Year Salary & 2nd Year Salary & 3rd Year Salary \\
\midrule()
\endhead
1 & 2,473.2 & 2,844.2 & 3,215.2 \\
2 & 2,212.8 & 2,544.7 & 2,876.6 \\
3 & 1,987.2 & 2,285.3 & 2,583.4 \\
4 & 1,791.6 & 2,060.3 & 2,329.1 \\
5 & 1,622.4 & 1,865.8 & 2,109.1 \\
6 & 1,473.6 & 1,694.6 & 1,915.7 \\
7 & 1,345.2 & 1,547.0 & 1,748.8 \\
8 & 1,232.4 & 1,417.3 & 1,602.1 \\
9 & 1,132.8 & 1,302.7 & 1,472.6 \\
10 & 1,076.2 & 1,237.6 & 1,399.0 \\
11 & 1,022.4 & 1,175.7 & 1,329.1 \\
12 & 971.2 & 1,116.9 & 1,262.6 \\
13 & 922.7 & 1,061.1 & 1,199.5 \\
14 & 876.5 & 1,008.0 & 1,139.5 \\
15 & 832.7 & 957.6 & 1,082.5 \\
16 & 791.1 & 909.7 & 1,028.4 \\
17 & 751.5 & 864.3 & 977.0 \\
18 & 713.9 & 821.0 & 928.1 \\
19 & 681.8 & 784.1 & 886.4 \\
20 & 654.5 & 752.7 & 850.9 \\
21 & 628.4 & 722.6 & 816.9 \\
22 & 603.2 & 693.7 & 784.2 \\
23 & 579.1 & 666.0 & 752.8 \\
24 & 555.9 & 639.3 & 722.7 \\
25 & 533.7 & 613.8 & 693.8 \\
26 & 516.0 & 593.4 & 670.8 \\
27 & 501.1 & 576.3 & 651.5 \\
28 & 498.0 & 572.7 & 647.4 \\
29 & 494.4 & 568.6 & 642.7 \\
\bottomrule()
\end{longtable}

\newpage

\hypertarget{nba-rookie-scale-ooos}{%
\section{1998-99 NBA Rookie Scale (\$OOO's)}\label{nba-rookie-scale-ooos}}

\begin{longtable}[]{@{}clll@{}}
\toprule()
Pick & 1st Year Salary & 2nd Year Salary & 3rd Year Salary \\
\midrule()
\endhead
1 & 2,679.3 & 3,081.2 & 3,483.1 \\
2 & 2,397.2 & 2,756.8 & 3,116.4 \\
3 & 2,152.8 & 2,475.7 & 2,798.6 \\
4 & 1,940.9 & 2,232.0 & 2,523.2 \\
5 & 1,757.6 & 2,021.2 & 2,284.9 \\
6 & 1,596.4 & 1,835.9 & 2,075.3 \\
7 & 1,457.3 & 1,675.9 & 1,894.5 \\
8 & 1,335.1 & 1,535.4 & 1,735.6 \\
9 & 1,227.2 & 1,411.3 & 1,595.4 \\
10 & 1,165.8 & 1,340.7 & 1,515.6 \\
11 & 1,107.5 & 1,273.7 & 1,439.8 \\
12 & 1,052.2 & 1,210.0 & 1,367.8 \\
13 & 999.6 & 1,149.5 & 1,299.4 \\
14 & 949.6 & 1,092.0 & 1,234.5 \\
15 & 902.1 & 1,037.4 & 1,172.7 \\
16 & 857.0 & 985.5 & 1,114.1 \\
17 & 814.1 & 936.3 & 1,058.4 \\
18 & 773.4 & 889.5 & 1,005.5 \\
19 & 738.6 & 849.4 & 960.2 \\
20 & 709.1 & 815.5 & 921.8 \\
21 & 680.7 & 782.8 & 884.9 \\
22 & 653.5 & 751.5 & 849.5 \\
23 & 627.4 & 721.5 & 815.6 \\
24 & 602.3 & 692.6 & 782.9 \\
25 & 578.2 & 664.9 & 751.6 \\
26 & 559.0 & 642.9 & 726.7 \\
27 & 542.9 & 624.3 & 705.7 \\
28 & 539.5 & 620.4 & 701.4 \\
29 & 535.6 & 615.9 & 696.3 \\
\bottomrule()
\end{longtable}

\newpage

\hypertarget{nba-rookie-scale-ooos-1}{%
\section{1999-2000 NBA Rookie Scale (\$OOO's)}\label{nba-rookie-scale-ooos-1}}

\begin{longtable}[]{@{}clll@{}}
\toprule()
Pick & 1st Year Salary & 2nd Year Salary & 3rd Year Salary \\
\midrule()
\endhead
1 & 2,885.4 & 3,318.2 & 3,751.0 \\
2 & 2,581.6 & 2,968.8 & 3,356.1 \\
3 & 2,318.4 & 2,666.2 & 3,013.9 \\
4 & 2,090.2 & 2,403.7 & 2,717.3 \\
5 & 1,892.8 & 2,176.7 & 2,460.6 \\
6 & 1,719.2 & 1,977.1 & 2,235.0 \\
7 & 1,569.4 & 1,804.8 & 2,040.2 \\
8 & 1,437.8 & 1,653.5 & 1,869.1 \\
9 & 1,321.6 & 1,519.8 & 1,718.1 \\
10 & 1,255.5 & 1,443.8 & 1,632.2 \\
11 & 1,192.7 & 1,371.7 & 1,550.6 \\
12 & 1,133.1 & 1,303.1 & 1,473.0 \\
13 & 1,076.5 & 1,237.9 & 1,399.4 \\
14 & 1,022.6 & 1,176.0 & 1,329.4 \\
15 & 971.5 & 1,117.2 & 1,262.9 \\
16 & 922.9 & 1,061.4 & 1,199.8 \\
17 & 876.8 & 1,008.3 & 1,139.8 \\
18 & 832.9 & 957.9 & 1,082.8 \\
19 & 795.5 & 914.8 & 1,034.1 \\
20 & 763.6 & 878.2 & 992.7 \\
21 & 733.1 & 843.1 & 953.0 \\
22 & 703.8 & 809.3 & 914.9 \\
23 & 675.6 & 777.0 & 878.3 \\
24 & 648.6 & 745.9 & 843.2 \\
25 & 622.6 & 716.0 & 809.4 \\
26 & 602.0 & 692.3 & 782.6 \\
27 & 584.6 & 672.3 & 760.0 \\
28 & 581.0 & 668.2 & 755.3 \\
29 & 576.8 & 663.3 & 749.8 \\
\bottomrule()
\end{longtable}

\newpage

\hypertarget{nba-rookie-scale-000s-3}{%
\section{2000-2001 NBA Rookie Scale (\$000's)}\label{nba-rookie-scale-000s-3}}

\begin{longtable}[]{@{}clll@{}}
\toprule()
Pick & 1st Year Salary & 2nd Year Salary & 3rd Year Salary \\
\midrule()
\endhead
1 & 3,091.5 & 3,555.2 & 4,019.0 \\
2 & 2,766.0 & 3,180.9 & 3,595.8 \\
3 & 2,484.0 & 2,856.6 & 3,229.2 \\
4 & 2,239.5 & 2,575.4 & 2,911.4 \\
5 & 2,028.0 & 2,332.2 & 2,636.4 \\
6 & 1,842.0 & 2,118.3 & 2,394.6 \\
7 & 1,681.5 & 1,933.7 & 2,186.0 \\
8 & 1,540.5 & 1,771.6 & 2,002.7 \\
9 & 1,416.0 & 1,628.4 & 1,840.8 \\
10 & 1,345.2 & 1,547.0 & 1,748.8 \\
11 & 1,277.9 & 1,469.6 & 1,661.3 \\
12 & 1,214.0 & 1,396.1 & 1,578.3 \\
13 & 1,153.3 & 1,326.3 & 1,499.3 \\
14 & 1,095.7 & 1,260.0 & 1,424.4 \\
15 & 1,040.9 & 1,197.0 & 1,353.2 \\
16 & 988.8 & 1,137.2 & 1,285.5 \\
17 & 939.4 & 1,080.3 & 1,221.2 \\
18 & 892.4 & 1,026.3 & 1,160.2 \\
19 & 852.3 & 980.1 & 1,108.0 \\
20 & 818.2 & 940.9 & 1,063.6 \\
21 & 785.5 & 903.3 & 1,021.1 \\
22 & 754.0 & 867.1 & 980.2 \\
23 & 723.9 & 832.5 & 941.0 \\
24 & 694.9 & 799.2 & 903.4 \\
25 & 667.1 & 767.2 & 867.3 \\
26 & 645.0 & 741.8 & 838.5 \\
27 & 626.4 & 720.4 & 814.3 \\
28 & 622.5 & 715.9 & 809.3 \\
29 & 618.0 & 710.7 & 803.4 \\
\bottomrule()
\end{longtable}

\hypertarget{bri-expense-ratios}{%
\chapter{BRI EXPENSE RATIOS}\label{bri-expense-ratios}}

Article VII. Section 1(a)(l)(v), (vi)

\begin{longtable}[]{@{}lc@{}}
\toprule()
Category & Ratio of Expenses to Revenues \\
\midrule()
\endhead
Novelties and Concessions & 50\% \\
Game Parking & Accountants to determine \\
Game Programs & 25\% \\
Team Sponsorships and Promotions & 34\% \\
In-arena signage & Accountants to determine \\
In-arena Club & Accountants to determine \\
\bottomrule()
\end{longtable}

Article VII. Section 1(a)(viii)

\begin{longtable}[]{@{}lc@{}}
\toprule()
Category & Ratio of Expenses to Revenues \\
\midrule()
\endhead
Sponsorships & 19\% \\
NBA Entertainment & 35\% \\
International Television & 22\% \\
Special Events & 100\% \\
\bottomrule()
\end{longtable}

\hypertarget{related-party}{%
\chapter{RELATED PARTY}\label{related-party}}

\hypertarget{related-party---tv}{%
\section{RELATED PARTY - TV}\label{related-party---tv}}

1993-94 SEASON

\begin{longtable}[]{@{}lccc@{}}
\toprule()
& CFS-NET & ADDITIONS & TOTAL \\
\midrule()
\endhead
Atlanta - Cable & \$900,000 & \$221,000 & \$1,121,000 \\
Chicago - Cable & \$985,000 & \$977,000 & \$1,962,000 \\
New York-All & \$4,860,000 & \$14,931,000 & \$19,791,000 \\
& \$6,745,000 & \$16,129,000 & \$22,874,000 \\
& & \$16,129,000 & \\
& & x 1.08 & \\
& & \$17,419,000 & \\
& & x 1.08 & \\
Addition 95-96 Season & & \$18,813,000 & \\
Lakers: & & & \\
Local TV & & \$12,585,382 & \\
Local Cable & & \$5,066,274 & \\
Local Radio & & \$2,139,388 & \\
& & \$19,791,000 & \\
\bottomrule()
\end{longtable}

\hypertarget{related-parties---other}{%
\section{RELATED PARTIES - OTHER}\label{related-parties---other}}

ALL NUMBERS NET

\begin{longtable}[]{@{}
  >{\raggedright\arraybackslash}p{(\columnwidth - 14\tabcolsep) * \real{0.0960}}
  >{\raggedleft\arraybackslash}p{(\columnwidth - 14\tabcolsep) * \real{0.0880}}
  >{\raggedleft\arraybackslash}p{(\columnwidth - 14\tabcolsep) * \real{0.2720}}
  >{\raggedleft\arraybackslash}p{(\columnwidth - 14\tabcolsep) * \real{0.1040}}
  >{\raggedleft\arraybackslash}p{(\columnwidth - 14\tabcolsep) * \real{0.1200}}
  >{\raggedleft\arraybackslash}p{(\columnwidth - 14\tabcolsep) * \real{0.1040}}
  >{\raggedleft\arraybackslash}p{(\columnwidth - 14\tabcolsep) * \real{0.1040}}
  >{\raggedleft\arraybackslash}p{(\columnwidth - 14\tabcolsep) * \real{0.1120}}@{}}
\toprule()
\begin{minipage}[b]{\linewidth}\raggedright
Team
\end{minipage} & \begin{minipage}[b]{\linewidth}\raggedleft
Programs
\end{minipage} & \begin{minipage}[b]{\linewidth}\raggedleft
Novelties \& Programs Concessions
\end{minipage} & \begin{minipage}[b]{\linewidth}\raggedleft
Signage
\end{minipage} & \begin{minipage}[b]{\linewidth}\raggedleft
Fixed Signage
\end{minipage} & \begin{minipage}[b]{\linewidth}\raggedleft
Sponsorship
\end{minipage} & \begin{minipage}[b]{\linewidth}\raggedleft
Parking
\end{minipage} & \begin{minipage}[b]{\linewidth}\raggedleft
Total
\end{minipage} \\
\midrule()
\endhead
ATLANTA & -0- & \$2,000,000 & -0- & \$ 969,000 & -0- & \$100,000 & \$3,069,000 \\
CHICAGO & -0- & (1) & 1,500,000 & 1,944,000 & -0- & -0- & \$3,444,000 \\
CLEVELAND & -0- & 350,000 & -0- & 1,180,000 & -0- & 100,000 & \$1,630,000 \\
DETROIT & -0- & 1,575,000 & 822,000 & 2,305,000 & -0- & 1,300,000 & \$6,002,000 \\
INDIANA & -0- & 750,000 & 320,000 & 1,559,000 & -0- & -0- & \$2,629,000 \\
LAKERS & -0- & 1,145,000 & -0- & 2,000,000 & -0- & 500,000 & \$3,645,000 \\
MINNESOTA & -0- & -0- & -0- & -0- & -0- & -0- & -0- \\
NEW YORK & \$188,000 & 2,375,000 & 3,000,000 & 1,600,000 & 786,000 & -0- & \$7,949,000 \\
PHOENIX & -0- & 1,074,000 & -0- & -0- & -0- & -0- & \$1,074,000 \\
PORTLAND & -0- & 1,179,000 & -0- & -0- & -0- & -0- & \$1,179,000 \\
SACRAMENTO & -0- & 1,662,000 & 665,000 & 660,000 & -0- & 1,200,000 & \$4,187,000 \\
UTAH & -0- & 1,278,000 & 551,000 & 440,000 & -0- & -0- & \$2,269,000 \\
WASHINGTON & -0- & 1,380,000 & -0- & 1,120,000 & -0- & 664,000 & \$3,164,000 \\
& \$188,000 & \$14,768,000 & \$6,858,000 & \$13,777,000 & \$786,000 & \$3,864,000 & \$40,241,000 \\
\bottomrule()
\end{longtable}

\begin{enumerate}
\def\labelenumi{(\arabic{enumi})}
\tightlist
\item
  Included in CPS Report
\end{enumerate}

\hypertarget{team-sponsor-categories}{%
\chapter{TEAM SPONSOR CATEGORIES}\label{team-sponsor-categories}}

\hypertarget{team-sponsor-categories-1}{%
\section{TEAM SPONSOR CATEGORIES}\label{team-sponsor-categories-1}}

\begin{itemize}
\tightlist
\item
  Airlines
\item
  Amusement Park
\item
  Apparel/Children's
\item
  Apparel/Men's
\item
  Apparel/Women's
\item
  Appliances
\item
  Armed Forces
\item
  Auto Supply Stores
\item
  Batteries
\item
  Beer
\item
  Beverage
\item
  Book Stores
\item
  Cable Systems
\item
  Commerical Banking
\item
  Computer Products/Services
\item
  Camera
\item
  Convenience Stores
\item
  Copiers
\item
  Courier Services
\item
  Dairy Associations
\item
  Discount Stores
\item
  Drinking Water
\item
  Drug Stores
\item
  Electronics/products
\item
  Electronics/Retailers
\item
  Exercise Equipment
\item
  Eye Wear
\item
  Financial Institutions
\item
  Film
\item
  Food Products
\item
  Furniture
\item
  Gasline/Oil
\item
  Grocery
\item
  Hardware/products
\item
  Hardware/Retailers
\item
  Health Care
\item
  Health and Beauty Aids
\item
  Health Clubs
\item
  Heating \& Cooling
\item
  Insurance
\item
  Investment Banking
\item
  Isotonic Beverage
\item
  Jewelry Products and Retailers
\item
  Local Television and Radio Outlets
\item
  State Lotteries
\item
  Mass Merchandisers
\item
  Medical Equipment
\item
  Moving Companies
\item
  Motorcycle
\item
  Music Retailers
\item
  Office Supplies
\item
  Paint and Wallpaper
\item
  Personal Care Products
\item
  Quick Service Restaurants
\item
  Rental Car
\item
  Restaurants
\item
  Regional Transportation Systems
\item
  Retailers (General)
\item
  Realty/Real Estate
\item
  Snack Foods
\item
  Soft Drinks
\item
  Sporting Goods Products
\item
  Sporting Goods Retailers
\item
  Shopping Malls
\item
  Sportswear
\item
  Travel Agencies
\item
  Telecommunications
\item
  Telephone Companies
\item
  Telephone/Beepers
\item
  Telephone/Cellular
\item
  Telephone/Long Distance
\item
  Telephone/Yellow Pages
\item
  Movie Theatre
\item
  Tire
\item
  Ticket Systems
\item
  Public Utilities
\item
  Visitors/Convention Bureau
\item
  Vitamin/Food Supplement
\item
  Video Retailer
\item
  Video Production
\end{itemize}

\hypertarget{nba-nbpa-grievance-escrow-agreement}{%
\chapter{NBA-NBPA GRIEVANCE ESCROW AGREEMENT}\label{nba-nbpa-grievance-escrow-agreement}}

\hypertarget{nba-nbpa-grievance-escrow-agreement-1}{%
\section{NBA-NBPA GRIEVANCE ESCROW AGREEMENT}\label{nba-nbpa-grievance-escrow-agreement-1}}

Escrow Agreement, dated February 13, 1997, among the National Basketball Association (the ``NBA''), for itself and its member teams, the National Basketball Players Association (the ``Players Association''), and IBJ Schroder Bank and Trust Company (the ``Escrow Agent''):

\begin{enumerate}
\def\labelenumi{\arabic{enumi}.}
\item
  \begin{enumerate}
  \def\labelenumii{(\alph{enumii})}
  \tightlist
  \item
    When a Player who has been fined and/or suspended files a Grievance with respect to such fine and/or suspension and funds are to be deposited with the Escrow Agent pursuant to Article XXXI of the Collective Bargaining Agreement between the NBA and the Players Association entered into on July 11, 1996 (the ``CBA''), the Team employing such Player shall transmit such funds (to the extent withheld from the Player's Current Cash Compensation, net of any applicable taxes that would have been withheld from such funds had they been paid to the Player, or paid by the Player directly to the Team) promptly to the NBA, which shall then deposit them with the Escrow Agent by wire transfer of immediately available U.S. dollars.
  \item
    On any date on which the NBA initiates a wire transfer pursuant to paragraph I(a), the NBA shall give a notice by facsimile to the Players Association and the Escrow Agent setting forth the amount transferred, the reference number for the wire transfer and the name of the Player involved in the Grievance. If the amount transferred is the first deposit by the NBA with respect to the applicable Player and Grievance, the notice shall be in the form of exhibit I(b)(i); if the amount transferred is an additional deposit by the NBA with respect to the applicable Player and Grievance, the notice shall be in the form of exhibit 1(b )(ii). On the business day immediately following its receipt of a notice pursuant to this paragraph I(b), the Escrow Agent shall deliver a copy of that notice to the NBA, the Players Association and the Team involved, acknowledging receipt of those funds and their deposit into an escrow account pursuant to this agreement (or, if applicable, a separate notice advising that those funds had not been received as of 12:00 p.m.(NYT) on such next business day). All amounts deposited by the NBA with the Escrow Agent under this agreement shall be held in an escrow account bearing the name of the NBA, the name of the Team involved, the name of the Player, the date of the initial deposit and the words ``Grievance Escrow Account,'' which account shall be separate from all other accounts created under this agreement.
  \end{enumerate}
\item
  The Escrow Agent shall invest the amounts deposited with it pursuant to this agreement in the Escrow Agent's interest bearing money market account. All interest earned on those amounts shall be held and disposed of by the Escrow Agent pursuant to this agreement.
\item
  At least once each month during the term of this agreement, commencing with the month following the month in which the first escrow account is established, the Escrow Agent shall deliver to the NBA and the Players Association a statement reflecting the investment activity with respect to each escrow account established under this agreement during the prior month.
\item
  If the Escrow Agent receives a writing signed jointly by a person certifying that he is an NBA Officer and a person certifying that he is a Players Association Officer and giving the Escrow Agent directions with respect to all or part of the funds in a particular escrow account, the Escrow Agent shall comply promptly with such directions. For purposes of this agreement, an ``NBA Officer'' shall be any of the persons listed on Exhibit 4A and a ``Players Association Officer'' shall be any of the persons listed on Exhibit 4B. Exhibits 4A and 4B may be modified from time to time by a notice from the NBA (in the case of Exhibit 4A) or the Players Association (in the case of Exhibit 4B) to the Escrow Agent and the other party. Whenever this agreement provides for a writing to be delivered by a party to the Escrow Agent, the Escrow Agent shall only rely on a writing signed by an NBA Officer, if given on behalf of the NBA, and a writing signed by a Players Association Officer, if given on behalf of the Players Association.
\item
  Except as otherwise provided in paragraph 4, the Escrow Agent shall hold the funds in a particular escrow account until receipt of a copy of a final determination of the Grievance Arbitrator pursuant to the CBA directing the Escrow Agent to dispose of those funds (including any interest earned thereon). The Players Association and the NBA shall, or shall cause each of the Player (in the case of the Players Association) and the Team (in the case of the NBA) involved, respectively, to, (i) request such a determination as part of its request for relief from the Grievance Arbitrator in connection with the Grievance, (ii) advise the Grievance Arbitrator that any interest constituting escrowed funds should be allocated to the parties in proportion to the other amounts of escrowed funds determined to be payable to them, and (iii) submit to the exclusive jurisdiction of the Grievance Arbitrator to make such determination. Upon receipt of any such determination with respect to the disposition of escrowed funds, the Escrow Agent shall promptly comply with its terms. For purposes of this agreement, the ``Grievance Arbitrator'' shall be the person identified on Exhibit 5, which Exhibit may be modified from time to time by a notice from the NBA and the Players Association to the Escrow Agent. The Players Association and the NBA shall cause each Player and each Team, respectively, that is entitled to a distribution under this agreement to provide the Escrow Agent with his or its taxpayer identification number prior to receiving that distribution.
\item
  This agreement and the duties of the Escrow Agent hereunder shall terminate upon receipt by the Escrow Agent of a joint written notice of termination from the NBA and the Players Association, which notice shall direct the disposition of any amounts then held by the Escrow Agent pursuant to this agreement.
\item
  The Escrow Agent shall have no duty or responsibility not expressly set forth in this agreement and shall not be bound by or have any responsibility with respect to any agreement among any other persons, including the CBA. The Escrow Agent shall incur no liability whatsoever to the NBA or to the Players Association or to any NBA Team or Player (including, but not limited to, on account of a loss incurred from an investment made in accordance with paragraph 2), except for liabilities arising from the Escrow Agent's bad faith, gross negligence or willful misconduct. The Escrow Agent will be protected in acting upon any writing believed in good faith by the Escrow Agent to be genuine and containing what purports to be authentic signatures, and the Escrow Agent shall have no obligation to verify or investigate the accuracy of any such writing. The Escrow Agent may consult with its counsel and shall incur no liability for any action taken or omitted in good faith in accordance with the advice or opinion of such counsel. In the event that the Escrow Agent shall be uncertain as to its duties or rights hereunder, it shall, without liability of any kind, be entitled to hold the escrowed funds pending the resolution of such uncertainty to the Escrow Agent's satisfaction, or the Escrow Agent may, in final satisfaction of its duties hereunder, deposit the escrowed funds with the Clerk of the United States District Court for the Southern District of New York.
\item
  The NBA and the Players Association jointly and severally shall hold the Escrow Agent harmless and indemnify the Escrow Agent against any loss, liability, claim or demand arising out of or in connection with the performance of its obligations in accordance with the provisions of this agreement, except for losses, liabilities, claims or demands arising from the bad faith, gross negligence or willful misconduct of the Escrow Agent. The foregoing indemnities in this paragraph shall survive termination of this agreement or resignation of the Escrow Agent.
\item
  The Escrow Agent's fee for its services under this agreement is set forth on Schedule 1; such fee shall be paid in the manner set forth on Schedule 1.
\item
  The Escrow Agent may resign at any time by giving written notice thereof to the NBA and the Players Association, but such resignation shall not become effective until a successor escrow agent shall have been appointed and shall have accepted such appointment in writing. If an instrument of acceptance by a successor escrow agent shall not have been delivered to the Escrow Agent within 30 days after the giving of such notice of resignation, the Escrow Agent may petition the Grievance Arbitrator for the appointment of a successor escrow agent; and the NBA and the Players Association hereby agree that the Grievance Arbitrator shall have the authority to appoint such successor escrow agent. In the event that a successor escrow agent shall not have been appointed and the Escrow Agent shall not have turned over to the successor escrow agent the escrowed funds within 45 days after the giving of such notice of resignation, the Escrow Agent may deposit the escrowed funds with the Clerk of the United States District Court for the Southern District of New York, at which time the Escrow Agent's duties hereunder shall terminate.
\item
  Except as expressly provided in this agreement, the Escrow Agent shall not be bound by any purported waiver, amendment, modification, termination, cancellation or rescission of this agreement, unless the same shall be evidenced by a writing jointly signed by an NBA Officer and a Players Association Officer. No amendment or modification of this agreement shall affect the Escrow Agent's duties or responsibilities hereunder unless the Escrow Agent's written consent thereto shall have been obtained.
\item
  All notices required or permitted under this agreement shall be in writing and shall be considered given when delivered personally or sent by facsimile (with a copy by any other means permitted for the giving of notices under this agreement, unless such notice is sent pursuant to paragraph l(b)), or one day after delivery by a reputable overnight courier, to the parties at the addresses set forth below (or at such other address as a party may specify by notice similarly given):\\
  If to the Escrow Agent, to:\\
  IEJ Schroder Bank and Trust Company\\
  One State Street\\
  New York, New York 10004\\
  Attention: Corporate Trust Administration\\
  Facsimile: (212) 858-2952

  If to the NBA, to:\\
  National Basketball Association Olympic Tower\\
  645 Fifth Avenue\\
  New York, New York 10022\\
  Attention: General Counsel\\
  Facsimile: (212) 888-7931

  If to the Players Association, to:\\
  National Basketball Players Association\\
  1700 Broadway\\
  New York, New York 10019\\
  Attention: Counsel\\
  Facsimile: (212) 956-5687
\item
  This agreement constitutes the entire agreement of the parties with respect to its subject matter and shall be governed by and construed in accordance with the law of the state of New York applicable to agreements made and to be performed entirely in New York. This agreement shall be binding upon and shall inure to the benefit of the successors and assigns of the parties. This agreement may be executed in two or more counterparts, which together shall constitute one and the same instrument.
\end{enumerate}

NATIONAL BASKETBALL ASSOCIATION,\\
for itself and its member teams\\
By: /s/ David I. Stern\\
David I. Stern\\
Commissioner

NATIONAL BASKETBALL PLAYERS ASSOCIATION\\
By: /s/ G. William Hunter\\
G. William Hunter\\
Executive Director

IBI SCHRODER BANK AND TRUST COMPANY\\
By: /s/\_\_\_\_\_\_\_\_\_\_\_\_\_\_\_\_\_\\
Name:\\
Title:

\hypertarget{schedule-1}{%
\subsection{SCHEDULE 1}\label{schedule-1}}

\hypertarget{fees-of-escrow-agent}{%
\subsubsection{Fees of Escrow Agent}\label{fees-of-escrow-agent}}

The fees of the Escrow Agent for its services pursuant to this agreement shall be an inception fee of \$2,500, which shall be paid 50\% by the NBA and 50\% by the Players Association upon the execution of this agreement. In addition, the Escrow Agent shall be paid an annual custody fee of \$5,000, which shall be paid 50\% by the NBA and 50\% by the Players Association on each anniversary of the date of this agreement (or a pro-rated portion of that amount on the date this agreement otherwise terminates).

\hypertarget{exhibit-1bi}{%
\subsection{EXHIBIT 1(b)(i)}\label{exhibit-1bi}}

{[}LETTERHEAD OF NBA{]}

\begin{longtable}[]{@{}l@{}}
\toprule()
\endhead
{[}Escrow Agent{]} \\
\_\_\_\_\_\_\_\_\_\_\_\_\_\_\_\_\_\_\_\_\_\_\_\_\_\_\_\_\_\_\_\_\_\_\_\_ \\
\_\_\_\_\_\_\_\_\_\_\_\_\_\_\_\_\_\_\_\_\_\_\_\_\_\_\_\_\_\_\_\_\_\_\_\_ \\
Attention:\_\_\_\_\_\_\_\_\_\_\_\_\_\_\_\_\_\_\_\_\_\_\_\_\_\_ \\
\_\_\_\_\_\_\_\_\_\_\_\_\_\_\_\_\_\_\_\_\_\_\_\_\_\_\_\_\_\_\_\_\_\_\_\_ \\
\bottomrule()
\end{longtable}

Re: Initial Deposit pursuant to the NBA-NBPA Grievance Escrow Agreement dated February 13, 1997 (the ``Grievance Escrow Agreement'')

Dear Sir or Madam:

Please be advised that pursuant to paragraph l(a) of the Grievance Escrow Agreement, the National Basketball Association has today initiated a wire transfer to you in the amount of (U.S.) \$ \_\_\_\_\_ (reference no. ), which amount is the subject of a Grievance involving the {[}NBA or Insert Name of Team{]} and {[}Insert Name of Player.{]} Such amount (and all interest earned thereon) is to be held by you pursuant to the Grievance Escrow Agreement.

Very truly yours,

\begin{longtable}[]{@{}l@{}}
\toprule()
\endhead
National Basketball Association \\
By: \_\_\_\_\_\_\_\_\_\_\_\_\_\_\_\_\_\_\_\_\_\_\_\_\_\_\_\_\_\_\_\_\_\_\_\_\_\_ \\
Name: \_\_\_\_\_\_\_\_\_\_\_\_\_\_\_\_\_\_\_\_\_\_\_\_\_\_\_\_\_\_\_\_\_\_\_\_ \\
Title: \_\_\_\_\_\_\_\_\_\_\_\_\_\_\_\_\_\_\_\_\_\_\_\_\_\_\_\_\_\_\_\_\_\_\_ \\
Address for Acknowledgment \_\_\_\_\_\_\_\_\_\_\_\_\_\_\_ \\
\bottomrule()
\end{longtable}

cc:

\begin{longtable}[]{@{}l@{}}
\toprule()
\endhead
National Basketball Players Association \\
1700 Broadway \\
New York, New York 10019 \\
Attention: Counsel \\
\bottomrule()
\end{longtable}

\newpage

ACKNOWLEDGMENT

The undersigned acknowledges receipt of the funds referred to in the above notice and has deposited such funds in an escrow account entitled NBA/{[}Insert Team Name{]} - {[}Insert Player Name{]} Grievance Escrow Account dated \_\_\_\_\_\_\_\_\_\_\_\_\_, \_\_\_\_\_\_.

Very truly yours,

\begin{longtable}[]{@{}l@{}}
\toprule()
\endhead
{[}ESCROW AGENT{]} \\
By: \_\_\_\_\_\_\_\_\_\_\_\_\_\_\_\_\_\_\_\_\_\_\_\_\_\_\_\_\_\_\_\_\_\_\_\_\_\_ \\
Name: \_\_\_\_\_\_\_\_\_\_\_\_\_\_\_\_\_\_\_\_\_\_\_\_\_\_\_\_\_\_\_\_\_\_\_\_ \\
Title: \_\_\_\_\_\_\_\_\_\_\_\_\_\_\_\_\_\_\_\_\_\_\_\_\_\_\_\_\_\_\_\_\_\_\_ \\
\bottomrule()
\end{longtable}

\hypertarget{exhibit-1bii}{%
\subsection{EXHIBIT 1(b)(ii)}\label{exhibit-1bii}}

{[}LETTERHEAD OF NBA{]}

\begin{longtable}[]{@{}l@{}}
\toprule()
\endhead
{[}Escrow Agent{]} \\
\_\_\_\_\_\_\_\_\_\_\_\_\_\_\_\_\_\_\_\_\_\_\_\_\_\_\_\_\_\_\_\_\_\_\_\_ \\
\_\_\_\_\_\_\_\_\_\_\_\_\_\_\_\_\_\_\_\_\_\_\_\_\_\_\_\_\_\_\_\_\_\_\_\_ \\
Attention:\_\_\_\_\_\_\_\_\_\_\_\_\_\_\_\_\_\_\_\_\_\_\_\_\_\_ \\
\_\_\_\_\_\_\_\_\_\_\_\_\_\_\_\_\_\_\_\_\_\_\_\_\_\_\_\_\_\_\_\_\_\_\_\_ \\
\bottomrule()
\end{longtable}

Re: Additional Deposit pursuant to the NBA-NBPA Grievance Escrow Agreement dated February 13, 1997 (the ``Grievance Escrow Agreement'')

Dear Sir or Madam:\\
Please be advised that pursuant to paragraph 1(a) of the Grievance Escrow Agreement, the National Basketball Association has today initiated a wire transfer to you in the amount of (U.S.) \$ \_ \_\_ (reference no. ), which amount is the subject of a Grievance involving the {[}NBA or Insert Name of Team{]} and {[}Insert Name of Player.{]} Such amount (and all interest earned thereon) is to be held by you pursuant to the Grievance Escrow Agreement in the previously established NBA/ {[}Insert Team Name{]} - {[}Insert Player Name{]} Grievance Escrow Account dated \_\_\_\_\_\_\_\_\_\_\_\_\_, \_\_\_\_\_\_.

Very truly yours,

\begin{longtable}[]{@{}l@{}}
\toprule()
\endhead
National Basketball Association \\
By: \_\_\_\_\_\_\_\_\_\_\_\_\_\_\_\_\_\_\_\_\_\_\_\_\_\_\_\_\_\_\_\_\_\_\_\_\_\_ \\
Name: \_\_\_\_\_\_\_\_\_\_\_\_\_\_\_\_\_\_\_\_\_\_\_\_\_\_\_\_\_\_\_\_\_\_\_\_ \\
Title: \_\_\_\_\_\_\_\_\_\_\_\_\_\_\_\_\_\_\_\_\_\_\_\_\_\_\_\_\_\_\_\_\_\_\_ \\
Address for Acknowledgment \_\_\_\_\_\_\_\_\_\_\_\_\_\_\_ \\
\bottomrule()
\end{longtable}

cc:

\begin{longtable}[]{@{}l@{}}
\toprule()
\endhead
National Basketball Players Association \\
1700 Broadway \\
New York, New York 10019 \\
Attention: Counsel \\
\bottomrule()
\end{longtable}

\newpage

ACKNOWLEDGMENT

The undersigned acknowledges receipt of the funds referred to in the above notice and has deposited such funds in the previously established escrow account entitled NBA/{[}Insert Team Name{]} - {[}Insert Player Name{]} Grievance Escrow Account dated \_\_\_\_\_\_\_\_\_\_\_\_\_, \_\_\_\_\_\_.

Very truly yours,

\begin{longtable}[]{@{}l@{}}
\toprule()
\endhead
{[}ESCROW AGENT{]} \\
By: \_\_\_\_\_\_\_\_\_\_\_\_\_\_\_\_\_\_\_\_\_\_\_\_\_\_\_\_\_\_\_\_\_\_\_\_\_\_ \\
Name: \_\_\_\_\_\_\_\_\_\_\_\_\_\_\_\_\_\_\_\_\_\_\_\_\_\_\_\_\_\_\_\_\_\_\_\_ \\
Title: \_\_\_\_\_\_\_\_\_\_\_\_\_\_\_\_\_\_\_\_\_\_\_\_\_\_\_\_\_\_\_\_\_\_\_ \\
Date: \_\_\_\_\_\_\_\_\_\_\_\_\_\_\_\_\_\_\_\_\_\_\_\_\_\_\_\_\_\_\_\_\_\_\_\_ \\
\bottomrule()
\end{longtable}

\newpage

\hypertarget{exhibit-4a}{%
\subsection{Exhibit 4A}\label{exhibit-4a}}

Names of NBA Officers

\begin{itemize}
\tightlist
\item
  Jeffrey A. Mishkin
\item
  Joel M. Litvin
\item
  Richard W. Buchanan
\item
  Robert Criqui
\end{itemize}

\newpage

\hypertarget{exhibit-4b}{%
\subsection{Exhibit 4B}\label{exhibit-4b}}

Names of Player Association Officers

\begin{itemize}
\tightlist
\item
  G.William Hunter
\item
  Ronald Klempner
\item
  David Mondress
\end{itemize}

\newpage

\hypertarget{exhibit-5}{%
\subsection{Exhibit 5}\label{exhibit-5}}

Name of Grievance Arbitrator

\begin{itemize}
\tightlist
\item
  John D. Feerick
\end{itemize}

\hypertarget{notice-to-veteran-players-concerning-summer-leagues}{%
\chapter{NOTICE TO VETERAN PLAYERS CONCERNING SUMMER LEAGUES}\label{notice-to-veteran-players-concerning-summer-leagues}}

\hypertarget{notice-to-veteran-players-concerning-summer-leagues-1}{%
\section{NOTICE TO VETERAN PLAYERS CONCERNING SUMMER LEAGUES}\label{notice-to-veteran-players-concerning-summer-leagues-1}}

An arbitration award issued on June 13, 1977, stated that, in order for an NBA Team to enroll veteran players in a summer pro league, it is necessary for the Team to have such players sign a form signifying that they have chosen to participate in the league on a voluntary basis, and to inform such players of certain provisions. Players participating in any summer league shall be informed as to the following:

\begin{enumerate}
\def\labelenumi{\arabic{enumi}.}
\tightlist
\item
  Under the Uniform Player Contract and the Collective Bargaining Agreement between the NBA and the Players Association, the Team cannot require players to participate in any summer league.
\item
  The failure of a player to sign such a form to participate in any summer league will not, by itself, prejudice or disadvantage such player in his Team standing or relationship.
\item
  The Team reserves the right to determine how many and which players it may enroll in any summer league.
\end{enumerate}

We would appreciate your signing and returning the attached form to:

\begin{longtable}[]{@{}l@{}}
\toprule()
\endhead
\_\_\_\_\_\_\_\_\_\_\_\_\_\_\_\_\_\_\_\_\_ \\
\_\_\_\_\_\_\_\_\_\_\_\_\_\_\_\_\_\_\_\_\_ \\
Name of Team \\
\bottomrule()
\end{longtable}

\hypertarget{form-regarding-name-of-appropriate-summer-league}{%
\chapter{FORM REGARDING (NAME OF APPROPRIATE SUMMER LEAGUE)}\label{form-regarding-name-of-appropriate-summer-league}}

\hypertarget{form-regarding-name-of-appropriate-summer-league-1}{%
\section{FORM REGARDING (NAME OF APPROPRIATE SUMMER LEAGUE)}\label{form-regarding-name-of-appropriate-summer-league-1}}

This is to acknowledge that I have freely chosen to participate in the (name of appropriate summer league) on a voluntary basis during the summer of ( ).

\begin{longtable}[]{@{}l@{}}
\toprule()
\endhead
\_\_\_\_\_\_\_\_\_\_\_\_\_\_\_\_\_\_\_\_\_ \\
(Signature of Player) \\
\_\_\_\_\_\_\_\_\_\_\_\_\_\_\_\_\_\_\_\_\_ \\
(Printed Name of Player) \\
\bottomrule()
\end{longtable}

\hypertarget{side-letters}{%
\chapter{SIDE LETTERS}\label{side-letters}}

July 11, 1996

Mr.~Alex English\\
Acting Executive Director\\
National Basketball Players Association\\
1775 Broadway, Room 2401\\
New York, New York 10019

Dear Alex:

This will confirm our agreement that a team's termination of a Uniform Player Contract by reason of the player's ``lack of skill'' (under paragraph 16(a)(ii) of the UPC) shall be interpreted to include a termination based on the team's determination that, in view of the player's level of skill (in the sole opinion of the Team), the salary paid (or to be paid) to the player is no longer commensurate with the team's financial plans or needs. This agreement shall not affect any post-termination obligation to pay salary that may result from salary protection provisions included in a Uniform Player Contract.

If the foregoing coincides with your understanding of our agreement, please sign this letter in the space provided below.

Sincerely,\\
/s/ David J. Stern\\
David J. Stern

AGREED TO AND ACCEPTED:

NATIONAL BASKETBALL PLAYERS ASSOCIATION

By:
/s/ Alex English\\
Alex English, Acting Executive Director

\newpage

July 11, 1996

Mr.~Alex English\\
Acting Executive Director\\
National Basketball Players Association\\
1775 Broadway, Room 2401\\
New York, New York 10019

Dear Alex:

This will confirm our agreement that the NBA and the Players Association will use their respective best efforts to negotiate and conclude a new NBA/NBPA Anti-Drug Agreement prior to the start of the 1997-98 NBA Season.

From this date until the conclusion of the new NBA/NBPA Anti-Drug Agreement, referred to above, the parties agree to be bound by the, terms of the NBA/NBPA Anti-Drug Agreement contained in the 1988 Collective Bargaining Agreement and all side letters relevant thereto, including the side letters dated June 2, 1989; October 5, 1989; February 26, 1990; and September 4, 1991.

If the foregoing coincides with your understanding of our agreement, please sign this letter in the space provided below.

Sincerely,\\
/s/ David J. Stern\\
David J. Stern

AGREED TO AND ACCEPTED:

NATIONAL BASKETBALL PLAYERS ASSOCIATION

By:
/s/ Alex English\\
Alex English, Acting Executive Director

\newpage

July 11, 1996

Mr.~Alex English\\
Acting Executive Director\\
National Basketball Players Association\\
1775 Broadway, Room 2401\\
New York, New York 10019

Dear Alex:

This will confirm our agreement that the three-year waiting period set forth in Article VII, Section 7(c)(2) of the Collective Bargaining Agreement (``Agreement'') shall not apply to any Renegotiation executed prior to the effective date of the Agreement that involved an increase in performance bonuses of more than 10\% but did not increase the player's Salary for the then-current Season.

If the foregoing coincides with your understanding of our agreement, please sign this letter in the space provided below.

Sincerely,\\
/s/ David J. Stern\\
David J. Stern

AGREED TO AND ACCEPTED:

NATIONAL BASKETBALL PLAYERS ASSOCIATION

By:
/s/ Alex English\\
Alex English, Acting Executive Director

\newpage

July 11, 1996

Mr.~Alex English\\
Acting Executive Director\\
National Basketball Players Association\\
1775 Broadway\\
Suite 2401\\
New York, New York 10019

Dear Alex:

This will confirm our agreement that representatives of the Players Association will meet with representatives of the NBA within the next ninety (90) days for the purpose of agreeing upon appropriate circumstances under which, pursuant to Article 35(c) of the NBA Constitution, a player may be fined for making statements prejudicial or detrimental to the best interests of basketball, the NBA or an NBA Team.

If the foregoing coincides with your understanding of our agreement, please sign this letter in the space provided below.

Sincerely,\\
/s/ David J. Stern\\
David J. Stern

AGREED TO AND ACCEPTED:

NATIONAL BASKETBALL PLAYERS ASSOCIATION

By:
/s/ Alex English\\
Alex English, Acting Executive Director

\newpage

July 11, 1996

Mr.~Alex English\\
Acting Executive Director\\
National Basketball Players Association\\
1775 Broadway, Room 2401\\
New York, New York 10019

Dear Alex:

This will confirm our agreement that, notwithstanding the provisions of Article IV, Section l(a)(l) of the Collective Bargaining Agreement made as of September 18, 1995 and entered into on July 11, 1996 (the ``CBA''), providing for the amendment in certain respects of the National Basketball Association Players' Pension Plan (the ``Plan'') beginning with the 1996-97 Season, the NBA shall have the right, in its sole and absolute discretion, either (a) to proceed with the amendment of the Plan in the manner described in Article IV, Section 1(a)( 1)(iii) of the CBA (the ``Pre- 1965 Amendment'') but to defer the adoption of such amendment until no later than July 10, 1998, or (b) not to proceed at all with the Pre-1965 Amendment.

In the event the NBA elects to proceed with the Pre-1965 Amendment but to defer the adoption of such amendment as provided for above, the Pre-1965 Amendment when made shall be retroactive to the beginning of the 1996-97 season.

In the event the NBA elects not to proceed at all with the Pre-1965 Amendment, it shall provide new or enhanced benefits to NBA players (additional to those provided for by Article IV of the CBA) at an annual cost (as determined on an after-tax basis) to the NBA and/or NBA teams equal to the annual cost the NBA and/or NBA teams would have incurred had the NBA proceeded to make the Pre-1965 Amendment to the Plan commencing as of September 1, 1996. The NBA and the Players Association shall agree upon the type( s) of new or enhanced benefits to be provided.

\newpage

Mr.~Alex English
July 11, 1996
Page 2

If the foregoing coincides with your understanding of our agreement, please sign this letter in the space provided below.

Sincerely,\\
/s/ David J. Stern\\
David J. Stern

AGREED TO AND ACCEPTED:

NATIONAL BASKETBALL PLAYERS ASSOCIATION

By:
/s/ Alex English\\
Alex English, Acting Executive Director

\newpage

July 11, 1996

Mr.~Alex English\\
Acting Executive Director\\
National Basketball Players Association\\
1775 Broadway, Room 2401\\
New York, New York 10019

Dear Alex:

This will confirm our agreement that (i) the attached accounting procedures are the procedures currently in effect for purposes of Article VII, Section 8 of the Collective Bargaining Agreement made as of September 18, 1995 and entered into on July 11, 1996, and (ii) such procedures shall be modified by agreement of the parties as soon as practicable.

If the foregoing coincides with your understanding of our agreement, please sign this letter in the space provided below.

Sincerely,\\
/s/ David J. Stern\\
David J. Stern

AGREED TO AND ACCEPTED:

NATIONAL BASKETBALL PLAYERS ASSOCIATION

By:
/s/ Alex English\\
Alex English, Acting Executive Director

\newpage

Minimum Procedures To Be Provided By The Accountants

General

\begin{itemize}
\tightlist
\item
  The Audit Report must be prepared in accordance with the terms of the Collective Bargaining Agreement (``CBA''), which should be reviewed and understood by all auditors.
\item
  The Basketball Related Income Reporting Package and instructions should be reviewed and understood by all auditors.
\item
  All audit workpapers should be made available for review by representatives of the NBA and Players Association prior to issuance of the report.
\item
  A summary of all audit findings (including any unusual or non-recurring transactions) and proposed adjustments must be jointly reviewed with representatives of the NBA and Players Association prior to issuance of the report.
\item
  Any problems or questions raised during the audit should be resolved jointly with representatives of the NBA and Players Association.
\item
  All estimates should be reviewed in accordance with the CBA. Estimates are to be reviewed based upon the previous year's actual results and current year activity. All estimates should be confIrmed with third parties when possible.
\item
  Revenue and expense amounts that have been estimated should be reconfirmed with the controller or other team representatives prior to the issuance of the report on August 15.
\item
  Where possible, team and NBA revenues and expenses should be reconciled to audited financial statements.
\item
  Auditors should be aware of revenues excluded from BRI. All revenues excluded by the Teams or the NBA should be reviewed to determine proper exclusion. Auditor should perform a review for revenues improperly excluded from, or included in, BRI.
\end{itemize}

Team Salaries

\begin{itemize}
\tightlist
\item
  Trace amounts to the team's general ledger or other supporting documentation for agreement.
\item
  Foot all schedules and perform other clerical tests.
\item
  Examine the applicable player contracts for all players listed, noting agreement of all salary amounts for each player, in accordance with the definition of Salary in the CBA.
\item
  Compare player names with all player lists for the season in question.
\item
  Discuss method used to value non-cash compensation with the controller or other representative of each team and conclude as to its reasonableness.
\item
  Examine trade arrangements to verify that each team has properly recorded its pro rata portion of the players' entire salary based upon roster days, and that any bonuses or salary increases payable to players have been properly accounted for.
\item
  Inquire of controller or other representative of each team if any additional compensation was paid to players and not included on the schedule, whether or not paid for basketball services. Also inquire if any business arrangements entered into by the team or team affiliate with players of their affiliates, including with retired players who played for the team within the past five (5) years.
\item
  Review performance bonuses to determine whether such bonuses were actually earned for such season.
\item
  Review signing bonuses to determine if they have been allocated over the guaranteed years of the contract.
\item
  Confirm that, where provided in the CBA, certain contracts have been averaged either upon signing of the contract or upon assignment of the contract.
\end{itemize}

Benefits

\begin{itemize}
\tightlist
\item
  Trace amounts to the team's general ledger or other supporting documentation for agreement.
\item
  Foot all schedules and perform other clerical tests.
\item
  Investigate variations in amounts from the prior year through discussion with the Controller or other representative of the team.
\item
  Review each team's insurance expenses for premium credits (refunds) received from Planet Insurance Ltd.~(owned by Teams) and Prudential Insurance (amounts can be obtained from League Office).
\item
  Review League Office supporting documentation as to the following expenses:

  \begin{itemize}
  \tightlist
  \item
    Players Pension (including ``Pre 65'' palyers)
  \item
    Severance
  \item
    Disability Insurance
  \item
    HIV/ AIDS Programs
  \item
    High School Basketball Camp
  \item
    Playoff Pool
  \item
    Joint Labor Management Committee
  \end{itemize}
\end{itemize}

Basketball Related Income - Schedule 3

\begin{itemize}
\tightlist
\item
  Trace amounts to team's general ledger or other supporting documentation for agreement.
\item
  Foot all schedules and perform other clerical tests.
\item
  Trace gate receipts to general ledger and test supporting documentation where appropriate.
\item
  Gate receipts should be reviewed and reconciled to League Office gate receipts summary.
\item
  Verify amounts reported as luxury suite revenues with supporting documentation from the entity that sold, leased or licensed such luxury suites.
\item
  Verify amounts reported as complimentary tickets and tickets traded for goods or services with supporting documentation from the team.
\item
  Trace amounts reported for novelties and concessions, game parking, game programs, Team sponsorships and promotions, arena signage and arena club sales to general ledgers and test supporting documentation where appropriate.
\item
  Where reported amounts include proceeds received by an entity related to an NBA Team, verify the amounts reported with supporting documentation from the related entity.
\item
  Examine the National Television and Cable contracts at the League Office, and agree to amounts reported.
\item
  Review, at League Office, expenses deducted from the National contracts in accordance with the terms of the CBA. Review supporting documentation and test where applicable.
\item
  Examine local television, local cable and local radio contracts.
\item
  Verify to amounts reported by teams.
\item
  When local broadcast revenues are not verifiable by reviewing a contract, detailed supporting documentation should be reviewed and tested.
\item
  All loans, advances, bonuses, etc. received by the League Office or its teams should be noted in the report and included in BRI where appropriate.
\item
  Schedules of international broadcast, market extension, copyright royalty revenues and expenses should be obtained from the NBA. Schedules should be verified by agreeing to general ledgers and examining supporting documentation where applicable.
\item
  Schedules of revenues and expenses reported by Properties for sponsorship, NBA related revenues from NBA Entertainment, and NBA Special Events should be obtained from the NBA. Schedules should be verified by agreeing to general ledgers and examining supporting documentation where applicable.
\item
  Net exhibition revenues and expenses should be verified to supporting documentation where appropriate.
\item
  All amounts of other revenues should be reviewed for proper inclusion/exclusion in BRI. Test appropriateness of balances where appropriate.
\item
  Determine the ratio of expenses to revenues for those categories of proceeds that come within the provisions of Article VII, Part A, Section 1(a)(3) of the CBA and determine the extent to which expenses should be disallowed, if at all, pursuant to the provisions of that Section.
\end{itemize}

Playoff Revenues - Schedule 4

\begin{itemize}
\tightlist
\item
  All sources of playoff revenues and expenses should be verified per the procedure outlined for Basketball Related Income.
\item
  Because of the late timing of the Playoffs, special attention should be given to revenue and expenses estimates.
\item
  Playoff gate receipts should be recorded net of admission taxes. Payments made to the Playoff Pool should not be deducted. Odd game payments should not be either deducted by the paying team or recorded by the receiving team.
\item
  Other playoff expenses should be reviewed in accordance with the terms of the CBA.
\item
  Team expenses paid by the League Playoff Pool, including travel expenses, should not be deducted by teams.
\item
  Review League Office supporting documentation as to expenses deducted from the Playoff Pool.
\end{itemize}

Ouestions concerning Related Entity Transactions

\begin{itemize}
\tightlist
\item
  Review with controller or other representatives of the team the answers to all questions on this schedule.
\item
  Review that appropriate details are provided where requested.
\item
  Prepare summary of all changes.
\end{itemize}

List of Related Entities

\begin{itemize}
\tightlist
\item
  Review with controller or other representatives of the team all information included on the schedule of related entities.
\item
  Prepare a summary of any changes, corrections or additions to the schedule.
\item
  Review supporting details of any changes.
\end{itemize}

\newpage

July 11, 1996

Mr.~Alex English\\
Acting Executive Director\\
National Basketball Players Association\\
1775 Broadway, Room 2401\\
New York, New York 10019

Dear Alex:

This will confirm our agreement that, notwithstanding the terms set forth therein, (i) any Required Tender previously made by an NBA team to a player selected in the first round of the June 1996 Draft shall be deemed to remain open until the first day of the 1996-97 NBA regular season, and (ii) any Required Tender previously made to a player selected in the second round of the June 1996 Draft shall be deemed to remain open until September 5, 1996.

If the foregoing coincides with your understanding of our agreement, please sign this letter in the space provided below.

Sincerely,\\
/s/ David J. Stern\\
David J. Stern

AGREED TO AND ACCEPTED:

NATIONAL BASKETBALL PLAYERS ASSOCIATION

By:
/s/ Alex English\\
Alex English, Acting Executive Director

\end{document}
